\chapter{理论部分总结\label{chap:midterm}}

不知道有多少读者读完了前三章的内容。可能有百分之一?千分之一?感谢你们能读下来,接受这种信息密度的枯燥文本不是一件简单的事。辛苦了。

还有一部分从各种地方跳转至此的读者,感谢你们没有弃书。在接下来的总结中,我将假设读者都是看完前言直接跳转过来的,会尽量清晰易懂地介绍前三章的整体思路,以便读者查阅相关的部分。

\vspace{10pt}
\hrule
\vspace{10pt}

本指南的目标是尝试解决\indicate{如何用过于简单的自我意识去面对和处理过于复杂的现实世界}的问题。简单来说,解决方法是\indicate{提升自身思维的系统性,逐渐使自己能够容纳对复杂现象的完整思考}。

理解这个目标需要以下几步:

\indicate{(1)理解每个概念都在说什么。}这句话中有两个必须解释的概念:“自我意识”和“简单\&复杂”。二者分别是第二章和第三章的基础概念。

\indicate{自我意识}:本指南为自我意识构建了一个三个层次的模型,最主要的部分见\hyperref[ref:意识的层次]{2.2节}。我们将\hyperref[def:行为]{\indicate{行为}}视为意识活动的最基础单位,并将多个行为组成的,在一段时间之内稳定存在,具有固定特点的现象称为\hyperref[def:行为模式]{\indicate{行为模式}}。不同的行为模式决定了人在面对外部环境时的具体行为。我们可能使用“能力”“性格”“态度”等词语来称呼行为模式。本指南认为\hyperref[def:意识现象]{\indicate{人的意识现象}}由多个行为模式组成,不同的行为模式(以及散在的行为)依照环境而触发,使人与外界交互。

\indicate{简单\&复杂}:本指南构建了\hyperref[def:复杂系统]{\indicate{复杂系统}}的模型来解释简单和复杂,具体介绍见\hyperref[sec:复杂系统]{3.2节}。我们引入了\indicate{微观现象}和\indicate{宏观现象}这一对概念,将微观现象之间通过相互影响,组成的更大的现象称为宏观现象,将这个过程称为\indicate{编织}。行为会编织出行为模式,而行为模式编织出人的意识现象。我们将“现实世界”拆分成一个个更具体的宏观现象来考察。

\indicate{(2)理解这个问题为什么会存在。}这需要解释三个方面:“简单的自我意识有多简单”、“为什么我们需要面对和处理复杂现象”、“用简单的自我意识处理复杂现象会出什么问题”。

\indicate{简单的自我意识有多简单}:本指南的标准是“所有的行为和行为模式都是从环境中被动习得,没有成功主动改变过自身的行为”。具体从环境中习得行为的机制见\hyperref[sec:刺激强度与外溢]{3.4.1小节}。本指南的所有方法介绍均以这种水平的自我管理能力为基础,更加有能力的读者可以依照自身情况参考书中需要的部分。我们不区分“没有改变的意识”“尝试过改变但失败”等情况,认为只要有阅读的能力,就有可能理解并掌握本指南中介绍的内容。

\indicate{为什么我们需要面对和处理复杂现象}:这出于两个现实原因:其一是因为现代媒体的高速发展使得任何一个人都会接触海量的信息。这些信息中有一大部分都和复杂现象相关,而其中又有一部分是我们必须面对的东西。其二是因为我们需要对自己的人生负责。脱离能力而空谈责任没有意义,具体论述见\hyperref[sec:责任与要求]{3.3.3小节}。必须具备处理现实事务的能力,才能处理这些客观存在,必然面对的东西。我们只能选择是先面对才知道自己要面对(可能意识到的时候已经是二十年后了),还是先发现再面对。

\indicate{用简单的自我意识处理复杂现象会出什么问题}:最大的问题就是无法正确认识世界,也无法达成目标。我们在\hyperref[sec:污染]{2.2.4小节}中将这种既不理解也不可控的东西称为\indicate{污染},并在第三章中具体介绍了污染的形成机制。我们容易过度简化地概括事务之间的联系,草率地触碰\hyperref[def:能力边界]{\indicate{能力边界}}之外的东西。这种现象在面对复杂系统时尤为明显,甚至越熟悉某个复杂系统,情况就会越糟,见\hyperref[sec:对复杂现象的分析]{3.2.4小节}。简单的自我意识相当混乱,会得出大量不靠谱的认知。这种情况下是否能完成目标,完全不取决于我们的努力。

如果读者的自身情况不符合以上三个基本条件,那么本指南的内容对您可能是无用的。如果对您来说还有其它方面的作用(如推荐给其他人),请按照自身实际情况便宜行事。

\indicate{(3)理解这个问题会有什么难以解决的难点。}难点集中在“简单的自我意识能力过弱”上,具体有两方面:

\indicate{简单的自我意识无法有效控制自身}:需要注意的是,\hyperref[def:意识现象]{人的意识现象}是一个复杂系统。简单的自我意识普遍地缺乏处理复杂现象的能力,无法有效控制自身意识现象的其它部分,具体讨论见\hyperref[sec:干扰与打断]{3.4.3小节}。这里的控制指的不是“完善地知道每一个细节”,而是“按照预期达成目标”。关于什么是“控制自身意识”,参见\hyperref[sec:自控]{2.3.3小节}。

\indicate{简单的自我意识无法用来理解世界}:简单的自我意识相当混乱,会包含大量不受控的行为和带有污染的认知。这些东西会主导我们的行为,具体见\hyperref[sec:全局行为模式]{3.4.2小节}。每当我们想做一些长期的事情,总会被它们打断。我们绝不可能在短时间内就充分认识一个具体的复杂现象,也绝不可能通过一次努力就解决一个深层次问题。过于简单的自我意识无法支撑我们完成复杂的工作。

\indicate{(4)明确解决问题的方法。}本指南将足够复杂的自我意识称为\hyperref[def:理性]{\indicate{理性}},将拥有理性所对应的能力称为\hyperref[sec:现实事务处理能力]{\indicate{现实事务处理能力}}。这一能力有以下四个方面:

\indicate{分析能力}:正确认识自身的能力边界,总是使用恰当的方式分析问题;

\indicate{调研能力}:明确并获取处理事务所需要的所有信息和知识;

\indicate{计划能力}:将复杂事务按照执行能力拆分成可以完成的简单事务;

\indicate{执行能力}:运用相应的知识体系完成目标中的简单事务。

除了上述提到的内容外,前三章中还有大量内容用于补充细节。本指南力求搭建一套完整的分析框架,使每位读者都能尽量顺畅地应用。这些内容可以补全读者在具体细节方面的问题。

对于状态较好的读者来说,前三章的内容应该已经足够用来构建理性。但状态不好的读者会受到自身意识的强烈干扰,这些抽象的理论不足以用于解决现实中的问题。再完美的理论落不了地也没用,本指南的后续内容会偏向于具体应用,希望可以对这部分读者有所帮助。

虽然也可以选择从此处继续向后阅读,但若读者有确切的自我改变需求,我推荐至少通读一遍前三章。阅读顺序没有严格的限制,读者可以从上面介绍过的概念中挑选自己感兴趣的开始阅读,但最好确认自己能理解相关段落的用词。如果最终的目标是熟练应用,那么应该不止会翻一遍这本书。在熟练掌握之前,本指南仍然可以起到辅助记忆、整理思路的作用。

前三章的总体可读性较差,仅能起到大致的观点介绍作用,在具体行文和整体安排框架上均有考虑不周之处。这是我经验不足所致,由此造成的阅读障碍我深表抱歉。其中的观点可能有考虑不周之处,甚至可能有严重错误,希望读者不吝指出。