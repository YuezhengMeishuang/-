\chapter{理论部分总结\label{chap:midterm}}

不知道有多少读者读完了前三章的内容。可能有百分之一?千分之一?感谢你们能读下来,接受这种信息密度的枯燥文本不是一件简单的事。辛苦了。

还有一部分从各种地方跳转至此的读者,感谢你们没有弃书。在接下来的总结中,我将假设读者都是看完前言直接跳转过来的,会尽量清晰易懂地介绍前三章的整体思路,以便读者查阅相关的部分。

% \vspace{10pt}
% \hrule
% \vspace{10pt}
\divider

本指南的目标是尝试解决\indicate{如何用过于简单的自我意识去面对和处理过于复杂的现实世界}的问题。简单来说,解决方法是\indicate{提升自身思维的系统性,逐渐使自己能够容纳对复杂现象的完整思考}。

理解这个目标需要以下几步:
\vspace{10pt}

\indicate{(1)理解每个概念都在说什么。}这句话中有两个必须解释的概念:“自我意识”和“简单\&复杂”。二者分别是第二章和第三章的基础概念。

\indicate{自我意识}:本指南为自我意识构建了一个三个层次的模型,最主要的部分见\hyperref[ref:意识的层次]{2.2节}。我们将\inote{行为}视为意识活动的最基础单位,并将多个行为组成的,在一段时间之内稳定存在,具有固定特点的现象称为\inote{行为模式}。不同的行为模式决定了人在面对外部环境时的具体行为。我们可能使用“能力”“性格”“态度”等词语来称呼行为模式。本指南认为\hyperref[def:意识现象]{\indicate{人的意识现象}}由多个行为模式组成,不同的行为模式(以及散在的行为)依照环境而触发,使人与外界交互。

\indicate{简单\&复杂}:本指南构建了\inote{复杂系统}的模型来解释简单和复杂,具体介绍见\hyperref[sec:复杂系统]{3.2节}。我们引入了\indicate{微观现象}和\indicate{宏观现象}这一对概念,将微观现象之间通过相互影响,组成的更大的现象称为宏观现象,将这个过程称为\indicate{编织}。行为会编织出行为模式,而行为模式编织出人的意识现象。我们将“现实世界”拆分成一个个更具体的宏观现象来考察。
\vspace{10pt}

\indicate{(2)理解采用这种模型的原因。}这需要解释两个方面:“为什么不采用更通用的心理学术语”和“为什么要单独为复杂性建立模型”。

\indicate{为什么不采用更通用的心理学术语}:这出于两个原因。一方面是这些术语在长期使用中,已经出现了很多歧义,会为讨论带来障碍。即使本指南明确给出定义,也难免会受到其它义项的影响。另一方面是一些术语本身较为模糊。将一切都归因于“情绪”“内耗”“自我意识”的方式无法精细描述客观存在的意识现象,必须使用其它方法来补足细节。

\indicate{为什么要单独为复杂性建立模型}:本指南需要刻画以下两方面的复杂性:一方面是成因复杂,由大量微观现象编织而成;另一方面是解决复杂,难以使用简单且固定的方法应对,具体讨论见\hyperref[sec:对复杂系统的分析]{3.2.3小节}。满足这两方面,且会频繁出现,持续对我们的生活造成影响的现象,便是我们需要面对和处理的复杂现象。
\vspace{10pt}

\indicate{(3)理解这个问题为什么会存在。}这需要解释三个方面:“简单的自我意识有什么特征”、“我们需要面对和处理哪些复杂现象”、“用简单的自我意识处理复杂现象会出什么问题”。

\indicate{简单的自我意识有什么特征}:本指南的标准是“所有的行为和行为模式都是从环境中被动习得,没有成功主动改变过自身的行为”。具体从环境中习得行为的机制见\hyperref[sec:并行与竞争]{3.4.1小节}。本指南的所有方法介绍均以这种水平的自我管理能力为基础,更加有能力的读者可以依照自身情况参考书中需要的部分。我们不区分“没有改变的意识”“尝试过改变但失败”等情况,认为只要有阅读的能力,就有可能理解并掌握本指南中介绍的内容。

\indicate{我们需要面对和处理哪些复杂现象}:现代媒体的高速发展使得任何一个人都会接触海量的信息。这些信息中有一大部分都和复杂现象相关,而其中又有一部分问题是我们无法有效处理,只能无助恐慌焦虑愤怒绝望。同时,我们需要对自己的人生负责。脱离能力而空谈目标没有意义,具体论述见\hyperref[sec:责任与要求]{3.3.4小节}。必须具备处理现实事务的能力,才能处理这些客观存在,必然面对的东西。我们只能选择是先面对才知道自己要面对(可能意识到的时候已经是二十年后了),还是先发现再面对。

\indicate{用简单的自我意识处理复杂现象会出什么问题}:最大的问题就是无法正确认识世界,也无法达成目标。我们在\hyperref[sec:污染]{2.2.4小节}中将这种既不理解也不可控的东西称为\indicate{污染},并在第三章中具体介绍了污染的形成机制。简单的自我意识容易过度简化地概括事务之间的联系,草率地触碰\inote{能力边界}之外的东西,无法也不去分辨自己是否真的知道具体情况,能力是否足够处理这些事。这种现象在面对复杂系统时尤为明显,甚至越熟悉某个复杂系统,情况就会越糟。

如果读者的自身情况不符合以上三个基本条件,那么本指南的内容对您可能是无用的。如果对您来说还有其它方面的作用(如推荐给其他人),请按照自身实际情况便宜行事。
\vspace{10pt}

\indicate{(4)理解这个问题会有什么难以解决的难点。}难点集中在“简单的自我意识能力过弱”上,具体有三方面:

\indicate{无法有效理解世界}:简单的自我意识相当混乱,会包含大量不受控的行为和带有污染的认知。这些\inote{大道理}会主导我们的行为,具体见\hyperref[sec:全局行为模式]{3.4.2小节}。我们只能不停地从大道理出发开始思考,最终得到新的大道理。这些大道理都脱离现实,要不然充满了认识错误,要不然是正确的废话。我们绝不可能在短时间内就充分认识一个具体的复杂现象。

\indicate{无法有效达成目标}:简单的自我意识生产的那些大道理会使我们得到错误的认知。我们会无法理解那些复杂的事情到底困难在哪里,觉得自己能成功但实际情况会失败,于是一边盲目乐观一边盲目悲观。过于简单的自我意识无法支撑我们完成复杂的工作,这种情况下是否能完成目标,完全不取决于我们的努力。解决那些深层次的问题靠的绝对不可能只是态度、热情、冲劲、一次努力,而必须是充足的认知、完善的判断、周密的计划、实在的执行。

\indicate{无法有效控制自身}:\hyperref[def:意识现象]{人的意识现象}是一个复杂系统。简单的自我意识普遍地缺乏处理复杂现象的能力,无法有效控制自身意识现象的其它部分。每当我们想做一些长期的事情,总会被它们打断。具体讨论见\hyperref[sec:缺失与外溢]{3.4.2小节}。需要注意的是,控制自身是一种目标,具体需要做什么按照具体情况而定,这里的控制指的不是“完善地知道每一个细节”。关于什么是“控制自身意识”,参见\hyperref[sec:自控]{2.3.3小节}。
\vspace{10pt}

\indicate{(5)明确解决问题的方法。}本指南将足够复杂的自我意识称为\inote{理性},将拥有理性所对应的能力称为\hyperref[sec:现实事务处理能力]{\indicate{现实事务处理能力}}。这一能力有以下四个方面:

\indicate{分析能力}:正确认识自身的能力边界,总是使用恰当的方式分析问题;

\indicate{调研能力}:明确并获取处理事务所需要的所有信息和知识;

\indicate{计划能力}:将复杂事务按照执行能力拆分成可以完成的简单事务;

\indicate{执行能力}:运用相应的知识体系完成目标中的简单事务。

除了上述提到的内容外,前三章中还有大量内容用于补充细节。本指南力求搭建一套完整的分析框架,使每位读者都能尽量顺畅地应用。这些内容可以补全读者在具体细节方面的问题。

\divider

解决某一个具体的问题或许较为容易,但这一问题的解决方法如果会加重其它问题或是产生新问题,则解决它未必是一件好事。这要求我们不能将每个问题孤立看待,必须整体考虑所有可能发生的情况。我们不止需要处理每一个单独的心理问题。它们虽然成因复杂,但大多数可以在短时间内(如每10个小时的心理咨询以内解决一个的速度)理清并解决。本指南的主要着眼点是那些全局性的,参与了性格主要组成,难以总结出,并且即使发现也没能力解决的问题。

但现实情况是,我们通常既缺乏发现问题的视角,又缺乏处理问题的能力。但绝不应该因此就觉得“遇不到什么很复杂的问题,可以不管”。以解决表面的简单问题为终点,只会失去对深层次复杂问题的认知。越是缺乏能力的人就越是会独自面对复杂问题,同时越是无法发现问题整体,只能发现“不断有新的小问题冒出来”。

读者可能会注意到,本指南在行文过程中着重描写“什么时候会处理不了问题,为什么会失败”,关于失败的解释和分析通常比关于成功的更全面细致。这有四方面原因:1、本指南的理论框架本身就是对待现实的有效思维模式,应该也算入对成功的讨论篇幅;2、如果成功了,就不需要再用到本指南了,只有当前的理论无效时,才需要继续阅读;3、我们需要正确认识这些问题的复杂性和困难程度,避免因为认识过于简单而采取无效甚至会起反面效果的操作;4、越简单的问题越好处理,处理不了的问题往往更复杂,需要更系统的分析视角。这需要更多讨论篇幅,而这些篇幅被安排在了更后面的位置。

对于状态较好的读者来说,前三章的内容应该已经足够用来构建理性。但状态不好的读者会受到自身意识的强烈干扰,这些抽象的理论不足以用于解决现实中的问题。再完美的理论落不了地也没用,本指南的后续内容会偏向于具体应用,希望可以对这部分读者有所帮助。

虽然也可以选择从此处继续向后阅读,但若读者有确切的自我改变需求,我推荐至少通读一遍前三章。阅读顺序没有严格的限制,读者可以从上面介绍过的概念中挑选自己感兴趣的开始阅读,但最好确认自己能理解相关段落的用词。如果最终的目标是熟练应用,那么应该不止会翻一遍这本书。在熟练掌握之前,本指南仍然可以起到辅助记忆、整理思路的作用。

我姑且认为那些走投无路的读者更有可能读完本指南,其它人不一定坚持得下来,会有其它东西分心;走投无路的读者也不一定是真的坚持下来了,大多数应该也只是没别的事好干,然后慢慢磨完了。如果您凭借渊博的知识、持久的习惯或坚韧的毅力读完了本指南,请接受我的敬意。

前三章的总体可读性较差,仅能起到大致的观点介绍作用,在具体行文和整体安排框架上均有考虑不周之处。这是我经验不足所致,由此造成的阅读障碍我深表抱歉。其中的观点可能有考虑不周之处,甚至可能有严重错误,希望读者不吝指出。

\newpage

\chapter{应用部分前言\label{sec:应用部分前言}}
为了方便各位读者,我们在进入应用部分之前,先简单罗列一下后文会频繁使用到的概念与分析思路:

\begin{itemize}
\item 本指南主要关注\inote{行为}与\inote{行为模式}层次的问题。人的意识活动是主观的,但“人有意识活动”是客观现象,可以观察和分析。由“主观”提供的复杂性在于“在同一情况下的行为因人而异”。我们如果想要\inote{客观}地分析意识现象,就必须且仅需确保自己从现实出发,对分析对象有充分了解,足以\inote{模拟}分析对象。
\item 本指南认为,我们总是以某个一个行为模式为出发点,来观察与分析。这些行为模式可能有不同的特点,因而会对同一现象产生不同的看法,具体讨论请见\hyperref[sec:解读]{4.4节}。为了方便起见,当我们分析另一个行为模式时,会将其简称为“人”,使用“对方”等称呼。仅在少数需要特别强调之处(如分析多个行为模式相互之间的影响时),我们会使用“对方的这一行为模式”这种更长的称呼。读者应时刻留意,切勿以偏概全。
\item 理论部分提出了\inote{人的意识演化基本模型},将“人的行为模式会随经历缓慢变化”这一现象的主要的机制归结为\inote{外溢}。外溢是指“在信号和行为之间错误地建立了关联”的现象。应用部分将会和大家一起分析“什么样的关联是不真实的”,并且参考这一机制,得到培养正确行为的方法。
\begin{explain}
需要注意的是,识别“什么样的关联是不真实的”不需要回顾这一关联形成的过程,改变相应认知也不一定需要。我们只需要让其\inote{可控}即可。但是培养新行为时,一定要充分\hyperref[def:理解行为]{\indicate{理解}}这种机制。具体讨论见第六章。
\end{explain}
\item 本指南推荐各位读者在阅读时,不带有任何先入为主的\inote{价值观},不要代入任何角色和身份,不要为例子和论述中的观点强加任何\indicate{责任}、\indicate{意义}和\indicate{作用},不要在充分了解之前就草率地下判断。我们应该优先关注能力方面的问题,即“当事人是否有能力解决所面对的事务”与“读者是否有能力客观地分析对应的情况”。我们唯一应该使用的价值观是\indicate{就事论事}。在应用部分中,本指南会将不符合这一标准的现象,如行为外溢、\inote{表面分析}等,称为“坏的”“不应该的”。请读者注意,这不是指“某一种具体行为不应该”,而是指“脱离了该行为的应用范围”这一现象不应该。
\item 本指南给出的定义中包含“是xxx的yyy”“xxx的yyy”“对xxx的yyy”“关于xxx的yyy”等表述,其中xxx是概念yyy的前提条件。在使用本指南中的概念时,请读者务必留意其适用范围。如果忽视前提条件,随意使用这些概念,就可能使结论外溢,从而失去意义。文中有时为了行文方便,会省略前提,读者应结合上下文以自行补全。
\item 本指南会有“将xxx视作yyy”一类的表达。这种表达的意思是“yyy所具有的性质xxx都有,所以讨论yyy时得到的结论在xxx处也可以用”。xxx的其它方面则不参与讨论。这并不是说“我们总可以把xxx当作yyy,xxx的其它方面不重要”,只是为了当前段落的简便。我们仍然需要在其他地方单独处理xxx的其它方面。
\item 本指南所介绍内容中,最重要的部分是\indicate{概念在分析框架中的应用}。分析框架是一个完整的系统,不应该将其中的某些内容单独拿出理解。读者如果觉得“某一句/段话很有道理”,请核实自己是否可以从本指南所介绍分析框架中,独立总结出相关观点。本指南不建议读者将特别看重某一句话。一句印象深刻的话只应起到提醒的作用,使得我们可以想起并使用分析问题的框架。
\item 很多概念,特别是和意识强相关的概念,因使用者的不同,所指代的具体内容可能有极大差别。为这些概念下一个客观定义是不可取的行为,无论怎么下都无法普遍地覆盖所有人的感受,总会有人觉得“不是这么回事”。于是,我们转而用“每个人自身的感受”来定义这些概念,并将这种定义方式称为\indicate{主观定义}。这些概念具有某些共同的特点,从而我们可以分析它们的影响。但在涉及到“主观定义的形成”等环节时,不应继续使用主观定义来分析问题。
\begin{explain}
\trained 可能更习惯\indicate{泛性质定义}、\indicate{现象学定义}等其它术语。本指南出于贴近日常用语的考虑,选择“主观”一词。
\end{explain}
\item 本指南会特别关注“一个人是否形成了某种认知”,只有明确的认知才能产生后续影响。在对行为归因时,本指南会尽量避免使用“其实是想被关注”“其实是想被回应”这一类用语。我们按照对应的方式来与其互动确实可以解决问题,但如果没有形成明确认知,这只应视为对应行为模式整体所体现出的特点,而不应该被视为原因。我们有必要更深入具体地研究这些行为模式,而不只是草率地表面归因。更详细的展开讨论请见\hyperref[sec:对竞争过程的解读]{4.4.3小节}。
% \item 为隐私考虑,在本章以及后续的所有内容中,凡是涉及案例分析的环节,事件主角的名称均使用\source{泠珞}替代。如果案例中会出现咨询师,则名称均使用\source{零羽}替代。如果案例中出现其它人物,会使用龙吟、辰柯、墨默等名称替代\footnote{这些名字都是《妄想症》系列的角色,\moegirl{妄想症}。}。读者请勿在此产生有关性别、身份或其它方面的刻板印象。
\end{itemize}