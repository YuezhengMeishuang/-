\begin{savequote}[400pt]
    \fontsize{8pt}{8pt}
    请赐予我平静,去接受我无法改变的;给予我勇气,去改变我能改变的;赐给我智慧,分辨这两者的区别。\footnote{出自雷因霍德·尼布尔的祈祷文。相关如下:grant me the serenity to accept the things I cannot change; \\\indent courage to change the things I can; and wisdom to know the difference. \\\indent 尼布尔并未命名,后世将其命名为《宁静祷文》。}
    \qauthor{雷因霍德·尼布尔(Reinhold Niebuhr)}
\end{savequote}
\chapter{事务处理的三要素}
%\titleformat{\section}{\bfseries}{}{0em}{}
\section{从买水开始说起}

为了提升事务处理的能力,我们首先需要明白应该如何理解“事务”。比如,大多数人会不可避免地曾经经历过这么一件事:\indicate{怎么买一瓶水?}

对大多数人来说这不是什么问题,大家都顺利地买过水。但是对于少部分人来说,这却是不可逾越的障碍。我们可以随意列出几种买水的方案:
\begin{itemize}
\item 柜台型:向店员说出自己的需求,然后接过店员递来的商品(这种买水流程在现如今已经较为少见,不过在饮品店依旧广泛存在);
\item 超市型:自己挑选好水以后,去收银台结账(在超市和小店均有这种类型);
\item 自选型:自己挑好水以后通过无人柜台结账(超市的无人结账区、自动售货机、无人超市均为这种类型);
\item 外卖型:直接下单,然后等待非接触送达。
\end{itemize}
以上四种都是可行的流程,大多数人也都能顺畅执行。但总有少部分人会遇上各种各样层出不穷的问题,对于一个社恐的人,前两种需要人际交互的流程算是没什么指望了;对于一个不熟悉电子设备的人,自如地使用后两种流程是不可能的。当我们想要教他们克服障碍的时候,我们经常发现他们无论如何都听不进去。

无论你是告知孩子“这没什么好怕的”,和孩子展开分析和讨论,或者直接命令孩子去做,孩子始终不会如你所想,克服了心理障碍,大大方方地去买水,而只是一如往常地低着头站在角落,看手机或者玩衣角;无论你是给长辈找教程,手把手地教长辈怎么点开app,还是贴心地帮长辈绑好银行卡和账户,长辈在买水的时候仍然永远不会想到网购,去超市的时候永远走人工通道,遇到电子设备就说“自己不会”。

对于不同的人,同一套操作的执行难度可能会有很大差别。上述两类人在买水的时候,都会遇到\indicate{不可解决的困难}。困难有可能是纯粹的心理因素,但如果不想着去解决困难,那么他们就不可能被教会。当第一遍教学没有起到效果时,后续重复的教学也将恰如第一遍一样不会起到效果。他们不是不知道买水的正确做法,而是无法执行。

当然,也有“不知道正确做法”类型的困难。对于“把一个无法顺利买水的人教会”的事务,如果不知道有效的教育方法,只是一味重复,那就不会成功;对于“克服自己的社恐”的事务,如果不知道自身恐惧的来源和正确的社交方式,只是一味害怕自己做错事,那就不会成功;对于“接触电子设备”的事务,如果不了解电子设备,也不积极接触,只是一味担心自己误操作导致不良后果,那就不会成功。

为了之后叙述方便,我们先不严谨地\footnote{毕竟现在我们既没有定义“事务”,也没有定义“复杂”和“能力”。这些定义会在稍后补上。}把事务划分为两类:\indicate{简单事务}和\indicate{复杂事务}。概括地说,简单事务是指在自身能力范围之内,自身能够顺利解决的事务;与之相对,复杂事务是指自身有能力缺失,从而无法顺利解决的事务。“买水”对于大多数人来说是一项简单事务,对少部分人来说是一项复杂事务;“把一个无法顺利买水的人教会”对大多数人来说是一项复杂任务;“克服自己的社恐”对于社恐的人是复杂事务;“接触电子设备”对电子文盲是复杂事务。

\begin{examples}
任何形式的“克服困难”都是复杂事务,因为困难本身就说明了这个人有能力缺失(但“学习”不一定是复杂事务)。对大多数人来说,“自己产生感情”一般而言算不上是事务,但“理解自己为什么会产生某种感情”是复杂事务,“理解他人为什么会产生某种感情”也是复杂事务。“规划自己的未来”是复杂事务。“找到人生意义”是复杂事务。“脱离原生家庭影响”是复杂事务。“追寻自由”是复杂事务。“获得坚定的意志力”是复杂事务。“制定并完成长期目标”是复杂事务。“学会如何画一幅好看的画”是复杂事务。“学会如何写一手好听的歌”是复杂事务。“挑选合适的礼物”是复杂事务。“改变别人”是复杂事务。“得到别人的爱”是复杂事务。“把某项技能教给别人”是复杂事务。“为别人操心”是复杂事务。“和别人合作”是复杂事务。“搭人脉走关系”是复杂事务。“带领团队”是复杂事务。“维持体系平稳运行”是复杂事务。
\end{examples}
面对复杂事务,直接动手处理不会有什么效果。不排除一些人在一些方面具有充足的能力,但对于大多数人来说,如果恰巧达成了目标,只能说明你运气比较好,无法达成目标才是正常情况。面对复杂事务总是失败,\indicate{不是态度问题,而是能力问题}。

本指南的前三章用于讲解事务处理所需的能力。其中本章讲解事务处理的通用流程,第二章讲解个人所需能力的具体内容,第三章讲解复杂性的由来和对待复杂性的方式。

\section{处理复杂事务的相关能力\label{sec:现实事务处理能力}}
本指南使用\indicate{事务}这一词汇时,一般有两种含义:一种是指“一个目标”,另一种是指“处理事务的所有操作”,其中\indicate{处理}是指“从当前情况出发,为实现目标而执行一系列操作”。在一些语境下,我们也使用\indicate{目标}、\indicate{改变}来替代\indicate{事务}。

在这些操作中,我们把可以实现目标的称为\indicate{有效操作},\indicate{解决}或\indicate{完成}则特指“执行有效操作”;把不能实现目标的称为\source{无效操作},“解决/完成/处理事务\indicate{失败}”特指“执行无效操作”。在其它时候,\indicate{有效}/\indicate{无效}也被用来指代“有助于/无助于解决事务的”。\indicate{尝试解决}这一搭配等同于\indicate{处理}。

想要解决复杂事务,唯一可行的方式是先提升自己的能力,在能力足够正确应对之后再着手处理。提升能力的方式可以大体分为两种:一种是外界的\indicate{输入},本指南将其称为\indicate{眼};另一种是内部的\indicate{分析},本指南将其称为\indicate{脑};而\indicate{对事务的实际处理},或者称为自身的\indicate{输出},则被本指南称为\indicate{手}。本指南将眼、脑、手这三个方面,称为事务处理的三要素。

\subsection{眼}
\indicate{眼}这一要素,是\indicate{从外界获取能力提升}的所有方式的总称,可能包括接受教育、跟随课程、观看教材、阅读资料、实地探访、实验取证等多种不同的形式。
\begin{examples}
因为要考试所以学知识,游戏出新角色了看看技能介绍和讲解,有了新工作让老员工带一带,上线了新系统翻一翻操作教程,结婚之前先观察对方的性格,诸如此类,都是眼的发挥空间。
\end{examples}
一些读者会注意到,上述的例子也都可以视为事务。其拥有非常明确的目标:获取解决原有事务所需的知识或信息\footnote{关于知识和信息的更详细讨论请参见\hyperref[sec:知识与信息]{第二章}。}。本指南将这一类型的事务称为原有事务的\indicate{调研}事务,将“从原有事务获取调研事务”的过程也称为调研。
\begin{examples}
上述例子中,“学习相关内容”是“考试”的调研事务,“看技能介绍与讲解”是“游戏上分(或其它类似目标)”的调研事务,“接受老员工的教育”是“承担新工作”的调研事务,“阅读操作教程”是“使用新系统”的调研事务,“观察对方性格”是“结婚”的调研事务。
\end{examples}
调研会从一个事务A出发,获得其调研事务B。我们一般希望解决B有助于解决A——起码B得比A简单。
\iffalse\begin{examples}
面对一个新的手机应用,想要了解它的用途的时候,我们一般只需要把每个按键单独点一点,看看文字说明即可。如果想执行一些复杂的功能,也只需要上网查看教程。“学会编程并且靠自己把该应用的代码逆向破译出来”在绝大多数情况下都不应作为实际选择,除了“想学习它的编写思路”或“想攻击该应用”等少数几种情况以外。
\end{examples}\fi
\begin{examples}
如果你需要完成一个项目,其中需要一些你无法产出的资源(比如绘画或者代码段),那就应该首先考虑寻求外部支持,以外包或友情帮助等方式获取这些资源。“自学相关领域”一般不作为“完成项目”这一事务的推荐选择,特别是在时间紧急的情况下。(目标就是“提升自己”的情况不在此事例讨论范围内。)正确的调研事务应为“寻找资源”。资源可能以网站、群聊、商单、AI等多种形式存在,依据不同的资源类型而有所差别。
\end{examples}
当原有事务较为简单时,它的调研事务通常仅有清晰的单一步骤。
\begin{examples}
在商场里找卫生间,看路牌或者问售货员都行;买了新工具,看说明书或者教学视频了解它的用法;为了打过某关,直接抄别人的答案;去营业厅办理业务,听从工作人员的指挥;遇到不知道的新闻,去搜一搜具体报道。只需要简单调研就能完成的事务在我们的生活中比比皆是。
\end{examples}
对于复杂的原有事务,它的调研事务大体上来说也会更为复杂。其复杂性主要分为两方面:一方面是调研事务不一定能完成,由此需要多种不同的调研事务,逐一尝试\footnote{对于\rigorous,可以将“调研”视为一种独立的事务,此时“当前情况”即为“原有事务”,而“目标”为“获取有效知识和信息”,“处理每种调研事务”视为“调研”这一事务的一次操作。};另一方面是调研事务本身还需要再次调研,由此产生调研的递归。
\begin{examples}
    对于事务A“学一门新技能”来说,它的调研事务可能包括B“看某一门课程”,也可能包括B'“看某一本教材”。而B的调研事务又可能包括C“了解需要什么配套教材”,C的调研事务有可能包括D“查找电子版教材”。而E“观看课程评价”可作为是否选择B的依据,也是A的调研事务之一\footnote{对于\rigorous,可以将E视为O“对事务A的调研”的调研事务之一。后续处理手法类似,不做重复演示。}。
\end{examples}
本指南将“有效地进行调研,从而在总成本尽可能小的情况下解决事务”的能力称为\indicate{调研能力}。这里的“成本”因人而异。希望大家注意,“时间”也是成本之一,并且很多时候是最主要的成本。

调研能力缺失是很常见的事情,不需要为自身的调研能力不足感受到过分焦虑。面对复杂调研,可能出现的情况包括但不限于“无法坚持长期调研(比如学东西半途而废)”、“无法获取有效信息/缺少获取有效信息的渠道”、“陷入无限递归/死锁”。

复杂调研是需要系统学习与训练的能力,并且可以通过系统学习与训练来培养,详细讨论见\hyperref[sec:识别方法与判断]{第二章的相关内容}。若完全没有调研能力,面对简单调研时也不知所措,或是完全没有调研的意识,则应该先从改正自身认知与行为做起——这是另一个复杂事务。

\subsection{脑}
\indicate{脑}这一要素,是\indicate{从内部获取能力提升}的所有方式的总称,可能包括思考、整理、规划、推演、分析、总结等多种不同的形式。与事务处理相关的主要有两方面:一方面是“整理来自外界的知识”,本指南将其称为\indicate{分析};一方面是“得出应当执行的操作”,本指南将其称为\indicate{计划}。与调研一样,分析和计划都可以看作事务。
\begin{examples}
    根据原有事务的不同,分析和计划环节可能会有不同的比重。对于某些简单事务(如使用某种新工具),在调研环节(查询新工具的使用方法)结束后,按照流程使用即可,此时几乎不需要分析和计划;“调试某个参数”一类的事务几乎不需要计划;很多需要长期投入的简单事务(如坚持锻炼)几乎不需要分析;大部分复杂事务既需要分析也需要计划。
\end{examples}
\indicate{分析}和原有事务没有直接关系,只与获取的信息有关,具体定义与讨论见\hyperref[def:分析]{2.2.3小节},本章不过多论述。读者现在仅需知道分析是“从已知信息出发,获得更深入的认识”的过程。\\
\indicate{计划}是“根据原有事务以及分析结果,产生新事务”的过程。我们将产生的新事务称为\indicate{计划事务}。与调研环节类似,我们同样希望计划事务比原有事务简单,解决计划事务有助于解决原有事务。

当原有事务较为简单时,计划事务通常仅有清晰的单一步骤。在强调“从原有事务产生\indicate{唯一}计划事务”,或“产生的多个计划事务只有\indicate{一个}重要”时,也将这个过程称为\indicate{转化},将该计划事务称为\indicate{转化事务}。
\begin{examples}
“在开始之前做好准备”是一类简单清晰的计划事务,可能包含“出门带好钥匙”“上车/游览前买好票/办好卡”“办理业务前带好相关文件”等。有时计划事务会与调研环节相结合,如“找新地方时先查好路线”“查找指南并依据指南整理相关资料”等。

\end{examples}
对于较为复杂的原有事务,计划事务通常不唯一。多个计划事务之间可能可以并行,也可能有先后次序,需要依次执行。本指南将“计划时产生多个计划事务”的过程称为对原有事务的\indicate{分解}或\indicate{细化}\footnote{对于\rigorous,可以将分解视为“从一个事务产生多个事务”的唯一方式,并据此重写调研事务的定义,将调研事务视为“调研”的分解事务。}。

在强调“从原有事务产生\indicate{多个}计划事务”时,也将这个过程称为\indicate{分解},将这些计划事务称为\indicate{分解事务}。我们把“解决一个分解事务”称为处理原有事务的一次\indicate{操作}或者一个\indicate{步骤}。
\begin{examples}
分解事务常见于长期规划。“读完一本书”是复杂事务,但是可以分解成数个“读一章”;“学完一门课”是复杂事务,但可以分解成数十个“听一节课”;“锻炼身体”是复杂事务,但可以分解成数百个“每天跑步”。
\end{examples}
无论是调研事务还是计划事务,我们统一将“从原有事务产生出的事务”称为\indicate{子事务},同时将原有事务对应地称为子事物的\indicate{母事务}。与调研环节类似,对于复杂事务,计划环节所产生的子事务同样有可能产生递归。更复杂的是,计划事务有可能自身需要调研事务,计划环节也有可能需要调研得到的信息,来确定具体的分解。
\begin{examples}
当你决定入坑并且好好去玩某一款游戏的时候,直接去看教程(典型的调研事务)通常收获较少。你总得先游玩一段时间,起码熟悉基本的键位和操作,对游戏流程有基础了解(这一环节有吸收知识的效果,但主要还是分析和计划)之后,再去观看教程,才能跟得上讲解。在此之后,仍然需要多次反复的听讲-练习的组合。当你水平足够高时,你才能明白什么样的教程对你有用,什么样的教程和你不对口,才能选择性地观看教程,更高效地吸收外界知识。
\end{examples}
与调研能力一样,计划能力缺失是很常见的事情,不需要为自身的调研能力不足感受到过分焦虑。面对复杂计划,可能出现的情况包括但不限于“无法坚持长期计划(比如锻炼)”、“分解与转化的方式无助于实现原有目标”、“陷入无限递归/死锁”。

复杂计划是需要系统学习与训练的能力,并且可以通过系统学习与训练来培养,详细讨论见\hyperref[sec:方法与过程]{第二章的相关内容}。但若完全没有计划能力,面对简单计划时也不知所措,或是完全没有计划的意识,则应该先从改正自身认知与行为做起——这是另一个复杂事务。

\subsection{手}
\indicate{手}这一要素,是\indicate{对事务的实际处理}的所有方式的总称。相比于眼与脑,手的覆盖范围要大得多:所有事务的处理都可以被视作手的一部分。前文提到的调研、分析和计划,因为都可以视作事务,严格来说调研能力、分析能力和计划能力都可以视为手的相关能力。不过除了这种广泛性的含义之外,还有一个语义上更精确的概念:\indicate{执行能力}。

\indicate{执行能力}特指“直接解决一个不继续分解和转化的事务”的能力\footnote{由此可以回头补足之前对“简单事务”和“复杂事务”的定义。}。它有清晰的结果导向判断标准:如果能解决原有事务,那就拥有(对该事务的)执行能力;如果不能解决原有事务,那就缺乏(对该事务的)执行能力。
\begin{examples}
长期规划,如“每天读书”、“每天听课”、“每天运动”等等,是一类种比较常见的主要需要执行能力的情况。在实际执行中还会遇到如“时间规划”、“突发事件”、“拖延推后”等一系列问题,但大多数人不会想着去专门解决这些,而是靠自律和意志力坚持。这就变成了纯粹对执行能力的考验。
\end{examples}
很明显,执行能力的水平与调研能力、计划能力的水平相挂钩:调研和计划的水平越高,就越能将复杂事务分解转化为简单事务,直接解决每个子事务就越容易,执行能力也就越强\footnote{这确实更改了需要处理的事务,但“直接处理复杂事务”和“直接处理每个子事物”都符合执行能力的判断标准,不符合判断标准的是“经过分解和转化处理复杂事务”。}。
\begin{examples}
《拉封丹寓言》中有一篇名为《老鼠开会》,讲了这么一个故事:一群老鼠开会讨论“如何不被猫抓走”,有一位老鼠提了建议,“只要把铃铛挂在猫脖子上,听铃铛的声音就可以提前躲避了”,但问遍所有老鼠,也没有谁想把铃铛往猫脖子上挂的。作者的本意是用来讽刺只会空谈而无法贯彻政策的官僚,但这同时也可以视作“因为计划能力不足,从而将事务错误地转化成了更难的事务”的例子。无论是转换前的“不被猫抓走”还是转化后的“给猫挂铃铛”,都不是老鼠的执行能力可及的范围。
\end{examples}
除了调研和计划能力外,执行能力还受到很多其它方面因素的影响,其中最为人重视的可分为三方面:技能水平、意志力、外部条件。意志力在此不展开讨论,请参考\hyperref[def:意志力]{2.3.3小节}。
\begin{examples}
技能水平对人执行能力的影响是有目共睹的。刚上小学的普通儿童绝不可能靠自己作出微积分题目;刚下车间的实习生也不可能有老师傅的熟练操作;给业界外人士讲解业内共识只觉得隔行如隔山;门外汉看着专业人士的成果只觉得难如登天。在某件事务上,因技能水平的差距导致的执行能力差距,绝不可能仅靠意志力就补足。意志力只能用在“提升技能水平”上,而不可能直接用于解决该事务。强行解决只会错漏百出,除了运气实在很好以外不可能有成功的机会。
\end{examples}
外部条件则具有补全个人能力不足的效果\footnote{如果回看关于“处理事务”的定义,我们会发现,外部条件更改了“当前情况”,于是某种意义上来说,可以认为是换了一个更简单的事务来处理。}。依照条件的不同,能起到的效果从“稍有帮助”到“完全解决”都有可能。
\begin{examples}
良好的教育有助于学生更深入地接触学科;行业规范有助于从业者更高质高效地完成任务;机器加工使得工人不必依赖手工熟练度;辅助系统使得飞行员的培训周期大幅度缩短;靠谱的乙方可以完全解决某个困扰已久的疑难问题;AI助手可以完全替代文书工作。
\end{examples}
由于技能有数不清的领域,每一个人势必在绝大多数领域缺乏执行能力——但这不会有什么影响,因为那些领域大都是一辈子都接触不到的。只有那些会造成显著影响(如经常接触,或是会对未来产生关键影响的事件)的重要领域,缺乏执行能力才会带来严重的后果。调研和计划能力会给人带来技能水平、意志力、外部条件三方面的优势,从而显著提升执行能力。要是换个表述方式的话,这就是“\indicate{磨刀不误砍柴工}”的道理。

\subsection{总结}
在这一小节中,我们定义了事务,将事务处理能力细分为了三个子方面:眼(调研能力)、脑(计划能力)、手(执行能力)。这些内容经常会出现在日常生活方方面面的讨论之中,对大多数读者而言,应该都是熟知的。应该也有不少读者在之前没有接触过这些,如果你们在看完了本小节以后感觉有所收获,本指南为你们感到高兴。

另一方面,应该也有读者听这些内容过多,认为这些不过是正确的废话,是空泛无用的大道理,对改善现状毫无帮助,进而认定本指南就是本纯粹的鸡汤书,没什么用可以扔了。对此,请允许本指南阐明如此安排的打算。将事务处理的三要素放到开篇来讲,不是为了让读者在看完这一小节后,就立即充分掌握并能熟练运用。这种安排旨在向读者展示本指南的\indicate{主要分析思路}。事务处理能力会作为全书的主要分析手段,随着分析框架的不断搭建而反复使用,读者的理解也会随之逐渐深化。作为一切思路的出发点,只有本身足够简单明晰,才能够被大多数人理解运用。为此,本指南尽量全面而准确地定义和讲解了需要用到的概念,力求每位读者都能够准确无误地理解。

选择事务处理能力作为本指南的主要分析思路,原因有二。其一是:虽然这在很大程度上是广为人知的大道理,但很少有人能够熟练运用。大多数人在实际处理事务的时候,从来不会按照这一流程行动。虽然每个人都会计划,但大多数人只是草率地将事务转化成了另一个自己无力解决的事务,称不上有计划能力。即使遭遇反复的失败,大多数人也不会系统总结反思,无法识别自身的能力和条件缺失,称不上有分析能力。大多数人不会主动寻求改变,不会有意识收集信息,称不上有调研能力。
\begin{examples}
先前提到的《老鼠开会》寓言就是一个很好的例子。在提建议的老鼠看来,它将“不被猫抓走”转化成了“听铃铛逃跑”,极大降低了实现难度。但它没有意识到,这么转化需要“有老鼠给猫挂铃铛”的前提,由此错判了实现难度。“无法解决简单事务”通常会被人们归结为态度问题,而“将复杂事物错判为简单事务”则会使人将能力问题错判为态度问题,从而会产生不满和指责。这种人与人之间的冲突毫无道理也毫无意义,应该尽力避免。
\end{examples}
其二是:事务处理能力确实是一种强大且应用范围十分广泛的能力。一切目标都可以视作事务\footnote{可以回看数页前对复杂事务的列举,以确认自身理解正确。},而目标来自各个构成十分不同的领域,这导致对应的调研、分析、计划环节也会因目标所属领域的不同而内容迥异。由此,我们从中提取的共性也就容易显得空泛,只有少数几条大道理。然而,即便只依靠这几条大道理展开分析,我们也足以在一些问题上得到十分有力的洞见(下一小节就是例子之一)。

本小节涉及的大道理分为两部分。一部分是三要素的具体划分与操作,在前文已详细展开,这里不再赘述;另一部分是事务本身的性质:对原有事务的处理会产生新事务,子事务和母事务最大的不同,在于它们\indicate{拥有不同目标}。目标不同可能导致操作完全不同,甚至有利于子事务的操作经常不利于母事务。学会区分不同的目标,尤其是在处理子事务的时候,不受母事务目标影响,是相当重要的能力——\indicate{就事论事}。这是换位思考的基础,也是理解感情,理解他人,理解社会的基础。

\section{实操:为什么我不建议你去看心理医生\label{sec:实操1}}
不少读者看这本书应该有着明确的需求,希望能通过这本书解决自己的痛苦、迷茫、绝望、抑郁等等一系列负面状态\footnote{其中有一部分应该还等着下一段谈论自杀的章节,不过这一节不是。在看过上一节以后,你们应该对“为什么自杀很困难”有了进一步的认识。我们现在的理论准备还不够充足,相关段落见\hyperref[sec:再谈死亡]{2.4.2小节}}。大多数人应该都听说过心理咨询,应该也有些人实际体验过。一些书籍和博主会将心理咨询视为解决心理问题的灵丹妙药,毕竟这确实是整个地球上最有效的方法。但另一方面,心理咨询却并不是一种对所有人都能起效的方式。

面对繁杂的心理问题和完全会把自己压垮的负面状态,如果你的目标是\indicate{尽快}获得相对正常稳定的生活,你就应该\indicate{果断地}去正规医院的精神科或精神专科医院,接受全面的专业诊治(千万不要自己诊断自己开药,即使你可以通过某些途径获得药品)。药物会立即使你摆脱超出自身承受范围的情绪,这是心理咨询完全做不到的。我们不应该将心理咨询视为诊治,虽然我们习惯性这么称呼。以下的篇幅中,请读者自觉使用咨询师代替心理医生的称呼,使用来访者代替病人的称呼。

如果你虽然状态很差,但因为财力、社会关系(如一些医院接诊未成年人需要监护人同意才会给出诊断和处方,而监护人对精神疾病有根深蒂固的偏见)等原因而客观上缺乏接受治疗的条件,或是已经在服药但仍然对当前生活相当不满,那就只能尝试自己解决自身的问题了——本指南可以在这方面帮到你,但仍然不建议你直接去心理咨询。

即使我们只考虑有充足能力\footnote{具体定义见\hyperref[def:心理咨询能力]{2.4.3小节}。}的心理咨询师,而不考虑由于职业水平不足和职业道德缺失而导致的无效咨询(一些有咨询师职位的人甚至完全不承认心理疾病的存在),心理咨询也不是百分百有效。心理咨询师无法直接干涉来访者的意识,只能采用一些间接手段,无法开药而只能沟通。

心理咨询主要有两方面问题:成本高和见效慢。成本主要分为金钱成本和时间成本。一次有明显成果的心理咨询,持续时间经常以月计甚至以年计。很多来访者无法坚持这么久,经常因为一些其他原因(如自身兴趣减弱、其它麻烦事情“总是”占用时间、对心理咨询或咨询师本身产生抗拒、家庭或学校或单位施压等)而不再参与心理咨询。一次持续一小时的咨询需要花费数百元,每周安排一两次的话,持续数月的咨询总花费很容易过万,这是一笔不小的支出,很多人难以负担。一旦中途放弃,前期的金钱和时间花费就全部打了水漂。

即使经济富裕且时间充足,你也不一定能收获一次成功的心理咨询。心理咨询的见效慢,由其内在特征决定——心理咨询是个复杂事务,不是对来访者而言,而是对咨询师而言。虽然来访者可能在挑选心理咨询机构的时候很费精力,在安排时间的时候处理了一大堆麻烦,为了稳定来访而大幅调整了生活模式,甚至将“得到有效心理咨询”本身视为了人生目标之一,但对某一次具体的,只有来访者和一位特定的咨询师参与的心理咨询而言,来访者只需要按节奏走,而咨询师需要考虑的就多了。

咨询师所要面对的事务是“解决来访者的负面状态”。这可以大致分为两个方面:其一是“帮助来访者摆脱来自过去的困扰”,其二是“使来访者能够健康地面对世界”。这两个方面的解决相辅相成,心理咨询和心理治疗的一切手段,包括但不限于谈话、催眠、沙盘,都是为了这些而服务。其中,前者不可避免地需要充分细致地了解来访者的过往经历和当前情况,而后者虽然看着和来访者的具体情况没有必然联系,但是这也几乎不可能在对来访者一无所知的情况下完成。

一名心理咨询师如果只会谈大道理,那么或许可以成为良好的陪伴者和精神支柱\footnote{另一方面,来访者对咨询师产生依赖也是非常不利的情况,一方面这会大幅影响咨询进展,另一方面若是咨询师没有良好的职业道德,来访者将会轻易地被欺骗和伤害,并且在相当多产生依赖的来访者看来,情愿被咨询师伤害也不愿远离。概括来说,我们不应允许任何形式的发生在咨询师和来访者之间的感情(普通情绪,如“感谢”“怀疑”,是可以允许的)。},但对解决问题没什么帮助——咨询师口中的大道理也可以从别处听到,“理解每一条大道理”都对来访者来说都是复杂事务,而来访者既然受到问题所困,那一定缺乏处理复杂事务的能力。有些来访者确实可以通过“与咨询师建立了长期且健康的深度关系,从而有了心灵归宿,对生活的态度改变了”的方式获得彻底的疗愈,但这一般而言持续时间过长。大多数情况下,这一类咨询师仅起到稳定心理状态的作用,来访者的疗愈还需要通过生活中偶遇的其它契机来达成。

另一部分咨询师会根据来访者诉说的情况,针对性地给出具体的方案。这些方案既包括“心理上如何对待”,也包括“行为上如何操作”(如果你去拿着自身遇到的问题咨询AI助手,那么AI就会给出这种风格的回答)。这一类的咨询师承担了“替代来访者处理事务”的功能,补足了来访者的能力缺失。对来访者来说,实践新方案也能切身体会咨询师给出的建议,并从中有效学习咨询师所使用的分析和处理手法。这最终会使得来访者彻底理解(经常是咨询师没有明说,但来访者自己总结出)大道理。对于一些情况较为简单,只面对某些特定种类问题的来访者,大道理足以补足自身的短板,由此产生的心理问题也得以彻底解决。

但不少人情况更为复杂,经常感觉整个人生都充满灰暗,每一个方面都无比绝望。他们会\indicate{因为某个根深蒂固的思考回路,在每次遇到问题时都重蹈覆辙}\footnote{具体展开请见\hyperref[sec:人的意识演化基本模型]{3.4节}}。在外人眼中无比相似的问题,在自己眼中却各有各的困难,每个都完全不可克服,只让人想着逃避。他们完全无法应用新方法,即使内心知道这么做是正确的,也总是想不起来也做不到。这种时候,向AI或者这类咨询师询问“特定问题的处理方法”,就只能每次都得到相同的回答,也就不会有什么用了。

阅读本指南的读者中,应该有不少都会自认为是情况复杂的这一类。心理咨询师想要解决复杂的情况,就必须对来访者有全面深入的了解。咨询师拥有这方面的专业技能,但这毕竟是一个复杂的调研事务。如果只靠咨询师的努力,这一过程注定相当缓慢。来访者自身的表达能力同样大幅度影响了咨询师了解的速度。

这里提到的表达能力不完全等同于说话量,它的核心在于“使对方能够听懂”而不是“自己把想说的话都说了”。对心理咨询来说,来访者需要向咨询师表达自身的情况。其所能表达的极限取决于来访者对自身的了解。这是两个复杂事务的综合,一个是“培养自己的表达能力”,一个是“了解自身情况”。这两方面都可以得到心理咨询师的辅助,但自身的表达能力越强,对自己了解越充分,心理咨询也就越快。而表达能力越弱,对自己了解越不充分,就越不适合心理咨询,去了就越有可能做无用功。
\begin{examples}
表达能力缺失有很多种不同的表现。最容易理解的是“面对咨询师完全一言不发”,但这还算是相对好处理的情况,来访者明确知道自己自身表达能力缺失。更为棘手的情况是“来访者很能说,但说出来的东西都是无效信息”。其中一种情况是“来访者经常性地重复某些相同的信息”,常见的比如说谈及“自身情绪”“对某些事物的观点”等等。来访者不自知自身在重复,也不知道自己说的对咨询师来说是无效信息(反而“来访者会在这方面重复”是个重要信息)。另一种情况是“来访者经常性地脱离当前事务”,同样,这经常体现为谈及“自身情绪”“对某些事物的观点”等等。来访者不知道自己过于发散,归纳整理散乱的信息也会拖慢咨询师的了解速度。
\end{examples}
在当今社会,一个人只要能接触到网络,就有条件接触到丰富的心理科普资源。如果你对这方面感兴趣,那更是会学到数不清的知识。你的书架里可能有上百本教你如何摆脱人生困境的书籍,你可能每天都会刷到那些很有道理的心理学概念。经常看这些的来访者会向咨询师说更多,但说的内容不一定是有效信息。来访者如果提前进行了自我诊断,只向咨询师询问“原生家庭不好”“习得性无助”的解决方法,咨询师也很难帮上什么忙。

当然,客观上来说,这些心理科普资源能促进对自身的了解。纯粹通过看书而摆脱负面状态的人也不在少数。大家能接触到的大多数科普文章或者科普视频,可以起到的效果大概可以等同于大道理式咨询师;而一本心理科普书籍能起到的效果,大概在大道理式咨询师和AI之间。如果接受系统的教育,可能会更有帮助,但同时也有可能因为教育质量问题而迷失在海量的概念中\footnote{对此的展开讨论参见\hyperref[sec:污染]{2.2.4小节}和\hyperref[sec:人的意识演化基本模型]{3.4节}}。本指南并不宣称“会比AI起到的作用更大”之类口号,而只是一次尝试:如果行文的基石不是“心理问题”而是“现实事务处理能力”,如果将心理问题本身视为应当被剖析的概念\footnote{有些知识面较广的读者可能认为这样的思路很精神分析,但本指南不宣称自己是一本精神分析书籍。},是否有助于读者更深入地了解自身?

意识\footnote{这里的“意识”作为一个统称,包含人类所有的意识现象,一些语境下的“潜意识”也包含于其中,读者可自由使用“精神”“灵魂”“自我”等自己觉得更合适的概念进行替换。具体展开参见\hyperref[sec:意识与自我]{2.3节}。}作为一种客观存在的现象,拥有其内在的规律。这些规律稳定且可以被认知。任何试图违背这些客观规律,仅凭命令与强迫而进行的,对一个人的意识塑造和改变,都只能无效或极端低效、事与愿违或因小失大。由于篇幅和聚焦点原因,本书仅讨论意识层面的内容%,并且以“处理事务”作为意识活动的基本单位\footnote{该定义可以覆盖“目标与努力”“情感”“条件反射”“邱奇-图灵命题”等多个对意识活动的不同观察视角,具体展开参见2.3节与第三章}
。人的生理活动还有其它方面,本指南只关注它们和意识活动有关的部分。
\begin{examples}
    例如,精神药物会对人体产生多方面影响,其对意识的影响直接影响可能有“更难沉浸于某个思考回路”“主动性更弱”等,间接影响可能有“注意到自己发胖懒惰并为之感到自卑”等。若服药时不在意自身的身材变化,那就没有“感到自卑”这一项;若下定决心锻炼,那就多了“下定决心锻炼”这一项。根据人的不同,同一原因对意识造成的影响可能有相当大的差异。
\end{examples}
人的所有主动行为都有意识活动参与其中(即使人可能对于其中相当多的行为没有自觉),研究意识活动在一定程度上等于研究人的所有活动,并且研究对象更为集中,研究手法更为统一,更容易理清思路。而另一方面,对自身意识的研究,是普通人能够接触到的,对自身心理状态调整最有效(很多时候是唯一有效)的方式。根据程度的不同,一个人的负面状态可以被细分为“情绪问题”“心理问题”“心理疾病”“精神疾病”等多个不同的层次\footnote{这里的划分并非学术和医疗规范,而是基于普通人经常讨论的语境而提取的概念。词语仅大致表明“一般人眼中疾病的严重程度”,所有词语都不严谨地用于指代“心理、精神、意识方面的病理性现象”整体,不表示本指南的划分方式。}。除了少数精神疾病是因物理/化学/遗传因素导致的神经系统生理损伤外,几乎所有负面状态都有长期而复杂的形成原因。一切心理问题都可归结为\indicate{内心与外界环境的不适配},一切心理疾病都是慢性病。想解决它们必然要涉及对意识这个复杂系统的深入认识。

本指南希望这样的尝试有助于更多人改善自身的心理状态。充分了解自身,再配合上合适的新处世方法,在很多时候,足以彻底解决自身的遇到的问题。如果成效没有这么好,起码在充分了解自身之后再去找心理咨询师,也更能讲述清楚自身所遇到的问题,更能得到咨询师的有效帮助。

再次强调,如果你觉得自己当前的心理状态令自己无法忍受(但又死不掉),脱离当前心理状态最快的有效方法就是药物治疗。这一步骤一定要听从专业医生的建议,从正规医院处获取诊断。药物治疗可能不治本,但它至少治标。药物治疗与其它方案并不冲突,药物可以使你远离频繁失控所带来的干扰,使其它方案发挥更深入、更广泛、更本质的效果。
