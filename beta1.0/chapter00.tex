
% % 重定义chapter命令,支持可选参数[名言],来自deepseek
% \newcommand{\chapterquote}{}
% \let\oldchapter\chapter
% \renewcommand{\chapter}[2][\empty]{%
%     \ifx#1\empty
%         \def\chapterquote{}%
%     \else
%         \def\chapterquote{#1}%
%     \fi
%     \oldchapter{#2}%
% }
% \titleformat{\chapter}[display] % shape参数
% {\normalfont\Huge\bfseries\raggedright} % format参数
% {}        % label参数
% {20pt}                                   % sep参数
% {\Huge}                                  % before-code参数
% [\vspace{-1em}\raggedleft\small\itshape\chapterquote] % after-code参数

\begin{savequote}[250pt] %250pt指定引言宽度
    \fontsize{8pt}{8pt} %两个参数分别指定字号和行间距
    人生若是二十七岁就能死掉,那么是摇滚救了我。\footnote{原文为“人生、二十七で死ねるなら、ロックンロールは僕を救った”。\\\indent \netease{1357953772}。}
    \qauthor{ヨルシカ《八月、某、月明かり》}
    不知道在世上活着的方法,隔壁家的姑娘昨天也自杀。\footnote{\bilibiliv{av38665329}\another\bilibilip{av80848116}。}
    \qauthor{純白《玛\ 德\ 琳\ 娜\ 电\ 塔》}
\end{savequote}
\chapter{序幕:你为什么活着?\label{chap:wedge}}


% \renewcommand{\chaptername}{序幕:} %在当前环境(secnumdepth=-1,chapter不计数)下无效果,取消secnumdepth后显示“序幕:1”,换行显示“你为什么活着”
% \chapter{你为什么活着?}


我觉得你一定想过这个问题:\indicate{你为什么活着}?在这个信息爆炸的时代,这个问题你应该没见过十次也见过九次,早就不新鲜了。我相信有相当一部分读者也已经有了自己的答案,有的想投身于数学物理信息科学,有的想为共产主义事业奋斗终生,有的想组一辈子乐队,有的想产一辈子同人粮。另一部分没有目标的读者自然过得浑浑噩噩,但既然有目标还会感到痛苦,那你一定不知道自己具体做什么才能实现自己的目标。本指南会详细地教你如何处理这个问题,不过在此之前,让我先问你另一个问题:\indicate{你怎么还没死}?

\begin{toparent}
    对于一部分读者而言,这个问题纯属无稽之谈。这些读者在“为什么不能死”这个问题上有着充足的理由,或者是感觉生活很美好,或者是有肩负的责任。我暂时称这部分读者为\indicate{正常人}。他们中的一部分在读到这里的时候,可能已经开始着手举报本指南,说它“带坏小孩子,教唆青少年自杀”了。我的这个问题不针对正常人,而是针对另一部分,认真有过“想死”的念头的读者。
\end{toparent}
无论你是半夜emo、遭受重大打击、无法完成原定计划,当你心情差到想死的时候,应该都会有这么一种感觉:无论你在心情比较舒畅的状态下多么热爱生活,多么想活着,有多少不能死的理由,它们在这种状态下都显得彻头彻尾地苍白无力。一方面,在想死的时候,你根本想不起来自己有什么要活着的理由;另一方面,即使有人跟你谈及你的兴趣爱好或者特长成就,你也不会被说动:要不然觉得自己不配,要不觉得那些东西毫无价值。既然不能死的理由完全不奏效,又有明确的死亡冲动,那么,\indicate{你怎么还没死}?

这个问题的答案其实很简单:\indicate{你还没死是因为你每一次都没死成}。更准确来说,你的每一次死亡冲动,每一次自杀计划,最终都没有落实。这看起来像是废话,一个已经死了的人不可能还在阅读本指南,但总体来说,这是一份对每个人都适用的回答。想要更深入的答案,就需要具体情况具体分析:

一种很常见的情况,是\indicate{脱离了想死的状态}。晚上躺在床上会觉得自己一无是处,但是随后就睡着了,第二天醒来,你的脑子里会想一些别的,比如今天有什么安排。

但也不是所有时候都能刷新状态。你要是失眠了一整晚,第二天起床的时候大概会更想死——只不过得去上学或者上班了。或者说,你一闲下来就会想死,只是还有事情做,总是闲不下来。这种情况可以总结为\indicate{想死的念头被打断}。无论你是需要上学上班,还是总有人找你聊天,还是有很多单子要打,哪怕只是和别人待在一起,或者是走在大街上,只要你还有事情需要做,还需要在别人面前维持一个还说得过去的形象,你就不会深入地考虑自杀计划。

大多数人“想死”的念头其实停留在这两种情况上。即使情绪持续性低迷,也不会持续性地想死。本身就没有坚定的想死意志的情况下,没死成是很正常的事情。正常人可能会将其称作“暂时性的想不开”(我不赞同这种看法,虽然这种看法在效果上不会出大问题,“忍一忍就过去了”确实是一种行之有效的处理方法,但这只是拖延而不是解决)。但同时我相信,本指南的读者中,不乏持续性地想死的人。如果你会对着手臂雕花刀,或者打开窗户望着楼下出神,如果你确实距离死亡只有一步之遥,那么,\indicate{你怎么还没死}?

有些读者身边会有形影不离的朋友或者长辈,以陪伴或者限制人身自由的方式阻止自杀行为,或者施展及时的救治,这种情况其实可以归为“想死的念头被打断”,不是当前讨论的重点。而另一些读者,确实尝试过自杀。割腕、上吊、吞药、烧炭、跳楼、跳河,你的自杀计划为什么没有奏效?

有相当一部分情况,是\indicate{没能达到死亡条件}。你想吞安眠药,但在下次醒来以后,才发现医生给你开的剂量根本达不到致死标准;你想烧炭,但却连去商店买炭的胆子都没有;你想上吊,但是找遍家里也找不到合适的房梁。

另一部分情况,是\indicate{被生物本能阻止了}。一个恐高的人不太可能跳楼,看着楼下就会腿软得无法动弹;一个怕疼的人不太可能割腕,下不去手只能草草了事;使用过一遍的自杀方式,下次再用总会有更大的心理障碍。

与之相反,那些条件简单,并且不触及生物本能或能克服生物本能的自杀方式,就更容易成功。如果有人递给你一把枪的话,朝自己脑袋来一枪是很轻松的事情;如果你在跳河的时候周围有人劝你“想想你的家人”,那你就顾不上怕,而只想逃开他们,于是就跳下去了。但类似的情况可遇而不可求,在这些自杀成功案例中,起决定性作用的是外部的其它因素,而不是你。如果你自己来尝试自杀,那你就会失败,几乎每次都是——因为\indicate{自杀是处于你能力范围之外的事项}。没有能力指定切实可行的自杀计划,所以只能碰运气,所以才每一次都没死成。并且,你还经常会因为心理保护机制而丢失自杀相关的记忆,新增一个无法触碰的心理阴影。

\begin{toparent}
可能一些正常人读者在看到这里的时候,已经对本指南厌恶至极,觉得本指南在教唆青少年自杀,是不折不扣的邪典,从而去举报封禁一条龙了。但正如上面的分析所说,讨论自杀的话题并不等于教唆自杀,这些“只是内向了一些/不知道在发什么疯/大概过了青春期就好了”的青少年,实际上缺乏自杀能力。告诉他们这一点也不会增加他们的自杀能力。我不否认有些读者在看了上述论述以后会尝试自杀,但总归他们不会成功(对于少部分读者来说,这个判断有可能刺激他们坚定自杀意志,但还是建议这些读者在整体看完本指南之后再做决定)。

如标题所述,本指南讲的是自救的方法,立场是让尽可能多的人幸福快乐地生活下去。但读者群体中相当一部分实际上不觉得自己有资格得到幸福。他们熟悉的是自我否定,而自杀几乎是他们每个人都想过的事情,从更熟悉的事项出发,更容易学到真实的东西。如您在后续阅读中仍然怀疑本指南的立场问题,请参考本段解释。如您在读完全书后仍然觉得本指南有害青少年心理健康,本指南尊重您的立场和行为。
\end{toparent}
成功的自杀方案需要缜密的设计,确保其中的每一步都可以执行。上述的四种失败情况,都可以从“方案设计与执行”的角度来考虑:“脱离了想死的状态”是一拖再拖,实际上等于放弃了计划,从来没有开始过;“想死的念头被打断”是无法排除周围环境干扰,导致不具备开始计划的条件;“没能达到死亡条件”是纯粹的能力不足,有无法克服(甚至无法察觉)的阻碍;“被生物本能阻止了”是意志力薄弱,被自身的软弱和恐惧压垮。

这些特征,与你失败的人生别无二致。你不止在自杀这件事上缺乏能力,同时也普遍地缺乏处理各种现实事务的能力,只能被纷至沓来的不利事项裹挟着,流向未知且恐惧的远方。\indicate{行动力够强的人已经死了,你因为干啥啥不行才活到了现在},并且也将继续痛苦地活下去。无论你是想要想要成功地自杀,还是想得到爱,还是把握住自己的命运,还是有什么更宏大的目标要实现,只要你想改变点什么,就必须走上\indicate{学习处理现实事务}的道路,搞清楚怎么做计划,怎么执行。这是唯一可行的自救之路。