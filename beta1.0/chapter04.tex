\begin{savequote}[450pt] %指定引言宽度
    \fontsize{8pt}{8pt} %两个参数分别指定字号和行间距
    音乐嘛,只有量身定做的才能打动人心啊!\footnote{原漫画因有妖气停止服务已下架。\moegirl{你什么都没看见}。}
    \qauthor{笛子Ocarina《你什么都没看见》第47话 塞壬}
\end{savequote}

\chapter{认知形成与信息交流}

\addtocounter{section}{-1}
\section{基础讨论与用语}
我们将“某个现象作为刺激,触发了某个行为模式,并形成了认知”的过程称为\indicate{信息传递},并将其描述为“信息从某个现象\indicate{输出},\indicate{输入}到某个行为模式中”或是“某个行为模式\indicate{接收}或\indicate{提取}了这一信息”,将形成的认知称为对该现象/信息的\source{解读}。
\begin{explain}
输出信息的现象不一定是另一个行为模式,也不一定是一个人。由于我们可以从任意现象中受到刺激(不过本章的讨论仍然会以人为主),这里的“输出”指的就是现象本身。一个现象可能分时间和方面输出多个信息。

而输入环节,因为本指南仅关心人的意识现象,故此处不讨论其它的信息传递途径,仅考虑输入到行为模式。因为外溢现象的存在,一个现象可能刺激出任何一种行为,形成任意一种认知。

如果没有形成认知,那么这个现象在刺激对应的行动结束后,便不再对该行为模式有直接影响。这不是本章关注的情况。形成的认知不一定可以使用语言来描述,记忆、情景、条件反射等其它形式也计算在内。

在一些情况下,本章所讨论的内容也可以套用到行为上,此时我们可以将信息传递的定义扩充为“形成了行为/认知”。读者可以自行替换相应段落。
\end{explain}
以具体语境的不同,我们将\indicate{一个信息刺激了某个行为模式,输出新的信息}的过程称为该行为模式\indicate{回应}了该信息;将\indicate{一个信息输入到某个行为模式,产生新的认知,并输出新的信息}的过程称为该行为模式\indicate{传达}/\indicate{传递}/\indicate{描述}/\indicate{表述}了该信息。我们将“行为模式A传递的信息触发了行为模式B”称为“A向B传递信息”。
\begin{examples}
所有的传递都是回应。回应和传递的区别在于\indicate{回应不要求产生新认知}。

一个信息可能会触发不止一个行为(比如说总是提起印象深刻的事情),此时我们将每个行为单独看做一次传递。同时,我们不将“行为模式内部的相互触发”视为回应,只将“产生了外部影响的行为”视为传递。

传递有很多种类。从形式上来说,新信息可能和原信息几乎一模一样(比如复述一句话),可能转换成了另一种形式(比如说向别人讲某个事件)。

以上两种传递都保持了认知的\note{内容}不变(后者可能有一些细节损失),我们将其称为\indicate{准确}\footnote{准确这一概念的正式定义请见\hyperref[def:准确]{4.4.1小节}。}的传递。但传递也有可能劣化。比如,当我们\note{表面分析}的时候,经常会得到错误的认知。此外,传递的输出不一定是认知。任何形式的行为/行为链,无论是自知的还是不自知的,只要被这个信息触发,都可以是传递。此时,传递信息的行为可能完全脱离了原意。

在本章讨论范围内,如果B总是可以准确传递A,那么我们就将B和A视为同一个行为模式,将“B(向C)传递A的行为”也称为“A(向C)传递”。
\end{examples}
如果输出信息的现象是某个行为模式,那么我们就将对应的行为称为该信息传递对应的\indicate{表达}。两个(或多个)行为模式的一次接触中可能有多次表达,我们将\indicate{一次}\inote{接触}{中的所有表达}统称为\indicate{沟通}或\indicate{交流}。
\begin{explain}
所有的行为都可视作表达,如语言、动作、神态等。使用“表达”这一称呼,表明我们仅关注两个行为模式的接触,不关注其它方面的影响。我们将对应的信息传递过程称为“输出者向/对输入者表达”。在一次沟通中,输出方和输入方的身份可能有多次变化。如果我们想只考察输出和输入的现象,就必须在表达的层次上分析。

由于信息输入会形成认知,认知会长期留存,并持续触发其它的行为,我们其实很难定义“一次接触”。本章的讨论中,沟通可能包含新形成认知的后续影响(这种沟通通常会在短时间内结束),或不包含新形成认知的后续影响(这种沟通可能持续很久),后者主要指“某个中断又恢复的话题”。读者应能自行区分这两者。

在第二、三章中,我们已经详细讨论了一类沟通现象:发生在同一个人身上,在不同行为模式之间的信息传递。这种交流现象会使行为模式之间相互触发,此处不再赘述。

大多数信息输出是不可控且不自知的。实际上,因为我们什么都可以解读,任何行为都可以视为输出了某些(只能激发某些特定行为模式的)信息。只不过大部分行为不会产生什么实际的影响。

人类传递自身认知的方法不是很多,大多数通过语言/文字叙述,动作、情景等情况占少数。本章中的讨论也会因此偏向语言沟通。
\end{explain}
我们将“一个信息完全不触发行为”的现象称为“行为模式未接收到信息”或是\indicate{无效表达};如果某次沟通/表达是实现某目标的一步操作(目标可以是“传递某个信息”“布置某个任务”或其它),但是该操作没有取得预料中的效果,就将这次沟通称为\indicate{关于该目标的无效沟通/表达},在不引起歧义的情况下也简称为\indicate{无效沟通/表达}。
\begin{explain}
我们有时将无效表达称为“行为模式没有留神/注意/关注该信息”。

信息的输入需要形成认知。在此定义下,很多日常语境下的交流(比如说相谈甚欢的闲聊)会被判定为无效沟通。对无效沟通的具体展开讨论见后文。
\end{explain}
% 如果有一个稳定的过程,可以给某个行为模式输入信息,那么我们就将这个过程称为一个\indicate{话题},并称该行为模式在这一话题上\indicate{可沟通/可交流}。
% 如果有一个稳定的过程,可以给某个行为模式输入信息,那么我们就将这个过程称为该行为模式的\indicate{沟通渠道}% ;如果有一个稳定的过程,可以接收某个行为模式的信息,那么我们就将这个过程称为该行为模式的\indicate{解读渠道}。
% \begin{explain}
% 注意,沟通渠道仅对于一个行为模式而言,仅针对其输入的
% \end{explain}
一个人的\note{想法}只能由自己的行为模式回应。
\begin{examples}
这句话听起来像是什么“做不被定义的自己”“只有自己才能决定自己的未来”之类的意思,但实际上不同。它所表达的仅是“想法不具有任何外部影响,所以只能刺激自身的行为”。想法只有作为动机,被表达为其它行为(可以是语言、表情、肢体动作或其它)时,才能被间接地注意到。这不应该算作是“想法直接刺激了外部”。

本指南仅考虑“行为模式之间的交流”,这是一种较为方便的安排。我们不关心是“自己的行为模式之间的交流”还是“两个不同的人在交流”,因为这两种情况只有“是否能察觉到想法”的区别。其它的任何不同,如“是否熟悉”“是否有盲点”“表达是否准确”之类的,都和“交流的是谁”无关,都可以归因到行为模式上。本指南将人视为一种\note{共识环境},这样就可以将采用统一的分析框架分析。关于共识环境的讨论见4.3.2小节。

和\hyperref[para:不研究人]{之前}一样,我们会使用“人”来简化叙述。本章中使用“人”“对方”“xx者”等概念时,有可能指人或行为模式。读者应默认指行为模式,除非出现“人拥有的行为模式”等概念。
\end{examples}
在行为模式A和行为模式B的某段沟通中,如果B没有得到任何输入,那么就称该沟通为A对B的\indicate{倾听}、\indicate{观察}或\indicate{观看}。如果同时A也没有得到对B的任何理解,那么就称其为\indicate{无效倾听/观察}。
\begin{explain}
B没有得到输入有很多种情况,如“A的所有回应都是想法,没有外在表现”“A仅采用‘嗯’‘点头’等方式回应”“A实际上刺激了另一个行为模式C”等。

此处定义的倾听不要求“A有倾听的意愿”,其语感类似于“B单方面表达”。同时,倾听不意味着增进理解,“双方各执一词,沟通毫无进展”的情况在本定义下算作“双方同时倾听对方”。
\end{explain}
某个思考回路如果\hyperref[def:理解行为模式]{理解}了某个行为,那么它就能准确传递这个行为;某个思考回路如果能\note{模拟}某个行为模式,那么就总是能准确地传递这一行为模式的行为。

本章的讨论重点就在于此。什么样的行为模式总是能准确传递另一个行为模式?什么样的思考回路能够理解另一个行为模式?如果不能,又是被什么原因干扰了?是产生了错误的认知,还是根本没接收到信息?

准确传递一个行为模式,是理解和改变的前提。而如果对这个行为模式的认知本身就是不准确的,那么进一步的理解和改变的尝试只能做无用功。

% 本节内容应该是本指南全书中和其它心理书籍最像的一部分,使用的方法论高度类似。毕竟这确实经过大量实践检验,相当行之有效。

% \begin{examples}
% 以下是一个本章会讨论的情景:一个人具有思考回路A,通过语言等行为模式B将其表达出来,尝试和另一个人的行为模式C交流。由于是不同的人,A无法直接和C表达。在此过程中,B的表达能力和C的理解能力都有可能不足,从而导致无法准确传达关于A的信息。
% \end{examples}

\section{作为副作用的沟通}
\begin{explain}
\note{副作用}是指“不会参与决策,成为行动的动机”的影响。本节中,我们将\indicate{拥有其它动机的行为模式}简称为\indicate{自身},将\indicate{受到行为影响的行为模式}简称为\indicate{其他人}。
\end{explain}
自身出于沟通以外的动机而行动时,“别人接收到信息”和“沟通”的现象就都是副作用。这种情况下,自身不将本次行动视为沟通。
\begin{explain}
这可能是因为“根本没意识到会对其他人有影响”,也可能是因为“能认识到会对其他人有影响,但是不在意”。
\end{explain}
不将行动视为沟通,就会有两个特点:没有“将信息有效输出给其它行为模式”的目标,也没有“接收其它人的回应”的能力。
\begin{explain}
特别地,如果一个行为模式被激发时,总是有沟通以外的其它动机,那么我们无法找到稳定可行的方式,给该行为模式输入信息。这使得除了倾听外,不存在其它与该行为模式的沟通方式。

% 两个有其它动机的行为模式相遇时,虽然看起来有可能很热闹,或是相谈甚欢或是要吵起来,但是实际上两人完全在各说各话。
\end{explain}
本节内容讨论各种不同的类型以及它们的成因。

\subsection{零散表达}
\noindent 如果一个表达不是任何思考回路的结果,我们就将其称为\indicate{零散表达}。
\begin{explain}
零散表达不经过思考回路,而是直接在刺激和表达之间建立了\note{关联},可以视为纯粹的条件反射。关联可能有很多种不同的成因:

最常见的一类是“模仿”。当一个人长期接触某个环境,环境中的其它人都以固定的方式对待固定的刺激时,自己就也慢慢形成了相同行为。这类情况包括但不限于“背诵/唱歌/跳舞”、“进球/听到高音就欢呼”、“重复某个词语/句子/称呼”\footnote{熟悉相关梗的读者应该知道我在指什么。为了避免写出来就过时,这里就不举例子了。}等。这种现象以及对应的行为应当视为污染。

另一类常见情况是“熟练”。当一个人长期使用某个行为去应对某个刺激时,也会在二者之间产生关联。可能在关联形成之前有完整的思路,但是关联形成之后的触发就和思路无关了。这类情况包括但不限于“说话风格/小动作/口头禅”、“各种情绪”、“看见某个东西就开始夸/骂”等。这种现象是行为的外溢(当然也是污染)。

两类情况的主要区别在于“污染源来自行为模式外部还是行为模式内部”。在一些情况下,如果污染源来自内部,那么自身会对这一行为的形成过程有明确认知(即“是否\hyperref[def:理解行为]{理解}这一行为”),从而方便对该行为的后续处理。不过实际上,“模仿”也可能知道这种方式整体的发展过程(可能伴随一些社会学研究),“熟练”也可能无法完整回想自己的思路。这使得这一区别变得不重要。

如果我们持有“只有思考回路以及更上层的意识结构才有意识”的看法,那么零散表达是无意识的表达,自身的行为不代表自身的意愿。虽然说了一句话,但这句话本身是背诵出来的,话是什么意思完全不知道。如果它是因熟练而无意识化的,那么它仅代表过去的意愿;如果它是因模仿而无意识化的,那么它完全不代表任何意愿。
\end{explain}
零散表达也可以编织成沟通,但这不代表其中有任何有效的信息传递。
\begin{explain}
在合适的环境下,零散表达会被反复触发。每一次触发都是一次独立事件,触发原因可能是环境中的刺激,也可能是上一个表达。我们必须对每个问题单独分析,不应因为“一个人在连续表达”的现象就将所有表达视为一个整体,草率地认为“因为上一次表达触发了下一次表达”。

另外,两个行为模式的表达如果分别能触发对方的零散表达,那么就会编织成一次沟通。这种沟通的总体氛围可以很融洽,或是很尖锐,沟通参与的双方或许也能得到“聊得很开心”“吵得很激烈”之类的认知。但这个认知是此次沟通本身的宏观特点,该信息由沟通输出,不可视为行为模式之间的信息传递。
\end{explain}

\subsection{转述} % 复述?
在行为模式的一次触发中,如果其中的每个行为都是对另一个事件/现象的表述,那么我们就称这次触发为对此事件/现象的\source{转述}/\indicate{复述}/\indicate{复现}/\indicate{模仿},将事件/现象称为这次转述的\indicate{内容},将行为模式称为此现象的\indicate{转述者}。
\begin{explain}
如果我们将此次触发视为“使用对应的认知”,那么被转述的事件/现象是此次触发的\note{参考事件/现象},转述的内容属于认知的\note{内容}。

一个行为模式有可能不总是在转述,但我们总可以将其中的转述视为一个子行为模式,并将这个行为模式的其它部分视为另外的行为模式。这种处理会为之后的讨论带来便利。

转述和零散表达不是互斥的两种表达类型,转述本身可能也与思考回路无关。另一方面,转述也可以完全可控,比如对知识体系的\note{应用}。参考现象可以有以下几类:

最常见的一类是“自己之前的思路”(这里的“自己”指一个人的意识整体)。当自己对某个现象思考过以后,再次遇到这一现象时,就有可能回想起之前的思路。这不只限于当前思考模式,只要是能想得起来的,都会产生影响(读者如果觉得“转述自己的思路”这种用词较为奇怪,请注意,思路和转述者是不同的行为模式)。在一些情况下,这被称为“第一印象”。

另一种是“之前了解过的现象”。具体来源多种多样,可能有“亲身经历”、“别人的一句话/一段论述”、“一段剧情中的表达”等。如果自身对此印象颇深,那么就会(可能无意识可能有意识地)反复观看/回忆对应段落,从而逐渐内化为自己的思路。如果现象本身没有明确记忆点,那么在一定时间后,我们可能忘掉其具体来源,将其完全转化为自己之前的思路。
\end{explain}
一个行为模式在转述一个现象时,能够对相关的刺激内容做出\indicate{回应},但这不代表它接收到了任何信息。
\begin{explain}
在遇到“询问”“情景再现”等一些刺激时,转述者看起来能够顺畅地交流。依照具体情况的不同,我们可能会用“来者不拒”“外向开朗”“积极地介绍自己的爱好”“亲切友善地回应”“打开了话匣子,滔滔不绝地讲述”“对这个问题很有见解”等来描述这种场景。如果当前事务是“倾听转述者,从它那里获取信息”“让转述者应用已有的行为模式解决问题”一类,那么进展就会很顺利。

但转述者无法接收信息。参考现象是已经固定的现象,就如同转述者无法更改故事的过程和结局一样,转述者也无法更改一个成型的思考回路。当看到某个信息时,转述者会做的事情是“检索参考现象,并在其中找到与刺激最接近的内容,并且将其转述出来”。这看起来像是在讨论这个信息”,但实际上说的还是参考现象。一些情况下,这可以被称为\indicate{路径依赖}、\indicate{自说自话}、\indicate{代入自己(过多)}等。这会导致如“无论怎么做都无法令转述者满意”一类的现象,因为转述者的行为模式中有可能只包含“表达需求/情绪/...”的行为,而不包含真实的需求,不包含满足的条件,不包含后续行动,不包含对其内容的任何理解。

我们同样也可以考察两个行为模式都在自己身上的情况。如果我们在面对某个现象时,同时触发了两个转述者(且只触发了这两个思考回路),它们具有不统一的想法,并且在价值判断上一个正面一个负面。它们会不停讲述自己的理由,而我们由于缺乏其它有能力的思考回路,无法将它们给出的信息梳理清楚。此时我们就会陷入纠结,还可能伴随有迷茫、烦躁、痛苦等情绪。

为转述寻找例子是很麻烦的事情,任何例子都会导致一部分读者深感赞同,另一部分读者感到愤怒恼火/窒息绝望。这会干扰读者对这一部分讨论的理解。确实有需要的读者可以参考龙应台《目送》中“我不爱吃鱼”的段落,或其它内容。
\end{explain}
\indicate{转述者是大多数人沟通时最常见的一类行为模式}。
\begin{explain}
这一结论基于两个假设。一个是“熟练的行为更容易触发,容易触发的行为更熟练”;另一个是“想法远比其它类型的行为更容易触发”。相应的讨论参考\hyperref[sec:想法的竞争与外溢]{3.5.1小节}。

外溢的参考现象和思考回路会大幅度干扰我们的思路,打断我们原有的行动。脑子里只有一种念头的时候,按着这种念头行事是顺理成章的。转述是纯粹的习惯堆积,不抱有任何明确的目的,也不体现原有的意愿。我们不应将其称为“享受自我表达”,也不应将其称为“渴望他人认同”。它自身会指引我们接下来的行动,一些场合下可以将其视为“新的意愿”。

触发转述时,如果原有思路(和意愿)能不被打断,那么它有可能依然可以接受信息,可以解读转述者的表达。但这不应该理解为“转述者体现原有意愿”,它和原有思路应当被看做两个行为模式。我们不可能时时刻刻清醒地反思自己的每一个行动和念头,转述者经常会成为唯一被触发的行为模式。

因此,“转述者之间的沟通”经常被人当做是沟通的参考样板。我们确实有的时候能从转述者那里接收到信息,但总体来说这是一种低效的方式。对此的展开讨论参见后文。
\end{explain}
转述者可能会以任意逻辑和次序表述参考现象。
\begin{explain}
这使得我们即使想接收转述者输出的信息,也不总是能成功。关于接收的具体讨论参见后文。

一方面,这是因为来自外界的刺激本来就随机;一方面,接收到刺激后,转述者可能在产生了数个想法后,才对外表达一次,具体表达哪个想法,很大程度上是随机的,即使内部思路很连贯,在外部也会体现出跳跃性;一方面,转述者之前可能进行过类似的表达,一些认知之间已经建立了额外的关联,这使得内部思路也有可能是跳跃的。

在合适的情景下,转述者有可能条理清晰地完整表述某一段参考现象。这种行为有可能再重复中不断固化,从而形成“转述者似乎很条理清楚”的表象。但实际上这只是处于行为这一层次的条理清楚,不代表转述者整体的特点。

如果转述者拥有“想到什么就立刻说出来”的特点,那么我们在外部也能观察到连贯的思路。但连贯的思路不等于顺畅的逻辑,转述有可能仍然充满了额外的关联和突然的跳跃。
\end{explain}

\subsection{拒绝沟通}
我们将\indicate{发现并不理会某个现象的刺激}\footnote{“不理会某个信息”指的是“不因其而产生行为”。对于\rigorous,“发现”一定会产生认知,和“不理会”矛盾。此处的处理方法是将发现视为另一个行为模式的行为。}的决策称为与该现象\indicate{拒绝接触},或是\indicate{拒绝}与该现象接触;如果该现象是行为模式,那么就对应地称为\indicate{拒绝沟通}。
\begin{explain}
拒绝接触一个现象时,自身一定对“接触该现象”有预料和相应的负面价值判断,如“接触本身有害(比如横穿马路)”、“对方本身有恶意(比如霸凌)”、“对方行为冒犯(各种不文明举止)”、“对方性格如此(还有别的方面可以接触)”、“对方和我不和(可能有宿怨)”、“对方不可沟通(对方是转述者的话确实)”等。

拒绝和自身的其它行为模式沟通也是常见的现象,我们经常会做出“想不明白,不想了”“不理会脑子里嘈杂的声音”“不要为这件事坏了我的好心情”“过去的事情就过去吧”“已经太迟了没办法了”等决定。

这些预料、价值判断不一定代表事实,只是自身的主观判断。自身可能只是转述了之前的记忆,而没有从客观出发分析。对应的决策也不一定在事实上有利的。
\end{explain}
拒绝接触这一行为本身和相关认知都有可能产生外溢,从而污染我们对接触和沟通的认知。
\begin{explain}
对于一些现象,不予理会是正常的选择,能够有效避免我们遭受影响/邯郸学步/裹足不前。对于沟通来说,这也可以使我们不被尖酸刻薄/不怀好意/煽风点火/嚼舌根传闲话/......的行为模式影响。

输出方面,沟通时最常遇到的行为模式是转述者。在面对转述者时,所有输出都没有作用,和它争论只是浪费时间,只会看到一些之前不知道见过多少次的重复行为。转述者压倒性的出现频率会让人很容易产生“所有的沟通都没有用”的全局认知。这个全局认知在一些例外情况下会被覆盖,比如说“和好朋友交流”等。

输入方面,大部分表达是零散表达或者转述,不包含什么有价值的认知。这使得“不理睬别人的风凉话/瞎操心/乱提意见”在大部分情况下是正确的决策,进而会让人很容易产生“不需要受别人影响,只要自信地走自己的路就好”的全局认知。这个全局认知在一些例外情况下会被覆盖,比如说“听某些老师讲课”等。

在思考沟通及其相关现象的时候,我们难以将全局的“沟通无效”认知和例外的“沟通有效”认知统一起来,大部分情况下是各论各的。在遇到新的沟通场景时,“沟通没有用”的认知和拒绝沟通的行为更容易触发和外溢。

这样的认知会干扰我们真正深入透彻地看待沟通这一客观现象,使我们既难以具体定位“无效的沟通为什么会无效”,也无法系统讨论“有效的沟通都做对了什么”。只有从中提取出原理,才有可能运用到其它方面,带来沟通能力的实际提升。
\end{explain}

\section{有效沟通的必要篇幅}
我们将\indicate{一次沟通中,内容相同的所有表达}称为一个\indicate{话题},将这些表达的\note{内容}也称为这个话题的\indicate{内容}。我们将\indicate{其中一个行为模式在此次话题中,所有(不重复的)认知}称为该行为模式在这个话题上的\indicate{篇幅}、\indicate{表达篇幅}或\indicate{讨论篇幅}。% 我们将\indicate{属于所有行为模式的篇幅的认知}称为这些行为模式在本话题中的\indicate{共识}。
\begin{explain}
篇幅和\note{出发点}是两个不同的概念,出发点的范围要更大。篇幅仅包含表达中涉及的认知,而不包含没有表达出的内容。我们只能从对方的表达篇幅中接收信息,所以有这种区分。

同一个话题中,不同行为模式的表达篇幅可能差别很大,两者可能完全是在自说自话,除了“使用同一门语言”以外找不出什么别的共识。

% 我们也可以对于某一些行为模式单独定义处于它们之间的共识,不过这对本节的讨论没什么帮助,故不额外定义。
\end{explain}
本节内容讨论因表达篇幅不足而产生的无效沟通,以及对应的处理方法。

\subsection{认知的形成与维持}
\begin{explain}
本小节中讨论的内容不仅限于沟通之中。出于简便的考虑,本小节会使用“一个认知”来指代一个/一组认知,读者应能自行补全。
\end{explain}
我们将\indicate{一个思考回路拥有且不被覆盖的认知}称为这个思考回路能\indicate{维持}的认知,将“之前维持,之后不维持”称为这个思考回路\indicate{丢失}了这个认知。
\begin{explain}
这里的“覆盖”指的是“在想起一个认知时,总会想起另一个认知,并且另一个认知覆盖了原有认知”。

虽然“维持”听起来像是个能持续时间很长的特点,但“思考回路能维持什么认知”必须在每一时刻分别判断。它经常处于快速变动之中,我们很多时候无法对于“某个沟通过程”或者其它的一长段时间来定义“一直维持的认知”。“花很大力气讲明白了一件事,然后转头就忘了”的事情时有发生。

对于一个理解途径,如果我们能维持它的每一个想法,那么在想过一遍理解途径后,我们就会临时组成一个“理解途径的转述者”的思考回路,认知也就能够维持,并且(在注意力转移之前)会越来越熟练。如果我们无法维持它的每一个(重要)想法,那么就总是无法理解这个认知。“一部分理解途径的转述者”可能会得出某个其它的认知或原认知的劣化版本。

维持认知不一定需要用原始的理解途径,一些认知完全可以由高度熟练的想法直接触发和维持(这个想法因此成为高度个人化的,无法让别人也使用的,高度外溢的理解途径)。如果在触发后有“验证它是否符合当前情况”的步骤,我们仍然认为它是可控的。
\end{explain}
我们将\indicate{某个认知的一个}\inote{理解途径}\indicate{的出发点}称为这个认知的一个\indicate{理解门槛}、\indicate{理解前提}、\indicate{理解条件}或\indicate{背景知识}、\indicate{背景信息},将理解门槛中的每一个认知称为这个认知的一个\indicate{前置认知}/\indicate{前置信息}。
\begin{explain}
如果某个思路同时得到了多个认知,那么也可以将其称为这些认知共有的理解途径,并以此来定义共有的理解门槛和前置认知。多认知共有的理解途径可能比其中一个认知的理解途径长,前置认知可能更多。

不是所有认知的形成都需要理解门槛,不由思考回路得到的认知(比如简单的模仿)就无法定义其理解门槛。

前置认知的范围很广泛,包括用词、指代对象等等基础而容易被忽略的东西,具体讨论参见\note{出发点}的相关内容。

在一个理解途径中,除了前置认知和目标认知以外,可能还有一些中间认知。如果这些中间认知可以在过程中自然地获得并维持,那么我们一般将其忽略;如果这些中间认知有可能无法维持(可能因为没有引起重视、比较复杂等情况),那么就应当将其也视为前置认知。如果有必要,可以将理解途径按照这些中间认知分为多段,将中间认知变为“上一段的目标认知”和“下一段的前置认知”,然后分别处理。在接下来的篇幅中,我们总是这样切分理解途径。

拥有背景知识不意味着就能够获取这个认知,还需要实际触发才行。有背景知识但没有对应认知,有可能只是纯粹的“从来没有想过”,也有可能是“有别的思考回路覆盖了对应的思考”。

为了方便后文的叙述,我们可以适当扩大定义:如果一个思路实际上没有得到某个认知,我们仍然可以将其视为理解途径,并且将这一认知本身(或是某些额外的前置认知)加入到理解门槛中,以使得添加后的思路称为该认知的理解途径。
\end{explain}
对于一个思考回路,我们将\indicate{一个认知的理解门槛中,该思考回路不能维持的部分}称为该思考回路(通过这个理解门槛)理解这个认知的\indicate{必要铺垫}。
\begin{explain}
必要铺垫实际上和三个东西有关:认知、认知的一个理解门槛(或是理解途径)、思考回路(能维持的认知)。

一个认知的理解途径可能不止一个,它们除了“都能得到认知”以外可能没有任何共同点,对应的理解门槛也可能没有共同点。这使得我们必须对于每个理解门槛去单独定义“必要铺垫”的概念,而无法一个认知定义统一的必要铺垫。不过在后文中为了简便起见,我们经常省略对理解门槛的强调,请读者自行补全。

必要铺垫中可能会包含“某些外溢的认知为什么不对”“是这种解读不是那种解读”等内容,这部分内容因人而异,有可能某个认知对大多数人来说不需要任何必要铺垫即可理解,但对于少数人来说完全不可理解。

除此之外,必要铺垫中的每个认知,都有可能还需要它自己的必要铺垫才能理解和维持。这种递归式的条件可能会带来无法克服的沟通困难,会使我们被迫在多个话题之间切换,并且经常忘了之前讲的是什么。
\end{explain}
如果\indicate{有一种方法,能使一个思考回路在一段时间内维持某个认知(的某个思考途径)的必要铺垫}\footnote{对于\rigorous,“维持必要铺垫”的意思是“维持这个必要铺垫内的所有认知”。如果写全会导致“认知”这个概念复用,有可能使后面的定义出现歧义,所以省略以避免这一点。},那么就称这个思考回路在这段时间内\indicate{可以理解}这个认知,或是有\indicate{理解(这个认知的)能力}\label{def:行为模式的理解能力},或是在这个认知上\indicate{可沟通};将这个过程称为对这个认知/理解途径的\indicate{铺垫},将方法称为这个认知的一个\indicate{铺垫方法}。
\begin{explain}
这里的“在一段时间内”依照具体语境确定,比如说“某次沟通结束前”。

不同于必要铺垫,我们确实可以对一个认知统一地定义理解能力,而不用对每种理解门槛分别定义。但将理解能力视为和理解门槛(以及理解途径)相关的概念会方便后续的分析,所以我们仍然这么做。同时,铺垫方法仍然是和理解门槛(以及理解途径)相关的概念。

而由于我们的定义是对于某个特定思考回路而言的,统一的定义不会使这个定义变成“大家最终总是能互相理解”的鸡汤式论断。“先用老思考回路接收信息,形成一个新的思考回路,然后再使用新思考回路来理解”的操作不算做老思考回路有理解能力。这使得我们可以放心地将转述者一类的行为模式从“可沟通的对象”中排除出去,因为转述者无法接收任何信息,无法形成任何认知。

如果铺垫方法基于沟通另一方的行动,那么可以称对方有\indicate{铺垫能力},不过我们在后文中不会经常用到这个概念。% 铺垫方法不一定要由对方来解释。如“观察”“试验”“实操”等操作,就完全不需要对方的行动。对方的作用只限于“提了一个建议”。

% 有理解能力不意味着在本次沟通中自身就可以理解这个认知,甚至不意味着在本次沟通中自身可以理解必要铺垫中的认知。大多数铺垫方法会超出对方的能力,并且沟通中也不一定触发了铺垫方法,并且沟通双方可能都未意识到“需要理解某个认知”。即使自身具备了理解能力,如果对方无法完整叙述对应的理解途径,那么也无法通过此次沟通而理解这一认知。

定义理解能力的主要目的是区分“因为没有理解能力而导致的无效沟通”和“因为没有使用铺垫方法而导致的无效沟通”。如果我们不从思考回路的层次来研究,而是将人视为一个整体,那么这两者就是不可区分的。拒绝向不可沟通的行为模式沟通传达信息,按铺垫方法和可沟通的行为模式沟通,才是正确的沟通方式。
\end{explain}

\subsection{资源}
如果思考回路A希望思考回路B通过“接触某个/类”事物的方式获得某个/类认知,那么我们就将这个/类事物称为(思考回路A视角下)这个/类认知的一个\source{资源}/\indicate{理解资源},将这一过程称为思考回路A对思考回路B\indicate{使用}了这个资源,将这个/类认知称为A这次使用资源的\indicate{内容}。
\begin{explain}
这里的思考回路A和思考回路B有可能是不同人的思考回路,也可能是同一个思考回路,或是同一个人身上的不同思考回路。

在不会引起歧义的情况下,我们也会使用“这个资源的内容”的表达,请读者自行补全。

理解资源可能有很多种形式,如“课程”“实验”“沟通”“对某理解途径的转述”等。一个资源包含的内容可能不止一个认知(比如课堂或书籍)。对于人创造的理解资源而言,资源的体量越大,一般来说包含的认知也会越多。

资源的内容完全是A的主观认识,“资源能使B获得认知”的认知可能有“A的亲身经历”“别人的经验”“B的亲身经历(此时可以视作A尝试让B回忆起来)”等多种来源。“A使用资源的内容”不一定包含于“A对这份资源的认知的内容”中,因为它没有对适用范围的限制。这里使用“内容”一词,是将“A的希望”视作客观现象,从而其内容就是它本身。相比起B实际因接触而形成的认知,内容既可能过多(B不具有某些认知的理解门槛/A本身理解错误或存在外溢),也可能过少(A对这份资源或是B没有充足认识)。

思考回路B不一定因接触资源而触发对应的理解途径。如果B无法维持所有的前置认知,那么有可能对其毫无感觉,并且有可能触发其它的行动。触发了理解途径后也可能因各种原因中断,无法完整经历理解途径,获得对应的认知。
\end{explain}
我们将“思考回路A对思考回路B\indicate{使用}了这个资源”也称为思考回路A(向B)\indicate{传播}了这个资源,B(从A)\indicate{接受}了这个资源。我们将A和B分别称为这个资源的\indicate{传播者}和\indicate{接受者}。
\begin{explain}
A使用的资源可以是由A制造的,比如“A自己给B讲”、“A写了一份材料”等方式;也可以使用已有的资源,比如“让B去别的地方学”、“用别人做好的教材”等方式。这也使得对于一次接受来说,传播者可能不唯一。

同样,对于一次传播来说,接受者也可能不唯一(比如讲课、视频、写书等),但这种情况通常可以视为多个独立的资源使用过程。
\end{explain}
我们将\indicate{一次接受中,所有传播者的内容中,B实际获得的认知}\footnote{对于\rigorous,这里和定义认知的内容时类似,也是先将所有传播者的内容取并集,然后再与B获得的认知取交集。}称为这次接受资源的\indicate{收获}。
\begin{explain}
一些传播者会在资源中加入一些额外信息(比如明确说明需要什么前置认知),以增加有效沟通的可能。这些额外信息也会成为内容的一部分。

B获得的认知不一定包含于任何传播者的内容中,因此不一定属于我们这里定义的收获。额外获得的那些认知有可能出于B对资源的深入学习和研究(以使B发现了前人没有发现的东西),也有可能出自B结合自身经历的体验(比如“一千个读者就有一千个哈姆雷特”),也有可能出于B自身已有的外溢(这样获得的认知可能没有实际内容)。

内容不一定是具体的认知,也可能是“关于某个特定领域的知识”这种模糊的判断。这让我们允许考察“在自身不掌握对应认知时使用资源”(比如自学)的现象,和其对应的收获。
\end{explain}

\subsection{话题的并行与切换}
如果某个事件会让某个思考回路/人丢失一些认知,那么我们就称这个事件\indicate{打断}或是\indicate{覆盖}了这些认知。
\begin{explain}
定义中的“人”指意识整体。此处的\indicate{打断}将“维持一个认知”视为一个独立的思考回路,与\hyperref[def:打断]{先前}的含义一致。这里的覆盖适当扩充了\hyperref[def:覆盖]{先前}的定义。

大体来说,覆盖分为两种情况:一种是事件本身产生了新认知,然后新认知覆盖了旧认知;另一种是触发了新的行为模式(或是增强对某个已有行为模式的刺激),从而产生行为模式层面的切换。本小节主要关注后面这一种。

能切换行为模式,覆盖认知的事件有很多种,如“自己突然想到了什么事情”、“突然有别人找”、“切换话题”等,不一定需要在沟通中,不一定源于沟通的双方。
\end{explain}
我们将\indicate{从一个信息中解读出不同的认知}的现象称为\indicate{分歧},也称其中一个认知是另一个认知的\indicate{分歧}。
\begin{explain}
在3.1.2小节中提及的\note{歧义}可以视作解读“词语”这种表达时产生的分歧。

分歧是很普遍的现象,它不局限于某次沟通之中。同一个行为模式随着时间的推移可能对某件事产生不同的看法;不同的人可能对同一个东西有很不同的反应和感悟。解读信息的方式和每个行为模式的出发点、认知的关联方式高度相关。

% 因为关联机制和\note{因果性破坏}的普遍性,任何会引起解读分歧的行为,其成因理论上都是不定的,分歧中的任何一种解读都有可能是成因。实际上的成因往往有某些特点,某些特定种类的事物会更容易比其它种类的事物成为成因,但不可因此就草率归因。
\end{explain}
一个思考回路已有的认知外溢会影响我们的解读方式,从而产生分歧。
\begin{explain}
“表达感谢”有可能被认为是正向情感交流,也有可能被认为是“敷衍了事,只停留在口头上”,也有可能被认为是“太有礼貌了让人很膈应”;“提供帮助”有可能让人十分感动,但也有人会认为自己被看不起了;“夸人坚强”时有可能会让人十分受用,也有可能让人想起自己孤立无援,觉得这是在说风凉话......以上的这些解读,在特定的场合下都是合理且唯一正确的解读,但是脱离了相应场合后,就只剩下了“这人咋这么别扭”。

个人过去的亲身经历、文艺作品中的剧情、他人对同一话题的探讨都有可能外溢并影响认知,这使得一个人对信息可能有任意的解读方式。因此,\indicate{任何表达形式都不能保证达成共识}。
\end{explain}
切换思考回路会覆盖一些认知。因此,如果在沟通中切换了思考回路,我们\indicate{应该将不同思考回路的表达视为不同的话题}。
\begin{explain}
这可能是因为“新思考回路不具备关于某现象的认知”,也可能是因为“新旧思考回路之间有分歧”。

如果有一些事件会触发新思考回路,进而覆盖认知,那么我们就也称这个事件覆盖了这些认知。

旧思考回路不一定就此停止,可能在一段时间后还能切换回来,那些认知也会恢复。但处于新思考回路的这段时间内,那些认知确实丢失了。更极端一些的情况下,两个(或更多)思考回路可以快速切换或并行,但是每个表达只出自某一个思考回路。这会使得交流变得非常混乱。
\end{explain}
在一次沟通中新获得的认知更有可能因话题切换而丢失。
\begin{explain}
新认知可能需要更复杂的刺激(当前思路)才能维持,并且这可能是得到新认知的唯一方式。相比之下,旧认知会和思考回路中的更多想法产生关联,更容易被想起,于是更容易维持。
\end{explain}
我们将\indicate{因沟通中的某个表达而触发的,内容不同的话题}称为当前话题的\indicate{子话题}或\indicate{衍生话题},将当前话题称为子话题的\indicate{母话题}或\indicate{原话题},将这种现象称为话题的\indicate{分化}或是\indicate{分散}。
\begin{explain}
我们已经见过很多衍生话题的例子,如“对话题中某一认知的前置认知的讨论”“某个表达触发了转述者的转述行为”等。

虽然对应的内容不同,但这两个话题通常不是无关的,甚至某个话题算不算子话题,需要依据“我们将什么视为内容相同”而定。对前置认知的讨论可以视为原话题的子话题。它们会共用很多东西,比如说讨论对象和用词。但同时,二者可能在词语的指代、侧重点等方面有不可忽视的差别。这使得原话题和子话题中经常存在很多分歧。如果讨论双方对“现在处于哪一个话题”有统一的认识,那么这种分歧不会造成影响。但如果讨论双方并不统一,或是两个话题在混杂着并行讨论,那么这种分歧就有可能造成更大的误解。
\end{explain}

\subsection{讲解}
\begin{explain}
本小节中,我们将\indicate{需要输出信息的思考回路}简称为\indicate{自身}、\indicate{我们}或\indicate{思考回路A},将\indicate{思考回路A觉得需要输入信息的思考回路}简称为\indicate{对方}或\indicate{思考回路B}。同时,本节会研究“自身观察对方以确定对方是否形成了认知”的过程,所以自身也会接受信息。读者可以使用“有意识维持有效沟通的一方”来代替“自身”的定义。
\end{explain}
% 一个认知仅在维持时才能触发其它行为。
% \begin{explain}
% 这是一个很显而易见的事实。如果不拥有这个认知,那么它也就不会产生任何刺激;如果这个认知被覆盖,那么它产生的刺激只会触发覆盖它的那个认知。

% 之所以强调这一点,是因为我们有的时候会混淆相关性和因果性,错误地将某些行为的触发归因于特定的认知。常见的情况包括:
% \begin{itemize}
% %\setlength{\itemsep}{0pt}
% \item 对方同时产生了认知和行为,并且都被我们(可能是这个思考回路本身)观察到。如果没有观察到对方的其它行为,那么就有可能认为是认知导致了行为。
% \item 对方产生了行为,同时我们对对方的思考产生了认知。对方未必真的具有这个认知,不具有的时候当然无法因此而产生行为。
% \item 我们在之前对其它行为模式的观察中,得到了“某认知会触发某行为”的结论,并且在观察到对方具有某行为后,认为对方也具有该认知,并且认知触发了该行为。
% \end{itemize}
% 我们在通过行为反推认知时,一定要谨慎,不要想当然。
% \end{explain}
如果\indicate{有一种方法,能使思考回路A判断思考回路B是否维持着某个认知},那么就将这个过程称为思考回路A\source{鉴别}了(思考回路B的)该认知,称思考回路A\indicate{可以鉴别}这个认知,或是有\indicate{鉴别(这个认知的)能力};将这个方法称为这个认知的一个\indicate{鉴别方法}。
\begin{explain}
鉴别方法和思考回路A有关,对相关领域更熟悉的思考回路A通常拥有更简单的鉴别方法;和认知有关,不同认知的鉴别方法复杂程度不同,有的认知仅需要观察已有行为或是提一个问题,而有的认知则必须通过数天或更长时间的高强度互动才能得出结论;和思考回路B有关,同一个认知在不同的思考回路下可能有不同的表现和叙述方式。

鉴别是一种客观过程。如果A有可能判断错,那么就不将这个方法称为鉴别,或是称为\indicate{无效}的鉴别方法。使用无效的鉴别方法仍然可以判断B是否具有认知,但是判断结果可能不反映客观现实。

由于“B具有某认知”/“B不具有某认知”是A的认知,鉴别过程中一定有从B到A的信息传递(如果没有信息传递,那么可以确定这一定不是一个鉴别过程)。

A在鉴别某个认知时,不需要自己也具有这个认知。但是当A不具有某个认知时,一般很难做到鉴别这个认知,很可能混淆不同认知的特点。
\end{explain}
对于一个认知和它的一个理解途径,如果思考回路A\indicate{可以鉴别思考回路B是否维持着所有前置认知,并且在B维持每一个认知的情况下,可以对B使用对应的资源,触发对应的理解途径},我们就称思考回路A\indicate{可以讲解}这个认知,或是有\indicate{讲解(这个认知的)能力};将鉴别和转述合称为这个认知的一个\indicate{讲解方法},将这两个步骤分别称为\indicate{鉴别环节}和\indicate{教学环节}。
\begin{explain}
讲解方法是一个和认知、理解途径、思考回路A、思考回路B都相关的概念,具体需要注意的点参见前文,此处不再赘述。

讲解方法不是让B获得对应认知的唯一方式,B也可以以其它方法接触资源得到认知。我们这里定义的出发点是“A已经为这个资源做好了所有可以做的准备,尽最大可能提高了成功率”。

鉴别环节是从B到A的信息传递,教学环节是从A到B的信息传递。两个环节不一定有严格的先后顺序,有可能交叉进行。缺失了其中一个环节的行为全部不可视作讲解方法。实际情况下,缺失鉴别步骤要更常见一些,像是“转述者的叙述”“公开的教学资源”等均不是完整的讲解方法(但它们仍然可以视为相应认知的理解资源)。

鉴别环节不一定需要当场进行,也可根据过往的接触来判断。如果讲解方法发现了“B没有维持某些前置认知”的情况,则应该转而使用对应的铺垫方法。铺垫方法里可能包含前置认知的铺垫方法和讲解方法,如果产生无限递归,或是这些方法打断了其它前置认知,则无法讲解成功。
\end{explain}
我们将\indicate{一个理解途径对应的铺垫方法和讲解方法篇幅之和}称为这个理解途径的\indicate{必要篇幅},将\indicate{思考回路A有讲解能力,并且思考回路B有理解能力的所有理解途径的必要篇幅中,最短的那一个}称为A向B传递这一认知的\indicate{必要篇幅}。
\begin{explain}
我们将铺垫方法、讲解方法、鉴别环节、教学环节也分别称作必要篇幅的铺垫环节、讲解环节、鉴别环节、教学环节。

对于同一认知的不同理解途径,“一个理解途径的所有前置认知”、“一个资源”、“讲解所有的前置认知(并将它们维持)”三者的复杂程度一般是此消彼长的,且都和这一认知本身的复杂程度有关。

在培训领域和游戏领域,存在“(某个技能/技巧的)学习成本/认知成本”等概念,这种概念只计算投入的时间/精力,不直接包含其它类型的成本(如经济开销等)。这些概念与必要篇幅类似,接触过这些词语的读者可以用于对比。

虽然这里也定义了统一的必要篇幅,但是其实际意义不大,A不总是能每次都精确选中必要篇幅最短的理解途径来讲解。并且,“最短”是一个较为模糊的概念,根据具体需求可以指代“认知数量”“持续时间”“所需成本”等多种内容。以后的运用中,还是以分理解途径定义的必要篇幅为主。我们在行文中也会省略必要篇幅所对应的理解途径,请读者自行补全。

“篇幅”仅指在一个话题上的不重复认知,而“传递一个前置认知”等行为被我们视为另一个衍生话题,其中如果出现了和原话题或另一个衍生话题重复的认知则需单独重复计算。由于确实会出现“因不太重视/记不住而丢失某些认知,而必须重新交流”的现象,这种规定是合理的。这种事情可能在“B丢失前置认知”“A丢失了关于‘B掌握多少认知”的认知”等情况下发生。

必要篇幅无限长代表无法让B通过这个资源获得该认知。在实际应用中还要更宽松一些,我们经常因为“必要篇幅充分长”(具体有多长根据情况决定,可能是“两句话说不清楚”“一节课讲不明白”“一辈子也听不懂”等)而放弃对当前资源,选择另一资源或是另一理解路径。

真实的交流中,经常会有其它因素干扰,比如“对同一现象的其它讨论”等,由此造成的打断和重新维持的过程不算在必要篇幅之内,由此而形成并维持的前置认知则需要算在必要篇幅之内(由此我们可以看出必要篇幅的“最短”含义)。如果读者有相关的讨论需求,可以将其它话题视为“同时触发的另一个行为模式”,并对新的行为模式B重新定义相关的必要篇幅。
\end{explain}

\section{巧合有效的沟通}
\begin{explain}
本节主要讨论未经过鉴别理解门槛就产生的认知,以及对应的沟通。
\end{explain}
\subsection{认知的巧合形成}
% \begin{explain}
% 本小节沿用上一小节的记号,将\indicate{需要输出信息的思考回路}简称为\indicate{思考回路A},将\indicate{思考回路A觉得需要输入信息的思考回路}简称为{思考回路B}。
% \end{explain}
如果\indicate{思考回路B形成了一个认知,而思考回路A没有鉴别B是否拥有理解门槛},那么就将“B形成认知”这一过程称为(对A来说的)\indicate{巧合}形成,或是“\indicate{巧合}的认知形成”。而如果该认知因必要篇幅而形成,那么就将其称为(对A来说的)\indicate{完整}形成,或是“\indicate{完整}的认知形成”。
\begin{explain}
“巧合”一词的意思是“不因A具有讲解能力而达成”。以下几种情况都算是巧合形成:
\begin{itemize}
\item A完全没有参与这一形成过程,B(关于A)完全独立地获得了该认知。
\item A知道形成过程本身存在,但没有直接参与,如“A让C给B讲”、“A知道B身处某个环境”等。
\item A直接参与,有可能观察到形成过程也有可能没有,有可能自身的行动对认知形成有影响有可能没有,但有其它主要因素。
\item 认知在A和B的沟通中形成,但A没有“让B形成该认知”的意愿。
\item A有意愿让B形成某认知,但B形成了另一认知。
\item A有意愿让B形成该认知,B也确实形成了,但过程中没有鉴别环节。
\item A鉴别了B拥有某理解路径的理解门槛,但B从另一条理解路径获得了这个认知。
\end{itemize}
一次沟通中,可能巧合地形成多个认知,可能巧合地形成多个不同方面的认知。

一个行为模式的巧合认知形成需要对另一个行为模式而言,这是不可省略的。B形成认知时,可能“对A是巧合,对B不是”,可能“对A不是巧合,对B是”,可能“对A和B都是巧合,但是对C不是”,这些差别在后续分析中十分重要。

如果一份资源不面向特定的接受者(如果使用下一小节中的概念,即“不面向某个共识环境”),而是面对基础不一的群体,那么资源的传播者实际上不可能鉴别每个接受者是否具有理解门槛,所有的“接受者获得认知”的现象对传播者来说都是巧合形成的。为了规避这种现象,传播者可以在资源中加入对前置认知的讲解,以方便接受者或另外的传播者鉴别。
\end{explain}
巧合形成是最常见的信息传递方式。
\begin{explain}
这主要有三方面原因:一方面是“完整展开某个认知的必要篇幅,需要消耗的时间精力过高”,一方面是“多数人不具备完整的讲解能力”,一方面是“一些情况下,沟通只是副作用”。

这一事实容易外溢成“所有的信息传递都是巧合形成”的认知。由此产生的“某个巧合形成的沟通过程是有效/无效的沟通方式”也都是劣质的认知。这个认知有很多不同的表述,比如说“只有少数几个人/没有人懂我”、“孩子/老人/客户怎么这么笨/迟钝/固执,教/说了多少遍都不会/不听”等。

这一事实的三方面原因相辅相成。比如,A如果长期缺乏完整的讲解能力,就会在“某个认知很重要,值得花费相应的时间精力给B讲明白”的时候,无法将其讲解清楚,只能选择其它方法替代。这会让A进一步加深“B总是不听我的”的认知,更加没有动力去提升自身的讲解能力。
\end{explain}
巧合形成的认知可以用于其它认知的完整形成。
\begin{explain}
我们将“其它认知”称为目标认知。

鉴别环节中,思考回路A通过B的某些表现,得到“B维持着某个认知”的鉴别认知,这一过程在多数情况下不包含B的鉴别环节(即“B确定A是否维持着鉴别认知的理解门槛”),而仅是“A对B的单方面观察”“A询问B回答”等简单流程。此时,“A形成鉴别认知”对B是巧合。

铺垫环节中,思考回路A需要向B传递某个前置认知,这在一些情况下可以简化为巧合形成,比如“A问B‘你知不知道这个’,然后发现B知道”、“A跟B刚讲了一个开头,B就想起来了”之类的情况。此时,“B形成前置认知”对A是巧合。

如果以上两环节中的所有认知都必须要完整形成,且不论这些完整形成的篇幅中还会引入新的认知,仅考虑这些篇幅本身,就通常是完全不可行的操作了。从外部因素来看,这会大量消耗时间精力,绝大多数情况下不值得这么做,并且也很容易被其它行动打断;从内部因素来看,这会引入多个衍生话题,从而使得某些前置认知无法维持,使得B不具备理解能力,反而不利于目标认知的形成。
\end{explain}

\subsection{共识环境}
我们将一个/一组认知/行为定义为一个\indicate{共识环境}。如果\indicate{在某环境中的所有行为模式都能维持这个/这些认知},我们也将这一环境称为这一个/一组认知的一个\source{共识环境},并且将这个/这组认知/行为称为该环境\indicate{维持}的一个/一组\indicate{共识}。
\begin{explain}
由于我们可以任意地定义环境,我们也可以(几乎)任意地定义共识环境:
\begin{itemize}
\item 可以是“一种语言/一种文化”。在本指南内,我们默认这是所有行为模式的共识环境。
\item 可以是“一门学科”。对此的具体讨论参见\hyperref[sec:知识与信息]{2.1.1小节}。
\item 可以是“某个现实场所/场景”。一些直接的所见所闻,比如说地理位置、人员构成、重要/规律性事件等信息,可以看做是该环境中的共识。但需要注意的是,“某一行为模式对这一环境的认识”不都属于共识,不同行为模式的侧重点和盲点经常有区别。
\item 可以是“某种特定的措辞/语境”。在一个稳定的讨论环境中,经常会出现这种所有人都熟悉的表达方式。这种表达方式的成因不固定,概括来说,由一些有意或无意的指代而逐渐形成。
\item 可以是“某几个行为模式及其相互接触”。这可以规避“有新行为模式进入环境后,该环境丢失了一些共识”的现象。这种定义下,上述情景可以被描述为“新行为模式和这个环境接触”。这在研究一些小圈子时很方便,此时小圈子经常既是触发环境又是共识环境。
\item 可以是“(某环境下)所有维持某个/某组认知的行为模式及其相互接触”。这种定义看起来完全是废话,但它是一种很方便的理论分析工具,以至于它经常外溢——很多人会有“处于同一环境中的所有行为模式维持着对应的共识”的认知。这会为交流带来一些麻烦。
\item 可以是“某次沟通”,此时参与的所有行为模式都在从这次沟通中获得认知。不过这些新获得的认知不一定是共识,有些行为模式可能不重视一些地方而没有获得,有些行为模式可能被覆盖或形成了别的认知。在使用这种定义时,一定要谨慎,不应草率地将某个行为模式的认知当成所有行为模式的共识。
\item 可以是“一个行为模式”,此时环境中的所有共识就是这个行为模式能维持的所有认知。不过这种环境的讨论限制较大,我们将其并入下一条一起讨论。
\item 可以是“一个人的意识活动”,此时环境中的所有共识是这个人同时触发的所有行为模式维持的所有认知。这是一种很重要的环境,因为其他人无法直接接触这一环境,会少获得很多信息。我们将其称为一个人的\indicate{意识环境}或\indicate{脑内环境}。
\end{itemize}
对前一种定义来说,“共识环境中的行为模式”定义为“维持这些认知/行为的行为模式”。我们在后文中不会明确指出使用的定义是哪一种,但读者应该总是能自行分辨。

共识环境不一定就是这些行为模式的触发环境。不过不同的行为模式之间越像(可能除了“属于不同的个人”以外没有其它区别),触发环境也就越像,相互之间的共识也就越多,触发环境就越有可能也是共识环境。
\end{explain}
我们将\indicate{一个人(因获得了这些共识而)从不属于这一共识环境变为属于这一共识环境}称为这个人\indicate{进入}了这个共识环境,将\indicate{一个人(因丢失了这些共识而)从不属于这一共识环境变为属于这一共识环境}称为这个人\indicate{脱离}了这个共识环境。
\begin{explain}
这里的术语和\note{行为模式}处的术语一致。

关于共识环境的讨论其实和对“很多行为模式”的讨论没有本质差别,在一些情况下,我们也可以将共识环境内的所有行为模式视为同一个行为模式,并且可以据此定义每个共识环境中的集体意识、集体潜意识、立场(也即出于这一共识环境的\note{价值判断})等概念。此处提出这个概念,主要是因为“环境”容易被我们(有意识或无意识地)当成一种独立的对象,判断“某个环境是不是共识环境”“某个行为模式是否属于某个共识环境”会比较方便。

只要一个环境会被人用来思考“属于这个环境的人都有什么特征,会去做什么”,我们就将其称为“将环境视为共识环境使用”,并且在这个意义上将其视为共识环境。由此,我们将“具有一些共同特征的集体”也视为共识环境。这么思考的人可能不清楚共识环境的概念,但是使用“某种身份”等措辞也会起到同样的效果。
\end{explain}
对于一个共识环境和一个认知的理解途径,如果有一种方法,能够使共识环境中的每一个行为模式都维持这个认知的所有前置认知,并且共识环境可以理解每个前置认知,那么就称共识环境\indicate{可以理解}这个认知,或是有\indicate{理解(这个认知的)能力}\label{def:共识环境的理解能力},或是在这个认知上\indicate{可沟通}。
\begin{explain}
这里的定义几乎是照搬“思考回路的理解能力”。该定义实际要求“共识环境中的所有行为模式使用同样的理解途径”,定义中出现的“理解前置认知”的嵌套也是出于此要求\footnote{这个定义不会导致“因为定义得过于严格,递归条件过强,导致共识环境没有任何理解能力”的情况。一个人在初次接触共识环境时,当然会基于自身情况产生认知。不同人的自身情况不同,这一阶段不会有共识。但随着接触逐渐加深,一些重要认知会显著更为熟练,最终成为不需要任何理解路径的条件反射。此时我们就可以以这些认知为起始点来递归定义理解能力了。}。

对于一个共识环境可以理解的认知,有一个行为模式使用某个资源理解了这一认知,只要对应的理解途径不是太难懂,那么这一环境中所有的行为模式都能使用同样的资源理解这一认知。这一现象很容易让这些行为模式产生“只要这么讲就能给人讲懂”的认知,将其视为有效的信息传递方式——但它仅在共识环境中保证有效。

共识环境中的行为模式也可以从资源中理解出其它认知,不过我们应该将其视为另一件事。如果这个行为模式的新认知覆盖了一部分共识,那么就应视为其已经脱离了这一共识环境。

% 如单独的行为模式一样,共识环境也有可能从不同的理解途径中获得同一认知,这使得“某一共识的一种理解途径”有可能不是共识的一部分,不同的行为模式会使用不同的理解途径。这有可能是源于对同一理解途径做了不同程度的简化/补充,也有可能是本身理解途径就不同。
\end{explain}
不属于共识环境中的行为模式接触资源时,不保证能得到相同的认知。
\begin{explain}
单看这一句话像是废话。在实际应用中,这个结论最容易出问题的地方是它的前提条件:我们在给资源的时候,经常不去区分行为模式是否属于共识环境。主要的疏忽可以归结为两种:
\begin{itemize}
\item 没有发现对应的共识环境,没有意识到资源需要在相应的共识环境内使用,将某一共识环境中有效的信息传递方式视为无条件下传递这个信息的有效方式。我们将“将某种其它类型的环境误认为共识环境,并让环境中的行为模式接触资源”也算进这一类中。
\item 发现了对应的共识环境,但将某一不属于该共识环境的行为模式误认为属于该共识环境,并使其接触对应的资源。这种情况例子很多,如“学习成绩差就是因为不专心听讲/做作业”“我说过的话别人就要听”等。
\end{itemize}
无论是哪种情况,一个行为模式A直接认为“某资源对B有效”,并让共识环境外的行为模式B接触资源,B如果形成对应认知的过程对A一定是巧合的。

在一些情况下,我们可以将共识环境视为资源/讲解方法的适用范围。共识环境一般比真正的适用范围(有理解能力的所有行为模式)要小,但“判断一个行为模式是否属于某个共识环境”在很多时候要比“判断一个行为模式是否能维持全部前置认知”要容易,要更为实用。
\end{explain}
我们可以得到一个\indicate{不可能三角}:不存在对\indicate{所有人}都\indicate{有效}的\indicate{简短}讲解方法。
\begin{explain}
如果不面向所有人,只在共识环境中讲解,那可以做到简短而有效;如果不求听众一定能获得该认知,那可以使用非常精炼高深的讲法;如果不求在短时间内讲完,那可以根据每个人已有的前置认知来因材施教。

在实际使用中,我们经常仅关注“在共识环境中的有效讲解”,并且从中得到认知和行为,比如“听不懂就是没认真听,态度不端正”等。但如果没意识到相应的共识环境,这就只是巧合的认知形成,这一方法本身也会在共识环境外失效,不是正确可推广的认知。
\end{explain}

\subsection{巧合行动}
对一个共识环境和一个不属于它的行为模式,如果它们会因同一个刺激而触发同样的\note{行动},那么就将这种现象称为行为模式关于共识环境的/行为模式和共识环境之间的\indicate{巧合行动}或是\indicate{巧合相同}的行动。
\begin{explain}
在实际使用时,我们经常先选定一个行为模式A和一个和A相关(可能是包含A,可能是A能模拟)的共识环境,讨论这一共识环境和另一行为模式B之间的巧合行动。在后文中,为了简便,我们会在不引起歧义时省略共识环境,而是用“两个行为模式之间的巧合行动”的措辞,请读者自行补齐。

两个行为模式的巧合行动一定是由不同行为导致的。虽然它们有同样的刺激和同样的结果,但触发的行为链不一样。定义中包含共识环境,就是为了刻画出两行为模式中,触发的行为链的差别。如果我们不关注具体的行为链,只关注刺激和行动,那么这就可以视作同一个行为。

认知的巧合形成不都可以视作巧合行动,有可能恰巧使用同一份资源触发了同一种理解途径获得了对应的认知。这里复用“巧合”一词,是因为巧合行动是巧合认知形成的重要原因。
\end{explain}
“A因为发现了B有巧合行动,就认为B属于共识环境”的现象是一种认知外溢。
\begin{explain}
单独的“B属于共识环境”的整体认知可能没什么影响。我们这里关注的是“认为B拥有共识环境中的行为和理解能力”的这种更具体的,精确到行为和认知层面的认知。这才是真正影响沟通的现象。为了叙述方便,我们暂且将“巧合行动”和“共识环境中的其它共识”分别称为“标志行动”和“其它行为”(这么命名是因为A会根据标志行动来识别共识环境)。

同一个客观存在的共识环境,会对A造成很多种不同的影响:
\begin{itemize}
\item A明确认识到并理解共识环境,并且有有效的方法判断B是否属于共识环境;
\item A错误地认为B属于某个共识环境(这一步可能以很多不同的形式实现,比如“认为B具有某种身份”),并且按照共识环境认为B拥有其它行为;
\item A从共识环境中得到了“有标志行动就有其它行为”的认知(此时A大都对共识环境没有清楚认识,会将这一认知视为普遍规律),并且根据这一认知认为B有其它行为;
\item A自身属于共识环境,并且自身会因标志行动而触发其它行为,并且将“拥有其它行为”的认知外溢到了B身上。
\end{itemize}
后三种情况都会造成“A认为B拥有其它行为”的结果。事实上,A在没有明确认识到这一共识环境时,因为缺少了验证认知的适用范围的步骤,反而更容易产生“B拥有其它行为”的外溢。
\end{explain}
如果B被误认属于的共识环境包含某个认知,那么这就会成为A关于此认知的一次无效的鉴别。
\begin{explain}
我们暂时将这一认知称为“目标认知”。会被误认的巧合行动包括但不限于以下这些:
\begin{itemize}
\item 触发行为的刺激实际是“在和对方沟通”的现象,而不是“对方话中的内容”。这可能包括“积极地响应和附和(但附和中不包含信息)”;“被问‘知不知道/听没听懂’时,不根据自身情况判断,而是直接回答‘知道/听懂了’”等情况。
\item 目标认知会触发一些行为,但B通过另一种不包含目标认知的行为链巧合触发了同一行为。这可能包括“A能确定B有某个认知,于是觉得B有某个前置认知”、“A发现B表达了该认知,实际上B只是在转述别人的观点”等情况。
\item B在某些环境下可以维持这一认知,但在另一些环境下不可以。环境可能包括“某个话题中”、“另一个共识环境”、“某种特定的措辞”等。
\end{itemize}
在这些鉴别方法失效时,有一些更细致的方法可以用来进一步验证B的情况,比如“让B自己完整地叙述”,但具体什么方法有效,还需要根据实际情况判断。

如果A在讲解时产生了这样的误认,那么鉴别环节就会失效,对应的认知也就成为了对A巧合的认知。
\end{explain}

\section{解读\label{sec:解读}}
按照\hyperref[def:解读]{4.0节中的定义},如果某现象作为刺激,触发了思考回路,并形成了认知,那么我们就将\indicate{由此形成的认知}称为对这一现象(信息)的\indicate{解读},或是从这一现象中\indicate{获得}的\indicate{解读/认知/信息}。
\begin{explain}
我们有可能从同一现象中获得很多种不同种类的解读:
\begin{itemize}
\item 解读可以是对现象本身的\note{特征提取}。提取的特征有可能被用于继续思考这一现象,产生其它解读。提取出的特点不一定正确,不一定全面,特点相关的认知也不一定适用。
\item 解读可以是对现象本身的\note{理解}。我们需要找到导致这一现象的过程(自身找到的过程不保证是真实的过程)。如果现象是人的行动,那么对现象的理解就变成了寻找这一行为的动机。对方有可能是有意为之,也有可能只是没多想,无意识地触发了某个行为。这一理解过程很容易产生认知外溢,给别人安上不真实的动机。
\item 如果现象具有\note{内容}(比如说“对某认知的转述”,或是“接受某种资源”),那么解读可以是对其内容的理解。这和上一条有不同之处,比如对于“老师讲课”这个现象来说,对现象本身的理解可能是“老师想教会学生”,对其内容的理解则是“知识是什么内容”。对具有内容的现象的解读总是会同时有这两种理解方向,从而会产生天然的分歧。
\item 解读可以是对现象本身的\note{预料}。这样的解读会影响我们对待这一现象的方式,指引我们的后续行动。预料不一定准确,后续行动也不一定是恰当的,具体讨论参照前文。
\item 解读可以是与该现象有关联的其它认知。这些认知(或是它们的内容)可能与该现象具有实际的关联,也有可能这种关联完全是由外溢产生(此时会产生\note{因果性破坏})。
\end{itemize}
在本节中,我们主要关注“理解现象本身”和“理解现象的内容”两方面。
\end{explain}

\subsection{对资源的解读}
\begin{explain}
本小节考察“思考回路A接触资源时,是否能获得某个/组特定认知”的情况。我们将这个/组认知称为思考回路A接触资源的\indicate{目标认知}\footnote{注意思考回路A不一定有要获取认知的主动性(所以目标认知不一定是A接触资源的内容)。目标认知也可能是其它思考回路的目标,或者是任何思考回路都没有目标,我们仅考察“A是否获得了认知”的客观事实。},并只关注“有助于获得该认知”的解读。如果还有其它的目标认知,那么我们将其视为另外的资源使用过程。

本小节主要关注对资源的内容的解读。如果一个资源本身有方便/不方便解读内容的地方,我们也会因此而解读资源本身的创造和设计。除此之外的部分不在本小节讨论范围之内。
\end{explain}
如果A可以在某些条件下接触同一份资源直到获得目标认知,那么就称这份资源在这些条件下对A是\indicate{稳定}的或是\indicate{可以重复使用}的。如果没有A可以达成的条件,那么就称这份资源对A是\indicate{不稳定}的。
\begin{explain}
根据资源的类型不同,这里的“某些条件”可能是“另一个人在场/有空(可以讲解)”、“自己在场/有空(听别人讲/自己练习)”、“有网(可以看公开课/下载资源)”、“付钱(各种类型的课程)”、“带在身边(书籍、电脑等资源)”、“场地和设备(实验或是收集数据)”等。

A无法获取的资源对A当然是不稳定的。除此之外,另一种常见情况是“A只有一次(少数几次)机会接触资源”,如课堂或是演讲。若A无法在这几次机会中稳定获得认知,这种资源对A就是不稳定的。我们也有将不稳定资源转化为稳定资源的方法,比如做笔记或者录像;而“找别的课/资料学习”应当被视为另一种独立的资源。

A即使一直无法维持所有的前置认知,一直无法获得目标认知,但如果可以一直接触资源,那么资源也算是对A稳定的。

资源应当能够(在满足理解门槛时)触发目标认知的某一理解途径(如果理解途径也属于目标认知,那么理解途径也应固定;否则可以在不同的接触时触发不同的理解途径)。如“虽然每天都在讲课,但每天讲的都是新内容”的情况不能被算作稳定资源;“定期复习”可以算作稳定资源(但它可能有其它问题,比如在A能够维持认知的时候复习,就会浪费时间)。

稳定的资源最大的优势在于可以多次使用,并且每次使用时获得不同的认知。这使得我们可以有“第一遍先学个大概,知道整体方向,第二遍再去关注细节”、“遇到不会的/不熟的先停下来,去找别的资源/往前翻/练习/复习到掌握后再继续”之类的方法,这可以在很大程度上保证我们稳定地获得目标认知。
\end{explain}
一部分学习方法实际上是在制作(稳定的)资源而非使用资源。
\begin{explain}
制作资源有很多种常见的方式,比如:
\begin{itemize}
\item\indicate{复制},如“背诵/记忆”、“抄写/拍照/复印/录制”、“笔记/日记/随笔”等。这些方式要不然直接保留了资源本身,要不然保留了资源的内容。
\item\indicate{整理},如“制作表格”、“编写教材”等。这些方式保留了资源的内容,但是更换了理解途径,使其更适配于某些使用者(比如“自己”“初学者”等)。
\item\indicate{索引},如“制作档案”、“大概留一个印象”、“知道怎么找专业人士”等。这些方式不直接保留认知的内容,但也提供了接触资源的途径。
\item\indicate{训练},如“刷题”、“按流程实操”等。这些方式让人可以维持更多的前置认知,以便“把思路顺下来”,达到其它更深入认知的理解门槛。
\end{itemize}
一些制作资源的方法同时也有获得认知的效果,比如说“训练同样可以巩固认知,查漏补缺”、“背诵本身一定记住了对应的内容”、“整理时或许可以促进整体理解”等。这些资源可以起到“辅助获取目标认知”的效果,所以获得的这些认知可以视为前置认知。但“通过制作资源从而获得目标认知”不是可以稳定成功的方法。“能否获取目标认知”由“是否达到了理解门槛”决定,制作资源只是一种巧合的方法。

同时,还有另一种“将制作资源获得的前置认知就当做教育的目标”的思路,比如“走入社会才知道语文书上写的是什么”“为人父母才知道父母的不易和爱”之类。虽然客观上确实有一些“缺乏社会经验/生活经历(对应的前置认知)从而无法理解”的障碍,但因此就将其视为教育所能做到的极限,是舍本逐末的行为。事实上,大多数这样的行为中,这样制作的资源对理解目标认知没有任何作用,当事人完全是根据另外的理解途径(比如自身经历)来获得目标认知的。将其作为教育的核心传承下去反而会起到反效果,在不相互理解的情况下强行相互包容,掩盖了真实存在的矛盾。具体讨论请见第六章。
\end{explain}
A接触资源时,如果对于目标认知的每个前置认知,都要不然可以获得前置认知本身,要不然可以获得前置认知的鉴别方法,那么我们就称这个资源是A获取目标认知的一份\indicate{铺垫资源},或是这份资源对A提供了\indicate{够用}/\indicate{充足}/\indicate{充分}的铺垫。
\begin{explain}
定义中的前置认知与理解途径相关,为了使叙述不过于臃肿,省略了这一点。

定义中的“获得”只关注“在接触资源后拥有”,而不关注“接触资源之前是否拥有”。这里定义的“充足”在语感上接近于“A接触资源后可以自行补足前置认知”。

为了方便后续讨论,我们一般在讨论“资源是否充分”时,会假设该资源对A是稳定的,以排除“因接触时间过短而产生的解读失败”情况。

我们也可以分别定义“资源可以让A获得所有前置认知”或是“资源可以让A获得所有前置认知的鉴别方法”,但这两种情况都比较少见到,最常见的还是二者的混合。我们既会经常遇到“有小细节看不懂得去查一查”的情况,需要用到其它的资源;也不太需要每一步都脚踏实地地鉴别,经常是遇到问题再去修正。

如果一份资源是由人精心制作的,质量很高,那么它通常既包含前置认知本身,又包含其鉴别方法。同时,资源中还可能包括多种不同理解途径的前置认知。当然,如果前置认知本身不属于创作者所关注的重点(这可能是进阶的材料),那么这份资源可能无法用于获取前置认知。我们还需要使用其它资源(一些更基础的内容)来获取。

但大多数资源还是需要A自己来分辨前置认知是什么。对于复杂的自然现象,需要事先掌握某些基础的自然现象;对于一段完整的工序,需要事先掌握每个单独的步骤;对于某段论述,需要事先掌握其中的用词指代和逻辑......只有当A自身的调研能力能够满足“获取这份资源的前置认知”的需求时,这份资源对A才是铺垫充足的。而调研能力不强的人就无法从资源中获取目标认知,如果强行接触资源,可能会获取其它认知。
\end{explain}
如果A获得的认知属于该资源的内容,那么就称A获取了(相对于该内容的传播者)\source{准确}的认知。
\begin{explain}
资源的内容是主观的,依据传播者的意图而定。一种常见且符合直觉的情况是,资源是由人创造的,或包含某些特定的知识。这种情况下A只有获取这些知识时,才是准确的。

另一种情况是“A自身作为传播者,要对自己使用某个资源”。这种情况下,资源的内容由A任意指定。
\begin{itemize}
\item 如果A有“获得任何认知都行”的想法,那么任何认知对A都是准确的;
\item 如果A完全没有此类想法,只是巧合地获得了认知,那么任何认知对A都是不准确的;
\item 如果A有“要得到准确的理解,获得真实的原因”之类的意图,那么只有符合实际情况的认知才是准确的。
\end{itemize}
如果没有特别说明,后文的篇幅总是认为A在对自己使用资源时,有“要得到真实的理解”的目标,并且据此来判断准确。
\end{explain}

\subsection{对人的解读}
\begin{explain}
本小节和下一小节考察“思考回路A接触行为模式B时,对人B的某行为/行为模式的理解”的情况。我们将B拥有的行为/行为模式称为思考回路A的\indicate{目标行为/行为模式},并只关注“有助于获得对该行为/行为模式的理解”的解读。B身上如果还有其它行为/行为模式,那么我们将其视为另外的解读过程。

本小节和下一小节主要关注行为/行为模式的形成和触发,不直接包含对行为/行为模式内容的解读。如果其内容有助于理解行为/行为模式的形成和触发,那么我们也会关注。除此之外的部分不在本小节和下一小节的讨论范围之内。
    
如果使用上一小节的概念,我们这里将行为/行为模式本身(或是其转述)视为“内容是行为/行为模式的资源”,并且A接触该资源的目标认知是“行为/行为模式的形成和触发”。本小节主要讨论稳定性,而下一小节主要讨论充足铺垫。
\end{explain}
如果目标行为/行为模式对A是解读的稳定资源,那么就将它们称为对A\indicate{稳定}/\source{可见}的行为/行为模式。
\begin{explain}
对于不同的行为/行为模式,稳定需要满足的条件也不同。如果是动作为主的,那么条件可能是“A在场旁观”、“录像”等;如果是口头表达为主的,那么“录音”也算;如果是技巧或是想法为主的,那么就需要B通过转述来表达自己的思路,此时可能需要A有一些询问技巧(这是直接对人的调研能力)。在接下来的篇幅中,为了叙述方便,在不引起歧义的情况下,我们总是忽略如“录像”“录音”等需要额外信息载体的间接解读方式。这些间接方式也符合相关论述,请读者自行补全。

并非B的所有行为和行为模式都对A可见。对于每一种特定的刺激,B在此刺激下确实能触发某些行为/行为模式(也有可能完全不触发),但A不一定能观察到所有的刺激类型。具体原因可以分为以下几类:
\begin{itemize}
\item A没有信息输入,即A没有观测到某一类刺激/行为的条件。比如“A不在场”“A缺乏前置知识”等;
\item B没有信息输出,比如“刺激和行动中间有由想法组成的行为链,A无法观测”“B的表达能力不足,没法说清楚自己的情况”等;
\item A产生了竞争和覆盖,即A对B的行为解读被自身的外溢认知干扰。这类外溢认知包括但不限于“我很了解B”“我很了解这个领域”“这个现象就是这个原因/就会这么发展”的情况。
\item B产生了竞争和覆盖,即A自身的行动(或是间接的信息记录)会更改刺激类型。“旁观”“询问”“指挥”“记录”“重复(比如总是教不会)”等行为有可能触发B的其它行为/行为模式,导致原本需要接触的行为/行为模式被覆盖。
\end{itemize}
如果A不具有良好的询问和分析能力,那么通常只能接触B的有限几个行为模式,也只有这么几个稳定的行为模式可以解读。
\end{explain}
\indicate{将对行为模式的解读视为对人的解读,是一种外溢。}
\begin{explain}
这是本指南一直坚持并多次强调的观点。

在接触环境固定的情况下,“将行为模式视为人”这一行为本身不会出现什么问题(这里不计“A对行为模式的解读有误/不足”的情况)。我们同样能对“这个人在这种环境下的所有行动”得到完全的理解,同样能充分地预测。本指南中也多次将行为模式称为人,以顺应读者的习惯性思路。

将行为模式视为人,最重要的问题是无法分析竞争过程。当A预测B在另一个环境中的行动时,如果只知道B的一个(或少量)行为模式,就只会在这些行为模式之中分析B最可能的行为(这个分析过程不一定是有意识的思考,也可能是无意识的联想)。这种对竞争过程的推断很可能不符合现实,从而会得出错误的结论。
\end{explain}

\subsection{对竞争过程的解读\label{sec:对竞争过程的解读}}
对于某个行动,我们将所有参与\note{竞争}的行为/认知称为这个竞争的\indicate{范围}或是这个行动的\indicate{竞争范围},将范围中的行为称为这次竞争/这次行动的\indicate{潜在行为}。如果范围有一些共同的特点,就将其称为此次竞争的一个\indicate{限制}。我们也称这个竞争是“在这个\indicate{范围}中的竞争”或是“\indicate{限制}在这个特点/这些行为/认知中的竞争”。
\begin{explain}
此处为了方便讨论,我们将没有参与竞争而直接触发的行为也视为参与了竞争,并且该竞争的范围中只有这一个行为。同时,我们仍然沿用“没有参与竞争”的表达。

限制可以采用一些比较宽泛的特点,比如说“符合格式和韵律”、“用不超过四个字指代一种事物”、“某些特定的符号和表达”等,都可以作为某些竞争的限制。在电影、舞蹈、绘画、音乐、游戏、诗歌等一些特定的文化领域,还会在特定的限制上发展出“服化道”、“镜头语言”、“肢体语言”、“空间构图”、“和弦走向”、“技能叙事”、“隐喻比兴”等多种不同的设计语言(解读时也可能出现“有很多种备选的意思,无法确定”“怎么解释好像都有道理”“过度解读”等情况,此处不展开讨论)。

以上这些例子中,竞争过程都是有意的设计,都是从某些特定的行动中,挑选出最能符合自己所想表达的内容的那一个。但这个过程同样也可以由“只是学到了这种特定的表达”的方式产生,此时它们来源于不同的竞争过程(后者竞争的范围中只有一个行为),不应视为同一种竞争过程,不应因“使用了高深的表达”就认为“有复杂的思想”,也不应因“使用了高深的表达”就认为“是在当名词党”。针对竞争过程的评价必须依据具体的竞争过程来判断。

一个人意识活动中的竞争有一个通用的限制:它只能在这个人已有的行为/行为模式中产生。同时,不同于上述例子中有意的设计,意识活动中大多数竞争过程是不自知的,特别是在“触发了某个范围很广的行为模式,以至于很多其它行为被覆盖”的情况下。只有在少数关键决策时,我们才能发现“自己在面对一个竞争过程”,但即使不计算只有单一行为参与的竞争过程,实际发生的竞争过程数量也要远多于我们能察觉的。
\end{explain}
理解一个竞争过程,需要的前置信息是“这一竞争的范围”和“范围中行为之间的覆盖关系”。我们将其称为这一行为的\indicate{逻辑}或简称为\indicate{行为逻辑}。
\begin{explain}
理解一个行动需要找到它的成因,而它的成因总可以视为一个竞争过程。因此,这也是理解一个行动(进而行为/行为模式/人)的前置信息。

在A试图理解B的行动时,这些前置信息不都是对A可见的。A有可能无法直接观测到一些行为,只能依赖“B的转述”等一些间接方法。甚至B在分析自己的行为时,也经常会遇到自身无法察觉的行为/想法。为此,我们可以将条件做一些弱化:如果某些行为总是会被覆盖,那么我们可以将其视为不参与竞争。我们只去观察那些有明显表现的行为即可。

有明显表现的行为有可能看起来和触发条件没有直接关系,或者是容易被视为性格或类似因素(如紧张、恐惧、自负、小动作等)。这些行为同样有对应的竞争过程。如果我们的目标是控制原行动,但理解了其竞争过程无助于控制,那么我们进一步对这些行为展开分析(或许还能产生递归的分析)可能会有所帮助。但仍需注意,原行动的竞争过程和“潜在行为的形成”的竞争过程是两个独立的竞争过程。即使理解后者对理解前者经常有帮助,对后者的理解也不可替代对前者的理解,同时对后者的理解也不属于对前者理解的一部分,它们是不同的认知。

熟悉博弈论、微观经济学或其它相关领域的读者可能会尝试在此引入专业的分析,通过相应学科的语言来分析出占优策略。这确实是可行的方式,这套分析框架理论上可以用于任何竞争、博弈和决策过程。上述的两个要求,在相应术语下变为“需要将哪些策略纳入考虑,将哪些策略排除”和“这些策略之间到底谁占优”。在分析时不一定能引入统一的策略收益(即使我们已经认为价值是主观的)有可能会出现“三种策略相互占优”,如剪刀石头布一样之类的情况;也不能认为不同的场合下(即使环境基本上没变,只是时间不同),同一个人的策略集一定不变。我们可以将这两种缺陷视为“人的非理性”的具体刻画。对于非理性的研究在对应学科已有论述,本指南在这里不展开。
\end{explain}
\indicate{对于任何行动,我们在尝试理解时,总是需要先得知它的竞争过程,再根据竞争过程来模拟。}任何不经过此步骤的理解都最多只能得到(对自己而言)巧合正确的认识。
\begin{explain}
最终的分析结果有可能是“仅有一种行为参与竞争,并且完全下意识地想做就做了”,有可能是“在两难中踌躇徘徊”,有可能是“经过了深思熟虑的决策”,所有的这些都有可能通过“猜测”“预感”“共情”“默契”“共鸣”等对不同人采信度不同的,不包含分析竞争过程的其它方式得到,它们也可能在某些共识环境下确实正确,但不可视为这种行动的统一原因。

对于结果相同的竞争,不同的竞争范围会产生不同的覆盖关系和不同的覆盖方法。其中一个比较好理解的例子是:因理解途径和必要篇幅不同,维持一个前置认知所使用的方法也不同。当我们需要排除某些行为(经常是负面且不太可控的,比如说偷懒、畏惧、片面、冲动)时,我们就需要用很多理由来说服自己。但如果另一个人没有这些缺点,那么就不需要克服这些缺点,自然也不会有意培养一个说服的行为来覆盖。而有时却恰好相反,我们使用的说服方法能说服我们自己,却无法说服对方。这两种情况下,“要求对方培养相应的说服能力”的行为不仅没有必要,而且浪费时间,并且分散话题,同时对方也可能无法理解,但它们的成因截然不同,解决方法也截然相反。

% 另一种例子是,同一种行为可以有很多种不同的特点,其中每种特点都可以作为决策的动机。有些特点十分宏观和抽象,比如“维护自身团体的利益”“象征着某种被压抑的需求”;有些特点只针对具体情况,比如“这么做就可以实现目标”“也不知道为什么,反正我就是喜欢这样”(这些可能是外溢行为/认知,但此处不关注来源)。

另一种类似例子是,“根本不存在某个行为”和“存在某个行为但是被覆盖”这两种情况会产生同样的影响,都会使我们产生“另一个人和我的做法不一样”的认知。我们在感觉上更容易将前者视作“无意的”,而把后者视作“有意的”,并且将覆盖视作主动决策。在一些语境下,我们会将前者称为“蠢”,而将后者称为“坏”。两种方向的误读均有可能发生:我们既可能在另一个人不存在行为(比如完全不知情,有另一套行为逻辑)的时候,认为对方存心对着干;又有可能在另一个人有明确认识和动机的情况下,用“还小”“不熟”“粗心”等方式理解。这些解读都既有可能对又有可能错。如果想教育对方,那么前者应该培养这一行为,后者应该消除覆盖的行为。方法错配就会起到反效果:对前者讲解“如何消除”会不知所云,而对后者讲解“应该怎么做”会加重抗拒的强度。
\end{explain}

\section{指挥\label{sec:指挥}}
\begin{explain}
虽然看起来不是很像,但是本节中的行为模式A、行为模式B、旁观者可以指代同一个人身上的不同行为模式,理论分析全部通用。我们可以用本节的理论来分析和处理“眼高手低”“半途而废”“无法自控地走神”“突然心就乱了”“知道了很多道理但仍然过不好这一生”等情况。自我指挥和解读的部分在下文中不会单独讨论,读者可以自行套用。
\end{explain}
\subsection{指挥}
我们将\indicate{行为模式A触发行为模式B的某个行动}的过程称为行为模式A\indicate{指挥}行为模式B(做这个行动)。如果可以确定具体的刺激(比如A的某个行为),我们也称A使用这一刺激\indicate{指挥}B,或是简称为这一刺激\indicate{指挥}B。
\begin{explain}
这里的指挥和第二章中定义的\hyperref[def:控制人]{控制}的适用范围不同。控制针对一个人具有的行为/行为模式,而指挥所针对的行动不一定对应着某个实际存在的行为/行为模式。多次指挥同一种行动,可能每次的刺激和行为链都不相同。

这里采用这种定义,是因为在很多情况下,我们不关心行动的刺激和成因,需要的仅是行动本身或其影响,如“完成某项任务”“获取某个认知”等。

\indicate{指挥是一种客观现象},不需要带有A的主观意图。因此,它的指代范围比日常语境下的指挥要更大,包含了很多A或B没有意识到的行动(很多情况下A和B双方都没有意识到)。它在这里的实际语义比较贴合“所有可能被认为是A指挥B的现象”。此处对指挥的定义不以“A和B是否有主观认识”来区分,仅关注二者的触发,以便建立更普遍的理论。和主观认识有关的部分在稍后单独定义并分析。

A指挥B不一定需要A行动,“A在场”等现象本身也可以提供刺激(比如说“老师维持纪律”“粉丝见面”“嫉妒别人”“见前恋人尴尬”等)。

A和B都分别可能对这类指挥现象有从一无所知到一手策划的不同程度的认识,有从积极迎合到无关紧要再到饱受其苦的不同程度的感受。本节内我们不关注这些主观评价。
\end{explain}
如果某个过程\indicate{需要B的某个行动才能进行},那么我们称这个过程具有指挥B这一行动的\indicate{需求},或简称为(对B的)\indicate{指挥需求};如果A\indicate{认为B需要执行某个行动},那么我们就称A有指挥B这一行动的\indicate{意图},或简称为(对B的)\indicate{指挥意图}。
\begin{explain}
\indicate{指挥需求是一种客观现象},完全根据过程的性质而定。过程可能是某个与意识无关的现象,比如操作机器;也有可能是“某个行为”或是“事务的某个步骤”这类虽然和主观意识有关,但可以独立分析的过程;也有可能是“B这么做能触发/避免A的某个行为”这类看起来完全是人际关系互动,既可以看成是“B指挥A”也可以看成是“B疲于应付A”的过程。

指挥现象与指挥需求的区别在于,指挥是“某个刺激导致了B的行动”,B的行动是果;指挥需求是“B的行动导致了某个过程”,B的行动是因。

\indicate{指挥意图是一种主观现象},完全是A的认知。该认知可能来自某些合理的思考与决策,比如说“因为目标有指挥需求所以要去指挥”;也可能完全是污染,只是“自己不知道为什么,但感觉一定得这样”。A一定自知自身的指挥意图,而B不一定对A的指挥意图有认识;B可以在没有发现指挥意图的情况下配合这一指挥意图。

A有指挥意图的情况下,“完成这一意图”的过程就有了指挥需求,但是其它的相关过程(比如“维持和A的关系”等)不一定;A有指挥意图不代表A有对应的行动,也不代表A的行动提供的刺激可以用于指挥B。
\end{explain}
指挥、指挥需求、指挥意图这三者可以在某些事件中同时存在,这也使得我们很容易出现“这三者是密不可分整体”的外溢认知。
\begin{explain}
比如说,经常会有人将这三者笼统地称为投射\footnote{这里指的不是投射的原始学术定义,而是其在社区讨论时因过度泛化而产生的外溢。投射被用来指代过多了的现象,已经产生了严重的歧义。}。这会为我们的分析带来明确的障碍,经常会让我们解读出某些实际上不存在的动机。所以,本指南全盘避免了引入投射性认同及其相关术语。

在A和B的接触中,对于B的某个行动及其产生的影响来说,是否具有以上三者,以上三者是否具有因果性,有数百种%\footnote{我们按照“二者之间有4中不同的因果性可能”来计算,则总共有$2^3\times 4^3=512$种情况。注意其中也有“因为不存在指挥所以产生了指挥意图”等情况。其中不少情况在实际中很难出现,但总能构造出符合条件的极端性例子。}
不同的可能。其中既有“因为发现了有指挥意图所以产生了指挥需求然后去指挥”的情况,也有“有指挥意图但是不会指挥”的情况,也有“因为无法指挥所以反而产生了指挥意图”的情况,也有“只根据‘B会被指挥’的预料而做了计划,B不配合就无法进入下一步”的情况......以上种种可能都既可能存在又可能不存在,在分析时必须依据实际情况判断。
\end{explain}

\subsection{响应指挥}
如果B执行了指挥/指挥需求/指挥意图对应的行动,那么那么就称B\indicate{响应}/\indicate{听从}/\indicate{服从}了这一指挥/指挥需求/指挥意图;反之则称B\indicate{没有}/\indicate{无法响应}(\indicate{听从}/\indicate{服从})这一指挥/指挥需求/指挥意图。
\begin{explain}
指挥只关注“B有行动”这一现象。B只要响应了指挥需求/指挥意图,那么就产生了一个指挥现象,B也就响应了这一指挥。这是三者关联性(以及关联性外溢)的一种重要产生途径。

如果我们严格按照定义,那么就会发现“B无法响应指挥”这种情况不对应任何实际现象,因为此时不存在指挥。我们在这里将“B无法响应指挥”重新定义为“某个刺激无法触发B的行动”的现象。比如在“可以指挥其它行为模式的刺激对B无效”,或是“存在指挥需求/指挥意图,但B没有响应”的情况下,使用“无法指挥”就是合适的措辞。
\end{explain}
如果B\note{可控}地响应了某指挥/指挥需求/指挥意图,那么我们就将这种响应也称为B\indicate{配合}了或是\indicate{服务}于这一指挥/指挥需求/指挥意图。
\begin{explain}
这里的“可控”如之前的定义,指“B经过自己的决策,决定是否响应”。

从表面上来看,“响应指挥”有可能是A发现了B的某个条件反射,然后在B不自知的情况下加以利用;而“响应指挥需求”和“响应指挥意图”却都需要B先发现这二者,看起来像是必须自知的决策。但指挥需求和指挥意图本身也可以通过竞争和关联机制而成为B行为的触发条件,从而变成不自知也不可控的潜意识行为。

分别根据A和B的主观意图,我们可以将指挥现象分为4类:
\begin{itemize}
\item A没有指挥意图,行动对B不可控。这种情况下,指挥其实可以视为一种完全自动,不出于任何人主观意图的现象。如果这样的指挥稳定存在,那么就可能会形成一个共识环境。A和B可能分别对这一共识环境有任何形式的看法和态度。
\item A有指挥意图,行动对B不可控。这就可以视作A对B的\hyperref[def:控制人]{控制}。这有“父母让孩子远离危险”“有经验的从业者传授技巧”等正面例子,也有“过度干涉”“限制自由”“剥夺隐私”“测试服从性”“支配”等例子。
\item A没有指挥意图,行为对B可控。此时我们就需要关注B的真实意图:可能是“完全不在意”,可能是“讨好/畏惧/迎合A”,可能是“对A有所图”,可能是“有备无患”,可能是“有自己的信念和追求”。不同的原因会对应着不同的评价和对待方法。
\item A有指挥意图,行为对B可控。此时除了上述对A和B行为的评价以外,还需考虑A和B之间的博弈。依据A和B水平的不同,博弈可能处于从完全没有策略到无限层嵌套之间的任意复杂程度。
\end{itemize}
如果B没有响应指挥,那么也可以按以上标准分为四类,并且展开对应的讨论。此处略去这一部分篇幅。

“只关注到指挥现象”或是“只发现了指挥现象(或是对应的刺激)”的现象,无论对旁观者还是A/B来说,都相当常见。如果我们只从指挥现象出发思考,而不关注其它交流,那么在A视角中不能直接获得“B是否有控制能力”和“B对行动的态度和意愿”的信息,第一、三种情况不可区分,第二、四种情况不可区分;在B视角中不能直接获得“A是否有指挥意图”和“A对行动的态度和意愿”的信息,第一、二种情况不可区分,第三、四种情况不可区分。而旁观者则四种情况都不可区分,只能笼统地分析。

不可区分的现象之间容易产生理解和归因的外溢,这种外溢还有可能影响对指挥现象和其它现象的解读。这会造成一些不利的后果,如“本来可以区分‘对方是否有意’的信息被覆盖”、“影响A或B对该现象的性质判断和后续的行动,激化冲突”等。
\end{explain}

\subsection{交流和讲解时的指挥需求\label{sec:交流和讲解时的指挥需求}}
对于“行为模式B获得某个认知”的过程,我们将该认知的\indicate{一个理解途径中,对B的所有指挥需求}称为这个理解途径关于B的\indicate{自理部分},将每个指挥需求称为一个\indicate{自理需求}。如果B能响应某个自理需求,那么就称B在这一自理需求上有自理能力;如果B能够响应所有的自理需求,那么我们就称B对这个理解途径有\indicate{自理能力}。
\begin{explain}
自理部分有很多容易理解的例子,比如说“形成/维持前置认知”“知道某句话的重点”“练习以巩固”“接触资源”等。之前已经讨论过,这里不再赘述。如果B有自理能力,同时响应了所有的自理需求,那么B就获得了这一认知。

自理能力的定义不需要有讲解,但本小节主要还是关注“A给B讲解”这一过程及其可能遇到的障碍。除了“让B获得目标认知”的指挥意图外,这一过程经常还会包含很多其它的指挥现象。这里将指挥需求重命名为自理能力,是为了防止在后续讨论中,“指挥”一词出现过多,进而产生不必要的理解障碍。
\end{explain}
如果A可以识别B在某一自理需求上是否有自理能力,那么就称A可以\indicate{鉴别}这一自理需求。
\begin{explain}
此处的鉴别比之前定义的\note{鉴别}涵盖范围要大,之前的鉴别只包括对“维持前置认知”这种特定的自理需求的考察。前文的讨论同样可以直接迁移到此处。

A的鉴别能力会在两种方面出问题:一种是“A认为B需要做的行为不属于真实的自理需求”,另一种是“A缺乏判断自理能力的方式”。这两种都会导致A产生额外的,和理解途径无关的指挥意图,并且成为B的负担。

前者一般发生在“A没有正确归因‘如何获得认知’”时。A可能通过某种表面归因获得了一些经验(比如说“要努力”之类,很多时候可以算作正确的废话),也可能通过无序的关联而得到了外溢的认知(比如“将结果当成原因”之类),也可能A处于某个共识环境内,B缺乏对应前置。

后者一般发生在“A和B的交流不畅”时。由于B没有输出、A没有输入,或是信息被覆盖,B的自理能力对A不\note{可见}。如果A完全无法/不去判断B是否具有能力,在A觉得B需要获得能力时,即使B已经获得了对应的能力,A仍然会一遍一遍地指挥B获得能力;在A觉得B已经掌握能力时,也会指挥B做一些需要该能力作为前置的行动,从而导致B无法完成。

此时,B经常出现“为了完成A的指挥而行动”或是“按照A评判的标准来行动”的现象。B本身没有掌握对应的内容,这对A完全是巧合行动。

这两种情况经常同时出现,此时会产生“A一见到B,就会说一些只有A自己会看重的话”(可以是在任意事情上任意程度的看重,比如说“某种现象的特定处理方法”,或是“A数十年来的人生经验”)的现象,此时A的讲解行为退化为纯粹的转述。A所说的话都会刺激A联想到目标认知,但这只是A自身的关联,对B则不是有效的理解途径。于是会产生“A觉得自己已经从多角度全方位地讲了很多遍,而B就是不听”的现象,而此现象在B眼中是“A又在唠叨了,(如果知道A在说什么)自己只要搭理就会惹上更大的麻烦/(如果不知道A在说什么)完全听不懂只是浪费时间还限制自由”。双方都会因此而产生不满和负面评价。

B有可能从另外的途径(比如亲身经历)获取和A相同的认知,这对A是巧合的。B此时可能会同时出现“感受到了A的爱”“辜负了A的良苦用心”等感受,但B如果只感到了这些,而没有想到或是覆盖了“A的讲解无效”的认知的话,就也会做出和A相同的举动,重复相同的无效沟通。这是不可取且无意义的。
\end{explain}
在实际情况中,我们经常能观察到这样一种情况:一些人(使用C指代)可以鉴别自身的自理能力,并且因此而获得认知。
\begin{explain}
根据语境不同,我们会将这种能力称为“学习能力”、“自学能力”、“察言观色”等,并将其评价为“聪明”“会学习”“开窍”“机灵”“圆滑”“八面玲珑”等。为了方便理解和方便叙述,以下例子将以学校作为背景,不过理论可以通用地迁移。

在所有有助于信息传递的能力中,学习能力是较为常见的一种。它是唯一不明确需求“解读他人的能力”这一前置,就能掌握的能力(即使对于“C很擅长社交”的情况也是如此,C可以在“只掌握了某种特定的社交方式”的情况下就能处理所有自己会接触到的社交内容)。相比于“需要明确鉴别对方是否具有前置认知”的讲解能力,学习能力更容易自然习得:C可以更简易地获得自己的认知情况,从而有针对性地调整。

习得了学习能力就可以保证“获取认知”的目标达成,于是就可以承担对应的“获得认知”的责任。而其它没有学习能力的人(使用B指代)无法承担这样的责任。对于不明就里的人(使用A指代)而言,很容易将其总结成“勤奋”“用心”等特点,并且得到“只要勤奋就能学好”之类的认知。

A如果对认知形成的过程没有充分的认识,并以“只要用心就能学好”的观点要求B,那么B就需要被迫同时处理“时间分配”“学习顺序”“应付A并接受惩罚”等多项超出自身能力范围之外的自理需求。除了少数B能够巧合地拥有学习能力,少数B能够巧合地按部就班补起来以外,这种要求对绝大部分B来说,只会浪费时间而没有其它收益。

我们如果想有效地教会B,甚至是给B培养出学习能力,就必须具体地分析B的行为逻辑,而不能笼统地推卸责任,将所有事情都称为“态度”。

当ABC其中的两者是同一人X的不同行为模式时,还会产生一些更为复杂的变化(此处仅讨论A的看法和行为):
\begin{itemize}
\item 如果B和C都属于X,那么A很容易据此认为X的偏科实际上是出于态度不端正的偏心/不专心,而因此对X展开道德和关系上的指责。部分A可能会产生“人总有长处和短处”之类的认知,并因此同时接纳X的优缺点,但这样的认知也有可能外溢,缺失判断标准地无条件地包容一些不应被包容,有明显实际害处的事情。

\item 如果A和C都属于X,那么X很容易据此认为学习没什么难的,无法观察到B的难处并提供有效的帮助,从而有可能产生有意识或无意识的歧视。即使X很热心也很有耐心,经常给B讲解,B也很容易什么都学不到(此时X可能同时是B关系最亲密,最依赖和信任的人,但是此处不讨论人际关系问题)。

\item 如果A和B都属于X,那么X获得了“C很努力刻苦地学习”的认知后,可能会吸取一些表面经验并以此来要求自身,但不得其法的学习很容易除了假努力外什么都没有。同时,X如果认为“刻苦努力”之类的认知有效,就可能会拿去指挥别人;也可能因为“自己努力了没有成功”,就演变成X自我攻击的素材。
\end{itemize}
\end{explain}

\subsection{指挥现象的一个实例解读\label{sec:指挥现象的一个实例解读}}
\indicate{为了加深对指挥现象复杂性的理解,我们重点考察这样一种情况作为例子:A原本有某个行为a,在B做了某个行为b之后,A的行为a消失。}
\begin{explain}
这里的行为a可以有很多种不同的类型:既可能有“闲逛”、“娱乐”、“分享”、“邀请”、“演示”等带有正面评价的行为,也可能有“唠叨”、“盘问”、“辱骂”、“责罚”、“威胁”等带有负面评价的行为。

我们对这两种情况的分析有时会不同:当行为b打断前者时,我们一般会将其称为“扫兴”,将其纯粹地视为打断;而行为b打断后者时,却有一种完全不同的分析:将行为b(或其影响)视为A的一种需求,将行为a视为A表达该需求的方式(有可能会使用“压抑在深层潜意识中的需求”之类的表达),将“行为a消失”视为A满足了自身的需求,将整个过程视为A对B的一次指挥(或是控制)。下方的讨论仅关注这种负面(并且B希望制止)的行为。
\end{explain}
\indicate{但这种分析仅代表了多种情况中的一种,可能的真实情况包括但不限于:}
\begin{explain}
\begin{itemize}
\item A完全没有指挥行为b的意图,只是被行为b转移了注意力,覆盖了之前的行为。此时严格来说,应该将这一过程视为B对A的指挥。A没有思考过“指挥B”的原因,B对A的理解不来自A自身,而是来自其它地方。%如果有必要,我们还可以考察B除了“制止A的行为”外还是否有其它的动机(动机可以很正当,比如说学习),此处不做展开。
\item A完全没有指挥行为b的意图,但是会为自己的行为找原因。这些原因包括但不限于“我只是希望B好”“我只是很难受/害怕”“我大意了/没想到”等。在A有相应表达(可能是主动说出,或是在别人询问时给出)时,在B(或者其它的旁观者)看来,这和“A说出了自己的指挥意图”不可区分。

但是A的“讲述原因”这一行为可能是由“被问到”“看到B”“行为a”等原因触发的结果,而A的“总结原因”这一行为也可能是由“被问到”“听到别人说”“自己想”等原因触发的结果。A所给出的原因可能并不代表行为a的动机(同时并非有意欺骗),而只是A因缺乏分析自身行为的能力而产生的外溢认知。如果行为a可以由其它刺激触发,那么基于A给出的原因进行的后续分析和处理就无法达到“制止A的行为”的效果。
\item A有指挥行为b的意图,但是未经决策,而是由指挥意图直接触发了行为a。指挥意图和行为a之间的关联有可能来自A之前的思考,也有可能来自某些其它人的讲解,也有可能是在纯粹模仿其它人的行为。如果A还记得完整的思考过程(无论是自己的还是别人的),那么A有可能有相应表达,此时在B(或者其它的旁观者)看来,这和“A说出了自己的决策过程”不可区分。但与上一种情况类似,和A讨论思路无助于打断“由指挥意图直接触发行为a”的过程。
\item A有指挥行为b的意图和明确的决策过程,并且据此行事。此时可以和A讨论这一过程是否合理,是否有助于完成目标。但A的决策未必可信,相关的出发点未必可讨论,这也不保证成功。
\end{itemize}
以上的分析中,为了简化考虑没有涉及“A因为B的劝说而触发其它行为”的现象(原因有很多,比如说“不想合作”、“感觉到背叛/变心”、“(在有指挥意图时)觉得B不听话/想偷懒/没眼力”、“只是单纯地应激”、“话题发散到讨论别的东西”等)。如果产生这种现象,则应将其视为一个单独的事项,独立分析和解决。

从我们对上述四种情况的讨论中也可以看出,B对后一种情况的认知会覆盖前一种。如果B去询问A的动机,那么就有可能从第一种变成第二种;如果B去询问A的决策过程,那么就有可能从第二种变成第三种。B得到的回答,可能只是触发了A的一个无关的转述。再加上“B视角下,无法区分A的某些行为出于相邻的哪种情况”的现象,最后很容易使B产生“B最终发现A这么振振有词,果然是第四种”的认知(同时A也会有对应的“B怎么这么难以交流”的认知)。真实的原因因此被掩盖。
\end{explain}
\indicate{无论我们的目标是“理解行为a”还是“制止行为a”,“A无意识地指挥B”的分析都不够用。}
\begin{explain}
对于“理解行为a”的目标来说,仅有第三、四种情况可以适用“A有压抑在深层潜意识中的需求”的说法,才有可能出现“A觉知了自己的深层潜意识”的现象,进而才有“接受自己”“悦纳自己”之类的进展。这不因我们将其称为“深层潜意识”“核心信念”“自证预言”还是什么别的名字而有所改变。

前两种情况和后两种情况的主要区别在于,后两种情况的“指挥B”作为A的指挥意图,是明确的目标;而前两种情况的“指挥B”只是A行为的影响。不加分辨地将影响直接视为A潜意识的一环是不妥当的行为,它会覆盖我们对A真实行为逻辑的分析,使我们无法找到触发行为a的真实原因和对应的竞争过程。

对于“制止行为a”的目标来说,如果“A无意识地指挥B”的分析过程能够强有力地说服A,让A每次触发行为a的时候都想起这个论述过程,并且因此决定不再执行行为a,那么无论是上述情况中的哪一种,这一目标都能实现。而如果没有这个在触发行为a之后的竞争和覆盖过程,而只能在触发之前加入竞争的话,如上述讨论,这四种情况都无法保证目标达成。它们的失败方式也各不相同:
\begin{itemize}
\item 对第四种情况来说,A的决策过程中可能包含一些B不认同的认知。在双方都无法改变认知的情况下,就会以观念的冲突(可以是搁置、吵起来等不同形式)结束一场无成果的讨论。
\item 对于第三种情况来说,说服的尝试仍然可能如上变为观念之争,但无论争论无论成功还是失败,都是无效的。即使成功,A的“由指挥意图直接触发行为a”的行为依然存在。如果A还记得相应的论述并且愿意为此而改变的话(这在一些比如“为了B好”的意图中很常见),就会遇到一个问题:A不一定有充足的自理能力,不一定知道怎样改才是B可以接受的。由此出现可能“A十分别扭,不知道怎么和B相处”,也可能“在A看起来改了很多,在B看起来没啥变化”等多种情况。在B的“你从来没听过我”和A的“你还要我怎么样”中,冲突不但没有解决,反而扩大了。
\item 对于第二种情况来说,说服也可能变成观念之争,也可能成功了但无效。但除了“A主动改变行为但不知道怎么改”以外还会遇到另一种情况。A的动机和指挥意图相关,但并不完全相同。它在大多数时候会更具体(比如说“成绩应该再高一点”“想找个人聊聊天”),从而A不能直接从这里联想到自己提出的动机。于是,A可能虽然同意改也准备改,但是找不到任何应该改的地方,从而没有任何改变。
\item 对第一种情况来说,除了上述三种情况外,还有另一种可能:A有另外的行为逻辑,不认为自己的行为和B有关,根本不认为它们会对B有影响(主要指负面影响)。这很容易让A产生对B的负面认知,错误归因B的动机,认为B就是过于敏感,从而无视B的反馈。双方虽然形式上有沟通,但没有任何的信息传递。
\end{itemize}
以上的四种情况中,均只关注A对行为a的处理。实际情况中,A不仅获得了行为a,而且还得到了其它很多行为(有可能能形成完整的行为模式)。这些行为可能只有在B眼中能体现某个统一特征,而对A来说每个都有不同的动机。一个个处理时有可能按下葫芦起了瓢,远比单个行为更加复杂和棘手。

以上的讨论中不包括A和B行为逻辑的动态变化。如果考虑这一点,无论是与外界交互还是自身产生的新关联。都会极大增加处理的复杂程度,收集到的有关行为逻辑的信息可能在一段时间后完全失效。
\end{explain}

\section{总结与讨论}
\subsection{本章总结}
本章内容围绕\indicate{信息传递}的过程而展开,讨论“形成了认知的沟通过程”以及这一类过程的具体原理。

本章中定义的有效沟通需要满足“有一方形成了认知”的条件,这使得它比我们日常生活中使用的“沟通”概念涵盖范围更小。为了防止读者在理解上出现分歧,我们首先讨论了几种不会被我们视作沟通的情况:\indicate{零散表达}、\indicate{转述}和\indicate{拒绝沟通}。我们仍然能从这些现象中获得认知,但这只是单方面的。有这种行为的行为模式只有输出而没有输入,即使自身有沟通意愿,也会被覆盖。在此情况下,沟通现象完全是单方面的副作用。其中,\indicate{转述}同时作为一种好用的理论分析工具和反例,在本章后续内容中被多次提及。

在此之后,我们便得以专心于讨论本章的主要内容。我们首先讨论了“认知的形成过程”本身。虽然相关内容在第二、三章中已有涉及,但是当时仅讨论了“已经形成的认知有什么特点”,而不涉及“形成认知需要什么条件”这一方面。本章从此处出发,将获得认知的过程抽象为“维持\indicate{前置认知}并触发\indicate{理解途径}”的过程。我们因此得以更加细致地描述认知形成,本章的很多重要概念都依据前置认知和理解途径(而不是“获得认知”这一整体)而定义。我们将可以触发理解途径的事物称为\indicate{资源},并着重强调了“同样的资源在不同人眼中不同”的客观事实。

从而,我们得以将两个行为模式的沟通视为获取认知的资源来处理。在沟通中,话题的\indicate{分散}不可避免,而由此产生的解读\indicate{分歧}会覆盖前置认知,打断理解途径,影响沟通。这是我们在所有沟通中所必须注意的事情。除此之外,认知的输出者是否对另一方有足够的了解,也决定了输出者能否有效调整\indicate{讲解方式}并最终成功地输入信息。我们引入了\indicate{鉴别}这一概念来描述“输出者确认另一方是否维持前置认知”的过程,并且以此定义了输出者的\indicate{讲解能力}。

在初步讨论完“有效的信息传递所需的要求”之后,我们需要来研究“虽然存在有效的信息传递,但并不通过稳定的方法实现”的情况。我们用\indicate{巧合}来指代这一类信息传递现象。如果我们从巧合现象中提取认知,总结规律,就有可能产生错误的判断。每一种讲解,每一个资源都有它的适用范围,而认知的适用范围就是“前置认知和理解途径”,我们也据此定义了\indicate{共识环境}的概念。资源只在对应的共识环境内有效。

在给出“信息传递的方法在什么时候会失效”的刻画后,我们下一步需要考察的是“如何有效地信息传递”。这需要再次深入研究获取认知的过程。我们从解读资源的过程中提出了\indicate{稳定}(可以一直接触资源)和\indicate{充分}(可以补齐前置认知)的概念。

本指南的向读者介绍的核心能力之一便是有效地分析人的意识现象这一复杂系统。另外,不管是“听从讲解”还是“鉴别对方是否具有前置认知”,我们都离不开对人的解读。故此处我们用了一些篇幅来详细展示“分析一个人所需的做法”。本指南认为,除了单独观察并总结人的每个行动/行为(行为模式)外,理解行为还需要一个必要步骤。竞争过程导致了行动/行为,而\indicate{准确认识行动/行为,需要找到真实的竞争过程}。

最后,我们具体分析了一种经常与沟通与讲解同时出现,并且原理十分相似的现象:\indicate{指挥}。当信息输出者自身的讲解能力不足时,经常会需要另一方在一定程度上\indicate{自理}来补齐讲解的短板,从而产生了指挥的需求。除此之外,我们解读指挥现象的时候也经常会得到不准确的认知,因此我们将指挥细分为\indicate{指挥}、\indicate{指挥需求}和\indicate{指挥意图}三种不同的概念,以澄清可能出现的误解。

本章的一、三、五节主要围绕“沟通为什么会失败”,而二、四节主要围绕“沟通为什么会成功”,从两方面出发,尽可能全面地讨论了信息传递和认知形成的现象。

\subsection{本指南的写作考虑}
读者容易注意到本指南在写作风格上很有特点。其中一些特点在\hyperref[sec:应用部分前言]{应用部分前言}中已有提及。

比如说,本指南会花很大力气去定义很多概念,并且花很大篇幅去辨析这一概念的适用范围。这样做的主要目的,是尽可能地使更多读者可以准确地理解这些概念,以方便之后的正确应用。这些概念大多较为贴近日常,本指南认为读者应该对其中的大多数至少有大体的认知。但贴近日常同时也会导致它们的歧义很重,这些歧义难免会干扰本指南后续应用这些概念时的信息传达。所以本指南将相当多一部分的篇幅放在了“排除这些歧义认知”上,甚至在某些小节会见到“除了定义和概念辨析以外什么都没有”的现象。对于任何一位的读者来说,其中大部分排除歧义的篇幅都是不必要的,但它们应该总是会对一些读者起作用。本指南尽可能地全面讨论了所有可能产生(有明显影响的)歧义的部分,希望没有太大的遗漏。

同时,本指南假设读者对于“精细操作概念”的做法不是很熟悉。由于我们客观上有“区分含义不同但用词相同的概念”的需求,我们需要做的是“区分每个不同的义项”(本指南使用“给每个义项分别起名字然后对比讨论”的方式)和“确定概念的适用范围”(如应用部分前言中提到的“对xxx的yyy”)两方面。前者只要起名字就可以解决,而后者则需要一定的训练。如果读者无法为概念配上适用条件,并从此处开始思考,那么在实际操作中就又会回归“带有歧义地讨论”的情况。为此,在概念的讨论中也包括了这些适用条件的具体用法,包括“什么时候可以省略”“什么时候应该注意”等内容,尽可能方便读者自学和自我鉴别。这一部分内容对于已经习惯于精细操作概念的读者是不必要的,而对于完全没有意识到必要性的读者来说,也无异于文字游戏。

以上两方面会不可避免地影响本指南的可读性。不过为了完整地展开分析,同时能准确地向读者传达分析思路,这些概念作为前置认知,是必不可少的。我个人能力有限,无法兼顾两方面,只能优先选择准确传达。本指南为因此而造成的阅读障碍表示由衷的歉意。

另外一个会影响可读性的方面是本书缺乏实际案例。读者经常可以在其它的书里见到“咨询师遇见一位来访者,通过交谈、分析和计划,逐步发现来访者的心理问题,并使用有效的方法改变和解决”的小故事,这些故事能使读者更容易理解咨询师所使用的方法论,容易理解所关注的重点和对应的解决思路。

——表面看起来是这样。但本指南认为,案例在实际上起到的作用,主要是“污染了读者,给读者灌输了不全面的认知”。这主要有两种可能的原因:一方面是“一些读者见什么信什么”,导致他们只学习到了“在这种特定案例下的结论”,而“通用的分析方法”被覆盖了(即使作者着重强调),进而他们会把特定结论草率地套到别处;另一方面是“读者没见过其它可能的成因”,而自身又没有充分掌握通用的分析方法,导致特定结论不可控地和其它东西产生关联\footnote{这两种原因也同样是“从效果来看,被覆盖和缺失不可区分”的一个例子。}。

为了避免先入为主,我们有必要展示完整的分析过程,涵盖所有可能的情况,也即“理解行为要从对应的竞争过程出发”(见\hyperref[sec:对竞争过程的解读]{4.4.3小节})。这使得我们在分析任何一个现象时,都需要细碎冗长地分类讨论。\hyperref[sec:情绪、感受与其分析与处理方法]{3.7节}和\hyperref[sec:指挥现象的一个实例解读]{4.5.4小节}中,我们分别按此方法讨论了一种情况。我们即使排除了很多相对关系不大的分支,但仍然不可避免地需要分很多类分别讨论。其中每一类都是可能真实发生的情况,都可以视为一种案例。相比之下,我们在\hyperref[sec:交流和讲解时的指挥需求]{4.5.3小节}中对学习能力的讨论就相当不完善,遗漏了很多种可能的情况。这种不可压缩的必要篇幅使得我们不可能举很多这样的例子,只能挑重点来集中演示。

读者如果确实需要一些例子来辅助理解,可以参考以下两方面内容:如果读者具有编程(尤其是运维)相关基础,可以使用找bug的思路来理解。在不考虑“重写整个业务逻辑模块”这一举动的情况下(毕竟我们没法对人这么做),我们总是要定位到具体的代码执行层面,才可以确定问题所在。问题可能是单个模块的设计漏洞,也可能是多个模块间的逻辑冲突,代码整体的复杂性产生了不可预知性,导致我们不可能保证在遇到问题时就直接定位。如果读者没有编程基础,那么可以参考(大学以下的)教育领域。教育领域是最为集中的信息传递环境,非常全面系统地梳理了所有常见的理解顺序。其中需要特别关注的是“总是有办法把差生带起来的老师”。他们需要处理各种不同学生的前置认知/自理能力缺失的问题,每个人都具有一套行之有效的定位问题的方式。这些方式可以用于参考。相比于心理咨询方面的资源,这二者的最大优势在于资源充足,量又大,处理的问题又集中,可以基本实现“覆盖所有可能的原因”的要求,避免出现先入为主的现象。

除此之外,本指南在写作过程中经常出现“想到什么写什么”的情况。这在一定程度上影响了逻辑的连贯,进而会打断思路,影响可读性。读者可能会发现,本指南经常有“废了好大劲,兜兜转转又绕回之前讨论过的话题”之类的情况。虽然这在一定程度上也有助于理解,但比起“认识的不断深化”之类的评价,本指南更愿意将其单纯地视为“对结构没有充分的掌握,导致无法充分地解耦”的现象。类似的现象也包括“前面定义了一个概念,后面又定义了没什么区别的另一个概念”等。本指南为因此而造成的理解障碍深表歉意。

\section{实操:日常沟通与社交}
\hfill\begin{minipage}{0.6\textwidth}
\fontsize{8pt}{12pt}\selectfont\fontsize{8pt}{12pt}

\raggedright 你是沉默的,只对黑夜倾诉更多。总是藏起话不说,要把自我吞没。\footnote{\bilibili{BV1r4411A77J}\another\netease{1865377022}。}

\raggedleft 闹闹丶\&果汁凉菜《无题》

\raggedright 满载思考的脑袋,偏爱沉默。盛不住心事的我,倾囊而出,不怕干涸。\footnote{\bilibili{BV1Hf4y1L7MF}\another\netease{1979007507}。}

\raggedleft vsinger团队\&果汁凉菜《夏虫》

\raggedright 谁人能够听到些微我的歌声吗?\footnote{\bilibili{av2711298}\another\sing{2871399}。\\}

\raggedleft COPY《回音》

\end{minipage}

可能一些读者从本章一开始就在想“怎么还没讲到这个话题”了。本章所讨论的沟通一直以“产生新认知”为前提,这看上去充满了目的性。而日常生活中的交流,大多不会有“一定要说服谁”“无止境的勾心斗角”之类的情况。如果不加限制地在所有沟通的场合下都这么厚黑地思考,机心也太重了。

对每次沟通都这么分析,是一种“不能充分掌握和熟练运用本章内容”的体现。在那些“明确可以判断出参与沟通的人都属于同一个共识环境”的情况下,该玩就玩,该乐就乐,完全不需要考虑这些。我们所需要特别关注的,只有一件事:什么情况下,参与沟通的人不属于同一共识环境。
% \begin{explain}
% 如果我们有一个标准的共识环境作为参考,那么可以将情况分为“自己不属于这一共识环境”(比如“觉得自己无法融入某个集体”、“觉得别人不应该这么做”之类的)和“别人不属于这一共识环境”(比如“(虽然自己推荐,但是对方仍然)对某一方面不感兴趣”、“”)
% \end{explain}
\begin{explain}
上一次还是兴高采烈地一起去玩,下一次就一个人都约不到,所有人都说自己没时间;某些人一直以来都是很热心且可靠,但在某一次却突然发飙说“我忍你很久了什么事都往我身上推”;自己明明是一片好心,却不知怎么地就碰到了别人的雷点,被人揪着细节一顿数落......

这些例子想举多少就能举多少。我们当然可以将这些现象统一称为“人心难测”,但到底有多难测呢?我们还有没有别的视角,来认识这些事情,进而解决这些事情?
\end{explain}
为了方便后续讨论,我们先大体上将所有日常接触分成两类:在一个话题中,如果\indicate{分歧越来越多},那么我们就将其称为\indicate{沟通不畅}、\indicate{不良沟通}或\indicate{不良话题},否则就将其称为\indicate{(话题/沟通的)良性维持}、\indicate{良性沟通}或\indicate{良性话题}。
\begin{explain}
这个概念既可以放在“两个人之间”,重点考察这一沟通过程;也可以放在“一个人和一个共识环境”中间,重点考察这个人对某一共识环境的理解和接受程度。

此处“良性/不良”的价值判断仅针对“话题维持”本身,若还有其它的目标(如“考验”、“隐瞒”等),则可以不关注这一价值判断。

这里的“分歧越来越多”的定义比较模糊,我们考虑的主要是“分歧会为沟通(和其它方面的相处)带来什么不利影响”。对于“会形成长期认知的分歧”,如果一个分歧通过后续的沟通得以消除,那么我们就将其视为“分歧减少了”,此时比较的是“产生分歧的速度和达成共识的速度哪个快”;对于“不会形成长期认知的分歧”(我们关注这种情况一般是因为“这一分歧产生了某些影响”,比如说“实际上没有必要的防备”“越解释越乱”之类的),如果在它被遗忘之前(可能是这个话题结束之前,或是一些类似的其它情况)没有被纠正,那么我们也将其视作永久性的“分歧增多”。

定义中的“在一个话题中”的前提条件不可省略。我们经常能遇到如“其它方面都聊得挺好的,就只有某个特定的话题极为固执,怎么说都说不通”“和另一个人只有某一方面的交互,其它方面没什么牵扯”的情况。在两个人(甚至是两个行为模式)的接触中,“某些话题是良性沟通,某些话题是不良沟通”的情况很常见,必须按话题分开来各自处理。相对地,“因为某个话题沟通不畅就认为这个人不好相处”、“因为能在某个话题上良性互动就十分信任这个人”之类的认知都属于外溢。
\end{explain}
日常沟通中,虽然我们不需要很明确的“一定要形成某个认知”的意识,但是随着交流,客观上确实也有一些认知(和行为)在潜移默化地形成。这些巧合形成的认知不一定有助于良性维持某一话题。本节的主要关注点即为“如何识别和处理因共识不足而导致的沟通不畅”问题。这可以大致分为三方面:
\begin{itemize}
\item 识别与进入共识环境:参与某些互动需要熟悉相关的对象,包括但不限于“某个文艺作品”、“某个操作流程”、“某人的内心活动”等。此处的共识环境主要指的是“在话题开始前就已经存在”的部分,不涉及话题中的演化。
\item 识别分歧:依靠某些现象,确定“参与话题的另一方和自己有分歧”,包括但不限于“自己/对方忽略了什么”、“自己/对方理解错了什么”、“自己和对方的重点不同”等。我们在这里重点关注话题中自然产生的分歧。
\item 消除分歧:在发现分歧后,通过一定的方式调整自己或对方的认知,以消除分歧。这些方式包括但不限于“分析和思考”、“解释”、“转移话题”等。
\end{itemize}
\begin{explain}
这样的行为可以有很多种不同类型的负面评价,包括但不限于“每天都在勾心斗角很累”、“这样做就是为了讨好别人,令人看不起”、“城府越深越不可信任”等。本指南持有以下观点(读者也会在后文中看到相关讨论):这种评价的一种主要成因恰恰是“不具有充足的有效沟通与达成共识的能力,从而才会疲于应付”。

以下篇幅将着重关注技术问题,不会专门讨论态度和动机方面。本指南不持“大家一定要在每个话题上都好好沟通”、“识别了分歧以后一定要加强沟通消除分歧”之类的建议,这部分决策由读者个人决定,如“识别了分歧以后发现不可沟通于是远离”等,也是可行(且在一定情况下推荐)的操作。此处列出这三方面,仅因为它们是重要的观察与分析角度。同时,有效地使用这些技术,也可以使我们尽可能少地将沟通不畅归因于不可改变的“态度和性格”因素,而是尽可能多地有识别和解决的方法。
\end{explain}

\divider

“识别与进入共识环境”这一方面的内容在\hyperref[sec:解读]{4.4节}中已有提及。此处会从另一个视角出发,讨论一些之前没有覆盖到的地方。

如果我们只关注某一个具体的事务,那么就可以直接评估它的难度。但是在处理覆盖面很广的一大类事务时,我们如果“以某个具体事务为例,以这一事务的难度来估计这一类事务的难度”,就容易把将这一类事务估计得过难或者过于简单,从而不然盲目乐观不然盲目悲观。为了全面客观地认识和讨论这一类事务,我们首先需要问的问题是:“它最简单可以有多简单?最困难又可以有多困难”?

对于“识别与进入共识环境”来说,最简单的情况当然是“什么都不需要做,本来就在共识环境里,一切都很融洽”。这没什么值得讨论的,所以我们略过它,直接考察第二简单的情况:顺畅地识别和进入了共识环境。
\begin{explain}
对于识别和进入,我们都能举出很多足够简单的例子。

我们在很多情况下可以自然地发现“我好像有什么不知道的”,比如说“前面好像有一群人聚在一块,不知道在干什么”、“最近好像有一个梗很火,到处都在用”、“到了一个没去过的地方,不知道卫生间在哪”、“对方和我想的不一样,有自己的意见”等等。虽然我们不一定能在第一时间就发现“这个共识环境的全貌”,但此时识别“有一个自己不在的共识环境”还是较为容易的事。

我们在很多情况下也可以很简单地进入一个共识环境,比如说“观察别人的反应看都在关注什么事情”、“去查一查有没有成系统的介绍/教程”、“直接找别人搭话问发生了什么/卫生间在哪”、“问一问对方是怎么想的”等等。如果仅靠这一次没有难度的接触他人或是接触资源,就可以完全掌握所有所需的信息,那么这就是足够简单的进入方式。

识别容易不代表进入容易,我们可以轻易举出“很容易就能发现自己能力不足,但是想要补起来得花数百个小时好好学好好练才行”之类的例子;进入容易不代表识别容易,我们也可以举出“之前一直觉得没什么,直到某件事以后才知道自己一直想错了”之类的例子。

这里的所说的“简单”基于每个人的主观认识,严格定义相对困难。我们甚至没办法使用“必要篇幅”来定义,因为有“一直都不知道,但是在合适的契机下马上就懂了”这种,从“理解信息”角度来看没什么难度(甚至成功经验可以复制),但从“实现目标”角度来看很有难度(不保证能遇到机会)的例子。区分简单和困难的主要目的是提醒读者保持全局视角,所以此处不多做讨论。
\end{explain}
而最困难的情况是“所有人都没能力做到,超出了人类目前的认知水平极限”。无论是“死语言”一类的共识环境,或是其它世界未解之谜,再或者是人类甚至都没有发现的一无所知的领域,都同样也没什么值得讨论的。所以我们也略过它,来关注第二困难的情况:只有少部分人可以,但大多数人不行。
\begin{explain}
当然,这里说的“少部分人可以”指的不是那种推动了人类进步和学科发展的天才或大师级别的成就,那种事例的性质更类似于“超出人类极限”的情况,也没什么好讨论的。我们这里仅讨论那些在日常交流中,只有少数人拥有的“善解人意”特征。

不可否认的是,有一些善解人意的特征完全是巧合。一个人A在进入某个共识环境(可能由于自身经历、文艺作品或其它途径)后,每当发现另一个人B(可以是自己)具有同样的行为,就会开始代入。这很容易让共情变成自我感动,会阻碍我们发现共识环境,从而使后续的分析与理解变得不可能。我们在\hyperref[sec:情绪、感受与其分析与处理方法]{第三章实操}中已经详细讨论了这种情况,此处不再赘述。“对对方的行为有自己的理解和处理,导致信号被覆盖”和“根本没注意到信号”是识别共识环境的两大主要障碍。

上面这种情况想要修正起来也比较简单:保持谨慎,不要脑补过度即可。一个经常有效的方法,是“询问对方B的动机”(如果是在分析自己,那么就回忆)。如果对方的回答和这种共识环境明显对不上,就可以确定自己想的不对,从而避免一次错误的代入了。

这种做法的难点在于,直接询问不一定能得到真实的结果。B可能无法准确概括自己的思路,或是B受到了另一个共识环境的污染,从而说出来的东西不代表真实情况。不过这不意味着原因就对A不\note{可见},A仍然可以通过别的方法(比如多次接触、旁观、换话题等)来收集信息以做出判断,进而进入对应的共识环境。这在之前也已经详细讨论过,此处不再赘述。这可能需要长时间观察,大量分析和排除,才能获得准确的结论。
\end{explain}
我们可以观察到,在以上“什么简单什么困难”的讨论中,我们不可避免地需要分析“为什么简单,又为什么困难”。尽管上面的分析也举了很多事例,对这些事例的分析也并非遵循统一标准,但这仍然可以带来一些很重要的信息,即“我们为什么会成功,又为什么会失败”。

用某一个事例的判断标准套用到其它事例上,如果能得到符合预期的结果(有的时候论述可能很简单,是“只要看到就能”),那么就说明这个标准比较好用;如果发现套不上去,或是即使能勉强套上去,理论也过于牵强,那么就需要回头反思是哪里出了问题,这个判断标准和对应的思路是否过于草率。

尽可能全面地选择与考察事例,并且把自己的每个思路都在所有事例上都过一遍,就能知道哪些思路比较好用,哪些思路用不成,每个思路分别适用于什么情况了。在面对一个一无所知的复杂系统时,这种分析方法比较高效,又尽力确保了准确。它不保证能得到所有有用的知识,不保证能排除所有的失误,不保证分析适用于所有可能,但至少在处理常见情况(常见情况会在想事例的时候想到)时基本够用。

\divider

遵循着这个思路,我们也对“识别分歧”这一方面提出同一个问题:它最简单可以有多简单?最困难又可以有多困难?

和上面一样,讨论“能有多简单”时,我们也忽略掉什么都不需要做的“没有分歧”的情况。并且,也不重复讨论那些“可以视为不在同一共识环境”的情况,只关心那些在话题中产生的分歧。
\begin{explain}
这样的分歧可以分为两类:一类“和当前话题没有因果关系”的,比如说“走神”等;另一类是是“和当前话题有因果关系”\footnote{此处的“因果关系”指的不一定是“一个是因,一个是果”,还有可能是“二者有共同的原因”这类更复杂的关系(但不可以是“二者共同导致了结果”)。}的,比如说“侧重点不同”等。我们将前者暂时称为行为分歧,将后者暂时称为话题分歧。

二者都可以举出很多发现起来很简单的例子。对于行为分歧来说,“走神”、“没有回应(问题、请求或者其它)”、“同时在干其他的事”、“离开”等行为可以让人清晰地分辨“对方心思不在这里”。对于话题分歧来说,也有“对方主动表示听不懂/听不明白”、“回答或分析错误”、“有明确的其它重点”等。很多这样的分歧只需要识别出来,就可以通过“提个醒”等简便的方式直接解决。
\end{explain}
而讨论“能有多困难”时,也需要忽略掉“无计可施”的情况。并且,也不重复讨论那些“因为信息缺失从而无法解读,无法识别分歧”(比如说因为对方没什么反应/自己不关注对方的反应,所以不知道对方是否走神)的情况。除了这些单独的分歧以外,我们还会面对一种更复杂,更系统的分歧,表现为“从一个话题中,源源不断地产生分歧”。
\begin{explain}
这些不断产生的分歧有很多不同的形式,有可能是“从一个话题开始,不断延伸和发散”,也有可能是“多次涉及这个话题,每次说完之后下次又和没说一样”。这些分歧可能有任意程度的解决,可能是“双方大吵一架,什么都没沟通到位”,可能是“每个问题说了一半就转到下一个问题了(有可能是当前问题的前置、另一个相关话题、或无关内容)”,可能是“每次都能良好地沟通,都能达成共识(但是下一次又忘了)”。

这种情况下,“根据当前情况(主要指已有的分歧)来调整自身的行为模式”的能力需求,超过了参与话题双方(或多方)的自理能力。真正的分歧来自某些更深入\footnote{这里的“深入”指“作为成因”,不包含“这来自潜意识”的意思。}的思考回路,而不是表面暴露出的问题。需要注意的是,这仅在某些特定目标下才是需要处理的问题,而在另一些目标下(比如“只是随便聊聊,聊到哪算哪”等)不需要理会。

发现这种分歧是极为困难的事情,这需要参与话题的双方能够完整复盘所有相关内容(复盘可以仅由一人完成,也可以由双方合作,如“我们来好好聊聊吧”地完成;复盘不一定带有目的性,可能在某次无目的/其它目的的回忆时巧合地注意到)。这会有以下几方面的困难(以下的篇幅中有时使用“我们”和“对方”分别指代主动方和被动方):
\begin{itemize}
\item 根据频率的不同,我们可能需要定位昨天/一周前/两个月前/......的某次发言。一般聊天中很少能清楚地记下所有细节,甚至很少能清楚记下“有这么回事”。
\item 而相对地,“模糊地有一个印象”的可能性要高上不少。我们可能不记得具体过程,而只是留有“总是会吵架”“总是不听”“看起来答应了但是转头就忘”的印象,或是虽然没有自知,但已经在长期的接触中形成了某种行为习惯。这些模糊的判断和对策会干扰其它方面,比如我们可能将这解释成“对方的态度和喜好问题”。
\item 我们会向对方反馈这个印象。我们可能明确知道(并且可能持有包括但不限于“想好好谈谈”、“求求你了”和“你必须给我一个交代”的任意态度),也可能是通过不自知,但外部可观察的行为(比如说回避)。但如果我们的分析不到位,那么仅靠这些反馈想要定位问题,难度和不给这些信息不会有差别,对方仍然需要使用和以前没什么区别的自理能力,去分析和以前没什么区别的情况。并且,如果这些反馈没有明确指定问题(比如说只是笼统地说“你从来就没在意过我说的话”),那么还可能产生新的理解障碍。
\item 这种不包含有效信息的反馈,最好的结果也只可能是“双方浪费了一些时间,一无所获”,大多数情况下总是会伤感情\footnote{这里将“一方单方面付出和改变”也视为伤感情,因为这样忍耐和改变大概率改不到点子上,还是会在很多相似的地方产生同样的问题。我们不应该将偶尔的巧合成功视为常态和目标。}。由于以上的现象实在是太常见了,所以导致另一种现象也会很常见:对方会抗拒这样的反馈。根据具体情况的不同,这些反馈可能会有“翻旧账”、“情绪输出”、“道德绑架”、“投射压迫”等很多种不同的称呼,也有可能对方没有明确认知,只是单纯莫名反感。
\end{itemize}
以上这些因素导致我们在绝大多数时候都会将这种分歧视为某种不可改变的状态,将其称为“性格问题”“两人不合拍”等等。虽然在事实上,相当多的此类问题只源于某个很容易解决的分歧(比如说“只是不知道某个东西有什么用/某个词是什么意思”),但定位问题的复杂程度已经高到几乎不可能靠某个巧合操作就试出来。在没有充足能力的情况下,即使双方都很友善,都愿意复盘,复盘流程也很容易被其它事情(比如相互道歉和相互原谅)打断,从而得不到真实原因,大概率也只能是相互包容相互谅解了事。这无法增进理解,不是最优的结果。

有效的消除分歧的方法,如之前所说,是“观察并解读所涉及的行为模式和竞争过程”。这也是我们需要完整复盘的原因之一——过去的事例会带来一些有效信息。当然,信息不一定足够,有可能还会需要一些询问以补充。这种询问如果操作不当,也容易被对方认为是“借机找茬”,毕竟在对方眼中可能这是完全不同的两件事。在调研的过程中,我们要尽量避免所有可能打断这一过程\footnote{我们此处只关注会影响分析过程,使我们无法得出结论的打断,像是“今天没时间了明天继续”这种中断不计算在内。}的内容,包括但不限于“提前得出结论”、“关注点转移”、“不耐烦”等等。

如果打断过于频繁,那么可能无法分析透彻这个分歧,此时我们应当转而关注会打断分析过程的这些行为。如果反复转换关注对象也不行,那么基本可以下结论“无法定位这个分歧”,此时根据具体环境的不同,我们可能有“绕开相关内容”、“远离这个人”等多种不同的处理思路。
\end{explain}

\divider

同样,我们也对“消除分歧”这一方面提出同一个问题:它最简单可以有多简单?最困难又可以有多困难?

在准确地识别了分歧之后,消除分歧的目标就变成了改变认知。如果需要改变认知的是自己,那么根据本章所介绍的方式处理即可,此处不再赘述。以下篇幅主要关注“需要改变对方的认知”的情况。
\begin{explain}
在一些简单情况下,这种改变可以通过短期接触(比如说直接提意见)、指挥(比如说给对方推荐某个资源)、甚至是对方完全自理(自己什么都没做,对方直接发现并改正)的方式达成。具体的例子和前两方面列举出的大同小异,我们在这里略过,直接开始讨论可能遇到的困难。

此处的省略不代表“所有这类问题都是困难问题”或是“我们应该对每件事都很正式地分析”之类的观点。在想起这些观点的时候,总该同时想到“简单易处理的情况也存在”,并且使用所有的分析方法,对每种情况具体问题具体分析。
\end{explain}
较为困难的情况,是对方不具有相应的理解能力。此时若没有合适的资源,就需要我们主动来铺垫和讲解。讲解的整体思路是:选定一个双方都在,并且可以理解该认知的共识环境,并且从这里开始讲。
\begin{explain}
以上的描述包含“对方虽然不在共识环境中,但可以顺利地进入共识环境”一类的“需要连续维持多个共识环境”情况。常见的例子比如“我们给对方讲一个故事/一段经历”,所需的共识环境仅为“语言相通”“有近似的价值观”等,与当前话题不直接相关。如果每次进入共识环境都很顺利,那么我们也可以将其视为消除分歧的简单情况。

而相对地,失败和困难也很清晰:没有进入共识环境。无论是“没有判断共识环境就直接开始讲解”还是“找不到可以进入的共识环境”,都会导致失败。如果我们严格按照有效沟通的步骤,此时应该转而去识别分歧。但在实际交流中,大多数情况下我们不需要这么麻烦,也可以继续维持沟通。我们有更简便的方法,可以同时做到识别分歧和消除分歧——直接继续。

对于同一个刺激(比如说提问或者指挥),是否处于相应的共识环境,本身就会导致不同的反应,而这就可以用来识别分歧。同时,大部分分歧易于识别,并且可以通过很方便的方法(比如当场问)直接解决。此时按部就班地做事是没有必要的。

但面对一些更复杂,以至于不能当场消除的分歧时,如果我们习惯了以上的操作,就很容易路径依赖。当我们发现“对方好像听不懂”的时候,我们会再尝试解释或者指挥一遍。但理解这些思路或者行动同样需要共识环境。我们属于这一共识环境,所以它们可以让我们获得对应的认知;而对方一直不属于这一共识环境,就一直无法理解。常见的情况比如“给基础不好的学生讲题”,或者是“自己情绪很激动的时候问别人意见”。这种情况下,对方在这一话题上能做的事情只有附和(对方本身可能自知或不自知),比如“装作自己听懂了”或者“同意以提供认同感”,而无法达成消除分歧的目标\footnote{本指南不推荐读者选择“熬过去就好了”、“提供情绪价值”等浮于表面的目标。我们可以做得更好,至少可以试试看。}。

同时,这也会使对方容易习惯这样的相处方式,进而使对方无法进入某些原先可以进入的共识环境。我们经常可以观察到“为了不让对方操心,选择不和对方说”等情况,这是其主要成因之一。对方甚至可以和陌生人谈这些,陌生人也真的有可能冷静理智地分析,但是对某几个特定的人就是做不到,因为相处方式已经固定了下来,而其中不包含交心的内容。
\end{explain}
在发现了这种分歧后,我们应该做的,是重新选择一个内容更少的共识环境,并尝试从此开始。
\begin{explain}
产生分歧的双方在消除分歧时,都可能涉及“放弃之前的认知(如‘对对方的评估’)”、“改变之前的行为”一类的操作。这些操作在一些语境下会被称为(逻辑和思路上的)后退。这和态度与立场上的“妥协与退让”是两种不同的事情。虽然有些操作(比如说“放弃某个要求”)同时符合这两点,但不可将这两点混为一谈,也不可以认为“妥协和退让一定能换来理解和认同”,这是两个独立的方面。

根据对方实际情况的不同,我们后退的幅度也会有所不同。比如说,对于基础不好的高中生,直接从这节课的知识点开始讲是没有用的,我们可能需要后退到初中才行,由此可能产生极高的沟通成本(无论是自己处理还是找其它资源都极高)。

不根据对方实际情况判断,而是直接认为“后退到某个程度一定有效”,是草率的结论。如果后退得过多,那么就会产生一大段无效沟通(我们经常可以在“唠叨”“聊天吹牛”等场景见到这样的后退);如果后退得过少,对方仍然不属于这一共识环境,那么也会变成无效沟通。为了确保有效,我们需要使用一些方法(比如提问和观察)来确定对方的基础,并且由此判断应该后退多少。我们可能在同一个话题内多次后退,以寻求有效的理解和共识。这也要求我们对目标认知和其理解过程有明确的认知。
\end{explain}

\divider

读者可能会注意到,以上的讨论中没有涉及“礼仪”方面的内容,比如“什么时候应该道歉”等。这是因为,所有的礼仪都只在某个共识环境内有效。不同的共识环境对于相同的需求(包括但不限于求助、道谢、致歉等),有可能有不同的习惯表达,从而在一个共识环境内得体的表达,在另一个共识环境内可能反而是失礼的。

在不是很熟的时候,可以使用一些较为广泛的共识环境中的简单表达。此时,无论是有礼还是失礼不应该被过度重视,不应将其视为“需要遵循的全部准则”。要产生合适的默契,总是需要先在充分了解对方的惯用思路和表达方式——这本身是因人而异的。面对同样的事情,有些人会在意,有些人不会在意,有些人只接受特定的表达方式,有些人会非常善解人意地“只要你有心就行”。这些都是在充分了解后才能掌握的分寸,不存在“可以普遍地对所有人都适用”的统一社交方式。