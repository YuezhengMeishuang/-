\begin{savequote}[450pt]
    承认吧,我们什么都不知道。如今才明白孤独并非孤独的全貌。我们所仰仗的那株脆弱的稻草,甚至比不上作家笔下的火柴牢靠。\footnote{\bilibili{av42228720};\netease{1343548780}。}
    \qauthor{DELA\_P\&雨狸《我对孤独一无所知》}
    正常来说,概念这种东西是没办法直接影响周围环境,伤害敌人的,不是说写下“大质量、大引力、高温、高热、聚变”这些词语就可以制造类似的效果。
    \qauthor{爱潜水的乌贼《诡秘之主》第八卷\ 第三十一章\ 概念化}
    你说人一样一样都没区别,在一堆一样一样自成一类。一样的世界,一样的感觉,一样地生搬硬套成了生活的情节。\footnote{\bilibili{av4894016}。}
    \qauthor{小旭PRO\&绛舞乱丸《一样一样》}
\end{savequote}
\chapter{复杂现象与复杂概念的组成方式}
%\titleformat{\section}{\bfseries}{}{0em}{}

可能有一部分读者看完第二章以后信心满满,觉得自己已经掌握了解决心理问题的方法,可以立即让自己挣脱污染,走向光明、希望和解脱。但很可惜,你高兴得太早了。事情哪有那么简单。

如果读者读过其它一些心理健康书籍,就会发现,前两节的内容和其它心理健康书籍所谈论的话题相差不大,本指南介绍的方法其它书里面也有。相比之下,其他书可能还写得更好一些,语言更亲切,有更丰富的例子来辅助理解,而本指南堆了一大堆理论辨析,十分让人头疼。但即使已经有这么多书了,而且有些人也看了不少书,收获却有限。

我们会被某些东西刺痛,我们会被某一句话打动,但是过不了一周,生活就又回到了原有的轨道中。到底欠缺了什么东西?为什么明明什么都懂,却还是无法面对?为什么明明听过很多道理,却依然过不好这一生?知识的积累和内化到底哪里有差距?本章所要做的,就是尝试回答这些问题。

要一句话回答这些问题也可以,那就是“事情没你想得那么简单,现实世界比你的理解更\indicate{复杂}”。但这一句话无法加深你对现实世界的理解,无法指导你解决实际事务。我们还需要更详细的讨论,用一整套分析框架来面对这个问题。
\begin{explain}
如果只是粗略阅读,那么读者可能会觉得本章所讨论的内容像哲学。有这种既视感是正常的,哲学家也会讨论同一类型的话题。但是,本章的思路和哲学无关,涉及这些内容完全出于实用性。

本章内容不需要哲学基础。无论读者的哲学基础如何,无论读者是觉得本章内容过难还是过易,都请读者尽量避免过多的哲学联想。如果确实有明显的阅读障碍(无论是看不懂还是无聊),可以先阅读本章总结,或者是本章后的整体梳理,再根据需要来回头阅读必要的部分。
\end{explain}

\section{概念与特点\label{sec:概念与特点}}
\begin{explain}
再次提醒:本节内容不是哲学,不要以“玄而又玄”的视角来看待。本节内容所涉及范围不超过中学语文课。
\end{explain}
\subsection{指代}
\begin{explain}
本小节关于笔的话题来源于知名meme《什么是笔》。
\end{explain}
“笔”在目前通行的语言环境中,主要指\indicate{用于书写绘画的工具}。
\begin{explain}
“笔”的原意是“书写”,但这一义项在后续演化中已基本消失。当今“笔”的“书写”义项多为引申义。本小节不做深入的语言学研究,仅略微提及。

别的语言环境下可能有所不同,如英语中没有只作为总称使用的词汇(pen还有额外的“钢笔”义项)。本小节仅以汉语为例。
\end{explain}
如果我们仅使用定义来看待“笔”,那么很多问题,如“这支笔你是在哪买的”,是无法回答的。我们在日常生活中,从来不会真觉得这个问题无法回答,是因为此时有一次\indicate{指代}。和“购买”有关的信息来源于“这支”:它锁定了交谈者附近(可能是身边,可能是手指的地方,可能其它)唯一具有“笔”的特征的东西。
\begin{explain}
\indicate{指代}必须结合环境。脱离环境的“这支笔你是在哪买的”问句无法回答。在没有语境的情况下,听到有人突然问了这么一句,多半会反问一句“什么?”。

而基于概念本身的提问,如“笔是干什么的”,则无需结合环境即可回答。
\end{explain}
我们将\indicate{通过部分特点锁定环境中的具体对象,以谈论它的其它特点}的行为称为\indicate{指代}。\\指代是人类在正常交流和思考时,不可缺失的一项重要能力。
\begin{explain}
我们永远都无法完全描述一件物体。一支笔处于什么位置,是什么材质、形状、颜色,为什么能书写,属于谁,现在还能不能用......只要想关心,总会有无穷无尽的信息可以获取。

但我们永远也不需要完全描述一件物体。通常只需要很模糊的信息(如“在你附近”“蓝色的”等),我们就可以完全锁定具体在指哪只笔。

仅靠部分信息,就完全锁定交谈对象,会不可避免地使用到指代。
\end{explain}
在一些情况下,\indicate{指代}可以不用完全符合定义给出的特点,词语于是有了引申义。
\begin{examples}
在使用“笔”的语境中,有些只在意笔的“记录信息”功能,如软件中的画笔工具;有些词组引申出了“书写”行为,如“执笔”;有些不在意功能,只关注笔的外形或其它方面,如“触屏笔”、“录音笔”、“坏掉的笔”等。
\end{examples}
指代经常导致词语含义的演变。
\begin{explain}
使用“书写”来指代“书写工具”,指代多了,就会使笔的含义也变为“书写工具”;在某个较为独立的环境,如具体学科、家庭、学校、公司中,也经常会有各自的“黑话”。

本指南不过多探讨词语在社会中的整体语义流变,仅关注在具体小环境下的变化。
\end{explain}

\subsection{概念}
我们所使用的每一个\indicate{概念}都具有如下特征:概念中提供了一组\indicate{特点},我们通过判断某个具体对象是否符合这些\indicate{特点},来判断是否可以用这个概念来指代它。我们将符合定义的具体对象称为这个概念的一个\indicate{实例},将用于判断的\indicate{一组特点}称为这个概念的\indicate{定义}、\indicate{含义}或\indicate{判据},将判断过程称为\source{识别}、\indicate{锁定}或\source{特征提取},将指代也称为\indicate{套用}、\indicate{概括}或\indicate{称呼}。

如果一个词语有多种定义,那么将每种定义称为它的一个\indicate{义项},此时称这个词语有\source{歧义}。
\begin{explain}
此定义的\indicate{概念}的定义较为宽松,只要是一个实词,都可以算做概念。甚至名称(针对某一特定个体的概念)也算。

对于有一定语言学或哲学基础的读者,此处的“概念”有能指和所指的混用,笼统地描述整个意指作用;“字”“词语”是能指;“特点”“含义”“定义”等表述是所指。本指南不深入讨论这部分内容,做了相应简化。

使用统一的概念可以大幅度缩减思考和沟通的成本,可以将原本使用数句/数段的描述简化为少数几个字。使用概念是人类在正常交流和思考时,不可缺失的一项重要能力。

本指南使用“特点”这一概念或其实例时,默认其没有歧义。行文时会尽量避免使用事实上有歧义的特点实例。

歧义总体来说是贬义的,但也有“同一个词同时表示多个不同义项”之类的用法,这在文学作品或其它领域是正常的运用方式。我们在此仅关心“因为表达不充分或理解不充分而产生歧义,导致交流出现障碍”的现象。这毫无疑问是负面的。
\end{explain}
%\begin{examples}
%大多数情况下,我们不需要很精确地知道自己的判断依据。我们在确定“面前的人是谁”的时候,绝对不会在脑子里清晰地把自己的理由都过一遍。很多判断(如通过外貌识别出眼前的人是谁)是完全不可控的行为。
%\end{examples}
在特定环境下,对于同一个概念,可能有好几种不同的判据都会识别出同样的对象。此时我们经常通过最方便的那一种来识别。
\begin{examples}
严格来说,名字用于指代一个具体的人,而不是某种特定的外貌和行为。但没有谁是通过“每时每刻观察一个人,确定这个人一直稳定存在”才能确定这个人的名字。靠外貌和行为通常就能锁定一个人。

靠外貌和行为的判断方式在一些情况下会失效,如双胞胎,或者恰巧外貌和行为相似;很久不见一个人后,外貌和行为都有可能发生大变化。

以上的误判可以通过“一个人亲口说出自己的名字”等方式来解决。这种判据也会在一些情况下(如神志不清时)失效,此时可以使用一些其它方法(如检查身份证件、查验DNA)确认。

我们极少通过“一个稳定存在的人”的原始定义来识别这个人。这多数情况下既无实践可能也无使用价值,只在少数罕见情况下(如有替身,需要持续监视)有必要使用。
\end{examples}
我们不一定明确知道自己使用的概念是什么定义。这有两种情况:一种是有确定的判据,但自己不清楚;一种是多种判据混用。我们将后者视为\indicate{指代},而前者不视为指代。
\begin{explain}
如果不清楚定义,我们在使用同一个词语时,就有可能在根据它的不同定义分别谈论。这些谈论有可能偷换概念,错误地关联起一些本来无关的事情,得到的结论也有可能无效。

可以通过增加具体描述来避免义项混用,将讨论的概念限制在“xxx的外貌”“xxx的(某个具体)行为”等具体特征上。
\end{explain}

\subsection{能力边界\label{sec:能力边界}}
定义只判断事物是否符合对应的特点,而不涉及这些特点的来源。使用概念来概括实例,就仅能分析它的一部分特点,遗漏了。这会带来不可避免的\indicate{信息损失}。\\
\indicate{存在信息损失时,分析是单向的,不可能由果推因}。
\begin{examples}
这里的“果”指“概括得出的特点”,“因”指“对应的实例的其它特点”。熟悉逻辑学的读者可以看出,这里指的是“若A是B的充分条件,那么可以由A推B,但不能由B推A”。

笔的特点仅有“能书写”。不同的笔书写原理不同,铅笔、蜡笔等使用摩擦留下痕迹,其它笔多使用墨水或颜料,而具体的方式也有不同。这不影响它们统一被称为“笔”。在通用含义下,“笔为什么能书写”这一问题等价于“能书写的工具为什么能书写”,我们不可能脱离语境,给出统一的答案。
\end{examples}
每个思考回路都会使用很多概念。我们将\indicate{某次思考时使用过的所有概念的所有特点}统称为此次思考的\source{出发点}或\indicate{依据},并且将出发点称为此次思考和所得到结论的\indicate{适用范围}或\source{能力边界}。
\begin{explain}
我们不一定需要分别定义每一个概念。实际上,我们很多时候只关心一些概念之间的联系。这些联系也是出发点的一部分。
\end{explain}
我们将“在某一范围内,所有有关的所有现实现象”依照具体语境的不同,称为一种\source{环境}、\indicate{情况}或\indicate{现实情况}。
\begin{explain}
具体的范围需要在使用这一概念前具体指定。在本指南中,若无明确说明,默认的范围是“会使讨论对象(如行为或结果)出现差异的相关因素”。
\end{explain}
如果我们在思考时,仅根据出发点来使用概念,而不使用指代,那就将这次思考称为\indicate{完全理论性的}、\indicate{和环境/外部无关的}或是\indicate{就事论事的}思考。相反,如果使用了额外的信息(也即使用了指代),那就将这次思考称为\indicate{和环境/外部有关的}或是\indicate{依赖直观}的思考。
\begin{examples}
如果问题是“钢笔为什么能书写”,就可以结合物理知识,使用“因为墨水会随着毛细现象渗进纸张”来回答。钢笔包含“自身结构特征”的信息,这是完全理论性的分析。

如果问题是“这支笔为什么能书写(指向一支钢笔)”,那么同样的回答对“笔”这一概念来说,就是和环境有关的的分析。如果听到这句话的人(比如小孩子)不清楚钢笔的相关知识,就有可能误用这一结论来解释其它笔的书写原理。而另一方面,这个回答对“这支笔”这一概念来说,是完全理论性的分析。

在极少数情况下,一个思考回路可以不包含识别方法,巧合地每次都就事论事地分析。除此之外的绝大多数情况下,总是在能力边界内分析的思考回路都包含对应的识别方法,都是\indicate{知识体系}。\footnote{\rigorous 可以验证此处的定义和\hyperref[para:知识体系]{2.1.1小节}中的定义一致。判断一般的行为模式是否是知识体系,仅需验证其子思考回路是否包含对应的识别方法。}
\end{examples}
如果我们的认知来自某个/某类具体事件/现象,那么就将其称为该认知的\source{参考事件/现象}。如果我们通过分析参考事件/现象得到认知,那么也将其称为该分析的参考事件/现象。
\begin{explain}
这里所区分的是“该认知是出发点还是推理的结果”。如果该认知的形式是“有信号就会有相应影响”,那么“对该信号的预料”就有对应的参考事件;如果该认知的形式是“有信号说明这是某一类事件,而这类事件会有影响”,那么它就没有参考事件,而“这类事件的判断标准”和“这类事件的影响”需要分别再判断是否有参考事件。一个认知可以有多个不同的来源,此时我们应该分行为模式看待。

参考事件可以是自身经历,也可以是文艺作品、听人讲道理等其它类型的事件。参考事件不一定是自知的,识别信号和后续行为有可能完全是下意识的反应。这一过程会产生不全面和不真实的认知,此时这种认知是污染,参考事件是污染源。
\end{explain}
如果某次思考时,明确知晓自身的出发点,并且只做完全理论性的分析,那么就将这次思考称为\indicate{在能力边界内}思考,或是\indicate{清醒地}思考。
\begin{explain}
出发点中的一些认知可能有自己的分析过程。如果此次思考并未涉及到那些分析过程,就可以不向前追溯;出发点中的一些认知有可能是污染,但只要此次完全理论性分析时知道自己使用了该认知,那么就仍然认为是在能力边界内;不同的思考回路涉及同一个词的时候可能使用了不同的定义,如果每次思考都清楚这次使用的定义,并且清楚之前的结论用了什么定义,那么也应认为这是在能力边界内。

和环境有关的结论在脱离环境(且无法模拟环境)后就会变为污染。这一过程也会发生在同一个人的不同思考回路之中。前后两段思考的出发点可能不同,前一段思考所使用的概念可能无法迁移至后一段思考中,前一段思考所得到的结论可能无法与后一段思考出发点的其它部分相容。
\end{explain}
如果一个思考回路能完全理论性地思考某个概念或进行某段分析,那么就称这个思考回路可以\indicate{容纳}这个概念/这段分析。
\begin{explain}
注意这里的\indicate{容纳}不是“包容、忍耐、忍受”之类的含义。它和具体的价值判断与感情倾向独立。

思考回路无法容纳的认知,对于这一思考回路来说,就是污染。类似于模拟,我们可能也只有在特定环境之内才能容纳某个认知。

如果出现“不使用某个概念,你就无法把思路顺下去”的情况,那么基本就可以确定你实际上没有掌握这个概念,自身不足以容纳对应的分析。能容纳分析时,我们能将概念替换成相应的描述,仅使用描述中的特点也可以完成分析。概念应仅起到简化的作用。
\end{explain}
我们将\indicate{因为忽视了前提,导致分析和结论无效的现象}称为\indicate{劣化}或是\indicate{想当然}。一段思考中使用了越多的劣化想法,就称这段思考越\indicate{劣质}。
\begin{explain}
越是不可避免地需要借助环境来思考,对自身思路的整体把控能力就越差,就越是无法容纳完整的思路,就越是需要借助环境中新的信息来思考。随着这一恶性循环,劣化的想法就越来越多。如果一个思考回路包含了足够多的劣化想法,随着不自知的偷换概念,有效的思维链条会很短,就会产生内部的矛盾。但同时,因为该思考回路对自身整体把控很差,它无法将自身整体纳入考量,只会判断“每条想法是否有问题”,然后得到“没问题”的结论。
\end{explain}

\subsection{应用}
\label{def:知识体系}一个知识体系会明确自身的出发点,使其不因具体应用而改变。我们将其称为知识体系的\indicate{基本假设}。
\begin{examples}
这些基本假设包括但不限于数学中的公理,物理中的定律、理想模型,以及一些其它学科中“对现实现象的归纳总结”等前提条件。

在19世纪,数学发生过一次重大的认识论转变,公理不再被看作“不证自明的命题”,而是被看做“某个研究领域的基本假设”。我们通过检验“数学对象是否符合公理”来判断“是否可以使用该领域的结论”。公理和对应的领域不负责阐述“符合公理的数学对象从何而来”,只从基本假设出发继续推导性质。一个明显的例子是,数学家们找到了欧几里得的很多依赖直观的漏洞,并且使用希尔伯特公理体系将其严格化(这不是双曲几何的故事)。

20世纪时,这种观点转变影响到了物理学,尤其是难以直接观测的量子力学。物理学家们不再使用“定律”一词,而是将薛定谔方程等内容称为“基本假设”。理论增添了“适用范围”的判断标准(如经典力学仅适用于宏观低速情况,中学物理还会要求“可以看做质点”等),在应用前需要判断对象是否符合要求。但这并不影响我们做完全理论性的推导。

\trained 可能可以看出,这里在讨论\indicate{范式}的一些特点。本指南不深入认识论和科学哲学内容,对此仅做简单介绍,以供读者参考。
\end{examples}
对某一具体领域展开研究的学科,可能因为历史发展,内部包含多个不同的知识体系。更换基本假设有三种可能的情况:发现某基本假设多余,可以由其它基本假设得到;发现了等价的基本假设,该假设在一些情况下更方便(或有其它优势);找到了更深入的基本假设,能推出原有基本假设,并且能解释一些新现象。
\begin{examples}
第三种情况涉及到研究领域的改变,是重大的研究突破。不可能只靠完全理论性的分析,就从原知识体系出发,得到更深入的基本假设。原有基本假设不包含它的信息。只有获得了新信息(或者是调研,或者是直接猜),才能得到新的基本假设。
\end{examples}通过明确基本假设,一个知识体系得以只研究某种特定层次的现象,而不用无穷无尽地向前还原,不用解释每一种现象的原因。
\begin{examples}
基本假设可能来自对现实的直接观察,也可能来源于对某些现象的深入分析和提炼,也可能就是凭空创造。无论哪种情况,都可以做纯粹理论性的分析。只要对象符合基本假设,那么就可以使用分析得出的结论。

%读者可能会注意到,本指南中对概念下的定义都很简短。对概念的其它讨论主要有以下三方面用意:向读者演示如何识别概念对应的实例;明确表述,消除可能存在的歧义;从概念出发做完全理论性的分析。如果读者能力足够,可以仅凭定义,就独立完成这些讨论。
\end{examples}
一个知识体系通常体量很大,内部包含的分析相当繁杂,通常难以穿过冗长的逻辑链条,回归基本假设。更常见的情况是\indicate{以知识体系的某些结论作为出发点,展开分析}。我们将这种思考称为\source{应用}这一知识体系。
\begin{examples}
一些学科的基本假设来自领域中已有的深入研究,对初学者可能过于抽象复杂,无法准确识别和正确使用。此时可以通过一些覆盖面更小,但同时更容易上手的基本假设,来辅助初学者接受这些概念。
\end{examples}

\section{复杂系统\label{sec:复杂系统}}
\subsection{编织}
我们将\indicate{由一个事件产生的现象}称为这个事件的\indicate{影响}或“这个事件\indicate{影响}了现象”,将\indicate{多个事件相互影响,组成更大的事件}的现象称为\source{编织}或\indicate{编织过程}。我们将编织过程中的每个事件称为\indicate{具体事件}、\indicate{微观现象}或\indicate{简单现象},将组成的更大事件称为\indicate{宏观现象}或\indicate{复杂现象}。我们将每个微观现象称为宏观现象的一个\indicate{组成部分}。我们允许“微观现象编织成宏观现象”的表述。

我们也将“微观过程组成宏观过程”\label{def:复杂过程}称为编织(注意过程是指一类具有共性的事件);也将“微观认知组成宏观认知”称为编织(注意“使用认知”可以视为事件)。
\begin{examples}
这里所定义的编织是一个非常广泛的定义。相互影响可以是各种形式的接连触发,可以并行、串行,或更复杂。可以是时间上的前后顺序,也可以是逻辑上的前后顺序。

前文中也涉及到很多可以被视为编织的概念,如行为编织成行为链,进而编织成行为模式,行为模式编织成人,以及事务编织成更复杂的事务、事件编织成更复杂的事件、行为模式编织成更复杂的行为模式。

其它学科中也有对应于编织的概念或例子,如专门的复杂性理论、系统论、统计力学、神经网络、格式塔、回声室、由个人编织成组织、由组织编织成更复杂的组织等。本指南不对这些内容展开深入讨论,仅作提及。
\end{examples}
我们将\indicate{只能从宏观现象中概括的特点}称为\indicate{宏观特点}、\source{表面特点}或者\indicate{表面现象}。
\begin{examples}
表面特点的定义也很宽松。它有可能是“某种稳定的状态”,比如说人的性格;也有可能是“会导致某种结果”,比如从某种思考回路中产生了某种情绪、使用了某个概念。

“\indicate{只能}从宏观现象中概括”的意思是“无法将其归因于复杂现象的某个组成部分”。会体现同一表面特点的不同复杂现象,可能包含完全不同的具体事件。我们只能从所有(主要的)微观现象出发,通过分析它们之间的相互影响,来推理出表面特点。不可能反过来,仅从表面特点出发,反向推导出复杂现象的具体组成。
\end{examples}
当我们称某个事物为表面特点时,若无特别说明,提到的复杂现象总是指产生该表面特点的复杂现象。
\begin{explain}
在同一个复杂现象中,某个特定的现象不可能既是表面特点又是微观现象。但复杂现象A的表面特点可以产生影响,作为微观现象参与复杂现象B的构成。

事实上,我们身边会遇到的绝大多数事件,要不然可以直接视为复杂现象,要不然可以通过少数几步归因,归到某个表面特点上。纯粹由简单现象组成的事件极为少见,通常只能在主动设计的介绍、教程等环节中见到。
\end{explain}

\subsection{涌现}
如果在某一类事件中,存在一些事件能够编织出宏观现象,那么就称这一宏观现象从这一类事件中\indicate{涌现}出来,称这一类事件为\indicate{低层次事件}或\indicate{深层事件},称涌现出的事件为\indicate{高层次事件}或\indicate{浅层事件}。

我们同时也称该宏观现象对应的宏观特点从这一类事件中\indicate{涌现}出来。我们将一类低层次事件和其中能涌现出的所有高层次事件和其宏观特点统称为一个\source{复杂系统}。
\begin{explain}
能从一类事件中涌现出的宏观现象数量不定。有可能一个都没有,也有可能涌现出多个。这一类事件的特性和相互之间的触发关系决定了它们能涌现出什么。这也导致某一个事件可能有多种不同的影响,想要找全“一个事件的所有可能影响”通常是很困难的事。

此处关于复杂系统的定义仅涉及两个层次的事件。也可仿照该定义,定义包含更多层次的复杂系统。本指南仅关注涉及两个层次的复杂系统,故不下更广泛的定义。

对于一个宏观现象的所有组成部分来说,它们所涌现出的内容可能不止这一个宏观现象。其中的一部分事件可能可以组成别的宏观现象。

每一个行为模式都是从行为和环境中涌现出来的现象。行为也可视为“从当前所有信号中选择了一个响应”这一涌现过程的结果。
\end{explain}
与编织不同的是,涌现并不要求每一个事件均起到作用。一类事件中可能有大量和宏观现象无关的其它微观现象。
\begin{explain}
对于一个宏观特点来说,不一定其对应宏观现象的所有组成部分都起到了作用。每个宏观特点可能仅和一些组成部分有关,不同的宏观特点所依赖的组成部分可能不相同。

这使得我们在面对涌现现象时,无法从“出现了某一类事件”就断定“一定会出现相应的宏观特点”。会导致宏观现象的事件可能具有另外的特点,与这一类事件并不重合;另一方面,另外的特点可能无法被良好地总结归纳,只能具体问题具体分析。
\end{explain}
如果一个概念的定义包含表面特点,那么就称其为\indicate{大词}、\indicate{复杂概念}或\indicate{宏观概念}。
\begin{explain}
思考回路由想法编织而成。大多数思考属于复杂现象,而使用概念和获得结论都是思考的表面现象。\indicate{人所使用的绝大多数概念都是复杂概念}。

我们日常使用概念时,大多数情况下不会很严谨地完全明确自身的出发点,而是会根据某些特点来套用。实际的使用方式,会从“所有和该概念相关的认知”中涌现出来。根据当前所处思考回路的不同,我们根据的特点可能很不稳定,会给很多不同的东西冠以同一个名字,用同一种结论套在不同的东西上。具体讨论见\hyperref[sec:人的意识演化基本模型]{3.4节}。

因此,任何一个复杂概念,都会不可避免地具有很重的\indicate{歧义}。如果不加以分辨就使用这些概念分析,得到的结果就会完全不可信。这些歧义有可能会让人感觉这些概念很\indicate{深奥},但“有歧义”和“确实很复杂”都会使人迷惑,都需要深入思考,一定要区分这两种情况,而不能将其笼统地看做“很有哲理,揭示了某种本质”。
\end{explain}

\subsection{对复杂系统的分析\label{sec:对复杂系统的分析}}
我们将\indicate{分析某个复杂现象时,仅将微观现象视为原因}的行为称为\indicate{微观分析}或\indicate{深入}、\source{客观}、\indicate{符合现实/实际}的分析。
\begin{explain}
微观分析的参考事件可能是微观现象之间的因果关系,或是编织过程整体。后一种可以用于分析“表面特点如何产生”。

分析微观现象时,有可能使用这些微观现象的某些特点来指代微观现象本身。我们也将其视为有效的分析。

分析编织过程的难度一般较高,需要对复杂现象有整体把握才能得出有效结论。这在很多时候需要掌握相应的知识体系才能做到。

“深入”一词可以指代很多不同类别的东西,如“全面”“准确”“有预测性”等。我们这里采用的定义大体上符合这些表面特点,但不可因此而使用这些表面特点来理解“深入的分析”,必须从定义出发。
\end{explain}
我们将\indicate{在分析某个复杂现象时,将它的某个表面特点视为原因}的行为称为\indicate{宏观分析}、\source{表面分析}、或\indicate{草率}、\indicate{武断}、\indicate{脱离现实/实际}的分析。
\begin{explain}
两种微观分析都会因为忽略了结论的适用条件,而劣化为宏观分析:

将考察微观现象时的结论误用于其它现象,通常是因为我们使用特点来指代微观现象,同时将微观现象的性质误认为特点的性质。这种情况可能在总结经验时,或者是举例、类比时出现。

将考察编织过程时的结论误用于其它现象,通常是因为我们过度简化了分析的流程,将表面特点简单归因于某个具体现象或者另一个表面特点,而不归因于整体的复杂现象,忽略客观存在的编织过程。这里有一种常见的很有误导性的情况:某个表面特点确实总是会导致另一个表面特点,但不同的实例中有不同的中间过程。此时我们很容易将其中的某个中间过程视为唯一可能的原因。

如果不能从微观现象中分析出表面特点,就不应认为自己理解了该复杂现象。不应认为宏观分析是对复杂现象的有效理解。如果我们脱离了宏观概念后就无法再对复杂现象展开讨论,那么先前使用宏观概念做出的所有分析和判断,就全都是\indicate{正确的废话}。
\end{explain}
对同一个复杂现象中的两个特点,在没有展开微观分析前,我们仅应认为这两个特点具有\indicate{相关性},而不应该认为这两个特点具有\indicate{因果性}。
\begin{explain}
这类因果性结论包括但不限于“xxx会导致xxx”、“xxx都是xxx”等绝对性的判断。

在研究简单现象时,如果我们确实控制好了其它所有变量,那么当有当两个信号总是先后出现时,就确实可以得出“二者具有因果性”的结论。

但我们在现实生活中遇到信号时,无从判断这个信号是出自简单现象还是复杂现象。如果两个先后出现的信号都出自复杂现象,都只是表面特点,那么它们只具有相关性,而不具有因果性。现实生活中的绝大部分现象都是复杂现象,如果没有确切的“当前信号出自简单现象”的证据,应该始终将其按复杂现象处理。\indicate{没有调查就没有发言权}。

如果我们总结出“出现前信号之后会出现后信号”的相关性结论,那就可以加以利用:比如在见到前信号以后,则可以提前准备,以利用后信号,或规避后信号的不利影响。但这仅限于被动地接受并处理现实,我们无法主动改变这一切。

如果我们总结出“出现前信号会导致出现后信号”的因果性结论,并且试图通过控制前信号来控制后信号,那就会出问题。由于前后信号没有因果关系,现实不会按照预料来进展。这种错误的归因已经是污染,而为失败另找理由还会产生另外的污染。
\end{explain}

\subsection{预料与副作用\label{sec:预料与副作用}}
我们将\indicate{认知中一个事件的影响}称为一个人在该事件上的\source{预料}。
\begin{explain}
“预料”指的是“将要出现的现象”,而不是“希望出现的现象”。“影响”可以仅关心“所重视的方面”,而不关心“是否观察到了事情的全貌”;“预料”仅关心“有这种认知”,而不关心“预料是否符合实际情况”。

以不同视角看来,同一事件有多个不同方面的结果,产生不同的影响。对同一事件的结果有多个判断是常见情况。比如“这事好麻烦,但很快就会结束,忍忍就过去了,但真的好烦”就包含两种不同的判断,它们基于不同的出发点,因此会让人纠结和烦躁。

有时会出现“虽然每一思考回路下只能得到一个判断,但是在多种场景中能得出不同的判断”的情况。每个思考回路都会认为自己是清醒的,直到因某些情况而将它们放在一起比较。
\end{explain}
我们对预料的判断会影响对事件本身的判断,进而成为对事件本身的判断,从而参与我们在该事件相关行为上的分析和决策。
\begin{explain}
考虑可能出现的影响会使我们的分析更加客观和实际。预料仍然可能变成污染,是因为这一过程往往是自知但不可控的,有两个可能出现问题的方面:一方面是判断本身就不真实/不全面,迁移到事件上后问题仍然保留;另一方面是有可能逐渐忘记了自己的预料,看到事件就直接条件反射般地产生判断,判断也因此成为污染。

这两种情况都会导致无法沟通的现象,具体讨论见第四章。简单来说,我们会因为观察到“别人无法产生相同的判断”而感觉“别人不清醒”,进而将其错误地总结为“别人无法沟通”。
\end{explain}
对于由行动而产生的事件,我们将\indicate{影响作为预料参与(某思考回路的)决策,从而成为该行动的动机}的过程称为“(该思考回路)\indicate{在意}该影响”,反之则称为\indicate{不在意}该影响。\indicate{事件的实际影响中,不在意的部分}称为“和动机无关的作用/影响”或“该行动的\source{副作用}”。
\begin{explain}
副作用的定义看起来有点绕。我们来具体看一下不同情况下的副作用:
\begin{itemize}
\item 对于不可控的行为,由于没有决策环节,行动直接被某个信号触发,此时该事件的所有影响都是副作用。
\item 预料可能不真实/不全面,此时错误和没有考虑到的方面就是副作用。如果这种预料被用于决策,那么这种操作可能无法实现目标。
\item 在针对某一目标做计划时,可能能预料到某步操作有负面影响,但因为还有其它更主要的正面影响,故仍然选择这步操作。此时负面影响是自知的副作用。
\end{itemize}
副作用是依据行动而判断的。对于“行动A有负面影响,从而采取行动B以消除/规避该负面影响”的情况,则只有触发行动B的思考回路在意这一负面影响,负面影响只算作行动A的副作用。对于\rigorous,可将“或是有负面影响,或是需要采取行动B”整体作为A的副作用看待(毕竟这两项中仅有一项是实际影响)。

一个行动可能产生不计其数的影响,从而也有不计其数的副作用。我们实际上无法完全认识到一个行动的所有副作用,不过好在我们也不需要这么做,只需要在特定的方面关注特定的副作用即可。
\end{explain}

\section{价值观\label{sec:价值观}}
\subsection{对人的表面分析}
行为模式是由行为编织而成的,行为模式的一次触发在绝大多数情况下都是复杂现象。因此,\indicate{人的绝大多数行为,及其所产生的影响,是表面现象}。
\begin{examples}
如前所说,行为本身可以作为微观现象,参与其它复杂现象(如另一行为模式)的构成。这不影响“行为本身可以是其它(更偏向无意识的)行为模式的表面现象”的结论。
\end{examples}
如果我们观察到了一个人的举动(可能是行为/行为模式),同时观察到了与该举动的某个影响,就有可能将影响视为这个人的目的。将\indicate{分析出的原因}称为该举动在\indicate{此次分析得到的动机}。这种分析在绝大多数情况下是表面分析。我们将此\indicate{通过表面分析得到动机}称为对该举动的\indicate{表面归因}。
\begin{explain}
分析得到的动机不一定是真实的\note{动机},有可能是该行为的副作用。这种分析方式的出发点是“人会依照自己的目的行事”。但这种草率的归因是完全不可信的,有以下几方面问题,可以与副作用的情况一一对照(以下将做出举动的人简称为对方):
\begin{itemize}
\item 一件事的影响有很多方面,每一方面都有可能是目的。对方不一定将其中的某个影响当做了自己的目的,有可能只是副作用。
\item 对方可能有自己的目的,但自身的分析有问题,从而导致举动不能实现目的。此时的目的不是任何一种影响,不可能通过这种方法分析出来。
\item 对方可能没有目的,而只是感受到了某些信号,并且条件反射式地做出了举动(同样地,在得到更多的信息之前,也不应草率地认为就是条件反射)。
\item 对于行为模式层次的举动,有可能对方在具体的行为上有明确的目的(或是感受到了信号),但是没有整体的行为模式层次的考虑。行为模式客观上从行为中涌现出来,但对方对此没有自觉。
\end{itemize}
以上的所有内容,在分析“自己的另一种行为模式/思考回路”时也适用。同样不要草率地概括自己的动机,不要草率地觉得自己的行为出自某种特定的潜意识,这会掩盖真实的原因。
\end{explain}
\indicate{通过表面归因得到的动机,对改变现状毫无帮助。}
\begin{examples}
这里的现状指代任何一种行为模式。导致举动的另有其它因素。无论是想要培养,还是想要去除,通过表面归因都起不到什么帮助。\trained 可以将其称为“形式主义”“主观主义”。本指南不引入这些概念。

比如,通过参考一些例子,我们可能会得到“吃苦才能成功”的结论。“吃苦”作为一个复杂概念,具有很重的歧义,如“做自身抵触/缺乏意愿的事”和“投入资源做事”。真正起作用的特征是“做会有成效的事”,二者都离其有一定距离,成功的人实际上也不一定真吃了苦。正确的思路是“对于一个确定的目标,和会有成效的事,即使需要强逼自己和投入资源,也需要去做”。但如果只看到了“需要强逼”,那么这不仅无助于成功,甚至无法起到磨炼意志力的作用。
\end{examples}

\subsection{大道理}
我们将\indicate{通过表面归因得到,并且可以指引行动的结论}称为\source{大道理}。
\begin{examples}
每一条大道理都是一条\note{认知},在对应场景下会变为动机。表面归因不提供对复杂现象的理解,所以所有的大道理均可视为污染。大道理仅能指引一个人按简单逻辑做事,无法起到辅助思考的作用。

污染不一定起到负面作用,按简单逻辑行事有时便足以维持某种状态。如“爱护环境”“勤俭节约”“遵守交通规则”等大道理,在大多数情况下都是良好品质,有利于社会的和谐与稳定。

但同时我们也能举出一些过犹不及的例子来,如“因为不舍得剩饭剩菜而食物中毒”“因为要等红灯所以堵住了救护车的通行”等情况。如前所述,这是因为忽略了这些大道理的适用范围,无条件地依据这些大道理而行动。

站在客观的立场上,我们能够轻易指出这些极端情况的谬误之处。但是如果以遵守大道理为优先,那么这就是大道理指引的行动,这种选择就是正确的。
\end{examples}
如果在应用知识体系或其它结论时,忽略、遗失或过度简化了结论的适用范围,那么结论就会劣化为大道理。
\begin{examples}
很多情况都会导致劣化,如“因使用过于熟练而不再检验前提条件”、“与他人沟通时因为篇幅不足而只能使用结论”、“无法熟练应用知识体系,未意识到其适用条件,仅觉得某一句话很有道理”等。具体机制可以参考2.2.3小节的讨论,此处不再赘述。

结论可以起到记忆锚点的作用,可以提醒一个人“此时该用这种知识体系来分析”。一个深入的结论能够有效地指引人思考和行动。这种作用不来自这句话本身,而来自它背后的整体分析框架。当我们面对某个现象,想到某个结论时,如果能从此出发,遵循已有的分析框架,从而验证结论确实正确,那我们就\indicate{理解}了这个现象。如果想到结论时总能具体地思考,那我们就\indicate{掌握}了这个结论。

而如果不能调用相应分析框架,那么就无法达成上述效果,此时该结论就只是大道理。无论看上去多么理所应当,多么符合自己的实际体验,\indicate{任何一句单独的话都是大道理,都不可信}(这句话也仅起到提醒的作用,不要以此为信条去否定一切规训,这超出了它的适用范围;本指南中着重强调的其它句子也仅起提醒作用)。就算是1+1=2一类的话也是如此。当它脱离数学语境,被拿去做“1+1>2”之类的比喻时,就不再显然正确了。我们必须重新完整地判断相关结论是否可信。
\end{examples}

\subsection{价值观}
我们将\indicate{比较一个事件“发生了”和“没发生”孰优孰劣}的行为称为\source{价值判断},或简称为\indicate{价值}或\indicate{评价}。
\begin{explain}
显然价值判断是判断。同时,每个价值判断都可视为一个认知。以下的行文中将不区分这二者,读者应能自行分辨。

一个价值判断可能仅关注属于某类特定过程的事件,而不关心其它的过程。一般地,一个价值判断关注的事件越多,涉及到的概念就越宏观,其判断也就越脱离现实。

价值判断是很常见的行为。有很多常用词汇包含价值判断的因素,如“好/坏”、“善/恶”、“对/错”、“应该/不应该”、“期待/抗拒发生”、“正常/不正常”、“恰好/多余”、“有无作用/意义/价值”、“是否本质”等。这些词汇都是复杂概念,以下使用“好/坏”来代指这些词汇。如果不定义“什么是好”,那么价值判断就是大道理。

有的价值判断是基于事件的某些特征,有的价值判断是基于事件的影响,出发点千奇百怪。按照是否自知出发点,价值判断可以分为两类,其中不自知的那一类价值判断均为大道理。但在自知的那一类中,价值判断的出发点可能是另一个价值判断,甚至可能出现“好几个价值判断互为出发点,循环论证”的情况。

为了确定是否自知“好”的定义,我们有必要将价值判断分为“可以归因于大道理以外的东西(比如说“是否有助于实现某个具体的目标”)”和“不可以归因于大道理以外的东西”两种。此处我们将“价值判断自身就是大道理”也视为“将价值判断归因于大道理”。
\end{explain}
有些词具有实际的含义,如“公平”“正义”“热心”“亲密”等。但如果这些词语劣化成了宏观概念,那么与之相关的判断就也会劣化为价值判断。
\begin{explain}
用大词容易让人容易忽略“这是在价值判断”,比如“没有意义”“不应该这么做”之类。但绝大多数人在下这样的判断时(无论判断自己的行为还是其它事件),都不是根据某个很严密的普遍终极答案而得出的结论。此时不能默认这种评判是恰当的,一定要确认清楚出发点。真的回头一想,大部分出发点是“某个很简单的思路,想达成某个很直接的目的”和“不清楚自己的出发点,价值判断是个污染”两种情况之一,有严密判断的情况极少。过于草率的价值观所产生的判断没有实际参考价值。
\end{explain}
我们将\indicate{参照价值判断的结果,进一步判断其它事物的好坏}的思考回路称为\source{价值观}。
\begin{explain}
多数情况下,价值观由多个价值判断与一些其它的想法组成。这些价值判断和其它想法越草率,包含越多大道理,价值观也就会越劣质。一个人可能因为各种原因被草率的价值观污染,而后该价值观就会持续生产脱离现实的结论。

比如将“达到要求”认为是“好”,“没达到要求”认为是“坏”,并根据“自己没有做到最好”认为自己是坏的;或者反过来,根据“别人没有做到最好”认为别人是坏的。具体的心态、行为和用词可能更多样,比如说“责骂/被责骂”“差劲”“无能”“没有希望”“失败”“不配活着”等。
\end{explain}
不同的价值观的“好坏”定义不同,脱离具体价值观谈论“好坏”没有实际意义。
\begin{explain}
此处的“没有实际意义”指的是“会引入歧义,从而无法得到有效结论”。更一般地,本指南中不说明前提时,所有默认价值判断全部都是以“是否会有歧义”“思路是否连贯”“推理过程和结论是否有效”“是否有助于目标”等判断为标准。

特别地,使用一个价值观的“好坏”去评判另一个价值观是否“正确”,是无意义行为。“正误”只能用于评价分析和结论,而无法用于评价出发点。
\end{explain}

\subsection{责任与要求\label{sec:责任与要求}}
我们将\indicate{一个人应该完成某个目标}的认知称为\source{责任}。我们也将这个认知表述为“这个人有完成目标的责任”。我们将“按照责任行事”称为“承担责任”,将“驱使某个人承担责任”称为\indicate{要求}。
\begin{explain}
每一个责任都可以视为一个价值判断。在不同的价值判断下,一个人会拥有不同的责任。

责任的来源多种多样,可以是“自己觉得自己应该做”,也可以是“别人觉得自己应该做”。在行为模式层次的视角下,这二者没有本质区别。
\end{explain}
如果某个责任是因为“觉得某个目标好”这一价值判断而产生的,那么就可以将这个目标视为一个\indicate{事务},将责任视为\indicate{处理方法}。我们将这个价值判断称为\indicate{该责任依照的价值判断}。
\begin{explain}
责任也是价值判断,它和其对应的价值判断不同之处在于,责任中针对“一个人”这一行为主体做判断,而其依照的价值判断对于“目标”这一和人无直接关系的现象做判断。

责任中的“一个人应该完成某个目标”和价值判断“觉得某个目标好”中的目标不一定是同一个,其中可以有“觉得目标A好→觉得实现目标B可以实现目标A→觉得应该完成目标B”,或更复杂的思考过程。

这里可以讨论两种问题:一种是“责任是否正当”,这取决于判断“目标A是否正当”,本指南不引入特定的价值判断,这不在本指南的讨论范围内;另一种是“责任是否有效”,这取决于判断“目标B是否可以实现目标A”,这是本指南着眼之处。如前所述,有效的责任在本指南中被称为\indicate{解决方法}。
\end{explain}
责任不一定有助于其依照的价值判断。必须有解决事务的能力,才能承担责任。
\begin{explain}
如果对现实的认知和对事件的分析很草率,那么分解目标的方式也就会很草率。正因如此,我们需要调研能力以清楚地认知事物的复杂性,需要分析能力以理清事物之间相互影响的方式,需要计划能力以编织有效的操作,最后需要执行能力以实现这些操作。如果没有处理复杂事务的完整能力,就无法分解并达成目标。具体论述参见\hyperref[sec:现实事务处理能力]{1.2节}。

从价值判断中直接得出责任,并且据此来提出要求,经常会遇到被要求的人没有对应能力的情况。\indicate{宏观特点只能作为某些复杂现象的结果,而不能作为目的和要求。}
\end{explain}
无视一个人的能力情况,就对一个人提出要求,是不负责任的。
\begin{explain}
这里表示本指南态度的“责任”指的是“应当完成‘完成目标’的目标”,遵循本指南的默认价值判断。

这里的“要求”涵盖很多内容,如“应当坚强”“应当成长”“应当优秀”“应当可以熟练应用某知识”“应当听得懂意思/指令”“应当明白自己的需求”等。读者可能觉得这种描述有些像是投射。严格来说有些不同,因为这里还涉及自我污染。本指南不引入投射的概念,故对此不做展开。如果读者有这方面的讨论需求,可以参考\hyperref[sec:指挥]{4.5节}。
\end{explain}

\section{人的意识演化基本模型\label{sec:人的意识演化基本模型}}
\hfill\begin{minipage}{0.55\textwidth}
\fontsize{8pt}{12pt}\selectfont\fontsize{8pt}{12pt}
\raggedright 思想越过六千层腐烂的美梦,浇筑成一台链接神经的水泵。\\
我的星球还没学会自西向东,就要被迫融进这条神的河流。\footnote{\bilibili{av9206055}\another\netease{466403600}。}

\raggedleft JUSF周存《欲》

\raggedright 傀儡之身,空有懵懂灵魂。是非不分,渴望人类的身份。\\
越是天真,越是无法放弃认真。\footnote{\bilibili{av2858071}\another\netease{34380080}。\\}

\raggedleft 鸟爷ToriSama《告别诗》
\end{minipage}

在系统介绍了复杂性的相关概念后,我们得以从更深入的视角出发,刻画人的意识现象。概括地说,第二章中的分析考虑简单现象如何编织成复杂现象,而本节的分析则考虑简单现象都能涌现出什么复杂现象。我们将本节介绍的内容称为\source{人的意识演化基本模型},或是简称\source{演化模型}。
\begin{explain}
本节的部分内容是\hyperref[sec:原生家庭]{2.5节的内容}在本章用语体系下的重新叙述。读者可以对比本小节与2.5节,以获得更全面的理解。

我考虑过要不要用“动力学”的来代替“演化”为这一模型命名,但考虑到大部分读者会对“动力学”这个术语不熟悉,所以选择了“演化”。不过读者仍然不应望文生义,需要将\indicate{人的意识演化基本模型}当成一个整体概念理解,将本节的所有内容均当做其定义。
\end{explain}

\subsection{并行与竞争\label{sec:并行与竞争}}
我们将\indicate{同时触发多个行为链/行为模式/思路/思考回路}的现象称为这些行为链/行为模式/思路/思考回路的\indicate{并行}或\indicate{并存}。
\begin{explain}
这是由于外部有多种不同的刺激,触发了多种不同的行为模式。即使刺激消失,这些行为模式也可能可以自己触发自己,因而维持下去。
\end{explain}
我们在同一时刻接收到的所有刺激,可能会满足多个行为的触发条件,而我们可能无法执行所有的行动。
\begin{explain}
刺激有可能来自外部环境,有可能来自当前处于的行为模式。

无法执行有可能是因为外界条件不允许(比如说一个东西只有一个用途),有可能是因为自身条件不允许(比如同一时间只能做一件事),有可能是因为计划和决策(比如克制某些不利行为)。

为了准确描述这样的现象,我们修改一下关于触发的定义:我们将“接收到信号”称为\indicate{触发},将“产生实际行动”称为\indicate{执行}。在不需要严格区分的场合下,本指南有时依然使用“触发”来指代整个过程,请读者自行分辨。
\end{explain}
我们依刺激最强烈的那个信号而行事。我们将这一现象称为刺激或行为的\source{竞争},并称被触发的刺激/行为\indicate{参与}了竞争,最终依照的刺激/行为\indicate{赢得}了竞争,\source{覆盖}或\indicate{打断}了其它刺激/行为。
\begin{explain}
我们也可以依此定义行为模式的竞争和相关概念。此处的“打断”将“从触发到执行”的过程视作行为链,与\hyperref[def:打断]{前文}定义相同。

\trained 可以将“参与竞争的所有行为(以及对应的刺激强度)”视为先验分布,并以此引入贝叶斯分析。本指南不过度深入这方面细节,在此仅作提及。

注意这里仅是非常模糊的描述,本指南之前只定义过\note{刺激},从来没有定义过“刺激的强度”。实际上,不深入生理过程,就无法精确描述“刺激”的具体机制。引入“刺激的强度”这一概念是为了之后叙述方便。以下将会引入另外几个假设,本指南认为信号的刺激强度仅会因这几个原因而改变。
\end{explain}
没有其它改变时,同一个信号在不同时刻带来的刺激程度大致相同。
\begin{explain}
由此,在面对相同的环境时,人的第一反应是相同的。我们将由此触发的行为称为该信号的\indicate{应对方式}。

刺激强度不以人的主观意志为转移,人的主观意志可以视为名为“意志力”的另一种刺激。如果你发现自己身上总是有改不了的坏习惯,那么说明这一习惯的刺激要强于意志力的刺激。在想办法降低习惯刺激(比如远离诱惑源)、提升意志力刺激(比如增强理性)、或引入新的刺激(比如遭遇突发事件留下深刻印象)之前,每次尝试都几乎必然失败。
\end{explain}
在短时间内,同样的信号刺激频率越高,总刺激强度越高。
\begin{explain}
这里的短时间只“从察觉到信号到做出行为”尺度的事件。依照具体行为的不同,时间尺度可能在几毫秒至几分钟内不等。

当前信号的刺激会与之前信号的刺激相叠加,从而印象更深。这一假设能够解释2.2.1节中对“\hyperref[para:记忆]{有记忆是形成行为的一个必要条件}”的论述:经历过并记住的相同事件越多,就越容易联想到,并在脑内预演可能的结果。每次想到结果都是一次信号,想得越快,信号出现的频率就越高,刺激也就越强。这产生了一个正反馈循环:刺激越强越容易触发该行为,越触发记忆越深,记忆越深刺激越强。行为在这一过程中逐渐形成。
\end{explain}
简单的东西刺激更强。
\begin{explain}
这出自两方面原因。一方面是简单的东西过脑子的速度更快;另一方面是简单的东西会在更多场合内出现。

当一个行为逐渐形成以后,除了“之前经历过的相同事件”以外,我们的脑子里还会多出另一种记忆:“见到信号就做出行为”。这种只有首尾的记忆比包含过程的回顾更简单,回顾这种记忆比回顾事件要更快,从而会提供更强的刺激\footnote{实际上不会分两步,而是渐进式形成。这里为了讲解而做了简化。}。当这一过程再次熟练以后,我们就多了一个无意识的条件反射。
\end{explain}
如果一个行为A触发了另一个行为B,并且行为A和行为B竞争,最终行为B覆盖了行为A,那么就称行为A\indicate{转化}、\indicate{转变}或\indicate{演化}为了行为B。
\begin{explain}
同理,我们也可以定义行为链和行为模式的转变。对于\rigorous,转变指的是“某一次行动时的触发和覆盖”。

转变这一概念仅针对有因果关系的A和B,不是所有的“行为B打断了行为A”都是转变。

如果行为A总是会转化为行为B,那么我们就可以认为它们已经形成了一个新的行为(行为链)。如我们上面所讨论的“复杂的行为链逐渐演变为简单的行为”,就是这样的一种演化过程。在不引起歧义的情况下,我们也会使用“行为A转化/演化为了新的行为”的表达。
\end{explain}

\subsection{缺失与外溢\label{sec:缺失与外溢}}
我们将\indicate{一种行为模式一次触发中行动的数量}称为该行为模式本次触发的\indicate{精力},将\indicate{一个人行动的数量}称为这个人的\indicate{精力}。
\begin{explain}
类似于\note{意志力}的定义,我们难以精确地将一个人的精力定义为“总量”“频率”等,仅笼统地将其称为“数量”。不同行为模式的精力特性有不同之处。有些可能以固定的频率行动;有些可能一天只能有一定数量的行动;刺激较足的时候可能精力也会多些;也可能因为有别的行为模式触发导致精力变少。请读者根据实际情况,为不同行为模式分别选择合适的理解方式。

这些行动不都是我们决策后行动的,不属于意志力部分的精力就是不可控的部分。它们会按照自身的发展机制演化,上一小节所说的“依刺激最强烈的信号做事”就是一个例子。我们也仿照之前的定义,将这一现象称为“行为/刺激在\indicate{竞争}精力”。
\end{explain}
我们将\indicate{无法使用正确的方式应对信息}的现象称为\indicate{缺失}、\indicate{缺乏},或“\indicate{缺乏}有效手段”。我们将\indicate{没有应对方式的信息}称为\indicate{盲点}。
\begin{explain}
缺失和盲点都不依赖于自知,并非“只有发现的缺失才是缺失”,也并非“只有没发现的信息才算盲点”。这里的“正确”取决于具体情况下的判断。

盲点一定缺乏有效应对手段,但缺失却不一定是因为对应信息是盲点,应对方式错误也算作缺失。

盲点普遍存在,在环境切换时会大量出现。在熟悉的环境下,很多信息会起到“提示你什么都不用做”的作用,依实际情况不同,这可能是盲点、有效手段或污染。

我们有时能观察到在盲点处发生的竞争过程。依照情感色彩的不同,我们可能使用“醍醐灌顶”“灵光乍现”“一念之差”“脑子一抽”等不同的描述来称呼这一过程。
\end{explain}
当我们同时遇到盲点和能刺激其它行为的信号时,盲点会因此与该行为建立条件反射,逐渐成为新的信号。我们将“随着多次竞争,应对某个信息的行为逐渐固定”的过程称为该行为\source{关联}了该信息,或是信息与行为之间建立/产生了\indicate{关联}。如果该行为不是正确的应对方式,那就将其称为\indicate{外溢}的行为,并将这一过程称为行为的\source{外溢}。
\begin{examples}
关联过程就是条件反射的形成。

竞争和外溢的过程可视为“行为\indicate{污染}了盲点”,此时充当污染源的是这一行为。该行为不一定是自己之前做过的,也可能是记住了之前其它人的操作,并在此模仿。

外溢可以被阻止。在污染还未形成时最为方便,仅需抢先建立正确的反应即可。污染已经形成则会比较麻烦,参见后文的论述。

\indicate{关联}是一个中性词汇,它指代所有行为的形成。而\indicate{外溢}特指其中处于应用范围之外的部分。
\end{examples}
外溢行为产生的刺激会为我们处理事务产生\indicate{干扰}\footnote{一个行为干扰一个行为模式指“它们同时触发,并且行为会覆盖行为模式中的行为”。}。如果外溢行为的干扰过强,那么即使我们当前还有别的安排,也会转而去按照外溢行为而行动。我们将\indicate{某行为产生刺激,使得我们脱离当前的行为模式}的过程称为这一行为\indicate{打断}\footnote{这里的打断和\hyperref[def:打断]{先前}的定义一致,只不过在这里我们更关注主语。}了行为模式。
\begin{explain}
外溢行为高度简单,高度熟练,在面对很多情况时,都是反应速度最快的那个。这使得它可以迅速给出简单、详细、可以立刻执行的操作。这些操作的正确性不得而知(比如说焦虑了就去刷手机),它们的唯一特点是会给人最强的刺激,吸引人去选择。
\end{explain}
竞争和关联是很随机的过程,需要视为宏观事件。
\begin{examples}
从很抽象的意义上来说,竞争过程与自然选择类似。一个信息会被什么行为污染,后续又会有什么改变,高度不可预测,需要具体调研具体分析。

随机性来自两方面:一方面是前提的任意性,即同一个刺激对不同的人来说,参与竞争的行为本身就会有不同;另一方面是过程的不可预测性,即竞争过程经常受到当前环境的明确(经常是主导性的)影响,过程经常是不可复刻的。因此,不同的人所拥有的外溢行为可能完全不同。

读者不应将其总结为“缺失是一切问题的本质”之类的观点,这只是正确的废话。我们不可能仅靠这种表面分析就彻底分析透彻这一宏观事件,不可能仅靠讨论成因就获得“一切问题的答案”。请读者将本指南中的概念与其它语境下的概念加以区分。本小节讨论缺失的唯一原因是缺失涉及因素较少,可以解耦出来独立讨论。它讨论起来较为简单,并且有些结论能在之后有所应用。

\trained 可以将“能指与所指之间关联的任意性”与此处观点对照理解。本指南不深入语言学内容,此处仅作提及。
\end{examples}
% 如果一个思考回路没有发现信号,我们的行为就触发了,那么将这个行为称为该思考回路下的\indicate{潜意识/无意识/下意识行为}或简称为\indicate{潜意识}/\indicate{无意识}。
% \begin{examples}
% 这里的定义相比术语进行了一些简化,定义仅在本指南范围内生效。外溢是很随机的过程,这也使得潜意识高度混乱。

% 本指南中使用的“自知”“有明确认识”等词语不仅用于区分行为,而是可以区分任何类型的认知。“有明确的目的,但没有注意到行为的副作用”等很多情况下,用“无意识”不能无信息损失地描述现象。实际上,所有仅关注行为本身,而不关注其影响的特征都只是表面分析。关于无意识行为的讨论不是本指南的重点,本指南会尽量避免使用这些术语。
% \end{examples}

\subsection{全局行为模式\label{sec:全局行为模式}}
我们将“某一行为/行为模式关联的的所有信息”称为这个行为的\indicate{触发范围}。
\begin{examples}
对同一个行为来说,外溢得越多,触发范围就越大。但是触发范围大不代表行为外溢得严重,属于触发范围但不属于应用范围的部分才是外溢的部分。一个行为/行为模式外溢得越多,它劣化也就越严重。
\end{examples}
行为的外溢会使得它在更多方面参与竞争,并且更容易赢得竞争,这会使其更容易进一步外溢。
\begin{examples}
盲点可能是具体事物,也可能是很抽象的内容,比如说“恐慌”“焦虑”“不知所措”。这种机制会导致“越焦虑就越刷手机”等现象,因为“刷手机”这一行为已经严重外溢,导致它在很多情况下都是最常被触发的行为,这进一步导致了“刷手机”继续外溢。掌握的信息越少越偏,面对事件时就越不容易拥有恰当的反应方式,盲点就越多,于是就越有可能产生污染。

每一次行为外溢都可以视为形成了一条\indicate{价值判断}。这样形成的价值判断大多是不自知的,人会先感觉到“我想这么做”,然后再将其用语言总结成“应该这么做”。可以为自己的行为找很多很充分的理由,但它的最初形成是污染(找理由后,有可能确实因为理由而培养出新的行为,这应当算作另一条行为)。
\end{examples}
同理,认知越具有普遍性,就越难正确地总结出来,但也越容易外溢。
\begin{examples}
“因为指代而使词语概念发生改变”的现象也算是“概念被词语污染”,这种机制是我们会将具体事件劣化为宏观概念的主要原因;“根据自身见识给复杂现象的结果归因”会经常归到错误的原因上,使其和结果错误地关联,得到“这一定就是因为某个原因”的外溢认知。
\end{examples}
行为的外溢持续下去,会导致某些行为几乎在任何时候都会触发。我们称这样的行为称为\indicate{全局行为}/\indicate{默认行为}。
\begin{explain}
全局行为只关注触发范围,而不关注触发范围与适用范围的差别,不关注行为是否外溢。全局行为也会参与竞争,如果同时有更强的刺激,它也会被覆盖。全局行为不包括呼吸、心跳等生理活动(因为它们不受其它事情触发),但包括“时时刻刻都小心翼翼屏息”、“因为病痛折磨而无法集中精神”之类的情况。全局行为较为少见,但讨论起来较为简单,可以用于展示一些特点。这些特点对稍后介绍的全局行为模式也适用。

有一些比较简单的现象可以视为全局行为。例如,当“想一想其他事情和我有什么关系”成为全局行为时,就会觉得万事万物都和自己有关,于是就会产生名为“全能自恋”的现象;当“发现事情是别人干的”成为全局行为时,就会把一些其它的类意识行为(比如推销电话或app)视为有人在操控,进而可能发展成被害妄想等。本指南不过度使用精神分析语言,故不做过多探讨。

瘾也是一类比较常见的全局行为,如烟瘾、酒瘾、性瘾、毒瘾等。需要注意的是,并非所有通常意义下的瘾都是全局行为,但它们被称为瘾,至少属于外溢行为。瘾很难戒除,是因为其有过多的触发条件,这也就是通常所说的“身瘾好戒,心瘾难戒”。

当自身存在全局行为时,进入新环境后,全局行为很容易污染新环境中的信息。手里有把锤子,看什么都是钉子。自己会真心实意地相信自己认知的正确,全心全意地践行全局行为。如果一个人的全局行为中不包含自我审视和反思,那么就丝毫不会怀疑其它全局行为是否恰当。

因为行为的外溢高度不可预测,全局行为因人而异。不同的全局行为之间很可能互斥,拆解一个全局行为培养一个新的,难度很高。本指南的目标是让读者将“唤醒理性”培养为自己的全局行为,以做到时刻清醒地认识世界,有效地行事。这难度很高。
\end{explain}
和行为类似,行为模式也有“刺激越强越容易触发”的机制,这一部分以及有关的定义不再赘述。如果一个行为模式在几乎任何环境下都会触发,就将其称为\indicate{全局行为模式}/\indicate{默认行为模式}。
\begin{explain}
也可仿照外溢行为,定义“外溢行为模式”。本节中的讨论同时适用于普通的外溢行为模式与全局行为模式,仅讨论全局行为模式是为了方便。读者可以通过加入“行为模式会触发的环境”的条件,来使以下讨论同时适用于普通的外溢行为模式。如果同时还有更强的刺激,全局行为模式可能被其它行为模式覆盖。

行为外溢得越严重,就越容易被触发。外溢的行为越多,这些行为之间就越容易相互触发,从而越容易形成一种严重外溢的行为模式。全局行为模式可以通过这种机制,直接由一些外溢的行为产生。之后为了叙述方便,我们将全局行为模式的组成部分也称为全局行为,读者应能自行分辨。

由于外溢的行为大都是高熟练度的无意识行为,全局行为模式也因此容易成为无意识的行为模式。
\end{explain}
类似地,如果一个思考回路在几乎任何环境下都会触发,那就将其称为\indicate{全局思考回路}/\indicate{默认思考回路}。
\begin{explain}
全局思考回路都是全局行为模式。也可类比定义“外溢思考回路”。

全局思考回路几乎每个人都有,一个明显的例子是“语言”。注意,语言的触发范围不都属于语言的外溢,一些概念、分析、认知是正常行为(虽然在一些情况下很难分辨到底是否是外溢)。本指南不过度深入哲学和精神分析,不对这方面展开讨论。

全局思考回路中的价值判断组成了一个价值观,该价值观会作为刺激,触发全局行为模式;同时,它会展示全局行为模式每一部分的正确性。
\end{explain}
我们不应将全局思考回路拟人化。
\begin{explain}
“将行为模式拟人化”是一种常见的外溢认知。在我们分析全局思考回路时,它会持续施加干扰。必须时刻注意。

具体来说,我们不应默认全局思考回路具有“保守”“懒惰”“抗拒改变”“主动”“固执”“控制欲”“指导欲”“想吸引注意力”等特点,不应该为全局行为模式赋予“本心”“本色”“底色”“灵魂深处”“骨子里”“精神内核”等感情色彩。这些词语都是行为模式层次,或人的层次的宏观概念,使用它们会带来不必要的污染。我们需要且只应该站在行为的层次来看待全局行为模式,将其视为一种需要研究的客观现象对待。

只有站在行为的层次,充分调研并分析我们所拥有的所有外溢行为和它们之间相互触发的方式,我们才有能力施加影响,从而主动改变全局行为模式。如果将其视为一个整体,就无法执行这么精细的操作。劣质的全局行为模式从来不是一个整体,没有什么确定的人格和倾向,没有什么统一且坚定的看法和信念。它只是一堆容易互相触发的外溢行为,在某一件具体的事上看起来可能很统一,但整体是混乱无序的。
\end{explain}
全局行为模式越是劣质,就越会以高频率发生不可预测且不可控的改变。
\begin{explain}
行为会依照行为的演化机制发展,不断因为新出现的刺激和记忆而更新。行为的外溢不可控,于是由外溢行为组成的全局行为模式也不可控。

这也导致两个不同人的全局思考回路之间差距非常大,可能完全无法交流。这种现象也可能发生在现在的自己和以前的自己之间。全局思考回路越是劣质,就越难以理解、控制、改变别的行为模式(包括自身)。这会给人以“人不可能相互理解”“做人只能靠自己”之类的错误结论。在大多数环境下这种看法甚至没什么问题,但它们本身不正确。详细讨论见第五章。
\end{explain}

% \subsection{清除外溢\label{sec:清除外溢}}
% 我们将\indicate{每次遇到一个信息时,都按照固定的方式操作,直到形成习惯}的过程称为\indicate{培养}。面对未被污染的信息时,我们可以较为轻易地培养出正确的应对方式。
% \begin{explain}
% 仅需遵循行为的产生机制,坚持重复使用正确的方法,直至产生关联即可。

% \indicate{这一切的前提是我们有意培养}。我们需要有获取正确应对方法的途径(无论是靠自己或外部)。如果没有主动培养,行为就会遵循上述机制,逐渐将反应固定为刺激最强的那个。
% \end{explain}
% 我们难以处理已经被外溢行为污染的信息。草率的处理手段\indicate{注定失败},只会使情况更糟。
% \begin{explain}
% 这里的\indicate{注定失败}带有一定的修辞效果,不意味着“一定不能成功”,但其极低的成功率使得成功不可复刻。这不是科学可重复的处理方式。本指南其它地方出现的“注定失败”“没有成功的希望”“只能看运气”等表述也是这种意思。

% 打断的机制使得难以修改并清除污染。如果没有充足的准备,无法拿出另一套简单、详细、可以立即执行的操作来代替原有的外溢行为,那么就很容易在坚持了几天以后半途而废,反而让外溢行为又污染了想要改变的尝试,使得下一次更容易半途而废。

% 这是处理“未被污染”和“已被污染”的两种信息时的本质不同之处。必须有区分这二者的能力,才能分清是偶尔还是习惯,才能确定改正或处理的方式。
% \end{explain}
% 如果有持续且强烈的刺激,全局行为模式会因此改变。但如果仅有刺激,改变的方向不可控。
% \begin{explain}
% 当刺激长期存在时,全局行为模式会不断被触发。在经历了充分长的时间后,要不然全局行为模式污染了刺激(此时归入先前的情况),要不然我们会获得“全局行为模式处理不了这一刺激”的新信息(总体来说,越善于总结经验就能越早发现这一点,越不善于总结经验就会越慢,只能指望最基础的行为养成机制)。

% 这一信息会在每一次触发全局行为模式时都出现以给予我们刺激,打断全局行为模式,使我们放弃接下来的行动(此时不同的人会感受到不同的情绪,包括但不限于急躁、失落、自责等。这不是讨论的重点所以略去。同时,如果这些情绪属于全局行为模式的一部分,或是因此而成为了全局行为模式的一部分,那么应该被视为“刺激被污染”的情况),转而尝试新的行为。

% 此时的情况类似于面对未被污染的信息时的情况,但仍有主要不同:全局行为模式仍然在持续触发,它会频繁打断我们的思考,从而使得我们难以维持培养的过程。这使得主动修改自身的全局行为模式和全局思考回路极为困难。

% 如果我们理性的速度和系统性不够,无法追上全局行为模式,只能提供刺激却不能提供完整的路线,就无法有效地改变,只能从一个污染跳到另一个污染。具体讨论见第六章。
% \end{explain}
% 我们应当将理性培养为自身的全局行为模式。
% \begin{explain}
% 理性的适用范围是所有行为模式,其不存在外溢现象,我们只需要担心“无法保持理性”的问题(但是每一种具体的处理手法都有可能超出其适用范围,产生外溢,请读者注意区分)。

% 最优状态应该是“使理性成为自身唯一的全局行为模式”,但那样难度过高,理性能够监督自身的意识活动就足够了。甚至在实践中,这一要求还是有点高,我们只需要在处理事务时能够唤醒理性,就已经足够了。

% 需要说明的是,理性不和情感互斥,不会出现“因为理性所以变成冷血机器人”的情况,过于冷漠正说明不够理性,身上有不可控的“压抑情感”的全局行为模式。非要找个对应的性格形容词的话,应该使用“豁达”“超脱”等。

% 掌握一种能力(知识体系),等价于在该能力的所有应用场合,都不被外溢行为模式打断(不受干扰更好,但更难以实现)。只有当我们不被劣质的全局思考回路干扰,思路不被打断,我们才能得到整体观察事物的机会,进而才能培养处理复杂现象的能力。

% 我们需要使用理性来修正自身的行为外溢现象,并且打断行为继续外溢。我们需要在每个场合都全面地接收所有信息,之后选用正确的看待方法。如果不知道怎么才算正确,那么就去研究或学习,在找到方法之前,控制住自身任何下意识的反应,不作任何草率的行动。
% \end{explain}

\section{认知外溢与因果性破坏}
\subsection{想法的竞争与外溢\label{sec:想法的竞争与外溢}}
\noindent\note{想法}远比其它类型的行为更容易触发。
\begin{explain}
触发难度上,想法可能只需要上一个想法就可以触发,而其它行为需要外部刺激,外部刺激通常没那么多;持续时间上,一个念头可能只会持续0.1秒,而大部分完整的行动需要数秒或更长;复杂程度上,我们的记忆可以使多个彼此差异很大的想法同时存在并且快速切换,而外界信息变化幅度通常较为缓慢,且关联性通常更高。
\end{explain}
想法会提供远比其它类型的行为更多的刺激。
\begin{explain}
大多数信号仅能触发一次行为,但其只要进入短期记忆,就会触发多次想法。即使是持续存在,能多次触发行为的信号,想法的触发频率也要远高于其它类型的行为。

这使得认知的形成可以非常快速,可能仅需一次接触就足以形成稳定的认知。在一些语境下,这被称为“第一印象”。同时,认知会刺激其它有关的认知,从而形成思路;如果能够循环触发,则会极大地加强这些认知之间的关联,有可能一次循环触发便能形成一个新的思考回路。之后每次触发这个思考回路,都会强化其中的关联。对于知识体系来说这叫融会贯通,对于脱离应用范围的认知来说就是加重污染。
\end{explain}
我们将“随着多次竞争,逐渐接收到信息就产生某个想法/联想到某个认知/触发某个思考回路”的过程称为该想法/认知/思考回路\indicate{赢得}了\indicate{竞争},\indicate{关联}了信息,\indicate{覆盖}了其它想法/认知/思考回路。如果该认知不正确/想法和思考回路与信息无关,那就将其称为\indicate{外溢}的想法/认知/思考回路,并将这一过程称为想法/认知/思考回路的\indicate{外溢}\label{def:外溢认知}。
\begin{examples}
盲点(和其它信号一起)出现时,我们会将受到刺激最强的认知套在盲点上。这一机制会在盲点和认知之间建立虚假的相关性。像不意味着真的有关系,除了“会先后出现”以外,它们不保证有任何共通之处。认知污染了盲点以后,就会妨碍对盲点本身信息的正确收集、理解、应用。
\end{examples}
我们思考时,会受到相关外溢思考回路的显著干扰。思考链越长,外溢思考回路的干扰就越大。如果某个认知会被其它想法覆盖,那么我们就称在此认知上存在\indicate{障碍}。
\begin{explain}
外溢思考回路会提供大量熟练且缺乏系统性和逻辑的想法。如果思考链被这些内容结论打断了,此次分析就会无效。思考链越长就越容易被打断。换句话说,只有在外溢思考回路不产生明显干扰的情况下,我们才能进行长时间、大规模的思考。

并且,我们很难去验证外溢思考回路给出的答案。外溢思考回路会非常熟练地证明自身每一个结论的正确性。如果没有其它得出结论的手段,便会又回归外溢思考回路的判断。除了少数情况有“无法辩驳的事实直接摆在眼前,让人印象深刻无法忘记”以外,其它刺激都会在某一次注意力被转移以后,再也想不起来。
\end{explain}
如果关于某个事件的认知/行为/...不能对事件的实际结果产生影响,那么就将这一判断称为该事件的\source{无效认知}/\indicate{行为}/...。我们也将无效认知称为\indicate{无关认知}或是\indicate{废话}。
\begin{explain}
读者可以自行验证,第一章中的\note{无效操作}与此处的无效含义一致。本指南中还会用到“无效努力”“无效精力”等概念,请读者自行套用。

如果认知是正确的,但也不产生影响,那么这就是\indicate{正确的废话}。

认知本身当然不能直接影响事件。这里指的是其指引的行为对结果产生的影响,例如“得到了有效的方法”就“按照方法达成目标”、“判断出一定没希望”就“放弃目标更改计划”等。而像是“感觉自己受摆布”“觉得自己运气很好”这种认知,如果没有进一步的行为,那么就不可能产生影响。我们无法仅从认知中判断它是否是废话,必须结合现实情况。

这里的“产生实际影响”的比较基准是“自身没有参与”。如“这事情气到我了,骂两句舒服多了”的“舒服多了”不算做影响了原事件,但是可以算影响了“气到我了”这一事件。由于我们总是将分析视为“自身在模拟另一个行为模式”,我们总可以将事件视为外部的,不会出现“自己没法不参与心理事件”的问题。

劣化或外溢的认知基本上都是废话。脱离应用范围的大道理是废话。大多数\note{预料}是废话。这不只是因为“预料会劣化和外溢”,还有另一种原因:我们不一定具有“参照自身的预料,做出有效的举动,影响事件的实际结果”的能力。
\end{explain}

\subsection{认知的内容}
我们将一个现象本身称为它自己的\indicate{内容}。我们将\indicate{获得某个认知时,相关的所有事物的内容中,符合这一认知适用范围的部分}\footnote{对于\rigorous,这里指的是“所有内容取并集”后再和适用范围取交集。}称为这一认知的\source{内容},将认知称为\indicate{关于}这一内容的认知。
\begin{explain}
由于“旧认知生成新认知”是一个递归的过程,我们这里也遵循同样的递归流程,来定义认知的内容。

如果一个认知的内容仅有唯一的具体事件/现象,那么这一具体事件/现象一定是该认知的\note{参考事件/现象}。如果的参考事件/现象在该认知的适用范围内,那么它就上该认知的内容。

一个认知(或者其它定义了内容的事物,如后文提到的\note{资源}和\note{转述})也可以视为现象,这导致它具有两种不同的内容:一个是它本身(作为现象),另一个是它的内容。比如“有人和我说楼下出了车祸”就同时有“有人和我说话”和“楼下出了车祸”两种内容。这两种内容需要分别处理,如果不做提及,本指南中我们默认不将它本身视为它的内容。

内容和拥有该认知的思考回路相关,内容有很多不同的类型。某个具体物体、某个实际过程、某种概念、某种特点,都有可能是某个认知的内容。如果认识得不全面,认知的内容会小于它的适用范围;对于能明确知道适用范围的认知来说,它的内容就是适用范围。

一个认知的内容有可能是另一个认知,比如“xx认知的适用范围”这类认知的内容就是“xx认知”(当判断这一认知是否适用于目标情况时)或“适用范围”(当判断目标情况都能使用哪些认知时)。这使得认知可以通过内容组成复杂的递归嵌套。区分认知本身和认知的内容有助于我们讨论这类嵌套。

如果在后续的发展中,一个认知产生了变化(如“突然意识到它的范围可以更大”),那么我们将其视为另一个新的认知,并且对其重新定义它的内容。更一般地,如果一个认知的内容要多于它的参考现象,我们就将其称为认知的\indicate{泛化}或者\indicate{抽象化}。
\end{explain}
和\note{理解途径}处的处理方法相同,我们在确定内容的时候,会忽视掉一些不重要的差别。
\begin{explain}
差别可以是大家都知道的省略,比如说一些情况下的“某种特点”和“拥有这种特点的所有事物”;“一种事物”和“这个事物可能直接参与的所有过程”等。也有可能是某些指代上的差别,比如“用词不同”,这包括“错误理解了一个概念,然后用错误的理解去描述现象,但如果将描述中的概念替换为对应的理解,那么描述的现象没有问题”一类的现象。展开的讨论见\hyperref[sec:解读]{4.4节}。

但是如果内容脱离了适用范围(如“总是用某个特定的参考事件去理解一类事物”),则差别不可忽视。不同人拥有的认知可能表面上看起来一致,仅从用词和描述中看不出差别。但如果内容有不可忽视的区别,仍然不应将二者视为同一认知。

这也是定义中要加入“符合应用范围”这一限制的原因。如果去掉这一限制,剩下的定义将会类似于“能联想到这一认知的其它内容”。我们仍然可以据此分析“这一认知指代的现象是什么”,但这与本小节的讨论无关。
\end{explain}
因污染而形成的认知有可能没有任何内容。
\begin{explain}
污染可能来自外部,是对某些行为的单纯模仿,如“学说话”、“过家家”、“背诵名言警句”等;也可能来自内部,是自身认知外溢的结果,如“因为熟练而省略中间步骤”等。具体机制在前文已有展开说明,此处不再重复。
\end{explain}

\subsection{内容顺序与触发顺序}
如果认知A会成为触发认知B的信号,那么我们就将其称为认知A和认知B之间的一个\indicate{触发顺序},或是简称为“A\indicate{可以触发}B”。
\begin{explain}
触发顺序具有方向性。两个认知之间的触发顺序一共有4种:相互不可以触发、只有A可以触发B、只有B可以触发A、A和B可以相互触发。

“A可以触发B”可以作为一个独立的认知,但也可以是不自知的。这种认知也可能是外溢(比如在“这种触发顺序实际不存在”时产生)。

A触发B后,B有可能参与到和其它行为/认知的竞争中,最终可能无法执行。我们在定义中用“可以触发”来强调这一区别。

如前所述,当A与B同时被触发时,它们之间有可能产生\note{关联},从而相互触发。这样形成的触发顺序是随机的,四种情况(如果只算存在触发顺序的情况,那就是三种)都有可能出现。除此之外,A还可以通过其它认知/行为,和B产生关联。如果这样的行为链有顺序,那么就会形成“只有A可以触发B”的情况。
\end{explain}
如果认知A的内容会导致认知B的内容,那么我们就将其称为认知A和认知B的一个\indicate{内容顺序},或是简称为“A\indicate{可以导致}B”。
\begin{explain}
虽然这里采用了“A可以导致B”的措辞,但实际上A和B都是认知,它们对应的实际内容不以主观意志为转移。此处的措辞仅起到简化表达的作用。

“A可以导致B”可以作为一个独立的认知,但也可以是不自知的。这种认知也可能因外溢,在这种内容顺序实际不存在时产生。

和触发顺序一样,内容顺序也有方向性。两个认知之间的内容顺序一共有4种:相互不可以导致,只有A可以导致B,只有B可以导致A,A和B可以相互导致。

但触发顺序和内容顺序也有明显不同,会相互导致的内容远少于会相互触发的认知。中思考回路内,认知的关联会新增触发顺序,但不会新增内容顺序。
\end{explain}
在两个认知之间,如果\indicate{一个触发顺序不对应实际的内容顺序},那么我们就将这种现象称为触发顺序对内容顺序的\source{因果性破坏}。
\begin{explain}
触发顺序是认知之间的因果性,内容顺序是内容之间的因果性,所以我们如此命名这一现象。

因果性破坏在一定情况下是好事。当我们需要决策的时候,经常会有“因为要完成某个目标,所以要达成某个前提条件/避开某种不利影响”之类的思路。如果我们只会顺着内容的顺序思考,而没有逆向思维,就无法做到(或是无法思路清晰地做到)很多事情。因果性破坏在一定情况下是坏,比如认知的外溢总是会导致因果性破坏。

两认知内容的相关性会导致认知触发的因果性,但认知触发顺序的形成不一定与其内容相关。如果将认知触发的因果性/相关性错误地当成了认知内容的因果性/相关性,就有可能得出“是结果导致了原因”“两个(实际上只是相关的)现象有因果关系”之类的结论。这是由于因果性破坏而得到的劣化认知\footnote{对于\rigorous,这里的具体细节如下:我们产生了“A和B内容具有因果性/相关性”的认知C,它可能是“A和B触发具有因果性/相关性”的认知D的外溢。但更有可能的是,我们并未得出认知D,而是直接从这一现象中错误总结出了C。}。
\end{explain} 

\section{总结与讨论}
\subsection{本章总结}
本章内容围绕\indicate{复杂系统}及其看待、处理方式展开。相比第二章,本章的讨论内容较为集中,除了复杂性的概念介绍,就是使用系统性的视角,来分析各个具有复杂性的事物。正确运用系统性的视角分析复杂问题,是我们处理现实事务的必备\indicate{能力}。

正式介绍复杂系统之前,我们需要先做一些准备,对我们认知世界的方式建模。这一部分虽然也属于人的意识活动,但与第二章的内容关系不大,与第三章联系更紧密,故将其放在第三章。我们引入了\indicate{概念}和\indicate{特点}作为基础概念,并且引入了\indicate{识别}和\indicate{指代}两种基础操作。据此,我们得以引入\indicate{出发点}和\indicate{能力范围},以界定在思考时是否产生\indicate{歧义},以使得思路\indicate{劣化}。

劣化是思考的主要危害,而对复杂性的认识不足,思考得过于概括粗疏,导致\indicate{信息损失},则是产生劣化现象的主要原因。在充分讨论劣化的相关内容后,我们得以在引入复杂系统的概念后,立即讨论其处理方法。

我们介绍了复杂系统中会出现的两种现象:更偏向从某一具体的\indicate{高层次事件}的视角看的\indicate{编织},与更偏向从全体\indicate{低层次事件}视角看的\indicate{涌现}。同时,我们还展开讨论了会导致思考劣化的\indicate{宏观分析}现象,仔细分析复杂系统为何会使得浮于表面的思考脱离实际。

讨论完复杂系统以及宏观分析的一般理论,就可以将这些理论应用于实际事物上。我们首先深入考察了事务处理方面的内容。如果对事务的认识不足,无法将它的本质困难纳入考量,只是因为觉得“应该这么做”就将其定位了目标,就会产生很多没有能力承担的\indicate{责任}。如果任由这种草率的\indicate{价值判断}组成\indicate{价值观},就会产生很多无意义的内耗。此处我们还声明了本指南所持有的价值观:解决现实事务。

最后,我们使用复杂性的视角,重新梳理了一遍人的基本模型。我们考察了\indicate{刺激强度}对\indicate{行为形成}的影响这一复杂系统,通过分析\indicate{缺失}的现象如何产生行为的\indicate{外溢},部分解释了不可控行为的产生原理,同时回顾了处理方法。我们提出了与\indicate{全局行为模式}的概念,并分析了它的复杂形成过程,从而确认了将其清除的困难程度。需要注意的是,此处的模型仍然是高度简化的版本,仅供本指南展开后续理论使用,有很多细节缺失。读者需要注意其应用范围。

作为基本模型的一个应用,我们分析了\indicate{认知的外溢},主要考察了其中的\indicate{因果性破坏}现象,展示了外溢机制如何在现象之间产生虚假的因果性,并因此使认知劣化,无法包含任何\indicate{内容}。

除此之外,本章中提及的一些概念,在之后的一些篇幅中也会频繁使用,一定程度上方便了行文与理解。

\subsection{澄清与叠甲}
大家在阅读本章的过程中,或许会产生一些联想。可能有些读者会觉得本章的内容和一些其它理论很像,也可能有些读者觉得本章的内容可以应用到一些其它领域。读者可能会疑惑,为何本章没有对这些内容展开更深入的讨论。这是本指南有意的安排,目的是尽量少地设计无关领域,将关注点聚焦在本指南的主题上,尽量避免因发散过多,讨论不充分,而产生污染。

\smalltopic{(1)关于应用范围}

读者在阅读本章内容时,尤其是介绍复杂系统时,会很容易将其套到社会现象上。我们似乎可以很轻易地就得出一大波有用的分析和洞见:社会中的组织由人编织而成,但起作用的不一定是人,组织可能只需要人的行为模式,这样就产生了异化;污染在人与人之间传播,于是形成了模因;不同人的行为可以在舆论场中相互激发,从而形成长久存在的行为链,形成回声室;这种纯粹的涌现现象,从表现上与行为模式无异,很容易被错认为是有人在背后布局;但实际上并不是这样,世界是个草台班子,大家都不知道它会变成什么样;大部分人根本就没有考虑长期问题的能力,都是乌合之众,都是巨婴;而使得大部分人的能力差成这样的社会更是万恶之源,是恶之花绽放的土壤......

以上所有这些,都是不负责任的暴论。本指南介绍的理论,不足以容纳全部的论述过程,也无法判断这些内容的正误和适用范围。任何从本指南出发的推导,思路中必然包含大量不客观内容。

“期望给事情一个简单的归因”是一个严重外溢的行为,必须警惕它的影响。社会是一个复杂系统,它的复杂程度远超人的意识现象,必须谨慎对待。读者可以衡量一下本指南介绍“理性的分析框架”所用的篇幅,以及将该分析框架熟练运用在自己身上的难度。这有助于破除“似乎掌握了社会运行的本质”的错误认知。在充分掌握理性之前,几乎不可能做到谨慎地采用科学方法从微观视角客观研究社会运行的具体规律,任何使用宏观概念的概括都会劣化成大话空话,任何自己得出的结论都会不可避免地带有偏见。

本指南所介绍的理论,仅用于分析每个具体的人的意识现象,不涉及更复杂的内容。这已经够难了。

\smalltopic{(2)关于其它理论}

阅读量比较多的读者可能会觉得本指南的论述在某些地方见过。我猜测这样的既视感应该集中于哲学和心理学领域,还有少量可能属于文学、社会学等领域。这种现象很正常,本指南没有介绍什么新东西。但如果将本指南中的内容按其它理论理解,可能会有害。

我们在评价理论的有效性时,经常会用到“现实情况符合理论预测”与“理论可以推出现实情况”两种表达。这样的表达同样可以用在理论上,但语感会有所区别。“理论A符合理论B”和“理论A推出理论B”是一个意思,它们都指“理论A的特点满足理论B的基本假设”,但语感上会借用之前表达中的“(普遍的)理论比(一个)现实情况更本质”的感觉,认为前者中理论B更本质,后者中理论A更本质。

给理论之间排“本质性”非常危险,本质是个宏观概念,同时还是个价值判断,不存在一个稳定,能经受住所有质疑的“本质”定义。理解一套理论仅能从其自身出发(如果自身没能严密到建立完整框架,那再另做处理,可以根据具体情况选择忽视或者是要求作者重新修补,不在当前话题讨论范围之中),不应使用别的理论的概念来代替对其本身的理解。这种行为会带来污染。

本指南仅介绍了一套适用范围较广的概念体系,它可以和其它领域的观点和论述相容,但不意味着本指南要更“本质”。读者可能看到概念部分的论述,感觉像是在讲结构主义或现象学;看到关于复杂系统的实事求是处理方法,感觉像是在讲辩证唯物论;看到价值判断的讨论,感觉像是在讲虚无主义;看到人的基本模型,感觉像是在讲精神分析与自由意志;或者看到一些别的地方,又有其它的既视感。如果要问“这么看对不对”的问题,答案是“这些东西确实算是这些领域的话题,相应的书籍中都有详细论述”;但如果要问“这么看有什么用”的问题,那大概没什么用。无法从这样的判断中,找出正确的废话以外的东西。你无法从中获得确切的知识,只能得到“我好像又学到和发现了东西”的幻觉。

% 本章所介绍的模型,乃至本指南所介绍的理论,是关于“因关联性而演化的复杂系统”的理论。它和具体生理现象无关,仅讨论意识现象本身\footnote{对于\rigorous,这里指的是“我们仅需要验证‘生理现象能涌现出关联性’即可”}。

\vspace{10pt}

对于那些真正深入理解这些话题的读者来说,本指南中所提到的这一点篇幅,又会显得过于简化,精华尽失。这是为了可读性而做的妥协。本指南是一本应用性的书籍,需要在保证完整介绍分析框架的前提下,尽量减少提及不必要的内容。完整阅读这一分析框架已经很困难,那些过于细节的论述会进一步增大读者的理解压力。我们选择将这些领域的观点一并作为本指南的出发点介绍给读者。

如果读者对其他领域有所研究,本指南欢迎读者在此基础上理解并评价。但请读者尽量避免概念的跨领域应用,尽量避免使用自身无法掌握的概念和结论。这会造成本可避免的麻烦。

\section{实操:情绪、感受与其分析与处理方法\label{sec:情绪、感受与其分析与处理方法}}
\hfill\begin{minipage}{0.7\textwidth}
\fontsize{8pt}{12pt}\selectfont\fontsize{8pt}{12pt}

\raggedright 把我的心都切碎了标上价码,卖掉它就能得到幸福吗?\footnote{原文为“心ってモノを量って切って、売れば幸せですか?”。\\\indent \bilibili{BV1r4411A77J}。}

\raggedleft 田中姬铃木雏《ヒバリ》(姬雏鸟)

\raggedright 大人说笑一笑解千愁,可是为什么,为什么,心中有火?
是不是我还太小,境界还不够?\footnote{\bilibili{av6107357}\another\netease{1377097873}。}

\raggedleft 绛舞乱丸《寿星街小结巴》

\raggedright 当我恻隐于哀恸,就听到心魔在教唆。\footnote{\bilibili{BV1yD4y117jU}。\\}

\raggedleft lemon夹子《行刑者赞歌与诡丽的戴罪人》

\end{minipage}

感情是一种我们极易感知到,但是又极难讨论和理解的意识现象。常见的几种定义都会有很多问题:
% \begin{explain}
\begin{itemize}
\item 我们可以从生理现象出发定义感情,比如说“哭泣了就是悲伤”“分泌肾上腺素就是恐惧/兴奋”。这样做最明显的问题就是“这么定义没多大作用”。生理现象一方面不直接能够代替心理现象(它很多时候作为表面特点而出现),一方面指标较为模糊,日常能够关注并分析的信号较少,难以刻画细致、微妙的感情。
\item 我们可以从“模因”出发来定义感情,这种处理方式的合理性主要来自“感情总是作为词语被我们认知,然后我们会把自己往上套,觉得自己会悲伤就真的会悲伤,觉得自己会愤怒就真的会愤怒,形成了自我实现的预言”。这样做的主要问题在于无法涵盖本人无法识别的情感(或者至少是周围人都没有识别出来的情感)。
\item 我们可以从“某种特定的心理机制”出发定义感情,比如说“恐惧和愤怒都是因为自身受到了外界的侵犯”“一切问题都能从潜意识中找到答案”之类。这样做虽然比从生理出发要更好观测,也可以直接用于分析一些简单的情况,但仍然不够精细。我们很难用这种方式定义每一种情感,通常只能定义几种主要的,然后声称“自己已经讨论了所有本质的情感,其它的都是变体”。但感情现象的关键正在于“感情怎么变”上。
\item 我们可以从“某种类型的具体经历”出发来定义感情,比如说带有正面情感色彩的“人生阅历”“境界”,或是带有负面感情色彩的“创伤”“原生家庭”。我们确实可以任意详细任意真实地某一个特定的人身上特定的一种情感,但这样做最大的问题在于我们讨论得不全面。在我们明白了某一个情感的生成原理后,就容易把故事套到其它人和其它情感身上,从而产生误导。
\end{itemize}
这些定义每个都有可取之处,每个都能解释一部分问题,甚至每个都真的能被人用来分析所有的感情现象。但这样得出的结论主要应该归功于个人的洞察力,仅靠定义本身出发就只能是盲人摸象。

我们能从中确定的是,感情是一类歧义很重的概念。每一种感情都能用于指代很多种不同的东西,而我们又能比较顺畅地在每个具体情境中理解具体的感情指什么。这看起来就像是“我们在根据关联性来理解和应用感情”。于是,我们不妨将关联和外溢的理论套用到感情上,看看可以分析出什么来。

\divider

\noindent 我们用以下的一个事例\footnote{这一融合了校园霸凌、家暴、ptsd的事例不代表现实情况。我们提高了冲突的激烈度,以便可以在更短的篇幅内包含所有要点。}来展示这一思路:
\begin{explain}
朱自清被选入语文课本的《背影》中有“父亲为作者翻过火车站台买橘子”的桥段。A自从初中学到这篇课文后,相关的骚扰\footnote{此处重点在于整体的流程,关于霸凌的具体分析见后文。我们这里采用的假设是“本来就有人在霸凌A,并且这些人学到《背影》以后就开始用橘子了”。}就充满了整个中学生涯。这样的事件深刻地参与了A的人格构成,使得A自此以后对橘子应激。

多年以后,A成家并生下了B。在B的朋友看来,B有个小毛病:不吃橘子。B自己也不知道为什么,但总之自己很讨厌橘子的外形和气味,碰到就赶紧离得远远的。B学到《背影》的时候,始终无法共情朱自清。
\end{explain}
\trained 可以一眼看出,这是一个“创伤在代际之间的传播”的小例子。B对橘子莫名其妙的讨厌,绝对是因为A在B小的时候的做法(无论是自己对橘子应激还是限制B都能产生同样的效果)导致的。接下来我们就需要做两方面,一共四件事情:1、此时我们应该怎么得知整体的真相?2、B应该如何解决问题?3、我们应该怎么理解这类现象?4、我们应该怎么处理这类现象?

一些方法只做表面功夫,比如说“直接将B的情况命名为‘柑橘恐惧症’,并且开始脱敏疗法”。这么做只关注问题2,无视了其它三件。而且,这么做对问题2也不一定有效。我们暂且把它们放在一边。

正经的心理治疗流派此时都会先试着开始找原因。考虑到这只是一个简单的情景,所有流派应该都能在短时间内让B发现“自己的讨厌来自童年创伤”。对于一些流派来说,只要得到了这个信息,就可以直接让B打开心结,尝试自由地吃橘子(当然,如果操作不当,可能会出现“B从此更加怨恨原生家庭(同时也仍然讨厌橘子)”等失败情况,相关讨论见下文)。到此为止,我们通过问题1的部分答案,就解决了问题2。

如果B仍然没法完全想通的话,我们还能在这个基础上做得更好一些。如果一切顺利,我们可以见到“B回家和A谈心,A回忆起了往事,(可能在一大堆跑偏之后)聊到了自己被霸凌的经历,从而家庭关系得到了修复、滋润和建设”的情节。这种成功经验自然是值得借鉴的,于是我们开始自然地尝试理解“为什么会成功”\footnote{此处可以视为对这个新的情节询问问题3。}。我们可以很容易地总结出一些B做对了的地方,比如说“使用沟通代替冲突”“使用包容代替对抗”“善于倾听从而共情”之类。但这么做会有两个问题:

其一,这些操作不保证成功。当B去和A沟通时,实际还可能出现另一些失败的情况,包括但不限于“A无法正常沟通(可能是语言或肢体上的暴力行为,可能是打断或回避)”、“A只记得霸凌相关内容,而不记得最初的源头是课文”等。我们不能保证“B一定有能力处理这些事情”,因为在这个事例中,A所具有的创伤要远比B所具有的创伤更复杂,需要更强的专业性才能应对。让B直接去和A沟通,性质上和“不自量力的见义勇为”差不多,很容易造成“除了让B受到伤害以外没有任何其它影响”的后果。“包容、理解、倾听、沟通、共情”的方法论中不包含“处理更复杂的创伤”的内容,B如果只学到了这些就很容易纯粹拖累自己。

其二,这些操作很难使用理论来概括。以上的一整套流程如果概括来说,都可以称为“觉知潜意识”。细分来说,有两次觉知:一次是“B发现自己讨厌橘子的原因”,一次是“A发现自己讨厌橘子的原因”。对于B来说,我们可以很确凿地下“只要觉知了潜意识,心理问题就可以解决”的结论;但是对于A则不行。A对橘子的厌恶只是“对霸凌的厌恶”的一小部分,而霸凌创伤远不是一次对话就可以解决的(如果产生了这样的事情,只能说明A本身已经想通/不在意了,当B问起来的时候顺便说一下。这种情况下,可能还同时有“A给B道歉”等环节)。如果A没想通,B去问的时候,A就会觉知出一大堆散乱不成体系的潜意识,无助于问题的解决。

即使我们不管A,只看B,仍然会有一些问题,比如说:我们的理论貌似过于丰富了。上面已经讨论了问题2的三种有效解决方案:通过脱敏治疗、通过觉知潜意识疗愈、通过亲密关系疗愈。而只要我们想,还可以在这个案例上给出至少十多种思路和操作都有所不同的解决方法,从“阉割”到“改变核心信念”一应俱全。

综合来看,这些方法总是会在一些方面上惊人地有效,而在另一些方面(据介绍别的方法的书的描述)毫无效果。那么这是为什么?它们为什么会成功,又为什么会失败?

\divider

首先我们可以注意到一个显然事实:所有的成功都有某种改变,而“解决原有问题”是这种改变的结果。改变可能是直接的脱敏,也可能是通过某些信息让B放下。但同时也不是所有改变都能带来成功,比如“B从此开始怨恨原生家庭”这种改变就不会起到什么作用。于是我们需要讨论两个问题:“怎么样才能有改变”和“什么样的改变才能有用”。

后一个问题有一个非常简单,完全从结果出发的回答:只要能解开B的心结,那么就是有用的。这句话看上去像是废话,它主要的作用是提醒我们“该关心具体情况”了。比如说,当B发现“讨厌橘子是童年创伤”以后,会有两种可能:一种是“发现这也没什么大不了,于是开始吃橘子”;另一种是“觉得原生家庭果然太黑暗了,于是更加讨厌橘子”。这直接说明“通过觉知潜意识来解决问题,不保证成功”。其它的任何一种方法也都能构造类似的例子以说明“它们不是通用方法”。

一些读者可能会认为“这是在钻牛角尖,极端个例不能否认整体的有效性”。这个观点本身过于概括,结论也过于草率。我们不需要借助“这种方法整体是有效的”的经验性结论,而是能够很轻松地说出“前一种情况为什么成功,后一种情况为什么失败”:因为前一种情况下,B只关心“讨厌橘子的原因”本身,而后一种情况下,B还关心“原生家庭”,注意力被转移走了。

“B的注意力从‘讨厌橘子’转移到‘讨厌橘子的原因’”,和“B的注意力从‘讨厌橘子的原因’转移到‘原生家庭’”,是两个性质相同的事件,只有感情色彩的区别。如果我们只关注“有用的前一半为什么有用”,那么就会很轻松地发现很多有效的方法,然后因为各种奇奇怪怪的原因,换个人就不灵了。

再将“转移注意力”抽象一些,我们就得到了“B在讨厌橘子时,激发了某个新的行为,这个新的行为覆盖了讨厌橘子的感受”的叙述。这一论述可以普遍地套用在每一个有作用的方法上。它不负责解释现象,只负责描述现象。
\begin{explain}
至于到底什么样的行为才可以让B覆盖讨厌橘子的感受,这是因人而异的:
\begin{itemize}
\item B有可能在某个C(通常是B很亲密的人)的影响下,潜移默化地开始吃橘子。最开始可能只是“C让B帮忙剥橘子”之类的,后来可能“打打闹闹就吃进去了”,然后彻底放开。或者也有其它很多过程不同的情节,喜欢看情景喜剧或是小甜文的读者应该可以举出很多例子。C可能有意帮B克服,也可能只是性格开朗,但这个相对不重要。
\item B可能记住了“有一个爱吃橘子的C”,但是并没有潜移默化地开始吃橘子。直到某个和C强相关的时刻(比如说分开很久了想C,或者见到橘子了想到C),然后或是在激烈的思想斗争下,或是在浓烈的情绪下,开始吃橘子。这种情况下C不一定是真实人物,也可能是文艺作品中的角色,但这个相对不重要。
\item B可能受到了某些观念的影响,决定要克服自己的缺点;或是因为有一些目标要完成(比如说在果园工作、成为吃播、演戏或其它)而有接触橘子的必要,从而在坚定的勇气和顽强的毅力下不再讨厌橘子。这种成长小故事会多少被人夸赞,从而夸赞代替了讨厌,成为新的感受。
\end{itemize}
\end{explain}
以上这些小剧情想列多少就列多少,但它们在其它理论下其实不好叙述,从而也就不太能当例子出现,只能作为心灵鸡汤喝一喝。它们在文艺作品中出现的频率要大大高于在心理书籍中的频率,只有现象而没有理论分析。而如果我们问“都有什么样的行为能覆盖讨厌橘子的感受”,就天然地需要考虑所有的情况,这样我们至少能够在理论上描述这些情况,不至于一上来就完全漏掉。

如上面所说,同样是基于“可能会被覆盖”,以上这些情况都也有对应的失败可能。比如“B有可能因为C的持续骚扰而讨厌C并产生冲突”、“B可能因为C喜欢橘子而直接远离C”、“B可能讨厌宏观叙事和规训”等。如果想细分,我们可以将其分为“完全没意识到”“因为反感而不去接触”“了解并且厌恶”等多种情况,但这在我们接下来的讨论中不起关键作用,我们仍然将所有情况都统称为“覆盖”(它们也确实都符合本书中\note{覆盖}的定义,区别只在于参与的竞争过程不同)。实际上,对于任何一种行为,我们都可以找到能覆盖它的其它行为\footnote{对于\trained,关联与覆盖的机制本身是图灵完备的。于是根据Rice定理,我们不可能得到它的任何非平凡性质。}。这使得任何一种方法都不能保证在所有情况下都奏效。

于是,我们对于“方法是否有效”的判断,变成了“该方法是否会被覆盖”。如果从抽象分析的角度来说,这和“方法有效的时候方法有效”没什么差别,但具体到人身上就很有价值了。最为明显的是,这提供了一个明确的判据,让我们得以识别“目前的方法无效,应该换方法了”。新方法可以是“解决那个会覆盖原方法的问题”,也可以是“完全另起炉灶”,它们都有可能生效。但我们需要注意,新方法是针对这一覆盖的对策,不应将其和原方法视为“同一方法的两个步骤”,或者是“原方法的深化”之类的东西。如果对另一个人一上来就使用新方法,反而有可能出另外的问题。

比如说,我们如果就此认为“所有的话聊全是为了制造出一个能覆盖的新方法”,并且据此设计出“构筑幻想以解决问题”的方法论,那也是不符合实际情况的。我们上面讨论的只是“因为某些因素而形成了行为后”的处理方法,而没有涉及“形成行为前”的处理方法。我们对后者的归因在很多时候被视作“复盘”,主要聚焦于“为什么这一系列行为会获得不如人意的结果”。这一系列行为不一定都属于潜意识,甚至可能每个都是决策,都有充足的理由。但因为没有全局视野,不会整体规划,才导致事情最终走向出人意料的不利方向。此时归因就真的能起到“在原因处更改决策,避免后续行为”的效果,而不是“在发现结果以后,通过新的思路避免后续影响”。

% 采用这种视角,还需要回答另一个问题:为什么某些方法(比如说觉知潜意识)普遍地对很多人有效。一个比较不负责任的回答是“因为他们在某个方面很相似”。这里,有一些分析方式会将相似视作人的某些普遍特征,比如说“俄狄浦斯情节”“投射性认同”“白骑士人格”之类的,归因于一大堆心理学名词;而另一些分析方式则会将相似视作相同的外部环境造成的影响,比如说“意识形态”“规训与惩罚”“模因污染”之类的,归因于一大堆社会学名词。但如上文的论述,这种单一归因很草率,每种情况都能构造出正例和反例。我们只能用“因为它不容易被覆盖”来回答这个问题,而这基本相当于什么都没说。

\divider

让我们把注意力放回到问题1上。当我们在溯因的时候,总是需要考虑这样一个问题:我们怎么才能确定我们掌握了真实的原因?

想要回答这个问题,其实相当困难。我们不总是有“判断这个思路是对的”的方法,也不总是有“判断这个思路是错的”的方法。在“回忆过去”这一方面,这两项都不是很稳定。如果有某个特别的记忆点和我们的分析一致,那么就可以确定分析对了;如果不一致,那么就可以确定分析错了。而如果已经啥都记不起来了,那就无法确认。

此时,我们的判断方法就变成了“这看起来对不对劲”。具体每个人的“对不对劲”都有区别:有些人注重“这是否合乎逻辑”;有些人依照某些通用的理论来刻画自己的行为;有些人给别人讲故事,以别人的标准来选择性地描述情节。这种情况下,即使我们问“是什么”,也只会得到“信什么”的回答。

这会为找寻原因带来相当多的隐患。比如说,A被霸凌的过程可能有多种不同的情况:
\begin{explain}
以下我们将霸凌者称为D。我们此处不在意D是一个人还是一群人,毕竟我们只关心D的霸凌行为,不关注其它方面。一个人或一群人对以下的分类不会产生什么影响。
\begin{itemize}
\item A可能自己发现了这个伦理桥段。D在注意到了A的某些表现(比如说“A自己往外说”、“A听到就一激灵”等)后,就开始专门根据这一点来刺激A。
\item D可能自己发现了这个伦理桥段,并且在A的面前有意模仿以刺激A。随着时间,D可能逐渐获得了一些新的主意。
\item D可能在发现以后,直接想到了一些新的主意,并且对A单方面取乐(比如说“硬把橘子塞A嘴里”(为了符合人设,读者可以认为橘子是烂的)之类的操作)。此时A不一定清楚D的动机。
\end{itemize}
\end{explain}
这样的分类有其现实意义。如果干预得够早,对于第一种情况来说,“让A收敛一些,看开一些”的方法可行,确实可以预防之后的霸凌行为(因其它原因而产生的霸凌行为另算);而对于后两种情况,则无法仅要求A是无效的,此时不可以使用“一个巴掌拍不响”的论述来转移责任。

如果A记得够清楚,并且情况够清晰,那么对前两种情况的反应大概会是“对没错就是这样”,对第三种情况的反应则会是“我就说D为什么会干这种事”。但这些情况本身可以复合,比如说最开始是情况1,D发现了以后无异于情况2,又发展成无异于情况3。A可能也自己想过“到底是为什么”,从而实际是情况3,也会被A归因成情况1。在混乱的记忆下试图归因,就不保证能得到正确的结果了。

如果我们仅关注问题1和问题2(准确来说是A在问题2的对应版本“A应该怎么处理自己的心理阴影”),那么归因的错误不会产生影响。任何一种A信服的解释都可以作为疗愈的素材。但对于理论分析来说,这还远远不够。我们必须要搞清楚,出现这种“无法溯因”的状况的原理是什么。

以上三种情况中,有一类值得注意的现象:\indicate{同一个认知演化的过程,可以在不同的人身上发生。}\footnote{此处的“不同的人”也可以代指“同一个人身上的不同行为模式”。我们需要微调后续的论述以符合这种情况,但同一个人的不同行为模式之间也可以产生下述的跳跃性演化现象。}比如说,“从《背影》中的情节联想到伦理桥段”可以是A想出来的,也可能是D想出来的;而“从叫爸爸到塞橘子”可能是D在霸凌A时想到的,也可能是D独自想到的,甚至有可能是A自己想到的(比如说看到橘子就犯怵,然后又被D发现)。如果我们只看认知和行为,那么以上几种情况分不出区别来,从而这也很容易形成完善的逻辑,进而变成无懈可击的分析。

但是如果我们具体到“演化发生在谁身上”上,这就会产生明显的区别。在第三种情况下,A如果完全不知道D的动机,那么他的恐惧就只能归因到“D的霸凌行为”上,绝对不能下“潜意识里,A害怕的是朱自清的橘子”的判断。我们在观察B的时候,也能看到同样的现象。B的创伤直接来自于A,而这和《背影》没有直接关系。同理,B在后续也可能会得到救赎或者更深的创伤(见前文讨论),而这些也都可以发生在A身上。

“只将某个单独的行为传递给另外的人,而不同时传递它的形成过程”的过程是可能并且相当常见的。不管前一个人中间有多少,多复杂的过程,后一个人都总是可以只学到起因和结果。至于中间的过程,可以以任何形式和触发顺序作为“这个行为相关联的其它东西”,而具体得到了多少,又破坏了多少因果性,是不可预知的。新行为的内容不再是老行为的内容,而就是老行为本身。认知和行为经过这种跳跃性的视角转换,得以跳跃性地演化。

再叠加上关联的任意性,就使得每一个通过这种方式传播的概念,都会不可避免地产生严重的歧义。不同的理解途径广泛而均匀地散布着,一个人的认知可能从任何几个碎片中拼凑出来。
\begin{explain}
\trained 可以把语言的演化当成一个例子来理解。比如,随着时间的推移,一些词语(和一些其它类型的表达)会不断产生引申义,但我们在学习语言的时候并不是从它的原始含义一路学过来,而是直接接触它当前的含义。它的演变过程可以作为趣味性或是专业性的内容,来加深理解和强化记忆,但直接接触引申义也是可行并且十分常见的操作。

再比如,当我们只知道“词语当前的义项”时,通常无法推断出什么是它的本义,又有哪些是引申义。虽然整体上可以总结出“由具体到抽象”等规律,但总是会有反例。我们总得借助一些其它的东西(比如说古时的预料,再比如说“该词语分地区演化出了不同引申义”、“如果一个东西在古时不存在,那么这个义项一定是引申义”等现象)才能判断。
\end{explain}
而此时,如果我们把“认知的某一种完整演化过程”讲给“只有部分演化过程的人”,那么这个人对该认知的看法就很容易被覆盖。这里的“讲”不一定是“由一个人给另一个人说”,也可能是“自己分析出一些结论”;“完整演化过程”也不一定就是真实发生的过程,标准可以低到“没有其它证据来反驳”(见上文)。接受这一思路的一瞬间,就形成了对自己的污染,就会觉得“这事情一直是这样的”。这使得我们无法通过这种方法区分“是一直有某种认知”还是“可以接受某种认知”。
\begin{explain}
    \begin{itemize}
        \item 这样的思路如果扩张一些,就会变成“人无法想象自己没见过的东西”之类的判断。如果我们仔细理解的话(比如允许“通过某些部分的认知造出整体,然后再去认识整体得到新认知”的过程),这句话还可以称得上是正确。
        \item 再扩张一些,就会变成“人在共情的时候都是在代入自身经历”之类的判断。如果我们强行让“自身经历”也包括“从别处听来的故事”之类的内容,那么这也能算是正确——只有能模拟才能理解,不理解就共情只是自我感动。
        \item 再扩张一些,就会变成“接受一个观点,是因为自己内心深处本来就认同”之类的观点。这就无论如何也圆不回来了,只要我们把注意力从人际交往方面放到教育方面,就能立刻发现反例。当然这仍然在很多方面有解释力和说服力,比如说“寻找自我”“别人不理解”的时候。
        \item 再扩张一些,就会变成“人会有某些行为,是因为人具有某些特质”。不管我们把特质叫“集体潜意识原型”、“天赋”、“性格类型”、“即时奖励”还是什么,试图这么给所有人共同归因,最后只能使心理学名词越来越多,“人的本质”也越来越多。
    \end{itemize}
    \noindent 以上这些,发散得越远,适用范围就越窄。发散得足够远以后,除了“可以指代一些现象”以外,已经无法以此为基础展开有效的分析了。每一步推理都会产生无穷无尽的反例,每一步动机都会在另一些人身上导致截然相反的行为。
\end{explain}

\divider

我们再回过头来,看看最初的那个问题:什么是感情?由于感情也在这样跳跃性地演化,导致我们“从内容角度对感情下通用定义”的尝试必然失败。我们不可能把所有感情都还原到生理反应上,因为确实有很多感情完全不涉及生理反应\footnote{对于\rigorous,这里省略了“将想法排除出生理反应”的讨论。这样的措辞应该不至于引起歧义,请读者自行补充。};我们也不可能把所有感情都还原到思想上,因为任何一种思路都可以被截断,最终只形成跳跃性的条件反射。

想要完全不出错,那只有一种合理的定义方法:把感情完全定义为主观认知,即“感情是所有‘会被人称为感情’的现象的总称”。这个定义天然地蕴含了感情的主观性,不同的人用同一种感情来指代不同的东西变得相当合理。它们之间并非完全不可交流,无论是经历、举动,甚至是词语本身,都可能触发某些人的行为和思路。我们甚至可以见到“看见悲伤这个字眼就会流泪”“见到有人描述空虚就会抗拒”“别人问‘这能忍得住’就冲上去了”之类的情况。这些情况既可能是完全浮于表面的为赋新词强说愁,也可能是深切厚重的回忆和创伤。在大多数情况下,我们就这么互动,也这么共鸣。

在一些情况下,我们其实不需要去理解感情。无论是没时间深入交流,还是有其它的事不需要顾及感情,还是因为各种原因交流无法进行下去,还是之前已经沟通过了,都不会涉及“理解感情”的部分。我们此时对于感情的思考、讨论,重点集中在“会有什么影响”上。避免触及他人的霉头可以更和谐,识别自己的心不在焉可以及时调整,这同样很有作用。此时因为有了明确的指代,我们在结合现实情况的同时思考,所以相对来说更靠谱一些;但是如果把“其他人情绪的影响”直接套用到当前事件上,就仍然有可能出问题。我们仍然需要从实际出发来分析和判断。

但是这样的交流并不能为我们带来理解。互动不是理解,共鸣也不是理解。想要确保理解,那就只能\indicate{从一开始就考虑所有情况,并且一直讨论所有情况,直到完全讨论清楚。}处理某一个具体的情况可能很简单,但处理任意一种不定的情况却相当困难。在此过程中要排除任何先入为主的思路,排除任何分析带来的污染,避免真相被淹没在合理的分析或者是真切的共情之中。

当然,我们也不是每次都要从这么根本的视角出发。在绝大多数情况下,我们分析“某个具体的感受是从哪来的”的时候,都不需要完整地走一遍流程。在具体分析时,上述论述的主要作用是“提醒我们不要过早地下结论”。我们应该首先尽可能全面地收集所有相关的信息(包括但不限于知识和记忆),并且尽量只从事实出发开始分析。必要的时候可以提出猜测并验证,但不要抱有任何一种偏向,以免对不符合推测的信息视而不见。只有这样,我们才能尽可能准确地逼近真相。
