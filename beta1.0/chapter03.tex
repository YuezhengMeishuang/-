\begin{savequote}[450pt]
    承认吧,我们什么都不知道。如今才明白孤独并非孤独的全貌。我们所仰仗的那株脆弱的稻草,甚至比不上作家笔下的火柴牢靠。\footnote{\bilibili{av42228720};\netease{1343548780}。}
    \qauthor{DELA\_P\&雨狸《我对孤独一无所知》}
    正常来说,概念这种东西是没办法直接影响周围环境,伤害敌人的,不是说写下“大质量、大引力、高温、高热、聚变”这些词语就可以制造类似的效果。
    \qauthor{爱潜水的乌贼《诡秘之主》第八卷\ 第三十一章\ 概念化}
    你说人一样一样都没区别,在一堆一样一样自成一类。一样的世界,一样的感觉,一样地生搬硬套成了生活的情节。\footnote{\bilibili{av4894016}。}
    \qauthor{小旭PRO\&绛舞乱丸《一样一样》}
\end{savequote}
\chapter{复杂现象与复杂概念的组成方式}
%\titleformat{\section}{\bfseries}{}{0em}{}

可能有一部分读者看完第二章以后信心满满,觉得自己已经掌握了解决心理问题的方法,可以立即让自己挣脱污染,走向光明、希望、解脱和自由。但很可惜,你高兴得太早了。事情哪有那么简单。

如果读者读过其它一些心理健康书籍,就会发现,前两节的内容和其它心理健康书籍所谈论的话题相差不大,本指南介绍的方法其它书里面也有。相比之下,其他书可能还写得更好一些,语言更亲切,有更丰富的例子来辅助理解,而本指南堆了一大堆理论辨析,十分让人头疼。但即使已经有这么多书了,而且有些人也看了不少书,收获却有限。

我们会被某些东西刺痛,我们会被某一句话打动,但是过不了一周,生活就又回到了原有的轨道中。到底欠缺了什么东西?为什么明明什么都懂,却还是无法面对?为什么明明听过很多道理,却依然过不好这一生?知识的积累和内化到底哪里有差距?本章所要做的,就是尝试回答这些问题。

要一句话回答这些问题也可以,那就是“事情没你想得那么简单,现实世界比你的理解更\indicate{复杂}”。但这一句话无法加深你对现实世界的理解,无法指导你解决实际事务。我们还需要更详细的讨论,用一整套分析框架来面对这个问题。
\begin{explain}
如果只是粗略阅读,那么读者可能会觉得本章所讨论的内容像哲学。有这种既视感是正常的,哲学家也会讨论同一类型的话题。但是,本章的思路和哲学无关,涉及这些内容完全出于实用性。

本章内容不需要哲学基础。无论读者的哲学基础如何,无论读者是觉得本章内容过难还是过易,都请读者尽量避免过多的哲学联想。如果确实有明显的阅读障碍(无论是看不懂还是无聊),可以先阅读本章总结,或者是本章后的整体梳理,再根据需要来回头阅读必要的部分。
\end{explain}

\section{概念与特点\label{sec:概念与特点}}
\begin{explain}
再次提醒:本节内容不是哲学,不要以“玄而又玄”的视角来看待。本节内容所涉及范围不超过中学语文课。
\end{explain}
\subsection{指代}
\begin{explain}
本小节关于笔的话题来源于知名meme《什么是笔》。
\end{explain}
“笔”在目前通行的语言环境中,主要指\indicate{用于书写绘画的工具}。
\begin{explain}
“笔”的原意是“书写”,但这一义项在后续演化中已基本消失。当今“笔”的“书写”义项多为引申义。本小节不做深入的语言学研究,仅略微提及。

别的语言环境下可能有所不同,如英语中没有只作为总称使用的词汇(pen还有额外的“钢笔”义项)。本小节仅以汉语为例。
\end{explain}
如果我们仅使用定义来看待“笔”,那么很多问题,如“这支笔你是在哪买的”,是无法回答的。我们在日常生活中,从来不会真觉得这个问题无法回答,是因为此时有一次\indicate{指代}。和“购买”有关的信息来源于“这支”:它锁定了交谈者附近(可能是身边,可能是手指的地方,可能其它)唯一具有“笔”的特征的东西。
\begin{explain}
\indicate{指代}必须结合环境。脱离环境的“这支笔你是在哪买的”问句无法回答。在没有语境的情况下,听到有人突然问了这么一句,多半会反问一句“什么?”。

而基于概念本身的提问,如“笔是干什么的”,则无需结合环境即可回答。
\end{explain}
我们将\indicate{通过部分特点锁定环境中的具体对象,以谈论它的其它特点}的行为称为\indicate{指代}。\\指代是人类在正常交流和思考时,不可缺失的一项重要能力。
\begin{explain}
我们永远都无法完全描述一件物体。一支笔处于什么位置,是什么材质、形状、颜色,为什么能书写,属于谁,现在还能不能用......只要想关心,总会有无穷无尽的信息可以获取。

但我们永远也不需要完全描述一件物体。通常只需要很模糊的信息(如“在你附近”“蓝色的”等),我们就可以完全锁定具体在指哪只笔。

仅靠部分信息,就完全锁定交谈对象,会不可避免地使用到指代。
\end{explain}
在一些情况下,\indicate{指代}可以不用完全符合定义给出的特点,词语于是有了引申义。
\begin{examples}
在使用“笔”的语境中,有些只在意笔的“记录信息”功能,如软件中的画笔工具;有些词组引申出了“书写”行为,如“执笔”;有些不在意功能,只关注笔的外形或其它方面,如“触屏笔”、“录音笔”、“坏掉的笔”等。
\end{examples}
指代经常导致词语含义的演变。
\begin{explain}
使用“书写”来指代“书写工具”,指代多了,就会使笔的含义也变为“书写工具”;在某个较为独立的环境,如具体学科、家庭、学校、公司中,也经常会有各自的“黑话”。

本指南不过多探讨词语在社会中的整体语义流变,仅关注在具体小环境下的变化。
\end{explain}

\subsection{概念}
我们所使用的每一个\indicate{概念}都具有如下特征:概念中提供了一组\indicate{特点},我们通过判断某个具体对象是否符合这些\indicate{特点},来判断是否可以用这个概念来指代它。我们将符合定义的具体对象称为这个概念的一个\indicate{实例},将用于判断的\indicate{一组特点}称为这个概念的\indicate{定义}、\indicate{含义}或\indicate{判据},将判断过程称为\indicate{识别}或\indicate{锁定},将指代也称为\indicate{套用}、\indicate{概括}或\indicate{称呼}。

如果一个词语有多种定义,那么将每种定义称为它的一个\indicate{义项},此时称这个词语有\indicate{歧义}。
\begin{explain}
此定义的\indicate{概念}的定义较为宽松,只要是一个实词,都可以算做概念。甚至名称(针对某一特定个体的概念)也算。

对于有一定语言学或哲学基础的读者,此处的“概念”有能指和所指的混用,笼统地描述整个意指作用;“字”“词语”是能指;“特点”“含义”“定义”等表述是所指。本指南不深入讨论这部分内容,做了相应简化。

使用统一的概念可以大幅度缩减思考和沟通的成本,可以将原本使用数句/数段的描述简化为少数几个字。使用概念是人类在正常交流和思考时,不可缺失的一项重要能力。

本指南使用“特点”这一概念或其实例时,默认其没有歧义。行文时会尽量避免使用事实上有歧义的特点实例。
\end{explain}
%\begin{examples}
%大多数情况下,我们不需要很精确地知道自己的判断依据。我们在确定“面前的人是谁”的时候,绝对不会在脑子里清晰地把自己的理由都过一遍。很多判断(如通过外貌识别出眼前的人是谁)是完全不可控的行为。
%\end{examples}
在特定环境下,对于同一个概念,可能有好几种不同的判据都会识别出同样的对象。此时我们经常通过最方便的那一种来识别。
\begin{examples}
严格来说,名字用于指代一个具体的人,而不是某种特定的外貌和行为。但没有谁是通过“每时每刻观察一个人,确定这个人一直稳定存在”才能确定这个人的名字。靠外貌和行为通常就能锁定一个人。

靠外貌和行为的判断方式在一些情况下会失效,如双胞胎,或者恰巧外貌和行为相似;很久不见一个人后,外貌和行为都有可能发生大变化。

以上的误判可以通过“一个人亲口说出自己的名字”等方式来解决。这种判据也会在一些情况下(如神志不清时)失效,此时可以使用一些其它方法(如检查身份证件、查验DNA)确认。

我们极少通过“一个稳定存在的人”的原始定义来识别这个人。这多数情况下既无实践可能也无使用价值,只在少数罕见情况下(如有替身,需要持续监视)有必要使用。
\end{examples}
我们不一定明确知道自己使用的概念是什么定义。这有两种情况:一种是有确定的判据,但自己不清楚;一种是多种判据混用。我们将后者视为\indicate{指代},而前者不视为指代。
\begin{explain}
如果不清楚定义,我们在使用同一个词语时,就有可能在根据它的不同定义分别谈论。这些谈论有可能偷换概念,错误地关联起一些本来无关的事情,得到的结论也有可能无效。

可以通过增加具体描述来避免义项混用,将讨论的概念限制在“xxx的外貌”“xxx的(某个具体)行为”等具体特征上。
\end{explain}

\subsection{能力边界\label{sec:能力边界}}
定义只判断事物是否符合对应的特点,而不涉及这些特点的来源。使用概念来概括实例,就仅能分析它的一部分特点,遗漏了。这会带来不可避免的\indicate{信息损失}。\\
\indicate{存在信息损失时,分析是单向的,不可能由果推因}。
\begin{examples}
这里的“果”指“概括得出的特点”,“因”指“对应的实例的其它特点”。熟悉逻辑学的读者可以看出,这里指的是“若A是B的充分条件,那么可以由A推B,但不能由B推A”。

笔的特点仅有“能书写”。不同的笔书写原理不同,铅笔、蜡笔等使用摩擦留下痕迹,其它笔多使用墨水或颜料,而具体的方式也有不同。这不影响它们统一被称为“笔”。在通用含义下,“笔为什么能书写”这一问题等价于“能书写的工具为什么能书写”,我们不可能脱离语境,给出统一的答案。
\end{examples}
每个思考回路都会使用很多概念。我们将\indicate{某次思考时使用过的所有概念的所有特点}统称为此次思考的\indicate{出发点}或\indicate{依据},并且将出发点称为此次思考和所得到结论的\indicate{适用范围}或\indicate{能力边界}\label{def:能力边界}。
\begin{explain}
我们不一定需要分别定义每一个概念。实际上,我们很多时候只关心一些概念之间的联系。这些联系也是出发点的一部分。
\end{explain}
如果我们在思考时,仅根据出发点来使用概念,而不使用指代,那就将这次思考称为\indicate{完全理论性的}、\indicate{和环境/外部无关的}或是\indicate{就事论事的}思考。相反,如果使用了额外的信息(也即使用了指代),那就将这次思考称为\indicate{和环境/外部有关的}或是\indicate{依赖直观}的思考。
\begin{examples}
如果问题是“钢笔为什么能书写”,就可以结合物理知识,使用“因为墨水会随着毛细现象渗进纸张”来回答。钢笔包含“自身结构特征”的信息,这是完全理论性的分析。

如果问题是“这支笔为什么能书写(指向一支钢笔)”,那么同样的回答对“笔”这一概念来说,就是和环境有关的的分析。如果听到这句话的人(比如小孩子)不清楚钢笔的相关知识,就有可能误用这一结论来解释其它笔的书写原理。而另一方面,这个回答对“这支笔”这一概念来说,是完全理论性的分析。

在极少数情况下,一个思考回路可以不包含识别方法,巧合地每次都就事论事地分析。除此之外的绝大多数情况下,总是在能力边界内分析的思考回路都包含对应的识别方法,都是\indicate{知识体系}。\footnotemark
\end{examples}
\footnotetext{\rigorous 可以验证此处的定义和\hyperref[para:知识体系]{2.1.1小节}中的定义一致。判断一般的行为模式是否是知识体系,仅需验证其子思考回路是否包含对应的识别方法。}
如果某次思考时,明确知晓自身的出发点,并且只做完全理论性的分析,那么就将这次思考称为\indicate{在能力边界内}思考,或是\indicate{清醒地}思考。如果一个思考回路能完全理论性地思考某个概念或进行某段分析,那么就称这个思考回路可以\indicate{容纳}这个概念/这段分析。
\begin{explain}
出发点中的一些认知可能有自己的分析过程。如果此次思考并未涉及到那些分析过程,就不应继续向前追溯;出发点中的一些认知有可能是污染,但只要此次完全理论性分析时知道自己使用了该认知,那么就仍然认为是在能力边界内。

和环境有关的结论在脱离环境(且无法模拟环境)后就会变为污染。这一过程在同一个人的不同思考回路之中也适用。前后两段思考的出发点可能不同,前一段思考所使用的概念可能无法迁移至后一段思考中,前一段思考所得到的结论可能无法与后一段思考出发点的其它部分相容。
\end{explain}
我们将\indicate{因为忽视了前提,导致分析和结论无效的现象}称为\indicate{劣化}。一段思考中使用了越多的劣化想法,就称这段思考越\indicate{劣质}。
\begin{explain}
越是不可避免地需要借助环境来思考,对自身思路的整体把控能力就越差,就越是无法容纳完整的思路,就越是需要借助环境中新的信息来思考。随着这一恶性循环,劣化的想法就越来越多。如果一个思考回路包含了足够多的劣化想法,随着不自知的偷换概念,有效的思维链条会很短,就会产生内部的矛盾。但同时,因为该思考回路对自身整体把控很差,它无法将自身整体纳入考量,只会判断“每条想法是否有问题”,然后得到“没问题”的结论。
\end{explain}

\subsection{应用}
\label{def:知识体系}一个知识体系会明确自身的出发点,使其不因具体应用而改变。我们将其称为知识体系的\indicate{基本假设}。
\begin{examples}
这些基本假设包括但不限于数学中的公理,物理中的定律、理想模型,以及一些其它学科中“对现实现象的归纳总结”等前提条件。

在19世纪,数学发生过一次重大的认识论转变,公理不再被看作“不证自明的命题”,而是被看做“某个研究领域的基本假设”。我们通过检验“数学对象是否符合公理”来判断“是否可以使用该领域的结论”。公理和对应的领域不负责阐述“符合公理的数学对象从何而来”,只从基本假设出发继续推导性质。一个明显的例子是,数学家们找到了欧几里得的很多依赖直观的漏洞,并且使用希尔伯特公理体系将其严格化(这不是双曲几何的故事)。

20世纪时,这种观点转变影响到了物理学,尤其是难以直接观测的量子力学。物理学家们不再使用“定律”一词,而是将薛定谔方程等内容称为“基本假设”。理论增添了“适用范围”的判断标准(如经典力学仅适用于宏观低速情况,中学物理还会要求“可以看做质点”等),在应用前需要判断对象是否符合要求。但这并不影响我们做完全理论性的推导。

读者可能可以看出,这里在讨论\indicate{范式}的一些特点。本指南不深入认识论和科学哲学内容,对此仅做简单介绍,以供读者参考。
\end{examples}
对某一具体领域展开研究的学科,可能因为历史发展,内部包含多个不同的知识体系。更换基本假设有三种可能的情况:发现某基本假设多余,可以由其它基本假设得到;发现了等价的基本假设,该假设在一些情况下更方便(或有其它优势);找到了更深入的基本假设,能推出原有基本假设,并且能解释一些新现象。
\begin{examples}
第三种情况涉及到研究领域的改变,是重大的研究突破。不可能只靠完全理论性的分析,就从原知识体系出发,得到更深入的基本假设。原有基本假设不包含它的信息。只有获得了新信息(或者是调研,或者是直接猜),才能得到新的基本假设。
\end{examples}通过明确基本假设,一个知识体系得以只研究某种特定层次的现象,而不用无穷无尽地向前还原,不用解释每一种现象的原因。
\begin{examples}
基本假设可能来自对现实的直接观察,也可能来源于对某些现象的深入分析和提炼,也可能就是凭空创造。无论哪种情况,都可以做纯粹理论性的分析。只要对象符合基本假设,那么就可以使用分析得出的结论。

%读者可能会注意到,本指南中对概念下的定义都很简短。对概念的其它讨论主要有以下三方面用意:向读者演示如何识别概念对应的实例;明确表述,消除可能存在的歧义;从概念出发做完全理论性的分析。如果读者能力足够,可以仅凭定义,就独立完成这些讨论。
\end{examples}
一个知识体系通常体量很大,内部包含的分析相当繁杂,通常难以穿过冗长的逻辑链条,回归基本假设。更常见的情况是\indicate{以知识体系的某些结论作为出发点,展开分析}。我们将这种思考称为\indicate{应用}这一知识体系。
\begin{examples}
一些学科的基本假设来自领域中已有的深入研究,对初学者可能过于抽象复杂,无法准确识别和正确使用。此时可以通过一些覆盖面更小,但同时更容易上手的基本假设,来辅助初学者接受这些概念。
\end{examples}

\section{复杂系统\label{sec:复杂系统}}
\subsection{编织}
我们将\indicate{多个事件相互影响,组成更大的事件}的现象称为\indicate{编织}\label{def:编织}或\indicate{编织过程},将其中的每个事件称为\indicate{具体事件}、\indicate{微观现象}或\indicate{简单现象},将组成的更大事件称为\indicate{宏观现象}或\indicate{复杂现象}。我们将每个微观现象称为宏观现象的一个\indicate{组成部分}。我们允许“微观现象编织成宏观现象”的表述。

我们也将“微观过程组成宏观过程”称为编织(注意过程是指一类具有共性的事件);也将“微观认知组成宏观认知”称为编织(注意“使用认知”可以视为事件)。
\begin{examples}
这里所定义的编织是一个非常广泛的定义。相互影响可以是各种形式的接连触发,可以并行、串行,或更复杂。可以是时间上的前后顺序,也可以是逻辑上的前后顺序。

前文中也涉及到很多可以被视为编织的概念,如行为编织成行为链,进而编织成行为模式,行为模式编织成人,以及事务编织成更复杂的事务、事件编织成更复杂的事件、行为模式编织成更复杂的行为模式。

其它学科中也有对应于编织的概念或例子,如专门的复杂性理论、系统论、统计力学、神经网络、格式塔、回声室、由个人编织成组织、由组织编织成更复杂的组织等。本指南不对这些内容展开深入讨论,仅作提及。
\end{examples}
我们将\indicate{只能从宏观现象中概括的特点}称为\indicate{宏观特点}、\indicate{表面特点}或者\indicate{表面现象}。
\begin{examples}
表面特点的定义也很宽松。它有可能是“某种稳定的状态”,比如说人的性格;也有可能是“会导致某种结果”,比如从某种思考回路中产生了某种情绪、使用了某个概念。

“\indicate{只能}从宏观现象中概括”的意思是“无法将其归因于复杂现象的某个组成部分”。会体现同一表面特点的不同复杂现象,可能包含完全不同的具体事件。我们只能从所有(主要的)微观现象出发,通过分析它们之间的相互影响,来推理出表面特点。不可能反过来,仅从表面特点出发,反向推导出复杂现象的具体组成。
\end{examples}
当我们称某个事物为表面特点时,若无特别说明,提到的复杂现象总是指产生该表面特点的复杂现象。
\begin{explain}
在同一个复杂现象中,某个特定的现象不可能既是表面特点又是微观现象。但复杂现象A的表面特点可以产生影响,作为微观现象参与复杂现象B的构成。

事实上,我们身边会遇到的绝大多数事件,要不然可以直接视为复杂现象,要不然可以通过少数几步归因,归到某个表面特点上。纯粹由简单现象组成的事件极为少见,通常只能在主动设计的介绍、教程等环节中见到。
\end{explain}

\subsection{涌现}
如果在某一类事件中,存在一些事件能够编织出宏观现象,那么就称这一宏观现象从这一类事件中\indicate{涌现}出来,称这一类事件为\indicate{低层次事件}或\indicate{深层事件},称涌现出的事件为\indicate{高层次事件}或\indicate{浅层事件}。

我们同时也称该宏观现象对应的宏观特点从这一类事件中\indicate{涌现}出来。我们将一类低层次事件和其中能涌现出的所有高层次事件和其宏观特点统称为一个\indicate{复杂系统}\label{def:复杂系统}。
\begin{explain}
能从一类事件中涌现出的宏观现象数量不定。有可能一个都没有,也有可能涌现出多个。这一类事件的特性和相互之间的触发关系决定了它们能涌现出什么。

此处关于复杂系统的定义仅涉及两个层次的事件。也可仿照该定义,定义包含更多层次的复杂系统。本指南仅关注涉及两个层次的复杂系统,故不下更广泛的定义。

对于一个宏观现象的所有组成部分来说,它们所涌现出的内容可能不止这一个宏观现象。其中的一部分事件可能可以组成别的宏观现象。

每一个行为模式都是从行为和环境中涌现出来的现象。行为也可视为“从当前所有信号中选择了一个响应”这一涌现过程的结果。
\end{explain}
与编织不同的是,涌现并不要求每一个事件均起到作用。一类事件中可能有大量和宏观现象无关的其它微观现象。
\begin{explain}
对于一个宏观特点来说,不一定其对应宏观现象的所有组成部分都起到了作用。每个宏观特点可能仅和一些组成部分有关,不同的宏观特点所依赖的组成部分可能不相同。

这使得我们在面对涌现现象时,无法从“出现了某一类事件”就断定“一定会出现相应的宏观特点”。会导致宏观现象的事件可能具有另外的特点,与这一类事件并不重合;另一方面,另外的特点可能无法被良好地总结归纳,只能具体问题具体分析。
\end{explain}
如果一个概念的定义包含表面特点,那么就称其为\indicate{大词}、\indicate{复杂概念}或\indicate{宏观概念}。
\begin{explain}
思考回路由想法编织而成。大多数思考属于复杂现象,而使用概念和获得结论都是思考的表面现象。\indicate{人所使用的绝大多数概念都是复杂概念}。

我们日常使用概念时,大多数情况下不会很严谨地完全明确自身的出发点,而是会根据某些特点来套用。实际的使用方式,会从“所有和该概念相关的认知”中涌现出来。根据当前所处思考回路的不同,我们根据的特点可能很不稳定,会给很多不同的东西冠以同一个名字,用同一种结论套在不同的东西上。

因此,任何一个复杂概念,都会不可避免地具有很重的\indicate{歧义}。如果不加以分辨就使用这些概念分析,得到的结果就会完全不可信。这些歧义有可能会让人感觉这些概念很\indicate{深奥},但“有歧义”和“确实很复杂”都会使人迷惑,都需要深入思考,一定要区分这两种情况,而不能将其笼统地看做“很有哲理,揭示了某种本质”。
\end{explain}

\subsection{对复杂现象的分析\label{sec:对复杂现象的分析}}
我们将\indicate{分析某个复杂现象时,仅将微观现象视为原因}的行为称为\indicate{微观分析}或\indicate{深入}、\indicate{客观}、\indicate{符合现实/实际}的分析。
\begin{explain}
微观分析大致可以分为两类:一类是\indicate{考察微观现象之间的因果关系},另一类是\indicate{考察编织过程}。后一种可以用于分析“表面特点如何产生”。

考察微观现象时,有可能使用这些微观现象的某些特点来指代微观现象本身。我们也将其视为有效的分析。

考察编织过程的难度一般较高,需要对复杂现象有整体把握才能得出有效结论。这在很多时候需要掌握相应的知识体系才能做到。
\end{explain}
我们将\indicate{在分析某个复杂现象时,将它的某个表面特点视为原因}的行为称为\indicate{宏观分析}、\indicate{表面分析}、或\indicate{草率}、\indicate{脱离现实/实际}的分析。
\begin{explain}
两种微观分析都会因为忽略了结论的适用条件,而劣化为宏观分析:

将考察微观现象时的结论误用于其它现象,通常是因为我们使用特点来指代微观现象,同时将微观现象的性质误认为特点的性质。这种情况可能在总结经验时,或者是举例、类比时出现。

将考察编织过程时的结论误用于其它现象,通常是因为我们过度简化了分析的流程,将表面特点简单归因于某个具体现象或者另一个表面特点,而不归因于整体的复杂现象,忽略客观存在的编织过程。

如果不能从微观现象中分析出表面特点,就不应认为自己理解了该复杂现象。不应认为宏观分析是对复杂现象的有效理解。如果我们脱离了宏观概念后就无法再对复杂现象展开讨论,那么先前使用宏观概念做出的所有分析和判断,就全都是\indicate{正确的废话}。
\end{explain}
对同一个复杂现象中的两个特点,在没有展开微观分析前,我们仅应认为这两个特点具有\indicate{相关性},而不应该认为这两个特点具有\indicate{因果性}。
\begin{explain}
这类因果性结论包括但不限于“xxx会导致xxx”、“xxx都是xxx”等绝对性的判断。

在研究简单现象时,如果我们确实控制好了其它所有变量,那么当有当两个信号总是先后出现时,就确实可以得出“二者具有因果性”的结论。

但我们在现实生活中遇到信号时,无从判断这个信号是出自简单现象还是复杂现象。如果两个先后出现的信号都出自复杂现象,都只是表面特点,那么它们只具有相关性,而不具有因果性。现实生活中的绝大部分现象都是复杂现象,如果没有确切的“当前信号出自简单现象”的证据,应该始终将其按复杂现象处理。\indicate{没有调查就没有发言权}。

如果我们总结出“出现前信号之后会出现后信号”的相关性结论,那就可以加以利用:比如在见到前信号以后,则可以提前准备,以利用后信号,或规避后信号的不利影响。但这仅限于被动地接受并处理现实,我们无法主动改变这一切。

如果我们总结出“出现前信号会导致出现后信号”的因果性结论,并且试图通过控制前信号来控制后信号,那就会出问题。由于前后信号没有因果关系,现实不会按照预期来进展。这种错误的归因已经是污染,而为失败另找理由还会产生另外的污染。
\end{explain}

\subsection{动机与表面归因}
行为模式是由行为编织而成的,行为模式的一次触发在绝大多数情况下都是复杂现象。因此,\indicate{人的绝大多数行为,及其所产生的影响,是表面现象}。
\begin{examples}
如前所说,行为本身可以作为微观现象,参与其它复杂现象(如另一行为模式)的构成。这不影响“行为本身可以是其它(更偏向无意识的)行为模式的表面现象”的结论。
\end{examples}
如果我们观察到了一个人的举动(可能是行为/行为模式),同时观察到了与该举动的某个影响,就有可能将影响视为这个人的目的。我们将做出该举动的\indicate{真实原因}称为该举动的\indicate{动机},并将\indicate{分析出的原因}称为该举动在\indicate{此次分析得到的动机}。这种分析在绝大多数情况下是表面分析。我们将此\indicate{通过表面分析得到动机}称为\indicate{表面归因}。
\begin{explain}
这种分析方式的出发点是“人会依照自己的目的行事”。但这种草率的归因是完全不可信的,有以下几方面问题(以下将做出举动的人简称为对方):
\begin{itemize}
\item 一件事的影响有很多方面,每一方面都有可能是目的。如果没有得到更深入的信息,就不应该草率地根据部分信息得出结论并相信。
\item 对方可能有自己的目的,但自身的分析有问题,从而导致举动不能实现目的。此时的目的不是任何一种影响,不可能通过这种方法分析出来。
\item 对方可能没有目的,而只是感受到了某些信号,并且条件反射式地做出了举动(同样地,在得到更多的信息之前,也不应草率地认为就是条件反射)。
\item 对于行为模式层次的举动,有可能对方在具体的行为上有明确的目的(或是感受到了信号),但是没有整体的行为模式层次的考虑。行为模式客观上从行为中涌现出来,但对方对此没有自觉。
\end{itemize}
以上的所有内容,在分析“自己的另一种行为模式/思考回路”时也适用。同样不要草率地概括自己的动机,不要草率地觉得自己的行为出自某种特定的潜意识,这会掩盖真实的原因。
\end{explain}
\indicate{通过表面分析得到的动机,对改变现状毫无帮助。}
\begin{examples}
这里的现状指代任何一种行为模式。导致举动的另有其它因素。无论是想要培养,还是想要去除,通过表面归因都起不到什么帮助。如果读者喜欢,可以将其称为“形式主义”“主观主义”。本指南不引入这些概念。

比如,通过观察一些例子,我们可能会得到“吃苦才能成功”的结论。“吃苦”作为一个复杂概念,具有很重的歧义,如“做自身抵触/缺乏意愿的事”和“投入资源做事”。真正起作用的特征是“做会有成效的事”,二者都离其有一定距离,成功的人实际上也不一定真吃了苦。正确的思路是“对于一个确定的目标,和会有成效的事,即使需要强逼自己和投入资源,也需要去做”。但如果只看到了“需要强逼”,那么这不仅无助于成功,甚至无法起到磨炼意志力的作用。
\end{examples}

\section{价值观}
\subsection{大道理}
我们将\indicate{通过表面归因得到,并且可以指引行动的结论}称为\indicate{大道理}。
\begin{examples}
每一条大道理都是一条认知。表面归因不提供对复杂现象的理解,所以所有的大道理均可视为污染。大道理仅能指引一个人按简单逻辑做事,无法起到辅助思考的作用。

污染不一定起到负面作用,按简单逻辑行事有时便足以维持某种状态。如“爱护环境”“勤俭节约”“遵守交通规则”等大道理,在大多数情况下都是良好品质,有利于社会的和谐与稳定。

但同时我们也能举出一些过犹不及的例子来,如“因为不舍得剩饭剩菜而食物中毒”“因为要等红灯所以堵住了救护车的通行”等情况。如前所述,这是因为忽略了这些大道理的适用范围,无条件地依据这些大道理而行动。

站在客观的立场上,我们能够轻易指出这些极端情况的谬误之处。但是如果以遵守大道理为优先,那么这就是大道理指引的行动,这种选择就是正确的。
\end{examples}
如果在应用知识体系或其它结论时,忽略、遗失或过度简化了结论的适用范围,那么结论就会劣化为大道理。
\begin{examples}
很多情况都会导致劣化,如“因使用过于熟练而不再检验前提条件”、“与他人沟通时因为篇幅不足而只能使用结论”、“无法熟练应用知识体系,未意识到其适用条件,仅觉得某一句话很有道理”等。具体机制可以参考2.2.3小节的讨论,此处不再赘述。

结论可以起到记忆锚点的作用,可以提醒一个人“此时该用这种知识体系来分析”。一个深入的结论能够有效地指引人思考和行动。这种作用不来自这句话本身,而来自它背后的整体分析框架。当我们面对某个现象,想到某个结论时,如果能从此出发,遵循已有的分析框架,从而验证结论确实正确,那我们就\indicate{理解}了这个现象。如果想到结论时总能具体地思考,那我们就\indicate{掌握}了这个结论。

而如果不能调用相应分析框架,那么就无法达成上述效果,此时该结论就只是大道理。无论看上去多么理所应当,多么符合自己的实际体验,\indicate{任何一句单独的话都是大道理,都不可信}(这句话也仅起到提醒的作用,不要以此为信条去否定一切规训,这超出了它的适用范围;本指南中着重强调的其它句子也仅起提醒作用)。就算是1+1=2一类的话也是如此。当它脱离数学语境,被拿去做“1+1>2”之类的比喻时,就不再显然正确了。我们必须重新完整地判断相关结论是否可信。
\end{examples}

\subsection{价值观}
我们将\indicate{比较一个事件“发生了”和“没发生”孰优孰劣}的行为称为\indicate{价值判断}。
\begin{explain}
显然价值判断是判断。同时,每个价值判断都可视为一个认知。以下的行文中将不区分这二者,读者应能自行分辨。

一个价值判断可能仅关注属于某类特定过程的事件,而不关心其它的过程。一般地,一个价值判断关注的事件越多,涉及到的概念就越宏观,其判断也就越脱离现实。

价值判断是很常见的行为。有很多常用词汇包含价值判断的因素,如“好/坏”、“善/恶”、“对/错”、“应该/不应该”、“期待/抗拒发生”、“有无作用/意义/价值”、“是否本质”等。这些词汇都是复杂概念,以下使用“好/坏”来代指这些词汇。如果不定义“什么是好”,那么价值判断就是大道理。

有的价值判断是基于事件的某些特征,有的价值判断是基于事件的影响,出发点千奇百怪。按照是否自知出发点,价值判断可以分为两类,其中不自知的那一类价值判断均为大道理。但在自知的那一类中,价值判断的出发点可能是另一个价值判断,甚至可能出现“好几个价值判断互为出发点,循环论证”的情况。

为了确定是否自知“好”的定义,我们有必要将价值判断分为“可以归因于大道理以外的东西(比如说一个具体事务)”和“不可以归因于大道理以外的东西”两种。此处我们将“价值判断自身就是大道理”也视为“将价值判断归因于大道理”。
\end{explain}
有些词具有实际的含义,如“公平”“正义”“热心”“亲密”等。但如果这些词语劣化成了宏观概念,那么与之相关的判断就也会劣化为价值判断。
\begin{explain}
用大词容易让人容易忽略“这是在价值判断”,比如“没有意义”“不应该这么做”之类。但绝大多数人在下这样的判断时(无论判断自己的行为还是其它事件),都不是根据某个很严密的普遍终极答案而得出的结论。此时不能默认这种评判是恰当的,一定要确认清楚出发点。真的回头一想,大部分出发点是“某个很简单的思路,想达成某个很直接的目的”和“不清楚自己的出发点,价值判断是个污染”两种情况之一,有严密判断的情况极少。过于草率的价值观所产生的判断没有实际参考价值。
\end{explain}
我们将\indicate{参照价值判断的结果,进一步判断其它事物的好坏}的思考回路称为\indicate{价值观}。
\begin{explain}
多数情况下,价值观由多个价值判断与一些其它的想法组成。这些价值判断和其它想法越草率,包含越多大道理,价值观也就会越劣质。一个人可能因为各种原因被草率的价值观污染,而后该价值观就会持续生产脱离现实的结论。

比如将“达到要求”认为是“好”,“没达到要求”认为是“坏”,并根据“自己没有做到最好”认为自己是坏的;或者反过来,根据“别人没有做到最好”认为别人是坏的。具体的心态、行为和用词可能更多样,比如说“责骂/被责骂”“差劲”“无能”“没有希望”“失败”“不配活着”等。
\end{explain}
不同的价值观的“好坏”定义不同,脱离具体价值观谈论“好坏”没有实际意义。
\begin{explain}
此处的“没有实际意义”指的是“会引入歧义,从而无法得到有效结论”。更一般地,本指南中所有不说明前提的价值判断,全部都是以“是否会有歧义”“思路是否连贯”“推理过程和结论是否有效”“是否有助于目标”等判断为标准。

特别地,使用一个价值观的“好坏”去评判另一个价值观是否“正确”,是无意义行为。“正误”只能用于评价分析和结论,而无法用于评价出发点。
\end{explain}

\subsection{责任与要求\label{sec:责任与要求}}
我们将\indicate{一个人应该完成某个目标}的认知称为\indicate{责任}。我们也将这个认知表述为“这个人有完成目标的责任”。我们将“按照责任行事”称为“承担责任”,将“驱使某个人承担责任”称为\indicate{要求}。
\begin{explain}
每一个责任都可以视为一个价值判断。在不同的价值判断下,一个人的会拥有不同的责任。

责任的来源多种多样,可以是“自己觉得自己应该做”,也可以是“别人觉得自己应该做”。在行为模式层次的视角下,这二者没有本质区别。
\end{explain}
如果某个责任是因为“觉得某个目标好”这一价值判断而产生的,那么就可以将这个目标视为一个\indicate{事务},将责任视为\indicate{处理方法}。我们将这个价值判断称为\indicate{该责任依照的价值判断}。
\begin{explain}
责任也是价值判断,它和其对应的价值判断不同之处在于,责任中针对“一个人”这一行为主体做判断,而其依照的价值判断对于“目标”这一和人无直接关系的现象做判断。

责任中的“一个人应该完成某个目标”和价值判断“觉得某个目标好”中的目标不一定是同一个,其中可以有“觉得目标A好→觉得实现目标B可以实现目标A→觉得应该完成目标B”,或更复杂的思考过程。

这里可以讨论两种问题:一种是“责任是否正当”,这取决于判断“目标A是否正当”,本指南不引入特定的价值判断,这不在本指南的讨论范围内;另一种是“责任是否有效”,这取决于判断“目标B是否可以实现目标A”,这是本指南着眼之处。如前所述,有效的责任在本指南中被称为\indicate{解决方法}。
\end{explain}
责任不一定有助于其依照的价值判断。必须有解决事务的能力,才能承担责任。
\begin{explain}
如果对现实的认知和对事件的分析很草率,那么分解目标的方式也就会很草率。正因如此,我们需要调研能力以清楚地认知事物的复杂性,需要分析能力以理清事物之间相互影响的方式,需要计划能力以编织有效的操作,最后需要执行能力以实现这些操作。如果没有处理复杂事务的完整能力,就无法分解并达成目标。具体论述参见\hyperref[sec:现实事务处理能力]{1.2节}。

从价值判断中直接得出责任,并且据此来提出要求,经常会遇到被要求的人没有对应能力的情况。\indicate{宏观特点只能作为某些复杂现象的结果,而不能作为目的和要求。}
\end{explain}
无视一个人的能力情况,就对一个人提出要求,是不负责任的。
\begin{explain}
这里表示本指南态度的“责任”指的是“应当完成‘完成目标’的目标”,和之前提及的“不说明前提的价值判断”表意一致。

这里的“要求”涵盖很多内容,如“应当坚强”“应当成长”“应当优秀”“应当可以熟练应用某知识”“应当听得懂意思/指令”“应当明白自己的需求”等。读者可能觉得这种描述有些像是投射。严格来说有些不同,因为这里还涉及自我污染。本指南不引入投射的概念,故对此不做展开。
\end{explain}

\section{人的基本意识模型\label{sec:人的基本意识模型}}
\hfill\begin{minipage}{0.55\textwidth}
\fontsize{8pt}{12pt}\selectfont\fontsize{8pt}{12pt}
\raggedright 思想越过六千层腐烂的美梦,浇筑成一台链接神经的水泵。\\
我的星球还没学会自西向东,就要被迫融进这条神的河流。\footnote{\bilibili{av9206055};\\\indent \netease{466403600}。\\}

\raggedleft JUSF周存《欲》

\end{minipage}

在系统介绍了复杂性的相关概念后,我们得以从更深入的视角出发,刻画人的意识现象。概括地说,第二章中的分析考虑简单现象如何编织成复杂现象,而本节的分析则考虑简单现象都能涌现出什么复杂现象。
\begin{explain}
本节的部分内容是\hyperref[sec:原生家庭]{2.5节的内容}在本章用语体系下的重新叙述。读者可以对比本小节与2.5节,以获得更全面的理解。
\end{explain}
\subsection{刺激强度与外溢\label{sec:刺激强度与外溢}}
我们每时每刻都会接收很多信息,其中有些信息会满足某些行为的触发条件,从而成为一个信号。但同时,同一时刻内我们仅能做一件事。
\begin{explain}
信息有可能来自外部环境,有可能来自自身的记忆和想法。

这里的叙述做了一定的简化,事实上我们可以手上做一件事脑子里想一件事,或者手上做好几件事脑子里想好几件事。但这不影响后续讨论,此处关注的是\indicate{信号数量远大于可执行操作的数量}的情况。这种情况下必定会忽略一些信号。
\end{explain}
我们依刺激最强烈的那个信号而行事。
\begin{explain}
注意这里仅是非常模糊的描述,本指南从来没有定义过“刺激的强度”。实际上,不深入生理过程,就无法精确描述“刺激”是什么。

这里真正引入的假设是“没有其它改变时,同一个信号在不同时刻带来的刺激程度大致相同”,从而“在面对相同的环境时,人的第一反应是相同的”。刺激强度不以人的主观意志为转移,人的主观意志可以视为名为“意志力”的另一种刺激。如果你发现自己身上总是有改不了的坏习惯,那么说明这一习惯的刺激要强于意志力的刺激。在想办法降低习惯刺激(比如远离诱惑源)、提升意志力刺激(比如增强理性)、或引入新的刺激(比如遭遇突发事件留下深刻印象)之前,每次尝试都几乎必然失败。

引入“刺激”这一词语是为了之后叙述方便。以下将会引入另外几个假设,本指南认为信号的刺激强度仅会因这几个原因而改变。
\end{explain}
在短时间内,同样的信号刺激频率越高,总刺激强度越高。
\begin{explain}
这里的短时间只“从察觉到信号到做出行为”尺度的事件。依照具体行为的不同,时间尺度可能在几毫秒至几分钟内不等。

当前信号的刺激会与之前信号的刺激相叠加,从而印象更深。这一假设能够解释2.2.1节中对“\hyperref[para:记忆]{形成行为是因为有记忆}”的论述:经历过并记住的相同事件越多,就越容易联想到,并在脑内预演可能的结果。每次想到结果都是一次信号,想得越快,信号出现的频率就越高,刺激也就越强。这产生了一个正反馈循环:刺激越强越容易触发该行为,越触发记忆越深,记忆越深刺激越强。行为在这一过程中逐渐形成。
\end{explain}
简单的东西刺激更强。
\begin{explain}
这出自两方面原因。一方面是简单的东西过脑子的速度更快;另一方面是简单的东西会在更多场合内出现。

当一个行为逐渐形成以后,除了“之前经历过的相同事件”以外,我们的脑子里还会多出另一种记忆:“见到信号就做出行为”。这种只有首尾的记忆比包含过程的回顾更简单,回顾这种记忆比回顾事件要更快,从而会提供更强的刺激\footnote{实际上不会分两步,而是渐进式形成。这里为了讲解而做了简化。}。当这一过程再次熟练以后,我们就多了一个无意识的条件反射。

这种机制是我们会将具体事件劣化为宏观概念的主要原因。
\end{explain}
我们将不能作为信号触发行为的信息称为\indicate{盲点}。
\begin{explain}
盲点普遍存在,在环境切换时会大量出现。在熟悉的环境下,很多信息会起到“提示你什么都不用做”的作用,依实际情况不同,这可能是盲点、正确对待方式或污染。
\end{explain}
当我们同时遇到盲点和带有强烈刺激的信号时,信号会触发行为,而盲点也会因此与行为建立条件反射,逐渐成为新的信号。我们将这一过程称为“行为\indicate{污染}了盲点”。
\begin{examples}
这里的污染和\hyperref[def:污染]{之前的定义}一致,此时充当污染源的是这一行为。使用认知可以视为行为,按此定义,“因为指代而使词语概念发生改变”的现象也算是“概念被词语污染”。概念和分析方式因此劣化。

这一机制会在盲点和行为/认知之间建立虚假的相关性。除了“会先后出现”以外它们没有任何共通之处。行为/认知污染了盲点以后,就会妨碍对盲点本身信息的正确收集、理解、应用。

每一次行为外溢都可以视为形成了一条\indicate{价值判断}。这样形成的价值判断大多是不自知的,人会先感觉到“我想这么做”,然后再将其用语言总结成“应该这么做”。可以为自己的行为找很多很充分的理由,但它的最初形成是污染(找理由后,有可能确实因为理由而培养出新的行为,这应当算作另一条行为)。
\end{examples}
不同行为对信息的污染是很随机的过程,需要视为宏观事件。
\begin{examples}
行为污染信息的整体过程与自然选择类似。一个信息会被什么行为污染,后续又会有什么改变,高度不可预测,需要具体调研具体分析。不同的人所拥有的外溢行为可能完全不同。
\end{examples}
我们将“行为污染信息”称为行为的\indicate{外溢},并且将会污染信息的行为称为\indicate{外溢}的行为。
\begin{examples}
盲点可能是具体事物,也可能是很抽象的内容,比如说“恐慌”“焦虑”“不知所措”。这种机制会导致“越焦虑就越刷手机”等现象,因为“刷手机”这一行为已经严重外溢,导致它在很多情况下都是最常被触发的行为,这进一步导致了“刷手机”继续外溢。掌握的信息越少越偏,面对事件时就越不容易拥有恰当的反应方式,盲点就越多,于是就越有可能产生污染。

外溢可以被阻止。在污染还未形成时最为方便,仅需抢先建立正确的反应即可(这涉及到如何指定正确反应,这应具体情况具体分析)。污染已经形成则会比较麻烦,参见后文的论述。
\end{examples}

\subsection{全局行为模式\label{sec:全局行为模式}}
行为的外溢会使得它更容易进一步外溢。这个过程持续下去,会导致某些行为几乎在任何时候都会触发。我们将这样的行为称为\indicate{全局行为}。
\begin{explain}
全局行为不包括呼吸、心跳等生理活动(但如果遇到“时时刻刻都小心翼翼屏息”之类的情况,则计算在内)。全局行为较难界定,更适合的研究对象是全局行为模式(见下一小节)。但全局行为讨论起来较为简单,故我们在此展开一些讨论。这些特点对全局行为模式也适用。

有一些比较简单的现象可以视为全局行为。例如,当“想一想其他事情和我有什么关系”成为全局行为时,就会觉得万事万物都和自己有关,于是就会产生名为“全能自恋”的现象;当“发现事情是别人干的”成为全局行为时,就会把一些其它的类意识行为(比如app)视为有人在操控,进而可能发展成被害妄想等。本指南不过度使用精神分析语言,故不做过多探讨。

瘾也是一类比较常见的全局行为,如烟瘾、酒瘾、性瘾、毒瘾等。需要注意的是,并非所有通常意义下的瘾都是全局行为,但它们被称为瘾,至少属于外溢行为。瘾很难戒除,是因为其有过多的触发条件,这也就是通常所说的“身瘾好戒,心瘾难戒”。

当自身存在全局行为时,进入新环境后,全局行为很容易污染新环境中的信息。手里有把锤子,看什么都是钉子。自己会真心实意地相信自己认知的正确,全心全意地践行全局行为。如果一个人的全局行为中不包含自我审视和反思,那么就丝毫不会怀疑其它全局行为是否恰当。

因为行为的外溢高度不可预测,全局行为因人而异。不同的全局行为之间很可能互斥,拆解一个全局行为培养一个新的,难度很高。本指南的目标是让读者将“唤醒理性”培养为自己的全局行为,以做到时刻清醒地认识世界,有效地行事。这难度很高。
\end{explain}
和行为类似,行为模式也有“刺激越强越容易触发”的机制,这一部分不再赘述。如果一个行为模式在几乎任何环境下都会触发,就将其称为\indicate{全局行为模式}。
\begin{explain}
也可仿照外溢行为,定义外溢行为模式。本节中的讨论同时适用于普通的外溢行为模式与全局行为模式,仅讨论全局行为模式是为了方便。读者可以通过加入“行为模式会触发的环境”的条件,来使以下讨论同时适用于普通的外溢行为模式。

行为外溢得越严重,就越容易被触发。外溢的行为越多,这些行为之间就越容易相互触发,从而越容易形成一种严重外溢的行为模式。全局行为模式可以通过这种机制,直接由一些外溢的行为产生。之后为了叙述方便,我们将全局行为模式的组成部分也称为全局行为,读者应能自行分辨。

由于外溢的行为大都是高熟练度的无意识行为,全局行为模式也因此成为无意识的行为模式。
\end{explain}
我们不应将全局行为模式拟人化。
\begin{explain}
准确来说,我们不应认为全局价值观具有“保守”“懒惰”“抗拒改变”“主动”“固执”“控制欲”“指导欲”等特点。这些词语可以用于形容人,或者人的某个行为模式,但我们需要站在行为的层次来看待全局行为模式。这些词语都是行为模式层次的宏观概念,会带来不必要的污染。

只有站在行为的层次,充分调研并分析我们所拥有的所有外溢行为和它们之间相互触发的方式,我们才有能力施加影响,从而主动改变全局行为模式。如果将其视为一个整体,就无法执行这么精细的操作。劣质的全局行为模式从来不是一个整体,没有什么确定的倾向,没有什么统一且坚定的看法。它只是一堆容易互相触发的外溢行为。
\end{explain}
全局价值观越是劣质,就越会以高频率发生不可预测且不可控的改变。
\begin{explain}
行为会依照行为的演化机制发展,不断因为新出现的刺激和记忆而更新。行为的外溢不可控,于是由外溢行为组成的全局行为模式和全局价值观也不可控。

这也导致两个不同人的全局价值观之间差距非常大,可能完全无法交流。这种现象也可能发生在现在的自己和以前的自己之间。全局价值观越是劣质,就越难以理解、控制、改变别的行为模式(包括自身)。这会给人以“人不可能相互理解”“做人只能靠自己”之类的错误结论。在大多数环境下这种看法甚至没什么问题,但它们本身不正确。
\end{explain}

\subsection{干扰与打断\label{sec:干扰与打断}}
外溢行为产生的刺激会为我们处理事务产生\indicate{干扰}。如果外溢行为的干扰过强,那么即使我们当前还有别的安排,也会转而去按照外溢行为而行动。我们将\indicate{某行为产生刺激,使得我们脱离当前的行为模式}的过程称为这一行为\indicate{打断}了行为模式。如果来自全局行为模式的刺激使我们转向全局行为模式,也同理将其称为\indicate{打断}。
\begin{explain}
这里的\indicate{打断}和\hyperref[def:打断]{先前}的定义一致,只不过在这里我们更关注主语。

外溢行为高度简单,高度熟练,在面对很多情况时,都是反应速度最快的那个。这使得它可以迅速给出简单、详细、可以立刻执行的操作。这些操作的正确性不得而知(比如说焦虑了就去刷手机),它们的唯一特点是会给人最强的刺激,吸引人去选择。
\end{explain}
每个全局行为模式会附带一个价值观。我们将其称为\indicate{全局价值观}。全局价值观会展示全局行为模式每一部分的正确性。
\begin{explain}
全局价值观是一种思考回路。

每一个外溢的行为都可以视作一个或多个价值判断。这些外溢的行为怎么相互触发,它们对应的价值判断就也会怎么相互触发,形成一个完整且能说服自己的论证过程。
\end{explain}
我们思考时,会受到全局价值观的显著干扰。思考链越长,全局价值观的干扰就越大。
\begin{explain}
全局价值观仅能提供逻辑链很短的,几乎直接出自其价值判断本身的结论。如果思考链被它给出的结论打断了,此次分析就会无效。思考链越长就越容易被打断。换句话说,只有在全局价值观不产生明显干扰的情况下,我们才能进行长时间、大规模的思考。

并且,我们很难去验证全局价值观给出的答案。全局价值观会非常熟练地证明自身的正确性,如果没有其它得出结论的手段,便会又回归全局价值观的判断。除了少数情况有“无法辩驳的事实直接摆在眼前,让人印象深刻无法忘记”以外,其它刺激都会在某一次注意力被转移以后,再也想不起来。

值得注意的是,先前提到过的“倾向于将行为模式拟人化”可能是全局行为模式自身包含的一种外溢行为。在我们分析全局行为模式时,它会持续施加干扰。必须时刻注意。
\end{explain}

\subsection{清除外溢\label{sec:清除外溢}}
我们将\indicate{每次遇到一个信息时,都按照固定的方式操作,直到形成习惯}的过程称为\indicate{培养}。面对未被污染的信息时,我们可以较为轻易地培养出正确的应对方式。
\begin{explain}
仅需遵循行为的产生机制,坚持重复使用正确的方法,直至产生关联即可。

\indicate{这一切的前提是我们有意培养}。我们需要有获取正确应对方法的途径(无论是靠自己或外部)。如果没有主动培养,行为就会遵循上述机制,逐渐将反应固定为刺激最强的那个。
\end{explain}
我们难以处理已经被外溢行为污染的信息。草率的处理手段\indicate{注定失败},只会使情况更糟。
\begin{explain}
这里的\indicate{注定失败}带有一定的修辞效果,不意味着“一定不能成功”,但其极低的成功率使得成功不可复刻。这不是科学可重复的处理方式。本指南其它地方出现的“注定失败”“没有成功的希望”“只能看运气”等表述也是这种意思。

打断的机制使得难以修改并清除污染。如果没有充足的准备,无法拿出另一套简单、详细、可以立即执行的操作来代替原有的外溢行为,那么就很容易在坚持了几天以后半途而废,反而让外溢行为又污染了想要改变的尝试,使得下一次更容易半途而废。

这是处理“未被污染”和“已被污染”的两种信息时的本质不同之处。必须有区分这二者的能力,才能分清是偶尔还是习惯,才能确定改正或处理的方式。
\end{explain}
如果有持续且强烈的刺激,全局行为模式会因此改变。但如果仅有刺激,改变的方向不可控。
\begin{explain}
当刺激长期存在时,全局行为模式会不断被触发。在经历了充分长的时间后,要不然全局行为模式污染了刺激(此时归入先前的情况),要不然我们会获得“全局行为模式处理不了这一刺激”的新信息(总体来说,越善于总结经验就能越早发现这一点,越不善于总结经验就会越慢,只能指望最基础的行为养成机制)。

这一信息会在每一次触发全局行为模式时都出现以给予我们刺激,打断全局行为模式,使我们放弃接下来的行动(此时不同的人会感受到不同的情绪,包括但不限于急躁、失落、自责等。这不是讨论的重点所以略去。同时,如果这些情绪属于全局行为模式的一部分,或是因此而成为了全局行为模式的一部分,那么应该被视为“刺激被污染”的情况),转而尝试新的行为。

此时的情况类似于面对未被污染的信息时的情况,但仍有主要不同:全局行为模式仍然在持续触发,它会频繁打断我们的思考,从而使得我们难以维持培养的过程。这使得主动修改自身的全局行为模式和全局价值观极为困难。
\end{explain}
我们应当将理性培养为自身的全局行为模式。
\begin{explain}
最优状态应该是“使理性成为自身唯一的全局行为模式”,但那样难度过高,理性能够监督自身的意识活动就足够了。甚至在实践中,这一要求还是有点高,我们只需要在处理事务时能够唤醒理性,就已经足够了。

需要说明的是,理性不和情感互斥,不会出现“因为理性所以变成冷血机器人”的情况,过于冷漠正说明不够理性,身上有不可控的“压抑情感”的全局行为模式。非要找个对应的性格形容词的话,应该使用“豁达”“超脱”等。

掌握一种能力(知识体系),等价于在该能力的所有应用场合,都不被外溢行为模式打断(不受干扰更好,但更难以实现)。只有当我们不被劣质的全局价值观干扰,思路不被打断,我们才能得到整体观察事物的机会,进而才能培养处理复杂现象的能力。

我们需要使用理性来修正自身的行为外溢现象,并且打断行为继续外溢。我们需要在每个场合都全面地接收所有信息,之后选用正确的看待方法。如果不知道怎么才算正确,那么就去研究或学习,在找到方法之前,控制住自身任何下意识的反应,不作任何草率的行动。
\end{explain}

\section{总结与讨论}
\subsection{本章总结}
本章内容围绕\indicate{复杂系统}及其看待、处理方式展开。相比第二章,本章的讨论内容较为集中,除了复杂性的概念介绍,就是使用系统性的视角,来分析各个具有复杂性的事物。正确运用系统性的视角分析复杂问题,是我们处理现实事务的必备\indicate{能力}。

正式介绍复杂系统之前,我们需要先做一些准备,对我们认知世界的方式建模。这一部分虽然也属于人的意识活动,但与第二章的内容关系不大,与第三章联系更紧密,故将其放在第三章。我们引入了\indicate{概念}和\indicate{特点}作为基础概念,并且引入了\indicate{识别}和\indicate{指代}两种基础操作。据此,我们得以引入\indicate{出发点}和\indicate{能力范围},以界定在思考时是否产生\indicate{歧义},以使得思路\indicate{劣化}。

劣化是思考的主要危害,而对复杂性的认识不足,思考得过于概括粗疏,导致\indicate{信息损失},则是产生劣化现象的主要原因。在充分讨论劣化的相关内容后,我们得以在引入复杂系统的概念后,立即讨论其处理方法。

我们介绍了复杂系统中会出现的两种现象:更偏向从某一具体的\indicate{高层次事件}的视角看的\indicate{编织},与更偏向从全体\indicate{低层次事件}视角看的\indicate{涌现}。同时,我们还展开讨论了会导致思考劣化的\indicate{宏观分析}现象,仔细分析复杂系统为何会使得浮于表面的思考脱离实际。

讨论完复杂系统以及宏观分析的一般理论,就可以将这些理论应用于实际事物上。我们首先深入考察了事务处理方面的内容。如果对事务的认识不足,无法将它的本质困难纳入考量,只是因为觉得“应该这么做”就将其定位了目标,就会产生很多没有能力承担的\indicate{责任}。如果任由这种草率的\indicate{价值判断}组成\indicate{价值观},就会产生很多无意义的内耗。此处我们还声明了本指南所持有的价值观:解决现实事务。

最后,我们使用复杂性的视角,重新梳理了一遍人的基本模型。我们通过考察\indicate{刺激强度}对\indicate{行为形成}的影响这一复杂系统,部分解释了不可控行为的产生原理,同时回顾了处理方法。我们提出了\indicate{外溢行为}与\indicate{全局行为模式}的概念,并分析了它的复杂形成过程,从而确认了将其清除的困难程度。需要注意的是,此处的模型仍然是高度简化的版本,有很多细节缺失,读者需要注意其应用范围。此处的模型仅供本指南展开后续理论使用。

除此之外,本章中提及的一些概念,在之后的一些篇幅中也会频繁使用,一定程度上方便了行文与理解。

\subsection{澄清与叠甲}
大家在阅读本章的过程中,或许会产生一些联想。可能有些读者会觉得本章的内容和一些其它理论很像,也可能有些读者觉得本章的内容可以应用到一些其它领域。读者可能会疑惑,为何本章没有对这些内容展开更深入的讨论。这是本指南有意的安排,目的是尽量少地设计无关领域,将关注点聚焦在本指南的主题上,尽量避免因发散过多,讨论不充分,而产生污染。

\smalltopic{(1)关于应用范围}

读者在阅读本章内容时,尤其是介绍复杂系统时,会很容易将其套到社会现象去。我们似乎可以很轻易地就得出一大波有用的分析和洞见:社会中的组织由人编织而成,但起作用的不一定是人,组织可能只需要人的行为模式,这样就产生了异化;污染在人与人之间传播,于是形成了模因;不同人的行为可以在舆论场中相互激发,从而形成长久存在的行为链,形成回声室;这种纯粹的涌现现象,从表现上与行为模式无异,很容易被错认为是有人在背后布局;大部分人根本就没有考虑长期问题的能力,都是乌合之众,都是巨婴;而使得大部分人的能力差成这样的社会更是万恶之源,是恶之花绽放的土壤......

以上所有这些,都是不负责任的暴论。本指南介绍的理论,不足以容纳全部的论述过程,也无法判断这些内容的正误和适用范围。任何从本指南出发的推导,思路中必然包含大量不客观内容。

“期望给事情一个简单的归因”是一个严重外溢的行为,必须警惕它的影响。社会是一个复杂系统,它的复杂程度远超人的意识现象,必须谨慎对待。读者可以衡量一下本指南介绍“理性的分析框架”所用的篇幅,以及将该分析框架熟练运用在自己身上的难度。这有助于破除“似乎掌握了社会运行的本质”的错误认知。在充分掌握理性之前,几乎不可能做到谨慎地采用科学方法从微观视角客观研究社会运行的具体规律,任何使用宏观概念的概括都会劣化成大话空话,任何自己得出的结论都会不可避免地带有偏见。

本指南所介绍的理论,仅用于分析每个具体的人的意识现象,不涉及更复杂的内容。这已经够难了。

\smalltopic{(2)关于其它理论}

阅读量比较多的读者可能会觉得本指南的论述在某些地方见过。我猜测这样的既视感应该集中于哲学和心理学领域,还有少量可能属于文学、社会学等领域。这种现象很正常,本指南没有介绍什么新东西。但如果将本指南中的内容按其它理论理解,可能会有害。

我们在评价理论的有效性时,经常会用到“现实情况符合理论预期”与“理论可以推出现实情况”两种表达。这样的表达同样可以用在理论上,但语感会有所区别。“理论A符合理论B”和“理论A推出理论B”是一个意思,它们都指“理论A的特点满足理论B的基本假设”,但语感上会借用之前表达中的“(普遍的)理论比(一个)现实情况更本质”的感觉,认为前者中理论B更本质,后者中理论A更本质。

给理论之间排“本质性”非常危险,本质是个宏观概念,同时还是个价值判断,不存在一个稳定,能经受住所有质疑的“本质”定义。理解一套理论仅能从其自身出发(如果自身没能严密到建立完整框架,那再另做处理,可以根据具体情况选择忽视或者是要求作者重新修补,不在当前话题讨论范围之中),不应使用别的理论的概念来代替对其本身的理解。这种行为会带来污染。

读者可能看到概念部分的论述,感觉像是在讲结构主义或现象学;看到关于复杂系统的实事求是处理方法,感觉像是在讲辩证唯物论;看到价值判断的讨论,感觉像是在讲虚无主义;看到人的基本模型,感觉像是在讲精神分析与自由意志;或者看到一些别的地方,又有其它的既视感。如果要问“这么看对不对”的问题,答案是“这些东西确实算,相应的书籍中都有详细论述”;但如果要问“这么看有什么用”的问题,那大概没什么用。无法从这样的判断中,找出正确的废话以外的东西。你无法从中获得确切的知识,只能得到“我好像又学到和发现了东西”的幻觉。

\vspace{20pt}

对于那些真正深入理解这些话题的读者来说,本指南中所提到的这一点篇幅,又会显得过于简化,精华尽失。这是为了可读性而做的妥协。本指南是一本应用性的书籍,需要在保证完整介绍分析框架的前提下,尽量减少提及不必要的内容。完整阅读这一分析框架已经很困难,那些过于细节的论述会进一步增大读者的理解压力。我们选择将这些领域的观点一并作为本指南的出发点介绍给读者。

如果读者对其他领域有深入研究,本指南欢迎读者在此基础上理解并评价。但请读者尽量避免概念的跨领域应用,尽量避免使用自身无法掌握的概念和结论。这会造成本可避免的麻烦。
