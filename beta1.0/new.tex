\documentclass[openany]{ctexbook}

\xeCJKsetup{AutoFakeBold=true} % 将pdflatex换成xelatex后,需要手动防止字体报错
\defaultfontfeatures{AutoFakeBold=2.5}
\setmainfont{Times New Roman}
\setmonofont{SimSun} % 设置等宽字体族,使\url可以调用以显示中文
% \setCJKmonofont{Noto Sans Mono CJK SC}
% \usepackage{ctex}
\usepackage{geometry}
\geometry{a4paper, margin=1in}
\usepackage{fancyhdr} % 用于自定义页脚
\usepackage{titlesec} % 用于自定义标题格式
\usepackage{quotchap} % 用于给章节标题加入引言
% \usepackage{epigraph} % 用于给子标题加入引言
% \usepackage{indentfirst} % 使章标题后的首行缩进(好像默认就有,不需要这么做)
\usepackage{hyperref} % 用于激活网址超链接
\usepackage{changepage} % 提供 adjustwidth 环境
% \usepackage{ulem} % 提供 \uline 命令
\usepackage{xeCJKfntef} % 提供可以给中文换行的下划线,包名是 CJK font effect 的意思
% \setCJKmainfont[
%   BoldFont = SimHei, % 或 Source Han Sans SC Bold
% ]{KaiTi}
% \setmainfont{Times New Roman}

% \includeonly{intro, chapter00}
% \includeonly{chapter00}
% \includeonly{intro, chapter00, chapter01, chapter02}
% \includeonly{chapter02}

\newcommand{\authorname}{乐正梅霜}
% \newcommand{\indicate}[1]{\textbf{\underline{#1}}}
% \newcommand{\indicate}[1]{\textbf{\CJKunderline*{#1}}} % 带*的会把标点跳过去,不带*的在标点下也有下划线
\newcommand{\indicate}[1]{\textbf{\uline{#1}}} % 好像这两种都行,应该是xeCJKfntef包重写过了,问过kimi后,这是因为xeCJKfntef包加载了\ulem包
\newcommand{\source}[1]{\indicate{#1}\label{def:#1}}
\newcommand{\note}[1]{\hyperref[def:#1]{#1}}
\newcommand{\inote}[1]{\hyperref[def:#1]{\indicate{#1}}}
\newcommand{\later}[1]{}
\newcommand{\divider}{\vspace{10pt}\hrule\vspace{10pt}}
\newcommand{\another}{;\\\indent}
\newcommand{\rigorous}{特别追求严谨性的读者}
\newcommand{\trained}{一些熟悉相关领域的读者}
\newcommand{\bilibili}[1]{b站:\url{https://www.bilibili.com/video/#1}}
\newcommand{\bilibilip}[1]{人声本家:\url{https://www.bilibili.com/video/#1}}
\newcommand{\bilibiliv}[1]{中v本家:\url{https://www.bilibili.com/video/#1}}
\newcommand{\bilibilid}[1]{b站(因不可抗力因素换源为纯音乐版本):\\\indent \qquad \quad \url{https://www.bilibili.com/video/#1}}
\newcommand{\netease}[1]{网易云:\url{https://music.163.com/\#/song?id=#1}}
\newcommand{\sing}[1]{5sing:\url{http://5sing.kugou.com/yc/#1.html}}
\newcommand{\moegirl}[1]{萌娘百科:\url{https://mzh.moegirl.org.cn/#1}}
\newcommand{\smalltopic}[1]{\vspace{5pt}\noindent \indicate{#1}\vspace{5pt}}
\newenvironment{toparent}{\begin{center}\begin{minipage}{0.9\textwidth}\kaishu}{\end{minipage}\end{center}}
% \newenvironment{examples}{\begin{center}\begin{minipage}{0.9\textwidth}\kaishu}{\end{minipage}\end{center}}
% \newenvironment{explain}{\begin{center}\begin{minipage}{0.9\textwidth}\kaishu}{\end{minipage}\end{center}}
\newenvironment{examples}{\begin{adjustwidth}{0.05\linewidth}{0.05\linewidth}\kaishu}
{\end{adjustwidth}}
\newenvironment{explain}{\begin{adjustwidth}{0.05\linewidth}{0.05\linewidth}\kaishu}
{\end{adjustwidth}}


\begin{document}
\begin{sloppypar}
%kimi输出结果
%\iffalse
\begin{titlepage}
    \centering
    \vspace*{\fill} % 上半部分
    {\Huge\bfseries 人间失格自救指南}\par\vspace{2cm}
    \raggedleft % 作者信息右对齐
    {\Large \authorname}\par
    {\small beta1.0 \\ \today \\ \url{https://github.com/YuezhengMeishuang/-}}
    
    \vfill % 用垂直填充分隔上下部分
    
    % 下半部分:名言警句
    \raggedright % 左对齐
    \begin{quote}
        美国心理学家斯科特·派克在他的著作《少有人走的路》中写道:“我们不能剥夺另一个人从痛苦中受益的权利。”\\
        \hfill—— 武志红《愿你拥有被爱照亮的生命》\footnote{我并未在《少有人走的路》中找到这句话的直接出处,但我觉得这句话很有道理,所以把它放到这。}
        \vspace{1cm} % 两句之间的间距
        
        但我们起码得告知另一个人从痛苦中受益的可能性与可行性。\\
        \hfill—— 沃茨基·硕德 \footnote{沃茨基·硕德(Walski Schölder),即“我自己说的”。}
    \end{quote}
    
    \vfill % 将注释推至页面底部
    
\end{titlepage}

\makeatletter
\let\cleardoublepage\clearpage % 将双页清理改为单页
\makeatother

\frontmatter                      % 书籍前文部分

% 使用\titleformat调整\chapter格式
\titleformat{\chapter}[display] % shape参数
{\normalfont\Huge\bfseries\raggedright} % format参数
{}        % label参数
{20pt}                                   % sep参数
{\Huge}                                  % before-code参数
[] % after-code参数
\titlespacing{\chapter}{0cm}{1cm}{1cm}
\large
\chapter{前言}                   
%\addcontentsline{toc}{chapter}{前言} % 添加到目录


\noindent 本指南主要讲述在现代社会中高质量地活下去的方法。书中的内容可能主要对以下三类人群有所帮助:
\begin{itemize}
    \item 感到痛苦、迷茫、不被周围人理解,同时缺乏长期目标的人。这类人是本指南的主要服务对象。本指南将会带领你细致地拆解生存中会遇到的问题,并对此给出详尽可行的处理方案。指南的主要篇幅将直接面对你。若无特殊指明,文中的第二人称指这一类人。
    \item 身边有第一类人,并且有动力改变他们的人。你是否觉得你的孩子性格越来越封闭,情绪越来越低沉?你是否觉得朋友需要和你报团取暖,以面对社会的不公?你是否在群里遇到了经常情绪崩溃,不停诉苦的群友?最重要的是,你的关心和安慰是否没有起到你预期的效果,对方并没有如你所希望的那样好起来,反而只是将情绪压在了心底,或者把你的话当耳旁风?
    
    本指南会为你深入剖析他们的思想状态,并且向你展示切实可行的帮助方法。这不是个例,而是一种广泛存在的症结:这\indicate{不是态度问题,而是能力问题}。对方并非如你所想“不愿积极地面对生活”,而是缺乏积极面对生活的能力。而作为帮助者,你也缺乏帮助他们处理问题的能力。如果只凭“想要好起来”而行动,那你的举动大概率对现状毫无帮助,只会使对方更加敏感脆弱。奉劝各位\indicate{不要在充分了解实际情况之前,草率地做出决策}。
    \item 社会领域的研究学者。我相信您的研究水准,故我仅在此声明我的核心观点:本指南中提到的关于一类人群状态的刻画,不应只被看做广泛存在的心理疾病与精神疾病,也不应被看做青少年不成熟的想法。这是一种稳定的意识形态,暂且称为\indicate{失格主义}\footnote{我尚未确定使用何种称呼来命名该意识形态,目前还有“排除主义”“放逐主义”“局外主义”等备选项。deepseek建议我使用“离栖主义”来称呼,但我觉得这也不贴切。}。
    
    这一意识形态的出现有两个条件,其一为发达的互联网建设与极大丰富的信息资源,其二为大量缺乏“能带来长期影响的短期收益”的人。失格主义在学生与青少年群体中十分流行,即使断绝了个体与互联网的接触,身边环境对个体的影响仍会使个体高频率接触并认同失格主义。它在社会中广泛存在,具有坚实的社会基础,绝不可能仅靠隔离而消除影响。
    
    失格主义的主要特征为“认为自己的生命和努力没有意义,在社会上很多余”,带有明显的虚无主义色彩,但与虚无主义有明显不同。其未从根本上否定意义的存在性,而是带有强烈的现实色彩,觉得自己无法融入社会,被社会排斥,或者融入社会也无法做出任何形式的贡献。
    
    客观来说这种认知有其合理性,随着科技的进步,能推动领域进步的人综合素质要求越来越高,而一般岗位所能创造的价值越来越多地被信息化所取代,无法再提供自我实现需要。经济条件无助于解决失格主义,认知方面的提升是刚需。
\end{itemize}

可能会有读者注意到,上面的三类人中不包含心理工作从业者。这有两方面原因:一方面是本指南中所展示的内容在心理学领域内并不新鲜,相当篇幅是人尽皆知的结论;另一方面是本指南采取的思路较为激进,有很强的实验性,很少在其它方面的心理书籍中看到。相关的讨论更多集中在文艺领域,而大量文艺作品充分参与了失格主义的形成。读者可能会从本指南中找到精神分析和哲学的影子,但本指南会尽量避免使用相关领域的术语,也不宣称自己是心理领域的书籍。请读者切勿自行诊断精神疾病,无论是对自己还是对身边的人。如果确实觉得自身或身边的人患有精神疾病,请前往医院寻求专业诊断。若未成年、经济、时间等条件限制而无法稳定就医,本指南在后文提供了一定的替代方案。因自行诊断而产生的一切问题,本指南概不负责。

本指南所要介绍的仅是现代社会中高质量地活下去的方法,指引读者识别、思考、处理生活中遇到的各式各样的问题。正文中的楷体字均为解释说明,包含概念辨析、解释说明、分类讨论等细节内容。若读者能力充足,或仅想大致了解某部分内容,可以跳过所有楷体部分,仅关注宋体段落。

本指南的第一部分包括第一章“事务处理的三要素”、第二章“认知的分类与从外界获取认知的过程”、第三章“复杂现象与复杂概念的组成方式”。第一部分会介绍本指南的基本分析框架,理论介绍较多,内容较为枯燥。如果对于这些内容不感兴趣,\indicate{可以先看每章的实操部分或第三章后的}\hyperref[chap:midterm]{\indicate{第一部分总结}},之后再去补感兴趣的部分。三章各自的内容相对独立,不过后文会使用前文介绍的一些概念。其中一部分概念可以望文生义,但建议最好参照前文的解释说明。

从第四章开始的部分,则是对前三章分析框架的具体应用。本指南会带领各位分析身边发生的事,【等写完后面的以后才能回来补的逐章节介绍】

在此希望各位读者,无论你想做什么事情,无论是想自杀还是想举报,我恳请各位在读完整本书以后再做。恳请各位\indicate{不要在充分了解实际情况之前,草率地做出决策}。想自杀的读者,虽然我充分尊重你的选择,但为了本指南能让更多人看到,请尽量控制一下自己的冲动。想举报的读者,虽然我充分理解您对孩子的担忧,但仍然希望您可以在看完本指南之后再决定是否举报。对于想将本指南分享给其它人的读者,请仔细判断对方的性格,尽量避免本指南会因此次分享而下架。

% \vspace{2cm}                    % 留出签名区域
\hfill
\begin{minipage}{0.3\textwidth}
    \begin{flushright}
        \authorname \\
        %于清华大学 \\
        2025年3月27日%\today
    \end{flushright}
\end{minipage}

\tableofcontents

\mainmatter
\setcounter{secnumdepth}{-1} % 在这一段(到secnumdepth计数器重新变回2为止)中chapter、section、subsection不参与计数,目录中不会出现编号,序幕可以保留“序幕”作为前缀 
% 发现了\addcontentsline命令,但是没搞明白,先搁置

\titlespacing{\chapter}{0cm}{4.5cm}{1.5cm}

% % 重定义chapter命令,支持可选参数[名言],来自deepseek
% \newcommand{\chapterquote}{}
% \let\oldchapter\chapter
% \renewcommand{\chapter}[2][\empty]{%
%     \ifx#1\empty
%         \def\chapterquote{}%
%     \else
%         \def\chapterquote{#1}%
%     \fi
%     \oldchapter{#2}%
% }
% \titleformat{\chapter}[display] % shape参数
% {\normalfont\Huge\bfseries\raggedright} % format参数
% {}        % label参数
% {20pt}                                   % sep参数
% {\Huge}                                  % before-code参数
% [\vspace{-1em}\raggedleft\small\itshape\chapterquote] % after-code参数

\begin{savequote}[250pt] %250pt指定引言宽度
    \fontsize{8pt}{8pt} %两个参数分别指定字号和行间距
    人生若是二十七岁就能死掉,那么是摇滚救了我。\footnote{原文为“人生、二十七で死ねるなら、ロックンロールは僕を救った”。\\\indent \netease{1357953772}。}
    \qauthor{ヨルシカ《八月、某、月明かり》}
    不知道在世上活着的方法,隔壁家的姑娘昨天也自杀。\footnote{\bilibiliv{av38665329}\another\bilibilip{av80848116}。}
    \qauthor{純白《玛\ 德\ 琳\ 娜\ 电\ 塔》}
\end{savequote}
\chapter{序幕:你为什么活着?\label{chap:wedge}}


% \renewcommand{\chaptername}{序幕:} %在当前环境(secnumdepth=-1,chapter不计数)下无效果,取消secnumdepth后显示“序幕:1”,换行显示“你为什么活着”
% \chapter{你为什么活着?}


我觉得你一定想过这个问题:\indicate{你为什么活着}?在这个信息爆炸的时代,这个问题你应该没见过十次也见过九次,早就不新鲜了。我相信有相当一部分读者也已经有了自己的答案,有的想投身于数学物理信息科学,有的想为共产主义事业奋斗终生,有的想组一辈子乐队,有的想产一辈子同人粮。另一部分没有目标的读者自然过得浑浑噩噩,但既然有目标还会感到痛苦,那你一定不知道自己具体做什么才能实现自己的目标。本指南会详细地教你如何处理这个问题,不过在此之前,让我先问你另一个问题:\indicate{你怎么还没死}?

\begin{toparent}
    对于一部分读者而言,这个问题纯属无稽之谈。这些读者在“为什么不能死”这个问题上有着充足的理由,或者是感觉生活很美好,或者是有肩负的责任。我暂时称这部分读者为\indicate{正常人}。他们中的一部分在读到这里的时候,可能已经开始着手举报本指南,说它“带坏小孩子,教唆青少年自杀”了。我的这个问题不针对正常人,而是针对另一部分,认真有过“想死”的念头的读者。
\end{toparent}
无论你是半夜emo、遭受重大打击、无法完成原定计划,当你心情差到想死的时候,应该都会有这么一种感觉:无论你在心情比较舒畅的状态下多么热爱生活,多么想活着,有多少不能死的理由,它们在这种状态下都显得彻头彻尾地苍白无力。一方面,在想死的时候,你根本想不起来自己有什么要活着的理由;另一方面,即使有人跟你谈及你的兴趣爱好或者特长成就,你也不会被说动:要不然觉得自己不配,要不觉得那些东西毫无价值。既然不能死的理由完全不奏效,又有明确的死亡冲动,那么,\indicate{你怎么还没死}?

这个问题的答案其实很简单:\indicate{你还没死是因为你每一次都没死成}。更准确来说,你的每一次死亡冲动,每一次自杀计划,最终都没有落实。这看起来像是废话,一个已经死了的人不可能还在阅读本指南,但总体来说,这是一份对每个人都适用的回答。想要更深入的答案,就需要具体情况具体分析:

一种很常见的情况,是\indicate{脱离了想死的状态}。晚上躺在床上会觉得自己一无是处,但是随后就睡着了,第二天醒来,你的脑子里会想一些别的,比如今天有什么安排。

但也不是所有时候都能刷新状态。你要是失眠了一整晚,第二天起床的时候大概会更想死——只不过得去上学或者上班了。或者说,你一闲下来就会想死,只是还有事情做,总是闲不下来。这种情况可以总结为\indicate{想死的念头被打断}。无论你是需要上学上班,还是总有人找你聊天,还是有很多单子要打,哪怕只是和别人待在一起,或者是走在大街上,只要你还有事情需要做,还需要在别人面前维持一个还说得过去的形象,你就不会深入地考虑自杀计划。

大多数人“想死”的念头其实停留在这两种情况上。即使情绪持续性低迷,也不会持续性地想死。本身就没有坚定的想死意志的情况下,没死成是很正常的事情。正常人可能会将其称作“暂时性的想不开”(我不赞同这种看法,虽然这种看法在效果上不会出大问题,“忍一忍就过去了”确实是一种行之有效的处理方法,但这只是拖延而不是解决)。但同时我相信,本指南的读者中,不乏持续性地想死的人。如果你会对着手臂雕花刀,或者打开窗户望着楼下出神,如果你确实距离死亡只有一步之遥,那么,\indicate{你怎么还没死}?

有些读者身边会有形影不离的朋友或者长辈,以陪伴或者限制人身自由的方式阻止自杀行为,或者施展及时的救治,这种情况其实可以归为“想死的念头被打断”,不是当前讨论的重点。而另一些读者,确实尝试过自杀。割腕、上吊、吞药、烧炭、跳楼、跳河,你的自杀计划为什么没有奏效?

有相当一部分情况,是\indicate{没能达到死亡条件}。你想吞安眠药,但在下次醒来以后,才发现医生给你开的剂量根本达不到致死标准;你想烧炭,但却连去商店买炭的胆子都没有;你想上吊,但是找遍家里也找不到合适的房梁。

另一部分情况,是\indicate{被生物本能阻止了}。一个恐高的人不太可能跳楼,看着楼下就会腿软得无法动弹;一个怕疼的人不太可能割腕,下不去手只能草草了事;使用过一遍的自杀方式,下次再用总会有更大的心理障碍。

与之相反,那些条件简单,并且不触及生物本能或能克服生物本能的自杀方式,就更容易成功。如果有人递给你一把枪的话,朝自己脑袋来一枪是很轻松的事情;如果你在跳河的时候周围有人劝你“想想你的家人”,那你就顾不上怕,而只想逃开他们,于是就跳下去了。但类似的情况可遇而不可求,在这些自杀成功案例中,起决定性作用的是外部的其它因素,而不是你。如果你自己来尝试自杀,那你就会失败,几乎每次都是——因为\indicate{自杀是处于你能力范围之外的事项}。没有能力指定切实可行的自杀计划,所以只能碰运气,所以才每一次都没死成。并且,你还经常会因为心理保护机制而丢失自杀相关的记忆,新增一个无法触碰的心理阴影。

\begin{toparent}
可能一些正常人读者在看到这里的时候,已经对本指南厌恶至极,觉得本指南在教唆青少年自杀,是不折不扣的邪典,从而去举报封禁一条龙了。但正如上面的分析所说,讨论自杀的话题并不等于教唆自杀,这些“只是内向了一些/不知道在发什么疯/大概过了青春期就好了”的青少年,实际上缺乏自杀能力。告诉他们这一点也不会增加他们的自杀能力。我不否认有些读者在看了上述论述以后会尝试自杀,但总归他们不会成功(对于少部分读者来说,这个判断有可能刺激他们坚定自杀意志,但还是建议这些读者在整体看完本指南之后再做决定)。

如标题所述,本指南讲的是自救的方法,立场是让尽可能多的人幸福快乐地生活下去。但读者群体中相当一部分实际上不觉得自己有资格得到幸福。他们熟悉的是自我否定,而自杀几乎是他们每个人都想过的事情,从更熟悉的事项出发,更容易学到真实的东西。如您在后续阅读中仍然怀疑本指南的立场问题,请参考本段解释。如您在读完全书后仍然觉得本指南有害青少年心理健康,本指南尊重您的立场和行为。
\end{toparent}
成功的自杀方案需要缜密的设计,确保其中的每一步都可以执行。上述的四种失败情况,都可以从“方案设计与执行”的角度来考虑:“脱离了想死的状态”是一拖再拖,实际上等于放弃了计划,从来没有开始过;“想死的念头被打断”是无法排除周围环境干扰,导致不具备开始计划的条件;“没能达到死亡条件”是纯粹的能力不足,有无法克服(甚至无法察觉)的阻碍;“被生物本能阻止了”是意志力薄弱,被自身的软弱和恐惧压垮。

这些特征,与你失败的人生别无二致。你不止在自杀这件事上缺乏能力,同时也普遍地缺乏处理各种现实事务的能力,只能被纷至沓来的不利事项裹挟着,流向未知且恐惧的远方。\indicate{行动力够强的人已经死了,你因为干啥啥不行才活到了现在},并且也将继续痛苦地活下去。无论你是想要想要成功地自杀,还是想得到爱,还是把握住自己的命运,还是有什么更宏大的目标要实现,只要你想改变点什么,就必须走上\indicate{学习处理现实事务}的道路,搞清楚怎么做计划,怎么执行。这是唯一可行的自救之路。
\setcounter{secnumdepth}{2}

% 使用\titleformat调整\chapter格式
% 用了ctexbook以后仍然不能删,ctexbook的默认格式中,正文里不带章号
\titleformat{\chapter}[display] % shape参数
{\normalfont\Huge\bfseries\raggedright} % format参数
{第\chinese{chapter}章}        % label参数
{20pt}                                   % sep参数
{\Huge}                                  % before-code参数
[] % after-code参数

\titlespacing{\chapter}{0cm}{3.5cm}{2.5cm}
\begin{savequote}[400pt]
    \fontsize{8pt}{8pt}
    请赐予我平静,去接受我无法改变的;给予我勇气,去改变我能改变的;赐给我智慧,分辨这两者的区别。\footnote{出自雷因霍德·尼布尔的祈祷文。相关如下:grant me the serenity to accept the things I cannot change; \\\indent courage to change the things I can; and wisdom to know the difference. \\\indent 尼布尔并未命名,后世将其命名为《宁静祷文》。}
    \qauthor{雷因霍德·尼布尔(Reinhold Niebuhr)}
\end{savequote}
\chapter{事务处理的三要素}
%\titleformat{\section}{\bfseries}{}{0em}{}
\section{从买水开始说起}

为了提升事务处理的能力,我们首先需要明白应该如何理解“事务”。比如,大多数人会不可避免地曾经经历过这么一件事:\indicate{怎么买一瓶水?}

对大多数人来说这不是什么问题,大家都顺利地买过水。但是对于少部分人来说,这却是不可逾越的障碍。我们可以随意列出几种买水的方案:
\begin{itemize}
\item 柜台型:向店员说出自己的需求,然后接过店员递来的商品(这种买水流程在现如今已经较为少见,不过在饮品店依旧广泛存在);
\item 超市型:自己挑选好水以后,去收银台结账(在超市和小店均有这种类型);
\item 自选型:自己挑好水以后通过无人柜台结账(超市的无人结账区、自动售货机、无人超市均为这种类型);
\item 外卖型:直接下单,然后等待非接触送达。
\end{itemize}
以上四种都是可行的流程,大多数人也都能顺畅执行。但总有少部分人会遇上各种各样层出不穷的问题,对于一个社恐的人,前两种需要人际交互的流程算是没什么指望了;对于一个不熟悉电子设备的人,自如地使用后两种流程是不可能的。当我们想要教他们克服障碍的时候,我们经常发现他们无论如何都听不进去。

无论你是告知孩子“这没什么好怕的”,和孩子展开分析和讨论,或者直接命令孩子去做,孩子始终不会如你所想,克服了心理障碍,大大方方地去买水,而只是一如往常地低着头站在角落,看手机或者玩衣角;无论你是给长辈找教程,手把手地教长辈怎么点开app,还是贴心地帮长辈绑好银行卡和账户,长辈在买水的时候仍然永远不会想到网购,去超市的时候永远走人工通道,遇到电子设备就说“自己不会”。

对于不同的人,同一套操作的执行难度可能会有很大差别。上述两类人在买水的时候,都会遇到\indicate{不可解决的困难}。困难有可能是纯粹的心理因素,但如果不想着去解决困难,那么他们就不可能被教会。当第一遍教学没有起到效果时,后续重复的教学也将恰如第一遍一样不会起到效果。他们不是不知道买水的正确做法,而是无法执行。

当然,也有“不知道正确做法”类型的困难。对于“把一个无法顺利买水的人教会”的事务,如果不知道有效的教育方法,只是一味重复,那就不会成功;对于“克服自己的社恐”的事务,如果不知道自身恐惧的来源和正确的社交方式,只是一味害怕自己做错事,那就不会成功;对于“接触电子设备”的事务,如果不了解电子设备,也不积极接触,只是一味担心自己误操作导致不良后果,那就不会成功。

为了之后叙述方便,我们先不严谨地\footnote{毕竟现在我们既没有定义“事务”,也没有定义“复杂”和“能力”。这些定义会在稍后补上。}把事务划分为两类:\indicate{简单事务}和\indicate{复杂事务}。概括地说,简单事务是指在自身能力范围之内,自身能够顺利解决的事务;与之相对,复杂事务是指自身有能力缺失,从而无法顺利解决的事务。“买水”对于大多数人来说是一项简单事务,对少部分人来说是一项复杂事务;“把一个无法顺利买水的人教会”对大多数人来说是一项复杂任务;“克服自己的社恐”对于社恐的人是复杂事务;“接触电子设备”对电子文盲是复杂事务。

\begin{examples}
任何形式的“克服困难”都是复杂事务,因为困难本身就说明了这个人有能力缺失(但“学习”不一定是复杂事务)。对大多数人来说,“自己产生感情”一般而言算不上是事务,但“理解自己为什么会产生某种感情”是复杂事务,“理解他人为什么会产生某种感情”也是复杂事务。“规划自己的未来”是复杂事务。“找到人生意义”是复杂事务。“脱离原生家庭影响”是复杂事务。“追寻自由”是复杂事务。“获得坚定的意志力”是复杂事务。“制定并完成长期目标”是复杂事务。“学会如何画一幅好看的画”是复杂事务。“学会如何写一手好听的歌”是复杂事务。“挑选合适的礼物”是复杂事务。“改变别人”是复杂事务。“得到别人的爱”是复杂事务。“把某项技能教给别人”是复杂事务。“为别人操心”是复杂事务。“和别人合作”是复杂事务。“搭人脉走关系”是复杂事务。“带领团队”是复杂事务。“维持体系平稳运行”是复杂事务。
\end{examples}
面对复杂事务,直接动手处理不会有什么效果。不排除一些人在一些方面具有充足的能力,但对于大多数人来说,如果恰巧达成了目标,只能说明你运气比较好,无法达成目标才是正常情况。面对复杂事务总是失败,\indicate{不是态度问题,而是能力问题}。

本指南的前三章用于讲解事务处理所需的能力。其中本章讲解事务处理的通用流程,第二章讲解个人所需能力的具体内容,第三章讲解复杂性的由来和对待复杂性的方式。

\section{处理复杂事务的相关能力\label{sec:现实事务处理能力}}
本指南使用\indicate{事务}这一词汇时,一般有两种含义:一种是指“一个目标”,另一种是指“处理事务的所有操作”,其中\indicate{处理}是指“从当前情况出发,为实现目标而执行一系列操作”。在一些语境下,我们也使用\indicate{目标}、\indicate{改变}来替代\indicate{事务}。

在这些操作中,我们把可以实现目标的称为\indicate{有效操作},\indicate{解决}或\indicate{完成}则特指“执行有效操作”;把不能实现目标的称为\source{无效操作},“解决/完成/处理事务\indicate{失败}”特指“执行无效操作”。在其它时候,\indicate{有效}/\indicate{无效}也被用来指代“有助于/无助于解决事务的”。\indicate{尝试解决}这一搭配等同于\indicate{处理}。

想要解决复杂事务,唯一可行的方式是先提升自己的能力,在能力足够正确应对之后再着手处理。提升能力的方式可以大体分为两种:一种是外界的\indicate{输入},本指南将其称为\indicate{眼};另一种是内部的\indicate{分析},本指南将其称为\indicate{脑};而\indicate{对事务的实际处理},或者称为自身的\indicate{输出},则被本指南称为\indicate{手}。本指南将眼、脑、手这三个方面,称为事务处理的三要素。

\subsection{眼}
\indicate{眼}这一要素,是\indicate{从外界获取能力提升}的所有方式的总称,可能包括接受教育、跟随课程、观看教材、阅读资料、实地探访、实验取证等多种不同的形式。
\begin{examples}
因为要考试所以学知识,游戏出新角色了看看技能介绍和讲解,有了新工作让老员工带一带,上线了新系统翻一翻操作教程,结婚之前先观察对方的性格,诸如此类,都是眼的发挥空间。
\end{examples}
一些读者会注意到,上述的例子也都可以视为事务。其拥有非常明确的目标:获取解决原有事务所需的知识或信息\footnote{关于知识和信息的更详细讨论请参见\hyperref[sec:知识与信息]{第二章}。}。本指南将这一类型的事务称为原有事务的\indicate{调研}事务,将“从原有事务获取调研事务”的过程也称为调研。
\begin{examples}
上述例子中,“学习相关内容”是“考试”的调研事务,“看技能介绍与讲解”是“游戏上分(或其它类似目标)”的调研事务,“接受老员工的教育”是“承担新工作”的调研事务,“阅读操作教程”是“使用新系统”的调研事务,“观察对方性格”是“结婚”的调研事务。
\end{examples}
调研会从一个事务A出发,获得其调研事务B。我们一般希望解决B有助于解决A——起码B得比A简单。
\iffalse\begin{examples}
面对一个新的手机应用,想要了解它的用途的时候,我们一般只需要把每个按键单独点一点,看看文字说明即可。如果想执行一些复杂的功能,也只需要上网查看教程。“学会编程并且靠自己把该应用的代码逆向破译出来”在绝大多数情况下都不应作为实际选择,除了“想学习它的编写思路”或“想攻击该应用”等少数几种情况以外。
\end{examples}\fi
\begin{examples}
如果你需要完成一个项目,其中需要一些你无法产出的资源(比如绘画或者代码段),那就应该首先考虑寻求外部支持,以外包或友情帮助等方式获取这些资源。“自学相关领域”一般不作为“完成项目”这一事务的推荐选择,特别是在时间紧急的情况下。(目标就是“提升自己”的情况不在此事例讨论范围内。)正确的调研事务应为“寻找资源”。资源可能以网站、群聊、商单、AI等多种形式存在,依据不同的资源类型而有所差别。
\end{examples}
当原有事务较为简单时,它的调研事务通常仅有清晰的单一步骤。
\begin{examples}
在商场里找卫生间,看路牌或者问售货员都行;买了新工具,看说明书或者教学视频了解它的用法;为了打过某关,直接抄别人的答案;去营业厅办理业务,听从工作人员的指挥;遇到不知道的新闻,去搜一搜具体报道。只需要简单调研就能完成的事务在我们的生活中比比皆是。
\end{examples}
对于复杂的原有事务,它的调研事务大体上来说也会更为复杂。其复杂性主要分为两方面:一方面是调研事务不一定能完成,由此需要多种不同的调研事务,逐一尝试\footnote{对于\rigorous,可以将“调研”视为一种独立的事务,此时“当前情况”即为“原有事务”,而“目标”为“获取有效知识和信息”,“处理每种调研事务”视为“调研”这一事务的一次操作。};另一方面是调研事务本身还需要再次调研,由此产生调研的递归。
\begin{examples}
    对于事务A“学一门新技能”来说,它的调研事务可能包括B“看某一门课程”,也可能包括B'“看某一本教材”。而B的调研事务又可能包括C“了解需要什么配套教材”,C的调研事务有可能包括D“查找电子版教材”。而E“观看课程评价”可作为是否选择B的依据,也是A的调研事务之一\footnote{对于\rigorous,可以将E视为O“对事务A的调研”的调研事务之一。后续处理手法类似,不做重复演示。}。
\end{examples}
本指南将“有效地进行调研,从而在总成本尽可能小的情况下解决事务”的能力称为\indicate{调研能力}。这里的“成本”因人而异。希望大家注意,“时间”也是成本之一,并且很多时候是最主要的成本。

调研能力缺失是很常见的事情,不需要为自身的调研能力不足感受到过分焦虑。面对复杂调研,可能出现的情况包括但不限于“无法坚持长期调研(比如学东西半途而废)”、“无法获取有效信息/缺少获取有效信息的渠道”、“陷入无限递归/死锁”。

复杂调研是需要系统学习与训练的能力,并且可以通过系统学习与训练来培养,详细讨论见\hyperref[sec:识别方法与判断]{第二章的相关内容}。若完全没有调研能力,面对简单调研时也不知所措,或是完全没有调研的意识,则应该先从改正自身认知与行为做起——这是另一个复杂事务。

\subsection{脑}
\indicate{脑}这一要素,是\indicate{从内部获取能力提升}的所有方式的总称,可能包括思考、整理、规划、推演、分析、总结等多种不同的形式。与事务处理相关的主要有两方面:一方面是“整理来自外界的知识”,本指南将其称为\indicate{分析};一方面是“得出应当执行的操作”,本指南将其称为\indicate{计划}。与调研一样,分析和计划都可以看作事务。
\begin{examples}
    根据原有事务的不同,分析和计划环节可能会有不同的比重。对于某些简单事务(如使用某种新工具),在调研环节(查询新工具的使用方法)结束后,按照流程使用即可,此时几乎不需要分析和计划;“调试某个参数”一类的事务几乎不需要计划;很多需要长期投入的简单事务(如坚持锻炼)几乎不需要分析;大部分复杂事务既需要分析也需要计划。
\end{examples}
\indicate{分析}和原有事务没有直接关系,只与获取的信息有关,具体定义与讨论见\hyperref[def:分析]{2.2.3小节},本章不过多论述。读者现在仅需知道分析是“从已知信息出发,获得更深入的认识”的过程。\\
\indicate{计划}是“根据原有事务以及分析结果,产生新事务”的过程。我们将产生的新事务称为\indicate{计划事务}。与调研环节类似,我们同样希望计划事务比原有事务简单,解决计划事务有助于解决原有事务。

当原有事务较为简单时,计划事务通常仅有清晰的单一步骤。在强调“从原有事务产生\indicate{唯一}计划事务”,或“产生的多个计划事务只有\indicate{一个}重要”时,也将这个过程称为\indicate{转化},将该计划事务称为\indicate{转化事务}。
\begin{examples}
“在开始之前做好准备”是一类简单清晰的计划事务,可能包含“出门带好钥匙”“上车/游览前买好票/办好卡”“办理业务前带好相关文件”等。有时计划事务会与调研环节相结合,如“找新地方时先查好路线”“查找指南并依据指南整理相关资料”等。

\end{examples}
对于较为复杂的原有事务,计划事务通常不唯一。多个计划事务之间可能可以并行,也可能有先后次序,需要依次执行。本指南将“计划时产生多个计划事务”的过程称为对原有事务的\indicate{分解}或\indicate{细化}\footnote{对于\rigorous,可以将分解视为“从一个事务产生多个事务”的唯一方式,并据此重写调研事务的定义,将调研事务视为“调研”的分解事务。}。

在强调“从原有事务产生\indicate{多个}计划事务”时,也将这个过程称为\indicate{分解},将这些计划事务称为\indicate{分解事务}。我们把“解决一个分解事务”称为处理原有事务的一次\indicate{操作}或者一个\indicate{步骤}。
\begin{examples}
分解事务常见于长期规划。“读完一本书”是复杂事务,但是可以分解成数个“读一章”;“学完一门课”是复杂事务,但可以分解成数十个“听一节课”;“锻炼身体”是复杂事务,但可以分解成数百个“每天跑步”。
\end{examples}
无论是调研事务还是计划事务,我们统一将“从原有事务产生出的事务”称为\indicate{子事务},同时将原有事务对应地称为子事物的\indicate{母事务}。与调研环节类似,对于复杂事务,计划环节所产生的子事务同样有可能产生递归。更复杂的是,计划事务有可能自身需要调研事务,计划环节也有可能需要调研得到的信息,来确定具体的分解。
\begin{examples}
当你决定入坑并且好好去玩某一款游戏的时候,直接去看教程(典型的调研事务)通常收获较少。你总得先游玩一段时间,起码熟悉基本的键位和操作,对游戏流程有基础了解(这一环节有吸收知识的效果,但主要还是分析和计划)之后,再去观看教程,才能跟得上讲解。在此之后,仍然需要多次反复的听讲-练习的组合。当你水平足够高时,你才能明白什么样的教程对你有用,什么样的教程和你不对口,才能选择性地观看教程,更高效地吸收外界知识。
\end{examples}
与调研能力一样,计划能力缺失是很常见的事情,不需要为自身的调研能力不足感受到过分焦虑。面对复杂计划,可能出现的情况包括但不限于“无法坚持长期计划(比如锻炼)”、“分解与转化的方式无助于实现原有目标”、“陷入无限递归/死锁”。

复杂计划是需要系统学习与训练的能力,并且可以通过系统学习与训练来培养,详细讨论见\hyperref[sec:方法与过程]{第二章的相关内容}。但若完全没有计划能力,面对简单计划时也不知所措,或是完全没有计划的意识,则应该先从改正自身认知与行为做起——这是另一个复杂事务。

\subsection{手}
\indicate{手}这一要素,是\indicate{对事务的实际处理}的所有方式的总称。相比于眼与脑,手的覆盖范围要大得多:所有事务的处理都可以被视作手的一部分。前文提到的调研、分析和计划,因为都可以视作事务,严格来说调研能力、分析能力和计划能力都可以视为手的相关能力。不过除了这种广泛性的含义之外,还有一个语义上更精确的概念:\indicate{执行能力}。

\indicate{执行能力}特指“直接解决一个不继续分解和转化的事务”的能力\footnote{由此可以回头补足之前对“简单事务”和“复杂事务”的定义。}。它有清晰的结果导向判断标准:如果能解决原有事务,那就拥有(对该事务的)执行能力;如果不能解决原有事务,那就缺乏(对该事务的)执行能力。
\begin{examples}
长期规划,如“每天读书”、“每天听课”、“每天运动”等等,是一类种比较常见的主要需要执行能力的情况。在实际执行中还会遇到如“时间规划”、“突发事件”、“拖延推后”等一系列问题,但大多数人不会想着去专门解决这些,而是靠自律和意志力坚持。这就变成了纯粹对执行能力的考验。
\end{examples}
很明显,执行能力的水平与调研能力、计划能力的水平相挂钩:调研和计划的水平越高,就越能将复杂事务分解转化为简单事务,直接解决每个子事务就越容易,执行能力也就越强\footnote{这确实更改了需要处理的事务,但“直接处理复杂事务”和“直接处理每个子事物”都符合执行能力的判断标准,不符合判断标准的是“经过分解和转化处理复杂事务”。}。
\begin{examples}
《拉封丹寓言》中有一篇名为《老鼠开会》,讲了这么一个故事:一群老鼠开会讨论“如何不被猫抓走”,有一位老鼠提了建议,“只要把铃铛挂在猫脖子上,听铃铛的声音就可以提前躲避了”,但问遍所有老鼠,也没有谁想把铃铛往猫脖子上挂的。作者的本意是用来讽刺只会空谈而无法贯彻政策的官僚,但这同时也可以视作“因为计划能力不足,从而将事务错误地转化成了更难的事务”的例子。无论是转换前的“不被猫抓走”还是转化后的“给猫挂铃铛”,都不是老鼠的执行能力可及的范围。
\end{examples}
除了调研和计划能力外,执行能力还受到很多其它方面因素的影响,其中最为人重视的可分为三方面:技能水平、意志力、外部条件。意志力在此不展开讨论,请参考\hyperref[def:意志力]{2.3.3小节}。
\begin{examples}
技能水平对人执行能力的影响是有目共睹的。刚上小学的普通儿童绝不可能靠自己作出微积分题目;刚下车间的实习生也不可能有老师傅的熟练操作;给业界外人士讲解业内共识只觉得隔行如隔山;门外汉看着专业人士的成果只觉得难如登天。在某件事务上,因技能水平的差距导致的执行能力差距,绝不可能仅靠意志力就补足。意志力只能用在“提升技能水平”上,而不可能直接用于解决该事务。强行解决只会错漏百出,除了运气实在很好以外不可能有成功的机会。
\end{examples}
外部条件则具有补全个人能力不足的效果\footnote{如果回看关于“处理事务”的定义,我们会发现,外部条件更改了“当前情况”,于是某种意义上来说,可以认为是换了一个更简单的事务来处理。}。依照条件的不同,能起到的效果从“稍有帮助”到“完全解决”都有可能。
\begin{examples}
良好的教育有助于学生更深入地接触学科;行业规范有助于从业者更高质高效地完成任务;机器加工使得工人不必依赖手工熟练度;辅助系统使得飞行员的培训周期大幅度缩短;靠谱的乙方可以完全解决某个困扰已久的疑难问题;AI助手可以完全替代文书工作。
\end{examples}
由于技能有数不清的领域,每一个人势必在绝大多数领域缺乏执行能力——但这不会有什么影响,因为那些领域大都是一辈子都接触不到的。只有那些会造成显著影响(如经常接触,或是会对未来产生关键影响的事件)的重要领域,缺乏执行能力才会带来严重的后果。调研和计划能力会给人带来技能水平、意志力、外部条件三方面的优势,从而显著提升执行能力。要是换个表述方式的话,这就是“\indicate{磨刀不误砍柴工}”的道理。

\subsection{总结}
在这一小节中,我们定义了事务,将事务处理能力细分为了三个子方面:眼(调研能力)、脑(计划能力)、手(执行能力)。这些内容经常会出现在日常生活方方面面的讨论之中,对大多数读者而言,应该都是熟知的。应该也有不少读者在之前没有接触过这些,如果你们在看完了本小节以后感觉有所收获,本指南为你们感到高兴。

另一方面,应该也有读者听这些内容过多,认为这些不过是正确的废话,是空泛无用的大道理,对改善现状毫无帮助,进而认定本指南就是本纯粹的鸡汤书,没什么用可以扔了。对此,请允许本指南阐明如此安排的打算。将事务处理的三要素放到开篇来讲,不是为了让读者在看完这一小节后,就立即充分掌握并能熟练运用。这种安排旨在向读者展示本指南的\indicate{主要分析思路}。事务处理能力会作为全书的主要分析手段,随着分析框架的不断搭建而反复使用,读者的理解也会随之逐渐深化。作为一切思路的出发点,只有本身足够简单明晰,才能够被大多数人理解运用。为此,本指南尽量全面而准确地定义和讲解了需要用到的概念,力求每位读者都能够准确无误地理解。

选择事务处理能力作为本指南的主要分析思路,原因有二。其一是:虽然这在很大程度上是广为人知的大道理,但很少有人能够熟练运用。大多数人在实际处理事务的时候,从来不会按照这一流程行动。虽然每个人都会计划,但大多数人只是草率地将事务转化成了另一个自己无力解决的事务,称不上有计划能力。即使遭遇反复的失败,大多数人也不会系统总结反思,无法识别自身的能力和条件缺失,称不上有分析能力。大多数人不会主动寻求改变,不会有意识收集信息,称不上有调研能力。
\begin{examples}
先前提到的《老鼠开会》寓言就是一个很好的例子。在提建议的老鼠看来,它将“不被猫抓走”转化成了“听铃铛逃跑”,极大降低了实现难度。但它没有意识到,这么转化需要“有老鼠给猫挂铃铛”的前提,由此错判了实现难度。“无法解决简单事务”通常会被人们归结为态度问题,而“将复杂事物错判为简单事务”则会使人将能力问题错判为态度问题,从而会产生不满和指责。这种人与人之间的冲突毫无道理也毫无意义,应该尽力避免。
\end{examples}
其二是:事务处理能力确实是一种强大且应用范围十分广泛的能力。一切目标都可以视作事务\footnote{可以回看数页前对复杂事务的列举,以确认自身理解正确。},而目标来自各个构成十分不同的领域,这导致对应的调研、分析、计划环节也会因目标所属领域的不同而内容迥异。由此,我们从中提取的共性也就容易显得空泛,只有少数几条大道理。然而,即便只依靠这几条大道理展开分析,我们也足以在一些问题上得到十分有力的洞见(下一小节就是例子之一)。

本小节涉及的大道理分为两部分。一部分是三要素的具体划分与操作,在前文已详细展开,这里不再赘述;另一部分是事务本身的性质:对原有事务的处理会产生新事务,子事务和母事务最大的不同,在于它们\indicate{拥有不同目标}。目标不同可能导致操作完全不同,甚至有利于子事务的操作经常不利于母事务。学会区分不同的目标,尤其是在处理子事务的时候,不受母事务目标影响,是相当重要的能力——\indicate{就事论事}。这是换位思考的基础,也是理解感情,理解他人,理解社会的基础。

\section{实操:为什么我不建议你去看心理医生\label{sec:实操1}}
不少读者看这本书应该有着明确的需求,希望能通过这本书解决自己的痛苦、迷茫、绝望、抑郁等等一系列负面状态\footnote{其中有一部分应该还等着下一段谈论自杀的章节,不过这一节不是。在看过上一节以后,你们应该对“为什么自杀很困难”有了进一步的认识。我们现在的理论准备还不够充足,相关段落见\hyperref[sec:再谈死亡]{2.4.2小节}}。大多数人应该都听说过心理咨询,应该也有些人实际体验过。一些书籍和博主会将心理咨询视为解决心理问题的灵丹妙药,毕竟这确实是整个地球上最有效的方法。但另一方面,心理咨询却并不是一种对所有人都能起效的方式。

面对繁杂的心理问题和完全会把自己压垮的负面状态,如果你的目标是\indicate{尽快}获得相对正常稳定的生活,你就应该\indicate{果断地}去正规医院的精神科或精神专科医院,接受全面的专业诊治(千万不要自己诊断自己开药,即使你可以通过某些途径获得药品)。药物会立即使你摆脱超出自身承受范围的情绪,这是心理咨询完全做不到的。我们不应该将心理咨询视为诊治,虽然我们习惯性这么称呼。以下的篇幅中,请读者自觉使用咨询师代替心理医生的称呼,使用来访者代替病人的称呼。

如果你虽然状态很差,但因为财力、社会关系(如一些医院接诊未成年人需要监护人同意才会给出诊断和处方,而监护人对精神疾病有根深蒂固的偏见)等原因而客观上缺乏接受治疗的条件,或是已经在服药但仍然对当前生活相当不满,那就只能尝试自己解决自身的问题了——本指南可以在这方面帮到你,但仍然不建议你直接去心理咨询。

即使我们只考虑有充足能力\footnote{具体定义见\hyperref[def:心理咨询能力]{2.4.3小节}。}的心理咨询师,而不考虑由于职业水平不足和职业道德缺失而导致的无效咨询(一些有咨询师职位的人甚至完全不承认心理疾病的存在),心理咨询也不是百分百有效。心理咨询师无法直接干涉来访者的意识,只能采用一些间接手段,无法开药而只能沟通。

心理咨询主要有两方面问题:成本高和见效慢。成本主要分为金钱成本和时间成本。一次有明显成果的心理咨询,持续时间经常以月计甚至以年计。很多来访者无法坚持这么久,经常因为一些其他原因(如自身兴趣减弱、其它麻烦事情“总是”占用时间、对心理咨询或咨询师本身产生抗拒、家庭或学校或单位施压等)而不再参与心理咨询。一次持续一小时的咨询需要花费数百元,每周安排一两次的话,持续数月的咨询总花费很容易过万,这是一笔不小的支出,很多人难以负担。一旦中途放弃,前期的金钱和时间花费就全部打了水漂。

即使经济富裕且时间充足,你也不一定能收获一次成功的心理咨询。心理咨询的见效慢,由其内在特征决定——心理咨询是个复杂事务,不是对来访者而言,而是对咨询师而言。虽然来访者可能在挑选心理咨询机构的时候很费精力,在安排时间的时候处理了一大堆麻烦,为了稳定来访而大幅调整了生活模式,甚至将“得到有效心理咨询”本身视为了人生目标之一,但对某一次具体的,只有来访者和一位特定的咨询师参与的心理咨询而言,来访者只需要按节奏走,而咨询师需要考虑的就多了。

咨询师所要面对的事务是“解决来访者的负面状态”。这可以大致分为两个方面:其一是“帮助来访者摆脱来自过去的困扰”,其二是“使来访者能够健康地面对世界”。这两个方面的解决相辅相成,心理咨询和心理治疗的一切手段,包括但不限于谈话、催眠、沙盘,都是为了这些而服务。其中,前者不可避免地需要充分细致地了解来访者的过往经历和当前情况,而后者虽然看着和来访者的具体情况没有必然联系,但是这也几乎不可能在对来访者一无所知的情况下完成。

一名心理咨询师如果只会谈大道理,那么或许可以成为良好的陪伴者和精神支柱\footnote{另一方面,来访者对咨询师产生依赖也是非常不利的情况,一方面这会大幅影响咨询进展,另一方面若是咨询师没有良好的职业道德,来访者将会轻易地被欺骗和伤害,并且在相当多产生依赖的来访者看来,情愿被咨询师伤害也不愿远离。概括来说,我们不应允许任何形式的发生在咨询师和来访者之间的感情(普通情绪,如“感谢”“怀疑”,是可以允许的)。},但对解决问题没什么帮助——咨询师口中的大道理也可以从别处听到,“理解每一条大道理”都对来访者来说都是复杂事务,而来访者既然受到问题所困,那一定缺乏处理复杂事务的能力。有些来访者确实可以通过“与咨询师建立了长期且健康的深度关系,从而有了心灵归宿,对生活的态度改变了”的方式获得彻底的疗愈,但这一般而言持续时间过长。大多数情况下,这一类咨询师仅起到稳定心理状态的作用,来访者的疗愈还需要通过生活中偶遇的其它契机来达成。

另一部分咨询师会根据来访者诉说的情况,针对性地给出具体的方案。这些方案既包括“心理上如何对待”,也包括“行为上如何操作”(如果你去拿着自身遇到的问题咨询AI助手,那么AI就会给出这种风格的回答)。这一类的咨询师承担了“替代来访者处理事务”的功能,补足了来访者的能力缺失。对来访者来说,实践新方案也能切身体会咨询师给出的建议,并从中有效学习咨询师所使用的分析和处理手法。这最终会使得来访者彻底理解(经常是咨询师没有明说,但来访者自己总结出)大道理。对于一些情况较为简单,只面对某些特定种类问题的来访者,大道理足以补足自身的短板,由此产生的心理问题也得以彻底解决。

但不少人情况更为复杂,经常感觉整个人生都充满灰暗,每一个方面都无比绝望。他们会\indicate{因为某个根深蒂固的思考回路,在每次遇到问题时都重蹈覆辙}\footnote{具体展开请见\hyperref[sec:人的意识演化基本模型]{3.4节}}。在外人眼中无比相似的问题,在自己眼中却各有各的困难,每个都完全不可克服,只让人想着逃避。他们完全无法应用新方法,即使内心知道这么做是正确的,也总是想不起来也做不到。这种时候,向AI或者这类咨询师询问“特定问题的处理方法”,就只能每次都得到相同的回答,也就不会有什么用了。

阅读本指南的读者中,应该有不少都会自认为是情况复杂的这一类。心理咨询师想要解决复杂的情况,就必须对来访者有全面深入的了解。咨询师拥有这方面的专业技能,但这毕竟是一个复杂的调研事务。如果只靠咨询师的努力,这一过程注定相当缓慢。来访者自身的表达能力同样大幅度影响了咨询师了解的速度。

这里提到的表达能力不完全等同于说话量,它的核心在于“使对方能够听懂”而不是“自己把想说的话都说了”。对心理咨询来说,来访者需要向咨询师表达自身的情况。其所能表达的极限取决于来访者对自身的了解。这是两个复杂事务的综合,一个是“培养自己的表达能力”,一个是“了解自身情况”。这两方面都可以得到心理咨询师的辅助,但自身的表达能力越强,对自己了解越充分,心理咨询也就越快。而表达能力越弱,对自己了解越不充分,就越不适合心理咨询,去了就越有可能做无用功。
\begin{examples}
表达能力缺失有很多种不同的表现。最容易理解的是“面对咨询师完全一言不发”,但这还算是相对好处理的情况,来访者明确知道自己自身表达能力缺失。更为棘手的情况是“来访者很能说,但说出来的东西都是无效信息”。其中一种情况是“来访者经常性地重复某些相同的信息”,常见的比如说谈及“自身情绪”“对某些事物的观点”等等。来访者不自知自身在重复,也不知道自己说的对咨询师来说是无效信息(反而“来访者会在这方面重复”是个重要信息)。另一种情况是“来访者经常性地脱离当前事务”,同样,这经常体现为谈及“自身情绪”“对某些事物的观点”等等。来访者不知道自己过于发散,归纳整理散乱的信息也会拖慢咨询师的了解速度。
\end{examples}
在当今社会,一个人只要能接触到网络,就有条件接触到丰富的心理科普资源。如果你对这方面感兴趣,那更是会学到数不清的知识。你的书架里可能有上百本教你如何摆脱人生困境的书籍,你可能每天都会刷到那些很有道理的心理学概念。经常看这些的来访者会向咨询师说更多,但说的内容不一定是有效信息。来访者如果提前进行了自我诊断,只向咨询师询问“原生家庭不好”“习得性无助”的解决方法,咨询师也很难帮上什么忙。

当然,客观上来说,这些心理科普资源能促进对自身的了解。纯粹通过看书而摆脱负面状态的人也不在少数。大家能接触到的大多数科普文章或者科普视频,可以起到的效果大概可以等同于大道理式咨询师;而一本心理科普书籍能起到的效果,大概在大道理式咨询师和AI之间。如果接受系统的教育,可能会更有帮助,但同时也有可能因为教育质量问题而迷失在海量的概念中\footnote{对此的展开讨论参见\hyperref[sec:污染]{2.2.4小节}和\hyperref[sec:人的意识演化基本模型]{3.4节}}。本指南并不宣称“会比AI起到的作用更大”之类口号,而只是一次尝试:如果行文的基石不是“心理问题”而是“现实事务处理能力”,如果将心理问题本身视为应当被剖析的概念\footnote{有些知识面较广的读者可能认为这样的思路很精神分析,但本指南不宣称自己是一本精神分析书籍。},是否有助于读者更深入地了解自身?

意识\footnote{这里的“意识”作为一个统称,包含人类所有的意识现象,一些语境下的“潜意识”也包含于其中,读者可自由使用“精神”“灵魂”“自我”等自己觉得更合适的概念进行替换。具体展开参见\hyperref[sec:意识与自我]{2.3节}。}作为一种客观存在的现象,拥有其内在的规律。这些规律稳定且可以被认知。任何试图违背这些客观规律,仅凭命令与强迫而进行的,对一个人的意识塑造和改变,都只能无效或极端低效、事与愿违或因小失大。由于篇幅和聚焦点原因,本书仅讨论意识层面的内容%,并且以“处理事务”作为意识活动的基本单位\footnote{该定义可以覆盖“目标与努力”“情感”“条件反射”“邱奇-图灵命题”等多个对意识活动的不同观察视角,具体展开参见2.3节与第三章}
。人的生理活动还有其它方面,本指南只关注它们和意识活动有关的部分。
\begin{examples}
    例如,精神药物会对人体产生多方面影响,其对意识的影响直接影响可能有“更难沉浸于某个思考回路”“主动性更弱”等,间接影响可能有“注意到自己发胖懒惰并为之感到自卑”等。若服药时不在意自身的身材变化,那就没有“感到自卑”这一项;若下定决心锻炼,那就多了“下定决心锻炼”这一项。根据人的不同,同一原因对意识造成的影响可能有相当大的差异。
\end{examples}
人的所有主动行为都有意识活动参与其中(即使人可能对于其中相当多的行为没有自觉),研究意识活动在一定程度上等于研究人的所有活动,并且研究对象更为集中,研究手法更为统一,更容易理清思路。而另一方面,对自身意识的研究,是普通人能够接触到的,对自身心理状态调整最有效(很多时候是唯一有效)的方式。根据程度的不同,一个人的负面状态可以被细分为“情绪问题”“心理问题”“心理疾病”“精神疾病”等多个不同的层次\footnote{这里的划分并非学术和医疗规范,而是基于普通人经常讨论的语境而提取的概念。词语仅大致表明“一般人眼中疾病的严重程度”,所有词语都不严谨地用于指代“心理、精神、意识方面的病理性现象”整体,不表示本指南的划分方式。}。除了少数精神疾病是因物理/化学/遗传因素导致的神经系统生理损伤外,几乎所有负面状态都有长期而复杂的形成原因。一切心理问题都可归结为\indicate{内心与外界环境的不适配},一切心理疾病都是慢性病。想解决它们必然要涉及对意识这个复杂系统的深入认识。

本指南希望这样的尝试有助于更多人改善自身的心理状态。充分了解自身,再配合上合适的新处世方法,在很多时候,足以彻底解决自身的遇到的问题。如果成效没有这么好,起码在充分了解自身之后再去找心理咨询师,也更能讲述清楚自身所遇到的问题,更能得到咨询师的有效帮助。

再次强调,如果你觉得自己当前的心理状态令自己无法忍受(但又死不掉),脱离当前心理状态最快的有效方法就是药物治疗。这一步骤一定要听从专业医生的建议,从正规医院处获取诊断。药物治疗可能不治本,但它至少治标。药物治疗与其它方案并不冲突,药物可以使你远离频繁失控所带来的干扰,使其它方案发挥更深入、更广泛、更本质的效果。


\titlespacing{\chapter}{0cm}{4.5cm}{1.5cm}
\begin{savequote}[250pt]
    \fontsize{8pt}{12pt}\selectfont\fontsize{8pt}{12pt}
    他们说要好好听话,又说乖小孩没想法。\footnote{\bilibili{av6636086};\\\indent 
    \netease{467394170}。}
    \qauthor{哦漏QAQ《他们说》}
    我所描绘的一切或许并非真实,然而我的感受绝无虚假。\footnote{\moegirl{妄想症Paranoia系列}}	
    \qauthor{雨狸《妄想症Paranoia》}
\end{savequote}
\chapter{获取认知的过程和意识现象的组成} %标题不能过长,超过一行会无法换页,具体原因不明

\section{知识体系与理解}

我们在第一章介绍“眼”这一要素的时候,将其定义为了\indicate{从外界获取能力提升}的所有方式的总称。我们能从外界获取的东西有很多:概念、思路、方法、工具、规则......
\begin{examples}
我们有很多种不同的视角来看待这些东西:比如说按很经典的“是什么/为什么/怎么做”划分;再比如按照第一章的观点,将它们完全视作事务,可以分为“有利于调研的/有利于分析的/有利于计划的/有利于执行的”;再比如根据具体应用范围,按学科划分,或者分为“理论层面/实操层面”......
\end{examples}

\subsection{知识与信息\label{sec:知识与信息}}
本指南从\indicate{系统性}角度出发,给出以下两个称呼\footnote{类似一开始划分简单事务/复杂事务的时候,我们这里给出的定义也是不严格的:既没有定义“系统”,也没有定义“从外界获取的东西”。提前区分能简化本章的行文,方便读者理解。严格定义会在稍后补上。}:我们所有从外界获取的东西统称为\indicate{信息},而将其中\indicate{包含于一个完整系统的信息}称为\indicate{知识}。当我们将信息和知识对立时,信息会偏向强调“不包含于任何完整系统”的意思。
\begin{examples}
一条信息是否可以视为知识,因人而异。比如“任何两个有质量的物体之间都存在万有引力”,如果一个人只是知道孤立的一条,那它只就是信息;如果这个人得到了充足的物理学基础教育,知道质量、运动、力等基本概念的的基础性质,能独立地使用它们来分析涉及万有引力的现象,那他就拥有了“高中物理(力学板块)”的完整系统,这一条就可以算作知识。
\end{examples}
“获取信息”是简单的调研事务,而“获取知识”则是复杂的综合事务:它由“获取一大堆相互关联的信息”的调研事务和“分析这些信息直到综合成一个完整系统”的分析事务组成。
\begin{examples}
从短科普中了解到的概念基本上都是信息,除非你对这方面有深入研究;听到的热点新闻基本上都是信息,除非你持续跟进知道来龙去脉;考前突击学到的基本上都是信息,除非你十分有天赋能看一遍就融会贯通。
\end{examples}
依照语境的不同,我们将一个\indicate{由信息形成的完整系统}称为一个\indicate{知识体系}、\indicate{体系}、\indicate{系统}、\indicate{能力}、\indicate{技能}或\indicate{分析框架}。

\begin{examples}
一个人可以拥有多个不同的知识体系;同一信息可以属于不同的体系。“任何两个有质量的物体之间都存在万有引力”可以属于“高中物理”体系,或是体量更大的“大学物理”体系;同时,它也可以属于另一些更不一样的体系:它可以属于“牛顿的生平”体系,也可以属于“物理学发展史”体系。但如果你在写什么励志文学,“牛顿发现了万有引力”被当成一个例子来用,就不认为它属于“写作”这个体系。这时它只作为孤立的事实而存在,发挥和信息一样的作用。
\end{examples}
“知识体系”的概念很宽泛,很多看上去很不同,具有相反特征的信息集合,都可以被称为知识体系。
\begin{examples}
不同的知识体系之间可能有很悬殊的大小区别,一些体系可以完全包含于更大的体系;较小的体系也可以通过学习和思考逐步扩大,多个小体系可以结合,一个大体系可以分解;一个知识体系可以只有深度,也可以只有广度;一些知识体系可能完全由命题组成(比如数学),一些知识体系可能完全由行为组成(比如跑步),一些知识体系可能完全由事实组成(比如新闻);一些知识体系每个人掌握的都差不多(比如交通规则),而另一些知识体系不同人掌握的千差万别(比如做饭)......
\end{examples}
这些例子表明,以上所提到的都不是知识体系区别于“一大堆散碎信息”的主要特点。作为一个完整的系统,而非一大堆散碎的信息,知识体系的唯一特点是:它可以区分\indicate{哪些事务它可以解决,应该使用什么方法解决。}\footnote{实际上本指南使用这个特点来定义“知识体系”\label{para:知识体系}。读者可以自行对照,确认前文的“事务处理能力”等概念确实可以视为“能力”。}。有很多不同的表述也在说同一个意思,比如“有自知之明”“知道自己的能力边界”“谨慎”等等\footnote{具体的定义参考\hyperref[def:能力边界]{3.1.3小节}。虽然这些表述通常被用来形容一个人的特征,但这种特征是由人拥有的知识体系决定的。}。
\begin{examples}
“一个体系可以解决的事务”来源于对这个体系的有效运用,比如说“认识客观现象”、“实施最优操作”、“对比已知信息”之类。“一个体系不能解决的事务”相对来说特征更为难以刻画,主要有一下这么几类:1、和体系完全无关的事务,比如跑步能力不能用于写歌;2、与体系有关,但缺少解决方法的事务,比如数学中的猜想;3、体系预测与现实情况相冲突的事务,比如说考试中的错题。
\end{examples}
一些知识体系有清晰明确的\indicate{标准}。\indicate{标准}也由相互关联的知识组成,个人可将自己的知识体系与之对比。如果存在大量“标准要求解决,但个人能力无法解决”的事务,则将这种情况称为(能力/知识体系/分析框架)\indicate{不足}/\indicate{缺失}/\indicate{缺陷}。有些时候我们将标准也称为\indicate{能力}或\indicate{知识体系}\footnote{注意此种用法的“能力”不因人的不同而有不同,它是一个固定的标准,具体展开见\hyperref[def:知识体系]{3.1.4小节}。因人而异的部分是“不同人能力缺失程度不同”。}。在结合上下文的情况下,读者应该能够区分“能力”的两种用法。

\begin{examples}
标准可以有多种来源。其可以使某门学科内部的共识,比如说生物学、经济学、建筑学;可以是某份公开且公认的文件,比如成文法律、规章制度、考试大纲;可以是客观事实本身,比如跑步速度、社会热点话题的整体时间线、案件侦破中的现场实际情况;可以是来自个人(或机构)的界定,如一本教材,或本指南对于“事务处理能力”的定义。该定义及其衍生的“调研能力缺失”等概念仅在本指南讨论范围内生效,不保证在其它讨论环境下仍然是清晰明确的标准。其它讨论环境下,同一词汇可能有不同含义。\footnote{更一般地,对于所有不来源于公认文件或客观事实的标准来说,脱离它的讨论环境时,都有可能产生歧义。具体展开参见\hyperref[sec:能力边界]{3.1.3小节}。}
\end{examples}

每有一个知识体系,或者一个标准,就会有它可以解决的\indicate{一类事务}\footnote{这里仅做描述,以方便读者理解。关于如何识别一类事务,请见下一小节}。这是本章的核心视角:对一类事务的统一研究和处理。

我们将“处理某一类事务的具有共性的一系列操作”称为\indicate{方法}或\indicate{流程}。如果这个方法能够解决这一类事务,则称为(这类事务的)\indicate{解决方法};如果需要强调“这个方法不一定能解决这一类事务,有明确的失败可能性”,则称为(这类事务的)\indicate{处理方法}。

方法不负责“判断某个事务能否属于自身的处理范围”。这种判断可以被视为另一种事务(调研事务或者分析事务)。我们将“判断某一个事务是否属于这一类事务”的解决方法称为这一类事务的\indicate{识别方法}。

\begin{explain}
一类事务的识别方法和解决方法可以组成一个完整的分析框架(知识体系)。如果缺乏解决方法,识别方法和“自身能力不足以处理此类事务”的判断也可以组成一个完整的分析框架。这两种搭配是最简单的分析框架,大部分分析框架具有更为复杂的结构,难以分解为清晰明确的“识别方法”与“解决方法”两部分\footnote{读者可以参考“调研事务”和“计划事务”的嵌套。识别方法与解决方法也有同样的嵌套。}。
\end{explain}

\subsection{识别方法与判断\label{sec:识别方法与判断}}

任何一个识别方法都由一些判断组成。我们将一个\indicate{一般疑问句}称为一个\indicate{判断}。
\iffalse 或\indicate{命题}\footnote{\rigorous 可能会要求命题可以判断正误,但这和个人具体拥有的知识体系相关,不同的人面对同一条判断时,有的人可以判断有的人不能判断,故在此不做这方面要求。}
\fi
判断同时可以作为动词,意为“回答这个一般疑问句”,其结果仅可能有三种:“符合该一般疑问句的描述”、“不符合该一般疑问句的描述”、“无法判断是否符合描述”。我们分别使用“真”“假”“不可判断”来称呼。
\begin{examples}
判断有很多种不同的类型。有一些涉及事务的某方面特征;有一些涉及事务的成因;有一些预测未来的发展......多条判断可以结合,以得到更详细的结论:比如“一种分类方式”就由好几条判断组成,其中恰有一条为真。
\end{examples}
人在判断时得到的结果,不一定就是真实情况。“不可判断”的结果必须由人主动给出,我们将这种“判断结果与真实情况不符”的情况不视为“不可判断”,而是称为\indicate{判断失误};而将“判断结果与真实情况相符”的情况称为\indicate{判断准确}。一个人在某个具体问题上是否判断准确,有些时候是不可得到的信息\footnote{可以在这里附加一个“判断自己的判断是否准确”的识别方法来套娃,但这样的套娃无穷无尽,套娃本身没什么特别的意义。}。
\begin{explain}
如果拥有足够的推理能力,识别方法足够强大,一个人可以尽量避免判断失误,并且修正自己的判断失误。常见的修正如“一段文字看到后面才意识到自己有个字看错了”。数个判断结合起来也可以修正一些判断失误,比如说“作业本在学校”和“作业本在家”不可能同时成立,一定有一个是错的。结合判断以修正判断/得到新判断的能力是分析能力的一部分,这可以通过一些手段培养,比如说接受逻辑学教育、阅读推理小说、观看剧情解析。
\end{explain}
解决一件事务的定义为“从当前情况出发,执行一系列操作,以达成目标”\footnote{这和之前的定义仅有用语上的差别,完全是一回事。}。据此,我们将\indicate{一类事务}定义为\indicate{当前情况具有共性,目标也具有共性的很多个事务组成的集合}。这里的\indicate{共性}由识别方法来判断。

不同的判断之间可以相互结合,通过分析以得到进一步信息,这些信息有助于下一步判断。对于“某个事务是否属于这一类事务”这个判断而言,如果存在一些其它的判断,将它们结合的信息足以使得在“判断为真”和“判断为假”时都不出现失误,那么我们就称这些判断为这一类事务的\indicate{识别方法}\footnote{对于\rigorous,该定义的严格表述为:如果对于某一类事务A,存在一个判断组成的集合S,使得对于任何事务a,“a是否属于A”这个判断都属于S,且在“判断为真”和“判断为假”时都不出现失误,那么就将S称为A的一个识别方法。}。识别方法是分析能力的重要组成部分,“使用识别方法判断”可以视作一种调研事务。
\begin{explain}
读者不难验证,这个定义和之前对于识别方法的定义是一致的。这个定义不允许“把不属于的事务判断成属于”、“把属于的事务判断成不属于”、“在无法确定的情况下以为自己能判断”,但允许谨慎地将很多事务分为“无法判断”,甚至允许将所有事务都分为“无法判断”。准确判断为“属于”或者“不属于”的事务越多,说明该识别方法的识别能力越强。

一些读者可能觉得识别方法很像是在玩\indicate{海龟汤}——还真没错,除了“没有主持人,你得自问自答,很多时候也没有汤底揭晓环节”以外,整个流程和海龟汤完全一致。
\end{explain}
很明显,对于“某个事务是否属于某一类事务”,当我们判断为“属于”的时候,我们就会用对应的解决方法来解决这个事务;当我们判断为“不属于”的时候,我们就不会用对应的解决方法来解决这个事务。而当我们判断为“不确定”的时候,我们也可以使用对应的解决方法,但此时这就只是一次\indicate{尝试}。尝试不保证成功。
\begin{explain}
    其它的文章可能有不同的表述。比如说很多人会有“我知道没希望,但我就是不甘心,想试一试”的表达,这种情况在本指南中被视为“不确定事务是否可以被某方法解决,并且尝试”。
\end{explain}
\begin{explain}
    同时,本指南不将“以前从来没成功过”视为“自己不具有解决的能力”的有效判断依据,而很多“我知道没希望”的说法其实表达的意思是“我以前从来没成功过”。在阅读本指南过程中,请优先使用本指南中给出的定义去理解涉及的概念。
\end{explain}
尝试的价值有两方面:一方面是可能会成功解决事务;另一方面,尝试很有可能带来有用的信息(这表明尝试也是一种有效的调研方法),从而增强下一次的识别能力。
\begin{explain}
    无论尝试成功还是失败,都有可能带来有用的信息。不能带来信息的情况有两种:一种是事务和方法过于复杂,而分析能力和已知信息不足,以至于无法有效处理;另一种是没有分析的意识,不主动去总结反思。大多数人的“不会总结经验教训”其实是前一种,但经常会被当成是后一种。
\end{explain}

\subsection{方法与过程\label{sec:方法与过程}}
我们将\indicate{一个原因和其对应的结果}统称为一个\indicate{事件}。我们也将原因称为\indicate{起因}、\indicate{前提}、\indicate{条件}、\indicate{前提条件}或\indicate{触发条件}。我们将\indicate{当前情况改变,满足相应原因,一个事件发生}称为\indicate{触发}这个事件。
\begin{explain}
和事务类似,事件也有复杂的递归和嵌套。几个简单的事件,如果前者的果是后者的因,就可以串联拼接成更复杂的事件;如果好几个果共同组成了另一个事件的因,就可以并联拼接成更复杂的事件。因此,在后文中,我们不区分“一个大事件”和“组成它的很多小事件”。我们将所有小事件前提条件的总和视为大事件的前提条件,把所有小事件结果的总和视为大事件的结果。
\end{explain}
我们将\indicate{一类原因和结果都具有共性的事件}称为\indicate{过程}\footnote{前文中提到的所有“过程”都是这个含义。},将“这一类事件中的一个”称为该过程的一个\indicate{实例},将\indicate{共性在实例上的具体体现}称为\indicate{特点},将\indicate{人能察觉到的特点}称为\indicate{信号},并将此称为“信号\indicate{刺激}了人”或“信号给人提供了\indicate{刺激}”。

我们称前提条件或者过程\indicate{导致}了结果,并将前提条件或者过程称为结果的\indicate{来源}。
\begin{examples}
在识别方法中,一步操作是“通过当前已有信息,得到判断的结果”;我们之前曾经将子事务称为母事务的一步操作,此处的“操作”可以视为“人主动选择了一个过程,触发相应的事件”,因为“解决子事务”可以“固定地实现子目标”;更一般地,“解决事务”本身也可视为过程。
\end{examples}
过程的唯一要求是“确定性的因果关系”\footnote{本指南不讨论“因果关系是否存在”的哲学话题,默认因果关系客观存在,但人认知到的不一定正确。对于\rigorous,本指南将所有“因果关系可能不存在”的情况全部视为“没有正确判断特点,从而无法在完全理论的环境下分析”,具体展开参见\hyperref[sec:概念与特点]{3.1节}。换句话说,本指南承认因果关系存在;人可以猜因果关系但不保证猜中;人可以交流自己猜的因果关系但不保证对面能听懂。}。如果具有某种共性的原因,总是会导致具有某种共性的结果,那这就可以视为一个过程。
\begin{examples}
    与事务不同的地方是,过程不要求“主动解决”,而是可以自然地发生。以下提到的都可以视为过程:完全的自然现象,如“水往低处流”;机械的运动规律,如“车燃烧汽油获得动力”;计算机的运行结果,如“应用程序的文字识别功能”;条件反射或肌肉记忆,如“流水线工人的熟练操作”;人/AI的事务处理。以上这些例子含有不同程度的智能和主动性。
\end{examples}
一些学科尽可能广泛考察并深入研究了具有某种特点的所有现象,这样总结出的过程足以被称为\indicate{规律}。但日常生活中的事特点太多,难以对每种共性都展开广泛深刻的分析。这时,“判断特点并总结共性,以认识过程”的过程,就只能由每个人自己来完成了。
\begin{explain}
    在提炼共性时,将一个现实事件按其特点分解成多个方面,每个方面分别分析,经常是一种有效的方法——控制变量法。仅考虑一个方面的分析通常会更简单;在一个方面得到的结果,也经常可以作为信息,辅助其它方面的分析。
\end{explain}
我们将\indicate{触发该过程的某一个实例}称为\indicate{触发}该过程。如果某个事务的当前情况符合某个过程的前提条件,同时这个过程的结果也是这个事务的目标,我们就能主动触发这个过程,以解决该事务。我们将“可以主动触发的过程”称为这个事务的一个\indicate{力所能及}的步骤/操作。

不管一个人有没有认识到某个过程,过程都客观存在。但是在处理某类事务的方法中,如果要主动选择一个过程作为操作,肯定需要先认识到这个过程。如果认识到的过程并不准确,那么这一步操作就有可能出现预料之外的结果,从而整体的方法就称不上解决方法,而只能是处理方法。
\begin{explain}
即使认识不够准确,也可能每次都能成功处理事务。这种情况应该被视为运气好,而不能视为掌握了解决方法。一个人总是成功处理某类事务,是因为运气好还是因为掌握了解决方法,不总能判断。但当一个人处理某事务失败时,可以判断这个人在当时没有掌握解决方法。一个人如果处理某事务总是失败,可以判断这个人没有掌握解决方法。
\end{explain}
解决方法可以使我们稳定地利用已知的过程,并且组合这些过程,来解决眼前的事务。触发已知的过程是执行能力的核心,组合这些过程是计划能力的核心。
\begin{examples}
掌握了某种解决方法,不一定就意味着有机会用它来解决实际事务,因为有的时候前提条件不具备。巧妇难为无米之炊。在沙漠中心不可能去某家便利店买水,虽然这在城里是可行的解决“口渴”的方法;在大海上没有手机信号不能打电话,虽然这在陆地上是可行的解决“思念”的方法;离开了某个岗位就不能调用某些资源,虽然这在担任这个岗位的时候是可行的解决“任务”的方法;一个小时后就得用的工具不可能网购,虽然这在时间充足时是可行的解决“需求”的方法。
\end{examples}
主动选择的过程仅需要确认前提条件和结果即可,不需要了解中间的具体步骤。这在某些语境下会被叫做\indicate{黑箱}。
\begin{examples}
可以将某个环节完全委托给另一个人做,常见的比如说约稿、外包项目、雇佣;可以借助某些机器的功能,使用冰箱的时候只需要知道它可以提供低温就行,不需要会制冷;可以只靠等待让事情发展到下一阶段,比如排队、加工、鹬蚌相争;可以只执行步骤而不懂每一步的具体原理,比如说按照教师要求练习、按操作使用机器;可以下意识地使用练习过的技巧,如游泳、绘画、歌唱。
\end{examples}
解决方法的子操作中,可以包含\indicate{尝试}。如果对于尝试的每种可能结果,都有下一步处理方法,都能最终实现目标,那么这当然也解决了事务。
\begin{examples}
去某个机构办理业务的时候,如果人生地不熟就有可能跑错,但一般也能在跑错的地方打听到正确的地点;考试没有通过可以明年再考、转换方向或者尝试就业;普通票抢不到的话可以考虑加钱。尝试有可能是因为信息不足,有可能是因为目标不单一:某种不保证成功的尝试有更大的收益,比如说更节约时间、更省钱、前途更好等等。
\end{examples}

\subsection{理解}
对于一个现象,如果一个人能\indicate{找到一个符合现实情况的过程,导致了该现象},那么就称这个人\indicate{理解}、\indicate{了解}或\indicate{归因}\footnote{有些读者可能会担心无限归因的问题。在本指南的定义中,对原因再进一步归因不算做对原现象的理解。理解仅需考虑当前现象即可。}了这个现象,将过程或者这个过程的原因,称为这个现象的\indicate{解释}或\indicate{归因}。如果\indicate{发现}/\indicate{知道}某个现象(掌握了对应信息)但没有理解,那只能称为\indicate{承认}或者\indicate{接受}。
\begin{examples}
并不是所有现象都能被理解。一些学科中所使用的基础概念,比如说经典物理学中的“质量”,就没有什么来源可言,我们只是承认了“物体都具有质量”“牛顿第二定律”等现象,并且加以应用。

相对地,物理学史中“物理学家定义了质量”“物理学家区分了质量和重量”等现象,有着清晰明确的来源:早期物理学家对日常现象和实验结果进行了总结归纳,从而提出了这些概念。

本指南中的“理解”不直接包含很常用的“理解一个概念”的含义。该含义在本指南中对应着“确认概念可以应用的边界”,请参考2.2.3节中关于理解和控制的定义。
\end{examples}
如果一种前提条件有可能导致多种结果,而某个现象只是这些结果中的一种,那么“找到这种前提条件”不能算了解了“这个现象”。但如果将“多种可能的结果”视为一种更大的现象,那么“找到这种前提条件”就可以算了解了这个更大的现象。
\begin{examples}
理解“涵盖所有可能结果的现象”有时候没什么用处,经常给人感觉像是废话一样,比如说“考试有可能考好也有可能考差”“目标有可能实现也有可能无法实现”之类的。但是在一些情况下,“只知道一个结果,而不知道其它结果”会造成重大的理解障碍,如“复习了就一定能考好”“努力了就一定能实现目标”。这样的错误理解会使得在制定计划时产生重大差错,而自己却无法发现具体错误。

“某个原因所有可能的结果”是一个基础但重要的信息,理解“所有可能的结果”比理解“其中一种结果”要容易。具体的结果需要更多前提条件才能理解。“什么样的附加前提条件才能导致一种确定的结果”很多时候是不可得到的信息。
\end{examples}
即使某些成对的现象总是先后发生,也不一定就能确定它们之间就有确定性的因果关系。之前的现象可能不足以导致之后的现象。但如果归因中包含背景,那么“之前的现象+背景”就可能足以导致“之后的现象”。
\begin{examples}
“种下橘子树,就能结出饱满鲜甜的橘子”不算完整的归因,还需要附加“在南方”的背景条件;“吹了空调,就会肚子疼”不算完整的归因,还需要附加“在自己身上”的背景条件。

将背景视为归因的一部分,相对来说不够深入,但得到的理解确保准确。当遇到超出背景范围的情况时,总是应该谨慎地重新审视新情况。“完整的背景”很多时候是不可得到的信息。
\end{examples}
只要理解了某个现象,那么就拥有了改变的可能。通过改变前提,可以使对自己不利的现象不再发生,也可以使凑出对自己有利的现象。当然,这一切建立在“自身有能力改变前提”的基础之上。不同的归因会使得需要更改的前提不同,从而难度和需要的能力也不同。
\begin{examples}
归因到环境上时,很多时候无法主动选择脱离环境来解决不利现象。在沙漠里口渴,也只能出沙漠再说;在大海上没手机信号,也只能靠岸了再说;北方种不出橘子,但土地是搬不走的;体弱不能吹空调,但身体是没法换的。
\end{examples}
为了有效地改变,或者更一般地说,为了保证能解决一类事务,很多时候我们只关注那些力所能及的归因。换句话说,所有的解决方法都会认定某个方面,并且只归因到这些方面。即便是有完善体系和公认标准的学科也是如此。因此,在特定语境下,“理解”也仅指“可以将现象归因到特定方面”。
\begin{examples}
相比于真实世界的复杂,物理学只研究有限的几个物理量,而更具体的部分被分给了各个不同的工程学。如果在中学物理中问“为什么桌子可以稳稳地放着”,回答“因为木工师傅的手艺很好”就不是有效理解。

相比于人类的复杂行动,经济学只研究经济的衍生现象,而其它更具体的假设衍生出了很多不同的经济学分支。在古典经济学中不总是能问“一个人为什么想要某件东西”,古典经济学不负责给“需求”归因。
\end{examples}
在本指南中,\indicate{理解}仅指\indicate{将现象归因到了力所能及的原因上}。根据各位读者的学识和能力不同,不同人力所能及的范围会很不一样。但对于意识活动来说,情况比较确定:我们不会将意识归因至生理活动上。你无法控制自己分泌多少激素,也无法使自己的心跳停跳。大多数人无法控制自己的情绪产生,最多只能控制自己情绪的影响。我们也不会将意识活动归因至环境上,大部分事情无法靠改变环境来解决,你也无法确定自己脱离了\indicate{一个}环境后就不会再遇到\indicate{同类}环境了。
\begin{examples}
我们没办法一个念头就增加肌肉、减少脂肪。想要做到这些,只能通过锻炼。身体如此,意识也一模一样。
\end{examples}
% 本指南将人的一切主动行为全部视作事务处理,这是我们的意识本身能掌控的边界。
意识活动的具体组成将在下节中讨论,我们将逐项研究意识中可控的部分,据此具体刻画出我们力所能及的范围。

\section{意识的层次\label{ref:意识的层次}}
我们的身上时时刻刻都在发生很多\indicate{过程}。有一些过程是自动的,我们无法主动控制,比如心跳、消化食物、免疫病原体;有一些过程也可以虽然自动,但想控制的时候可以控制,比如呼吸、眨眼、咽口水\footnote{如果这导致读者切到了手动挡,本指南深表抱歉。};有一些过程虽然一开始是有意识控制的,但熟练了就变成自动的了,比如打字、打游戏、流水线工作;有一些过程一直得有意识地做,比如做题、写文书、谈判。

\subsection{行为}
依照语境和具体限制的不同,我们将\indicate{一个人身上的过程}称为这个人的\indicate{行为}\label{def:行为}、\indicate{反射}、\indicate{习惯}、\indicate{瘾}、\indicate{条件反射}、\indicate{肌肉记忆}或者\indicate{下意识的}行为/举动/.....。%\indicate{自动发生}指的是\indicate{在过程中没有额外的主动触发}。
\begin{explain}
这种用词可能比较奇怪,毕竟我们确实不常见到“一个人有呼吸心跳的习惯”之类的说法,这听上去不像人话。“习惯”“条件反射”“肌肉记忆”看起来有明确的“从外界习得”的意思。

本指南不使用“从外界习得”来定义“行为”,是因为我们很多时候分不清“这个行为到底是从外界习得还是天生就有”,经常会产生错判。我们能确定的只有“某个过程会稳定地发生”。这词就凑合用一用吧。

脱口而出的一句话,突然涌上心头的情绪,看到问题就想到的思路,这些都被我们算作行为。行为的定义仅有“是一个过程”,偏向生理和偏向心理的都算。
\end{explain}
绝大多数行为都是后天习得的。其中有些是主动培养的,有些是被动受环境塑造的。本指南有时会倾向于将后天习得的行为称为\indicate{习惯}。
\begin{examples}
本指南不做过多的认知科学讨论,不从很深入的神经可塑性角度出发论证这一观点。如果读者对此感兴趣,请参考《刻意练习:如何从新手到大师》或更深入的书籍;也不很深入地研究具体案例,如斯特拉顿的翻转视觉实验,或@SmarterEveryDay的自行车握把翻转实验\footnote{原视频链接:\url{https://www.youtube.com/watch?v=MFzDaBzBlL0};中文翻译:\url{https://www.bilibili.com/video/BV1zx411m7cx}。}。本指南仅概括性地引用结论。
\end{examples}
\label{para:记忆}概括来说,我们能培养出行为,是因为人有\indicate{记忆}。当我们重复经历某一过程的时候,之前的经历会作为额外信息,辅助我们计划和决策。这使得我们的思考越来越快,越来越熟练,直到最终,只要察觉到某个信号,就会完全自动地作出一系列操作,决策也就变成了习惯。
\begin{examples}
相反地,如果在重复中无法参考之前的经历,那就无法培养出行为。这可能有很多原因,比如无法理解或者错误地理解了之前的经历;比如注意力在别的方面,没意识到这是相同的现象/没想起来之前的经历;比如没有找到可以主动控制的环节,过程自动地结束了;比如空闲时间没有思考分析,实际发生的时候来不及现场判断......
\end{examples}
如果在一个行为触发前,我们进行了分析\footnote{分析的定义见\hyperref[def:分析]{2.2.3小节}。},并在这之后决定触发这个行为,那么就称这个过程为\indicate{(主动)控制}或者\indicate{过脑子},反之则称为\indicate{自动发生}。我们将“一个人主动控制的(无论自身还是外部的)过程”称为这个人的\indicate{决策}\footnote{对于\rigorous,“一个决策”指的是“主动触发一次行为”而不是“一个可控的行为”。}。
\begin{explain}
我们察觉到的信号不一定是完整的归因。环境变化后,做同样的操作不一定能得到同样的结果。是否触发一个行为很多时候由“是否察觉到信号”决定,而是否触发决策一定经过了“是否能得到结果”的判断。
\end{explain}
行为有可能对我们有益,也有可能对我们有害。它自动发生的时候是不可控的。当环境变化的时候,原本的益处可能变成害处,原本的害处可能变成益处。
\begin{examples}
熟悉了一款游戏的键位,换到另一款游戏上的时候就会出问题,尤其是那些不太经常使用的功能快捷键或者连招,会更容易按错;熟悉了和一些人的交往模式,换到另一些人身上就很可能触霉头,有可能把别人惹到了,自己却还一无所知;甚至那些纯粹的生理过程也会在某些环境下带来麻烦,比如受伤的时候心跳反而加快从而失血增多;细胞因子风暴的致命性可能远大于病原体本身。
\end{examples}
一些行为可以作为知识体系的一部分,几乎每一次触发都是自动发生。
\begin{examples}
打(动作)游戏的时候,除了新手期,不可能每按一个键都过脑子。要是按键不靠肌肉记忆的话,你根本跟不上游戏的节奏。阅读的时候,你肯定不会每个字都想上好几秒才能知道意思。刚开始学一门语言的时候可能这样,但这样根本没法读下来一整本书。走路的时候,你完全不是在随时控制每根肌肉何时收缩,走路这个动作在婴幼儿时期就已经练会了。
\end{examples}
我们之前提过,\indicate{理解}一个现象,是指找到一个真实的过程,导致了该现象。对于行为来说,有两种现象需要理解:一种是“为什么会养成这个行为”,一种是“每一次行为为什么会被触发”。\\
理解“自己养成了一个行为”,也就是找到自己过去养成行为时,所追求的目标。
\begin{explain}
由于我们不总能知道自己的行为是否是后天习得的,这个问题不一定能找到答案。但对每个行为都思考一下“目标是什么”总是有帮助的,至少总不会比“从来没想过”更无知。
\end{explain}
理解“行为的每一次触发”,也就是需要给行为归因。按照本指南的约定,我们需要将其归因到事务处理上:只有能够主动控制能否触发,才算是理解了行为。如果只是找到了无法避免的原因,那只能算是\indicate{发现}/\indicate{知道}了这个行为(此时我们称这个行为为\indicate{自知的}行为),而不能算是理解。
\begin{explain}
有些行为不太好界定是否可控,比如说“看到了某个现象以后想到需要做的事”,这里前半段的“看到”不是决策,而后半段的”想到“是决策。此时我们应该将其拆成两部分分别看待。

如果一个行为总是不会触发,我们也认为自己理解了它的每一次触发,认为它是可控的。这在后续的分析中会带来便利。
\end{explain}
我们将“理解行为的触发”改称为(行为)\indicate{可控},而仍然将“理解行为的养成”称为\indicate{理解}。如果既理解又可控,那么就称为\indicate{掌握}了这个行为。
\begin{examples}
我们总是掌握\indicate{决策}所对应的行为,至少在决策后的短时间内如此(直到忘掉自己的思考过程)。我们不掌握绝大多数生理反应,理解生理反应需要相应的生物学/医学/认知科学知识,大多数生理反应不可控制。

注意,“理解”和“可控”都不包含“知道某个行为有什么用”的部分。这是因为行为具体有什么用,是好是坏,取决于具体环境。我们应该使用知识体系来判断,然后进行决策。%即使是很简单很初等的过程,也会在不同情况下有不同的作用。比如“心跳”,正常情况下其可以“为了维持血液流动,给器官提供物质和能量”,但有大伤口的情况下,反而会使人失血过多,加速人的死亡。
\end{examples}
即使某些行为既无法理解也不可控,也不代表它一定有害。
\begin{examples}
我们认识成千上万的字,每看到一个字,“想起它的含义”都可以当成一个行为。你肯定不记得自己是怎么学会每一个字的(只是可能会对少数几个字很印象深刻),最多有统一但模糊的“被父母老师教”的印象;每次阅读的时候,也肯定不是“先决定自己要看懂一篇文章,然后再一个字一个字想含义”(最多对少数生僻字会回忆一下)。当看到一段文字的时候,你就已经开始了阅读,触发毫无疑问来自外界,不受自己控制。

但会识字比不会识字要好太多了。“识字和阅读”是一个知识体系,这些不可控的底层行为为我们提供了可控的“读书”“理解句意”等完全可控的行为作为能力。真遇到少数情况,我们也能快速地纠正自己的误用。

而相反,当你初学另一门语言,掌握的词汇和语法还不足以构成完整的知识体系的时候,尽管你更会去有意识地想每个词的含义,但整体看来这不可控,会使你无法正确表达自己的意思,可能造成严重的沟通障碍。
\end{examples}

\subsection{行为模式\label{sec:行为模式}}
\noindent 我们将一段“连续触发的行为”称为一个\indicate{行为链}。%之前定义中提到的\indicate{稳定}是指“会有很多相关的行为和决策连续触发”。
\begin{examples}
这里的“连续发生”有可能是因为外部环境会持续地提供刺激,比如“嘈杂的周边环境”“不断到来的工作”“连续的办事流程”“无所事事”之类;也有可能是外部的刺激引起了自身强烈的反应,比如“做完事情以后的成就感”“突然遇到crush的一见倾心”“被恐怖电影吓得一晚上睡不着”之类;也可能和外界没什么关系,纯粹是自身内部引起的,比如“焦虑感”“抑郁感”“死亡冲动”之类。在“一个高度重复的环境”等极端情况下,一个固定的行为被反复触发,也可能组成一个行为模式。
\end{examples}
每一种环境中都有很多种特点,这些特点中的一些会被我们作为信号察觉到,从而触发行为,引起分析和决策。我们将\indicate{一个人在某个环境下所有可能发生的行为链}称为这个人在该环境下的\indicate{行为模式}\label{def:行为模式}、\indicate{状态}或\indicate{侧面}。
\begin{explain}
不同环境下的行为模式,会有相同的部分和不同的部分。在所有环境或大多数环境下都不变的行为和决策,很多时候会被叫做“性格”或“情绪”(但是按本指南的定义,会出现“心跳是性格的一部分”之类如果不是在青春浪漫文学中就没在说人话的句子。所以本指南没有引入这个概念)。
\end{explain}
我们将“对应的行为链正在触发”称为一个人\indicate{处于}这个行为模式中。我们将“之前不处于,之后处于”称为\indicate{触发}或\indicate{进入}这个行为模式,将“之前处于,之后不处于”称为\indicate{打断}\label{def:打断}或\indicate{脱离}这个行为模式,将“脱离一个行为模式,进入另一个”称为\indicate{切换}行为模式。
\begin{explain}
进入行为模式有可能通过触发某个行为达成,也有可能没有明确的行为触发;脱离行为模式有可能是因为触发了新的行为模式、触发某个行为/决策从而打断了当前行为链、所有行为链都自然结束等多种情况。
\end{explain}
我们可以轻易在自己身上找到成百上千个行为,但行为模式却没那么多。一个人拥有的独立行为模式,通常只有几个或者十几个。
\begin{examples}
虽然我们会遇到很多不同的环境,但这并不意味着每种环境下的行为模式不同。例如,对很多人来说,无论是在听课,还是在等车,还是晚上躺在床上,还是一段闲暇的午后,所面对的环境都和“时间空闲且有手机”没什么不同,只会在一些细节上(比如说上地铁需要过安检,而其它场合没这回事)有少量的区别。

一个行为模式内部通常包含多种行为链,根据具体环境的不同,触发的行为链也会不同。
\end{examples}
行为模式也具有和事务、过程类似的嵌套与递归。如果算得很仔细的话,行为模式也可以有成百上千个。但我们一般不去研究不同行为模式之间的细微区别。
\begin{examples}
我们可以将“做题”算作一种统一的行为模式,也可以将每一科的题目单独算作一种行为模式;面对自己无法处理的事务的时候,除了“尝试”或“放弃”以外,同时还有可能有“焦虑”“迷茫”“好奇”等多种子行为模式;“休闲”是一种统一的行为模式,每个游戏、软件或者其它项目都有对应的子行为模式。
\end{examples}
一个行为模式可以完全由知识体系组成,所有的行为全部是清晰的决策;也可以完全由不自知的行为组成,所有的行为全部是下意识反应。但更多情况下,行为模式是这两方面的混合。
\begin{examples}
在尝试处理某件事务的时候,越是能意识到眼前事务的专业性和困难性,行为模式就一般越偏向于决策(一个人在这种情况下会越认真),比如“考试”这个环境下。而偏向于行为的可能性有很多:比如眼前的事务不困难,自身已经能熟练解决;又比如缺少对应的知识体系,无法决策,只能慌张。通过决策来拆分事务,使得低层次的步骤可以熟练解决,是混合组成行为模式的常见方式。
\end{examples}
% 不同于习惯或决策,过程或事务,行为模式不一定有“明确的一类结果或目标”。习惯的结果和决策的目标不一定能上升到行为模式层次。
% \begin{explain}
% 行为模式可能没有什么目标(只有很多习惯和决策在互相触发),可能有很多个不统一的目标(这些目标有可能是不同方面的,相互之间可能相容,也可能有冲突),也可能确实存在一类明确的目标(比如纯粹由知识体系组成的行为模式,一切以“解决某一类特定的问题”作为目标)。
% \end{explain}
类似于行为,对于行为模式来说,也有两种现象需要理解:一种是“为什么会形成这个行为模式”,一种是“每一次为什么会进入某种行为模式”。类似于行为,我们将前者称为\indicate{理解},将后者称为\indicate{可控}。

\indicate{理解}一种行为模式(无论是自身的还是他人的),需要发现其所包含的所有行为(不要求理解每一个行为),并且分析清楚这些行为和决策如何组合和接续,以形成行为链。
\begin{explain}
一般来说,理解一个行为模式的难度要远高于理解一个行为。尽管我们不需要在理解了每一个行为之后才去理解行为模式,但仅仅是发现所有相关的行为,就往往已经比透彻地理解一个行为更复杂,况且之后还有“搞清楚行为和决策之间的组合”的步骤。
\end{explain}
一种行为模式是\indicate{可控}的,是指“其包含的每一种行为链中,都有一个可控的行为”。注意这并不需要发现所有的行为链才能做到。我们将“行为链发生时,控制那个可控的行为”称为\indicate{控制}这个行为模式。
\begin{explain}
一个可控的行为在行为模式中的价值,取决于“它还会触发多少个行为”。如果可控的行为模式是行为链中的最后一环,那控制它有可能没什么用。

不采用“控制每个行为链的第一个行为”的定义,是因为人很难完全确定自己的行为链都包含哪些行为,由哪个行为开始。这种定义没有实际价值。

不采用“在重要的事件之前控制一个行为”的定义,是因为这样还得额外定义“重要的事件”,而这个因人而异。如果读者想要更有实际价值的定义,可以根据自己的实际情况确定重要的事件,并且据此定义“可控的行为模式”。这种定义替换在本指南中不会产生任何论述差异。
\end{explain}
如果只是总结出了“自己在某些环境下的状态”“自己的性格特点”之类的事情,则不算理解了行为模式,只算\indicate{发现}/\indicate{知道}了这个行为模式(此时我们称这个行为模式为\indicate{自知的}行为模式)。
\begin{explain}
理解一种行为模式需要自知作为前提条件,但控制一种行为模式则不需要。我们可以巧合地控制住一种行为模式的每一种行为链。这与行为的情况略有不同,理解和控制一种行为都需要知道这种行为。
\end{explain}
如果既理解又可控,那么就称为\indicate{掌握}了这个行为模式。
\begin{explain}
我们总是掌握\indicate{知识体系}。知识体系的要求比“可理解且可控的行为模式”更高,在知识体系中,我们要求其中每一个行为要不然是决策,要不然在触发后能发现做得是否合适。
\end{explain}
即使某些行为模式既无法理解也不可控,也不代表它一定有害。
\begin{examples}
当你初次接触某一门课程时,你一定不知道每一个概念、每一步操作在这项技能中到底有什么作用(尤其是一些应用类课程)。你可能在学到一定阶段的时候突然开窍,但也有可能直到结课的时候依然只会死记硬背。即使是只会死记硬背,一般也能通过考试,老师不会出什么真正超纲的内容;并且有时候也能顺利完成工作,只要不总是被分配到超出能力范围的事情。

当外界有其它人、机构或系统掌握了一个行为模式时,就可以有意识地培养一些拥有这一行为模式的人。在可控的环境下,这不会有什么问题。即使没有掌握后的人在有意识培养,如果运气够好,外界环境仍然有可能正好适配这一行为模式。

而相反,当外界环境变化时,这种行为模式就有可能出问题。当面对更复杂的环境,或者更深入的知识的时候,就有可能完全解决不了眼前的问题,并且也完全找不到原因。
\end{examples}

\subsection{思考回路、认知、决定}
我们将起因和结果都与外界环境无关的行为称为\indicate{想法},并将由想法组成的行为链称为\indicate{思路}。
\begin{explain}
如之前所说,如果一种行为涉及到了外界,那么可以通过将其拆成几个组成部分的方式,从中分离出想法来。那些不属于想法的行为,总会包含“通过感官从外界获取信息”或者是“根据自己的决定去行动”的环节。将它们分离出去以后,就得到了只在意识内部发生的想法。

我们不一定能意识到自己都思考了些什么,尤其是当思考已经高度熟练的时候。福尔摩斯能轻松判断出一个人的身份,但华生问他为什么的时候,反而得思考好半天。我们的任何一种情绪背后都有可能有相当复杂的思路。
\end{explain}
我们将每一行为模式中\indicate{由想法组成的子行为模式}称为其所对应的\indicate{思考回路},将触发一次思考回路称为一次\indicate{思考}。
\begin{explain}
不是所有的想法都能组成思考回路,很多想法只短暂存在于一套行为链之中。不是所有的行为模式都有对应的思考回路。运动、电子游戏、体力工作等行为模式可以没有思考回路。对应地,也有一些行为模式可以完全或几乎完全就是思考回路,如一些知识体系、所有识别方法,或者一些情感。
\end{explain}
我们将思考回路的得出的结果称为\indicate{认知}或\indicate{认识}。
\begin{explain}
认知是一种\indicate{信息},而应用认知来分析是一种\indicate{行为}(可能是决策)。当我们谈论认知是否可理解/可控时,我们相当于在谈论“应用认知”这个行为是否可理解/可控。一个认知\indicate{可理解},是说自己明确知道这个认知的形成过程;一个认知\indicate{可控},是说自己明确知道这个认知的应用范围,并且总是根据应用范围来决定是否使用该认知。

信息不都是认知,有些信息不依赖思考回路得到,比如眼睛直接看到的影像。但像是认知那样,我们也能定义信息是否可理解/可控。可控的信息是知识。
\end{explain}
我们将\indicate{使用知识体系得到认知}的过程称为\indicate{分析}\label{def:分析}。
\begin{explain}
分析得到的认知都是\indicate{知识}。它天然地是当前知识体系的一部分(也可能同时属于其它知识体系)。分析不是获得知识的唯一途径。经过他人或自己有目的的培养,一些行为可以整体结合成一个知识体系。一些行为也可以巧合地组成一个知识体系。

分析中可能包含识别方法,此时的思考回路就是知识体系。
\end{explain}
在先前的讨论中,我们将可控笼统地定义为了“经过分析后,决定是否触发行为”。而现在我们可以看得更深入些:\indicate{决定依赖于思考回路},不同的思考回路的分析和得到的结果不同,决定也会因此而不同。
\begin{explain}
当我们做\indicate{决策}的时候,一定有一个\indicate{目标}(虽然我们不一定清楚自己的目标;目标可能依思考回路的不同而不同)。思考回路需要判断“触发这一行为,得到的结果是否有助于实现目标”。思考回路中可能包含错误的认知,于是可能判断错误;思考回路有可能自身不可控,控制行为完全是下意识举动。但这些都不会影响“这种行为是可控的”,这是思考回路的问题而不是行为的问题。
\end{explain}

\subsection{污染\label{sec:污染}}
\hfill\begin{minipage}{0.55\textwidth}
\fontsize{8pt}{12pt}\selectfont\fontsize{8pt}{12pt} %/selectfont使得英文字体也调整字号
像飞龙掠过我,转眼间规则都颠破。\footnote{\bilibili{BV1vW4y1H7kH}。}

\raggedleft lemon夹子《放生十万个地球》

\raggedright 如果不曾向天空仰望,也不会发狂。\footnote{\bilibili{av7772529};\\\indent \netease{450387538}。}

\raggedleft DELA\_P\&雨狸《海星人鱼》

\raggedright 我握紧这份痛,演出偏执的兽,被困在畸形的悲剧段落。\footnote{\bilibilid{av9214515};\\\indent \moegirl{九重现实}\\}

\raggedleft DELA\_P\&雨狸《九重现实》
\end{minipage}

\label{def:污染}我们将自身所拥有的\indicate{不理解或不可控的行为、行为模式、认知}称为\indicate{带有污染的}行为、行为模式、认知,或简称为\indicate{污染}。如果一些东西(如外部环境、自身经历、记忆、信息、知识、行为等)会\indicate{使我们形成污染},我们就将其称为\indicate{污染源},并将这个过程称为污染源\indicate{带来}、\indicate{造成}或\indicate{传播污染},或是污染源\indicate{污染}我们,我们\indicate{被污染}或\indicate{受到污染}。
\begin{explain}
“污染”带有明确的负面含义,而它的定义从表面看起来并不是那么负面。这是因为,虽然可以直接用“污染”一词来指代“有害的行为和行为模式”,但“是否有害”取决于具体的环境和判断标准,因人而异,不利于展开分析。采用了更为客观简单的定义后,我们就可以从外部判断行为和行为模式是否属于污染。略微扩大定义范围,使得污染有了更加统一的应对和解决方法。

如果强调“有益”,则可改称为\indicate{熏陶}、\indicate{耳濡目染}、\indicate{信条}等褒义词汇。我们在前几小节中已经见到了污染可能有益的情况,这里不再赘述。在其它语境下,污染也会被称为\indicate{混乱}、\indicate{盲目}、\indicate{规训}等。

“污染”带有文学色彩,它是从克苏鲁体系中借来的词汇。作为近年的流行词,它无论是从感情色彩还是从实际内涵上,都相当贴切。
\end{explain}
污染的收益随外部条件而变化。如果任由污染自动触发,我们就会承受损失。如果我们能掌握,那么就可以避免这些损失。
\begin{explain}
可控和理解分别可以减少这些损失。

如果某个污染是可控的,那么它所带来的损失也是可控的。可控的行为(决策)压根不会带来损失;可控的行为模式有可能带来损失,我们不一定能在行为链开始时就切断。比如说“哭半个小时就停”,也会有半个小时的煎熬。但不管如何,可控的行为模式不会带来无穷无尽的损失。

如果我们理解了某个污染,我们就能得知“哪些损失可以避免,哪些损失不能避免”。得到这个重要的信息后,就可以省下很多无效操作,加强可控性,并且可以针对不可避免的损失提前做好准备。比如说“知道了自己昏迷是因为有心脏病”,就不会像常人一样参加剧烈的体育运动来强身健体,而是转为康复锻炼,同时随身携带速效救心丸。
\end{explain}
以上是关于“污染的收益”的讨论。接下来的内容关注“污染的形成”。\\
\indicate{接受了不完整的教育,就会受到污染。}
\begin{explain}
如前所述,在接受教育的初期,几乎一定会受到污染。一套完整的教育可以教会学生完整的知识体系,但如果缺少环节(特别是缺少了识别方法),学生就无法获得完整的知识体系,只能获得带有污染的行为模式。不完整的教育有很多种情况,其中比较常见的有以下两种:

\indicate{得到了某种完整教育的一部分},如“只有教科书但没有听课”“三门前置课程少上了一门”“只上了一半课就不上了”之类。这种情况下,没有外部力量保障一个人接受完整教育,尤其是教育的直接提供者(比如老师,甚至是书)不知道完整教育都包含什么的时候。这或者是因为教育提供方没有特别关注某个学生(甚至完全没注意到有这个学生,比如说买书自己看的),或者是因为学生可以控制自己停止/放弃继续学习。

\indicate{得到的教育本身已经残缺},如“老一辈人坚守的信条”“一场激动人心的演讲”“没办法讲清楚自己为什么优秀的高手”之类。这种情况下,即使教育提供方可能可以维持学生持续学习,但本身教育体系的残缺会使学生永远无法掌握。同时,教育提供方往往没有意识到自身提供的教育有所残缺。学生或许可以从别的途径学会,或许可以依靠自身强大的调研能力学会(这一般被称为“天才”)。但这与残缺的教育体系无关。

有些时候我们无法区分具体是哪种情况,比如说“因为看见了书里的一句话而深感认同”,有时候不能确定书是不是完整教育的一部分,但这句话已经种下了带有污染的认知。
\end{explain}
\indicate{遗忘会使原本能掌握的东西变成污染。}
\begin{explain}
遗忘有很多种不同的机制。概括来说,遗忘发生在“脱离了之前曾进入过的思考回路”时。

为了简便考虑,我们重点关注行为和认知的形成。随着遗忘产生的污染,既有可能是我们自知的认知,也有可能是我们不自知的行为。限于篇幅,这里的两种情况讨论不是很严谨,仅供读者参考。

当我们处于某种思考回路时,我们会从一些特定的思路中获取一些认知。当我们脱离这一思考回路时,我们可能仍然记得这个认知,也会继续根据它来决策,但却无法再回想起当时的思路了。我们忘掉了这个\indicate{认知}对应的前提条件后,它就成为了污染。

新接触一件事情的时候,我们起初会认真思考和决策。随着我们在这件事上慢慢地熟练,它的决策用时和步骤会越来越短,最终完全不出现在我们的意识中出现。我们培养出的\indicate{行为}可能是不自知的,自然也谈不上理解和可控。此时,自己身上就增添了一个新的污染。

不只有“无法再次进入的思考回路”才会产生污染,只要脱离了之前的思考回路就会。当思考回路切换时,很多认知也会在掌握和污染之间切换。详细讨论参见\hyperref[sec:意识与自我]{2.3节}。
\end{explain}
\indicate{长期被迫响应某一环境的刺激,就会受到污染。}
\begin{explain}
如我们之前所说,处理同一件事逐渐熟练,就会形成不自知的行为。

如果自己的\indicate{能力足够应对那种情况},形成的行为就会偏向行动,如熟练的手艺、不假思索地寻求帮助、脱口而出的观点表达、对他人理所当然地要求等等。如果涉及到和他人的互动,那么这种行为就可以视作残缺的教育,也就会使他人受到污染。

如果自己的\indicate{能力不足以解决那种情况},形成的行为就会偏向思考,如不断担心是否会出问题、焦虑地寻找解决方法、习得性无助、完美主义等等。如果以自己感受到的信号来评判他人的行为,就会变得敏感脆弱难以接近,从而自身作为一种会给人带来刺激的环境,也就会使他人受到污染。

限于篇幅,以上的两种情况讨论不是很严谨,仅供读者参考。读者可以借此大致了解污染在不同人之间的传播方式。
\end{explain}
\indicate{使用污染来处理事务,就会受到进一步的污染。}
\begin{explain}
当事务处理有污染参与时,就会产生“没有事务需要处理”的错觉——要不然是“没有发现/找错了事务”,要不然是“认为事务可以解决”。很多时候会同时产生两方面的错觉。

用来处理事务的污染,有可能是完全不自知的行为,见到了信号就不自主地做了事;也有可能是完全自知的错误认知,手里拿着锤子,眼里都是钉子。只要产生了“没有事务需要处理”的错觉,同时环境中确实有事务需要处理,就一定会产生错误的归因和新的行为。

如果是\indicate{没有发现事务}的情况,那么新的污染也会偏向不自知,也就是以“新的行为”为主。在固定的环境下,行为累积多了就会成为行为模式。可能需要花上很久,才能在某天突然反应过来“我之前的状态出问题了”。

如果是\indicate{认为事务可以解决}的情况,那么新的污染也会偏向自知,也就是以“错误的认知”为主。和行为需要慢慢养成不同的是,认知一形成,立刻就会影响自身的决策。同样,可能需要花上很久,才能意识到“自己之前想错了”。

错误的认知在一些语境下会被称为\indicate{自我欺骗}。本指南不引入这一术语。

受到“不自知的行为”的影响,和“响应某一环境的刺激”机制类似;受到“错误的认知”的影响,和“接受了不完整的教育”机制类似。这是污染在同一人体内的演化方式。
\end{explain}
污染是本指南的核心概念之一。概括地来说,污染代表了自身能力无法覆盖到的部分,是事务处理能力的反面。它是一种相当好用的分析工具,在接下来的篇幅中会被频繁提及。
\begin{explain}
有些读者可能看这些很像\indicate{投射性认同}。确实如此。虽然在词义上略有差别,但\indicate{污染}基本上就是\indicate{认同}的实际表现,一样有\indicate{投射}和\indicate{內摄}等机制。本指南的重点不在于此,故不正式引入投射性认同的概念(以及与其相关的整个体系)。
\end{explain}

\section{意识与自我\label{sec:意识与自我}}
\hfill\begin{minipage}{0.55\textwidth}
\fontsize{8pt}{12pt}\selectfont\fontsize{8pt}{12pt}
\raggedright 曾幻想长大展翅飞翔,如今一副瘦肩膀。\footnote{\bilibili{av10154377}。}

\raggedleft vsinger团队\&和田野《未来的我》

\raggedright 我是最无能的主宰。\footnote{\bilibilid{av4188543};\\\indent \netease{407764392};\\\indent \moegirl{四重罪孽}。\\}

\raggedleft DELA\_P\&雨狸《四重罪孽》
\end{minipage}

%终于,在经过了前面漫长的铺垫以后,我们要进入重头戏“对意识和自我的讨论”了。
我们将一个人自身的所有\indicate{信息、知识、知识体系、行为、决策、行为模式等}统称为\indicate{意识活动}或\indicate{意识现象}\label{def:意识现象}。
\begin{explain}
可能有读者在看前面的时候已经注意到,本指南以一种很别扭的方式,规避了“潜意识”这一概念。除了偶尔用一下“不自知”“下意识”以外,我们把所有潜意识都跳过去了。“行为”这一概念和“潜意识”有显著的区别。本章中不定义潜意识,本指南中也很少提及潜意识。

意识现象是一个复杂系统\footnote{“复杂系统”的定义见\hyperref[def:复杂系统]{3.2.2小节}},潜意识是其中产生的现象,我们现在还没有准备好能处理复杂系统的工具。关于潜意识的讨论见\hyperref[sec:人的基本意识模型]{3.4节}。

引入潜意识无助于构建本章的主要分析框架,只会使讨论变得更混乱。只有自知是不够的,甚至有时自知了反而会产生更坏的后果,比如加剧内耗。对我们影响更大的是理解和控制。
\end{explain}
本节讨论的内容是“人拥有多种行为模式”的事实以及这些行为模式相互的作用。人可以同时处于好几个行为模式之中,但很难同时处于自身拥有的所有行为模式之中。当处于不同行为模式时,我们的能做到的会有差别,能思考的会有差别,得到的认知也会有差别。
\begin{explain}
一些行为只有在特定的行为模式中才能触发。如果状态不对,我们就无法记起某些事情、进入某些思路、使用某些方法、理解某些现象、得出某些结论。大多数情况下这没有到“大脑的自动保护机制”的程度,实际状态更贴近于“注意力不在这方面”“没有耐心”“没意识到”之类。

面对同样的东西,我们在不同的行为模式下,会得出不同的评价。它们有时能解决有时无法解决,有时可控有时不可控,有时可理解有时不可理解。本节描述的一些现象可能看起来像是思维奔逸,甚至人格障碍、精神分裂等,但请勿自我诊断,那只是描述相似而已。
\end{explain}
本指南不深入讨论“自我如何产生”的具体生理原理,只聚焦于“人有自我认知”这一现象及其影响。
\begin{explain}
本节的讨论不涉及“胼胝体切除手术对人的影响”(裂脑人实验)等生理现象带来的启示,也不涉及“自我如何从婴儿期开始不断成熟”的发展心理学理论。一个人不可能因为知道了裂脑人就能将大脑分开思考,也不可能因为“被人告诉自己的心理年龄处于婴儿期”就能成熟起来。我们只关注那些力所能及的现象。
\end{explain}
% 我们将人的意识现象分为两类:将\indicate{直接与外界环境交互}的意识现象称为\indicate{行为现象},将\indicate{不直接与外界环境交互}的意识现象称为\indicate{心理现象}。
% \begin{examples}
% 在事务处理三要素中,负责处理外界输入的\indicate{眼}(调研)总是行为现象,而负责内部分析的\indicate{脑}(分析、计划)总是心理现象。根据具体事务的不同,负责直接处理事务的\indicate{手}(执行)有时是行为现象(如需要改变环境中的某处),有时是心理现象(如需要得出结论)。

% 信息和知识总是心理现象。行为有可能是心理现象,也有可能是行为现象。行为模式大多数情况下是心理现象的行为和行为现象的行为的混合,故本指南不笼统地给行为模式贴标签,而是只判断其组成比例。
% \end{examples}

\subsection{自我}
我们将\indicate{使用思考回路产生对某个人的认知}的过程称为\indicate{评估},将被评估的人称为评估的\indicate{对象}。
\begin{explain}
不是所有的思考回路都能评估某个人,只有少量的思考回路有这个功能。而且,一些思考回路只能评估某些特定的行为模式,而无法评估那些不具有这种行为模式的人。

当使用的思考回路不同时,评估产生的认知也会不相同;当评估对象处于不同的行为模式时,评估产生的认知也会不相同。当我们想到一个人时,总是在使用对这个人的认知来思考。

会评估出不同的认知,是本指南将人按行为模式划分的重要动机之一。在特定的环境下,我们可能只会接触到一个人的某个侧面,而接触不到整体。能评估出复杂、立体的人物形象,反而是不太容易做到的事,要不然需要专业能力,要不然需要机缘巧合。
\end{explain}
如果评估对象是自己,那就将其称为\indicate{自我评估}。我们将\indicate{由自我评估得到的所有认知}称为\indicate{自我}或\indicate{自我认知}。
\begin{explain}
自我评估有些时候会触发两种行为模式,有的时候会使用一种行为模式的思考回路对这种行为模式自身来评估。

每当使用“我”这个人称代词的时候,我们就不可避免地会用到自我认知来思考。那些和自我强相关的东西,如(自己的)命运、需求、目标、倾向、过去、未来、情感、生活等等,也不可避免地会让我们借鉴自我认知。

自我认知可能是经过谨慎的思考得来的,但这并不常见。大多数人都不会像哲学家那样,很彻底地反思“自我”到底是什么。

更多时候,自我认知是完全是下意识的反应,或者是草率得到的结论,和现状高度相关。心情好就会有正面的自我认知,心情差就会有负面的自我认知,想起过去就会有过去的自我认知,想起别人就会有别人的自我认知。
% 
% 如果你正在一路顺风,自我认知就会包含光辉与胜利;如果你正在咬牙坚持,自我认知就会包含顽强与不屈;如果你正在处处碰壁,自我认知就会包含迷茫与无助。如果你正在回忆过去,自我认知就会包含过去的思考;如果你正在想象未来,自我认知就会包含未来的期望;如果你正在代入某个角色,自我认知就会模仿对应的情节。
\end{explain}
不同行为模式下的自我认知有可能截然不同,甚至相互矛盾。
\begin{examples}
你可能在假期刚开始的时候兴致满满地安排自己的活动,但一觉起来却只觉得躺在床上刷手机是最好的;你可能在深夜觉得自己一事无成也看不见一点方向,但第二天继续干活的时候就又鼓足了勇气和精神;你可能看到讲解底层逻辑的视频就觉得自己全都懂了,但面对现实时却又觉得自己根本什么也做不到;你可能在吵架的时候情绪激动非要讨个说法不可,但吵完就会开始发现自己对感情和陪伴的需求;你可能独处的时候觉得最重要的事情就是在一起一辈子,但一见面就会被深深地恶心到发现你们俩并不适合......

这些截然不同,甚至相互矛盾的感受,并没有客观的谁高谁低,并没有客观的“理智与冲动”和“应该与过错”之分。它们是你所拥有的不同的行为模式对你产生的不同影响。
\end{examples}
当你在某种行为模式中使用的自我认知是从一种\indicate{没有发现当前行为模式的思考回路}中得到的,那么这个自我认知就是\indicate{污染}。
\begin{explain}
一种自我认知的适用范围是“形成该自我认知时,发现并加以分析了的行为模式”。只有在对应的行为模式中,才应该去使用对应的自我认知。

如果在使用一种自我认知时,忽视/忘记/没有意识到它的适用范围,那根据它得到的结论和做出的计划很可能是无效的,做事的结果也会偏离预期。你可能高估或者低估了自己的行动力、毅力、专注程度,或其它任意一方面能力。你可能过于乐观或者悲观地判断自己的进度、可用资源、周围人的态度,或其它任意一种外部环境。

相对地,发现了该行为模式,并不意味着自我认知就不是污染。它可能涉及到错误的归因。
\end{explain}

\subsection{理性}
\noindent 我们总是\indicate{使用某种思考回路}去理解其它行为模式。
\begin{explain}
在不同的思考回路下,面对同样的意识现象,我们能理解的部分有所不同,不同的行为模式理解难度不同。如果当前的行为模式不包含思考回路,我们就无法理解任何东西。

如果你不清楚自己某些行为模式的触发条件,发现不了某些关键的思路,那你就无法理解自己的这一部分。当你真正处于这一行为模式,能够观察自己的所作所为时,却又很可能从来不会想到“要理性思考,理解自己”。这会导致你无论何时都无法理解自己的某个行为模式。

在之后的篇幅中,当我们使用“xxx是否可理解”的表达时,一定有“在某种思考回路/行为模式下”的前提条件。在不引起歧义的情况下,也有时会省略掉作为前提条件的思考回路,或者仍然使用“一个人能理解”的习惯表达。读者应该有能力自行补全。
\end{explain}
\indicate{理解}一个人(的意识现象),就是要发现这个人所拥有的所有行为模式(不要求理解和控制每一个行为模式),并且分析清楚这些行为模式如何组合和交替出现。
\begin{explain}
这个定义和理解行为模式的定义如出一辙。行为如何组成行为模式,行为模式就如何组成一个人。它们具有高度相似的层级结构。具体内容参考第三章。

这个定义没有区分“理解自己”和“理解别人”,实际上也没有必要区分。它们遵循着相同的原理,只在一些具体的细节上有所区别。我们用思考回路来分析行为模式,而不在意行为模式具体是谁的。同样,行为和行为模式的理解也不区分是自己的还是他人的。

我们会听到“没有人比自己更理解自己”,也会听到“当局者迷旁观者清”,理解自己和理解别人没有确定的“谁难谁易”。理解的难度取决于你当前应用的思考回路和对方所处的行为模式(这里的“对方”也可能是你自己)。

这听上去像是精神分析,实际上也差不多就是。但本指南尽量避免提及专业概念,故不引入成体系的精神分析术语。
\end{explain}
当理解了一种行为模式后,我们就可以根据其中的行为对待某种环境时的反应,逐步分析出这一行为模式的举动。我们将其称为(在某种环境下)\indicate{模拟}这种行为模式。
\begin{explain}
当理解了一种行为模式以后,我们就可以设想“如果在某种环境中处于该行为模式,会发生什么”。模拟可以选择任意环境,不必要在原本就会触发该行为模式的环境中。

如果这种行为模式同时是可控的,我们就可以根据模拟的结果来决定是否触发某种环境下这种行为模式。在一些情况下,这比“在未知环境下一律控制某个行为模式不触发”要更好。

同时,模拟也不限于自身,可以模拟别人有但自己没有的行为模式。这有助于我们理解别人。能模拟不代表能拥有,可能会因为知识缺失导致某些步骤无法自己完成。
\end{explain}
如果一个人拥有一种思考回路,能够发现并分析自己所拥有的每一种行为模式,并且得到不含污染的自我认知,那么我们就将这种思考回路称为\indicate{理性}\label{def:理性}或\indicate{理性思考},将“使用理性来分析自己的行为模式”称为\indicate{反思}。
\begin{explain}
如果读者喜欢的话,也可以叫\indicate{自我精神分析}。

不同人所拥有的理性是高度相似的。它总是包含一个完整的知识体系(由事务处理能力等内容组成),在此之外还根据个人特征有些许差别。

我们这里定义的理性不直接包含“科学研究的方法论”等内容,这些不在本指南的讨论范围之内。实际上,我们这里定义的理性就是“将科学研究的方法论运用到自我意识上”的行为模式。

可以分析一种行为模式不代表可以理解,如果能力不足,理性有可能得不到任何有效的自我认知。但这只是能力问题或者是现实条件限制,只需要继续丰富自身的知识体系即可。

理性所得到的自我认知,仅对理性来说不是污染。若自身处于别的行为模式中,可能无法理解这些认知。在别的行为模式看来,这些认知和理性本身可能都是污染。

理性有可能只在某些特定环境下存在,包括但不限于面对心理医生时、和朋友谈心时、看书时、深夜、发呆时。
\end{explain}

\subsection{自控的两种方式:意志力法与自我编织\label{sec:自控}}
\noindent 我们总是\indicate{使用某种思考回路}去控制其它行为模式。
\begin{explain}
在不同的思考回路下,面对同样的意识现象,我们能控制的部分有所不同,不同的行为模式控制难度不同。如果当前的行为模式不包含思考回路,我们就无法控制任何东西。

在之后的篇幅中,当我们使用“xxx是否可控”的表达时,一定有“在某种思考回路/行为模式下”的前提条件。在不引起歧义的情况下,也有时会省略掉作为前提条件的思考回路,或者仍然使用“一个人能控制”的习惯表达。读者应该有能力自行补全。
\end{explain}

类似于“行为模式的可控是每一行为链都有行为可控”,我们也可以用同样的方式来定义人是否可控:一个人在某种思考回路/行为模式下\indicate{可控},是指这个人的每种行为模式都在这个思考回路/行为模式下可控。我们将“行为模式触发时控制这个行为模式”称为\indicate{控制}一个人。
\begin{explain}
和前面“理解”时一样,这个定义也无需区分“控制自己”和“控制别人”,这二者没什么本质差别。同样,行为和行为模式的控制也不区分是自己的还是他人的。

如果能有效打断或开启另一个人的每一种行为模式,那就能控制一个人。这常见于父母对小孩、教师对学生、狱警对囚犯等场合。详细讨论参见2.5节。

“控制”本身是一种中性的行为。如果是因为污染而控制,那么它就会偏向负面;如果是因为正确决策而控制,那么它就会偏向正面。
\end{explain}
我们将“用自己的思考回路来控制自己”的行为称为\indicate{自控},本小节后面的内容仅讨论自控。
\begin{explain}
如果某一思考回路A不直接控制行为模式C,而是通过触发行为模式B,使用B控制行为模式C,这对C当然也是有效的控制。如果一个人同时处于好几个思考回路A、B、...中,每个思考回路可以控制一部分行为模式,这也算是这个人可以自控。
\end{explain}
想要达到“控制所有行为模式”的目标有两种方法:一种是站在行为模式层次的“控制每一种行为模式”,另一种是站在意识现象层次的“只控制一部分行为模式,并且清除其它不可控的行为模式”。

\label{def:意志力}我们将\indicate{控制每一种行为模式}的自控法称为\indicate{意志力}自控法,其中,一种思考回路对一种行为模式的\indicate{意志力}是指\indicate{在该行为模式下,这种思考回路可做决策的数量}。
\begin{examples}
控制不同的行为模式所需要的决策数量有很大差别。当你处于某个充满诱惑的环境中时,你可能需要频繁地(比如每搁5秒就)做一次决策,打断自己“想吃东西”“想玩游戏”“想睡觉”的冲动;当你处于某个充满干扰的环境中时,你可能需要频繁地(比如每隔30秒就)做一次决策,强迫自己“继续学习”“继续干活”“继续思考”。而在“只需要触发另一种行为模式,就可以脱离当前行为模式”的时候(如在心情糟糕的时候点杯奶茶喝,或是只要脑子里面想点别的别闲下来)或者“只需要脱离当前环境,就可以脱离当前行为模式”的时候(如看书看累了起身走一走,或是回家/离家),一共就只需要做一次决策。

不同思考回路的意志力特性有不同之处。有些思考回路可能只能应付一定频率以下的决策;有些思考回路可能一天只能做一定数量的决策;有些思考回路只能坚持少数几次决策就会信心崩溃,从此都无法再使用;有些思考回路越做决策越专心,越自控就状态越好。

因此,这里对于意志力的定义比较模糊,仅笼统地称为“数量”,而不具体定义成“总量”“频率”等。请读者根据实际情况,为不同行为模式分别选择合适的理解方式。
\end{examples}
意志力自控法的主要缺点有两个:一是“每次决策都有代价”。行为模式会反复地被触发,从而也会一直需要支付代价。二是“无法补齐能力缺失”。补齐能力的方式参见后文“培养行为模式”。
\begin{explain}
代价有很多种可能的形式,包括但不限于“消耗时间”“分散注意力”“打断思路”“心情变差”“失去兴趣”“错过时机”等。

在处理某一个行为模式时,有可能在坚持了一定时间后,就逐渐脱离了那个行为模式,于是可以专注于自己的目的。靠意志力达到心流状态都是有可能的。这一缺点只在于“无法一劳永逸,每次触发行为模式都需要单独处理”,而不在于“时刻会被打扰”。
\end{explain}
我们将\indicate{选择某些行为模式,并将自身调整为由这些行为模式组成}称为\indicate{自我编织}。当选择的行为模式都是可控的行为模式时,自我编织就是一种自控法。
\begin{explain}
自我编织是一个事务。完成它需要做两方面事情:培养所选的行为模式,去除未被选择的行为模式。培养/去除每一个行为模式都是它的一个子事务。

培养一个行为模式,需要理解这个行为模式,并且针对性地培养组成这一行为模式的行为。去除一个行为模式有两种方法:一种是控制环境,使得这一行为模式总是不会被触发(这种方法仅需发现这个行为模式);一种是消解这一行为模式,即通过改变或去除某些行为,使得行为链不再持续触发,行为模式不再存在(这种方法需要理解这个行为模式)。更详细的讨论参见\hyperref[sec:原生家庭]{2.5节}与\hyperref[sec:人的基本意识模型]{3.4节}。

\indicate{编织}是一种普遍存在的现象,定义见\hyperref[def:编织]{3.2.1小节}。
\end{explain}
自我编织自控法的主要缺点在于“难度高”。我们需要使用一整套完整的理论,才能够稳定地实现目的(但也不能保证实现目的)。
\begin{explain}
不同于意志力自控法,自我编织要求发现自身所有的行为模式(只有这样才能保证按选择调整自身),所以仅有理性才有能力实现自我编织。如果能做到培养/消解某个行为模式,就能一劳永逸地解决这一方面的自控问题。

可能有读者会担心如果把自己的行为模式都换过一遍,那自己还是不是自己。本指南不讨论意识连续性问题(毕竟每个人的自我评估都不一样),只想请读者注意一件事实:人的行为模式本身就会不断地更改,相当多行为模式只会持续很短的时间(比如说一个月)。“无法共情过去的自己”是相当常见的现象。实际上只有“不可控的改变”和“可控的改变”两种选项,而没有“不改变”的选项。
\end{explain}
即使无法全面完成自我编织,即使是做到了培养/消解某个行为模式,也会增加自身的可控程度。在实际操作中的自控,更多的是两种方法的结合。
\begin{examples}
先通过意志力控制住一个浪费时间精力的行为模式,等到时间久了它可能慢慢就消解了;对于更顽固的行为模式,需要专门分析,以找到可以破坏的节点。

而分析行为模式同样有助于找到意志力可以控制的行为,省下更多时间精力用于其它方面的目标。
\end{examples}

\section{总结与讨论}
\subsection{本章总结}
\begin{explain}
本小节准备了两种风格的总结,希望不同口味的读者都能加深对本章的理解。
\end{explain}
\smalltopic{(1)知识点串讲}

我们从外界获取\indicate{信息},分析这些信息就会得到\indicate{认知}\footnote{\rigorous 可能需要确定是先有认知还是先有行为模式。答案是先有初级的行为模式,然后更高级的认知和行为模式才在此基础上交替出现。初级的行为模式由动物性的行为组成,如“进食”“休息”“记忆”“模仿”等。}。信息和认知会影响我们的行为,使我们更加偏向于某种选项,直至形成\indicate{行为}。在合适的环境下,一些行为会连续触发,从而形成\indicate{行为模式}。以上这些,是一个最简单的关于意识的模型。这在很多动物身上也能见到,是一种低层次的\indicate{动物本能}。

但我们是人。我们不想失败。我们不想只能在夜里偷偷哭泣。我们不想让过去的回忆褪色。我们不想只能过毫无价值的一生。我们不想屈服于自己的动物本能,只能做出事与愿违的事情。当我们回顾自身,\indicate{发现}自己的问题的时候,就会去想着\indicate{控制}。人类的\indicate{自我意识}驱动着我们总结出这样的结论:我们相信,只要能解决问题,就不会再经受失败,就会达成所愿。带着这样的\indicate{认知},我们又有了一些新的行为,又形成了一些新的\indicate{行为}和\indicate{行为模式}。

很不幸,事情不一定会向着我们希望的方向发展。对自己和对外界的认知不完善,所有改变的尝试就全都只能是\indicate{污染}。就和那些动物本能一样,我们照样没办法控制这些新东西。自我意识是盲目的,它只能看到目标,但看不到实现目标的道路。靠\indicate{意志力}只能在漆黑一片的荆棘丛中行走,承受着痛苦的折磨,却不知道路还有多长,甚至总是原地打转。路真的会有尽头吗?尽头就是目标了吗?我们不知道。

我们能知道的只有一件事:有其它人\footnote{实际情况可能更复杂,这里使用“人”是为了行文简便。}走到了路的尽头,实现了自己的目标。他们是幸运还是强大,对我们无关紧要;你是见贤思齐、羡慕嫉妒、自惭形秽、愤世嫉俗,也无关紧要。我们只关心一件事:他们是怎么成功的?我们也能向他们那样走向成功吗?于是我们开始了\indicate{归因},开始了模仿。模仿的风险很大,错误的归因也会给自己带来\indicate{污染}。只有那些正确总结了因果关系的\indicate{过程},才是应当前进的正确道路。将这样的道路一段一段拼接起来,就能抵达目标。

这时,我们才终于\indicate{理解}了如何实现自己的目标。但理解了不代表就有能力实现,目标越遥远,越困难,需要补充的能力就越多。为了获取所需的能力,我们首先\footnote{实际情况下,清除污染、总结规律、培养能力三方面行动会交替发生。这里为了叙述流畅做了简化。}需要\indicate{发现}并\indicate{清除}一些\indicate{错误的认知}。这些认知来自于我们身上的污染,为了不再重蹈覆辙,我们也有必要一并\indicate{清除}这些\indicate{污染}。

接下来,我们需要培养新的行为模式。新的行为模式完全由正确的归因组成,我们能够完全理解,充分控制,并且熟练运用。这样的\indicate{知识体系}使我们能够正确地\indicate{分析},正确地\indicate{决策},从而有充足的能力来实现自己的目标。而在这个过程中,我们也慢慢培养出了自己的\indicate{理性}。

\smalltopic{(2)写作思路与分析框架}

第一章中我们详细讨论了事务处理,后续需要详细展开的有两方面:一方面是关于\indicate{能力}的讨论,另一方面是关于使用能力的主体,也就是\indicate{人}的讨论。本章建立的分析框架主要偏向对\indicate{人}的讨论,已经涵盖本指南所有需要用到的概念和思路。而\indicate{能力}方面仅做了简要叙述,完整展开见第三章。

本章围绕\indicate{理解}与\indicate{控制}两个重点,将人的意识活动由低到高区分了\indicate{行为}、\indicate{行为模式}、\indicate{人}三个层次,分别展开了讨论。

本章最重要的切入点是“能发挥出什么能力,取决于一个人在什么状态”。为了区分一个人的不同状态,我们需要明确定义\indicate{行为模式}并据此展开讨论。考虑到思考和认知的重要性,我们还细分出了\indicate{思考回路}这种纯粹在脑内发生的行为模式。

一个人身上不同的行为模式很多时候各行其是,没有什么协调性可言。它们会给出截然相反的判断,做出相互冲突的行动。这是每个人都能在自己身上感受到的事情。做出的计划要是无法执行,那就和没做没啥区别。我们需要稳定地维持能解决事务的行为模式。为此,我们需要\indicate{理解}行为模式的相关原理。

行为模式的组成相当简单清晰:环境和行为的相互触发。相互触发的具体细节通常会很复杂,没有通用的规律,一般需要每个行为模式具体分析。本章讨论了\indicate{行为},外部环境将在下一章讨论。\indicate{行为}的形成也很简单清晰,它是某些信号反复刺激,逐渐熟练后的结果。

理解了行为模式的相关原理后,我们就可以着手细化我们的目标了。如果想要稳定地维持能解决事务的行为模式,而不被其它行为模式干扰或打断,我们就需要\indicate{控制}这些有关的行为模式。行为模式由行为组成,于是想要控制行为模式,只能也只需通过控制行为来实现。行为由信号触发,想要控制行为,就只能也只需通过控制信号来实现。

我们将“维持”细化为了“控制”,但这不代表目标更容易解决了。这一步细化的唯一作用是\indicate{明确了实现目标的难度}。我们必须要发现行为模式,发现关键的行为,并且确认这种行为的触发条件,才能实现有效的控制。这在一些时候可以通过运气或是意志力实现,但这只能应对一些特殊情况。我们需要一种通用的\indicate{分析}方法,来处理所有需要处理的行为和行为模式。于是,我们据此定义了\indicate{理性}。分析能力的强弱,直接决定了我们能否理解和控制每一个不同的行为模式。

也因此,在关于\indicate{能力}的讨论中,我们将能力(\indicate{知识体系})定义为“哪些事务它可以解决,应该使用什么方法解决”。看完第二章的读者应该可以明白,这等价于“其中包含的认知和行为可理解并且可控”。

为了方便搭建分析框架,行文中还引入了“方法”“识别”“过程”等概念。这些概念较为底层,主要为了简化描述和精确规定重要概念,较少在其它地方出现。

本章中还引入了一个较为特殊的概念:\indicate{污染}。这一概念在搭建分析框架时没有太多存在的必要,而在后续的讨论中,应用分析框架时,会体现出其方便之处。

\subsection{再谈死亡\label{sec:再谈死亡}}
\hfill\begin{minipage}{0.55\textwidth}
\fontsize{8pt}{12pt}\selectfont\fontsize{8pt}{12pt}
\raggedright 嗫嚅着故事的终章,迷人的馥郁芬芳。\footnote{\bilibili{av19566463};\\\indent \netease{495834939}。\\}

\raggedleft 汤《时间日记》

\end{minipage}

\smalltopic{(1)关于求死的行为模式}

运用第二章的概念,我们已经能对死亡和自杀这一话题展开更详细且深入的讨论了。

我们在\hyperref[chap:wedge]{序幕}处谈了一大堆和死亡相关的事情。本章的概念可以让我们给出一个更精确的表述:你想死,是因为你有一个行为模式,会在你遇到困难的时候想到“死了就好了”(我们将其称为\indicate{求死}的行为模式);没有死成,是因为你所拥有的行为模式,不足以让你完成“自杀”这一行为。

\indicate{求死}由多种不同的思路结合而成,可能包括恐惧、无助、逃避、绝望、解脱、爱恋、渴求、回忆、梦想、温柔、回答、离去、花束、夏天、歌曲、谎言、晨风、晚霞、乌鸦、酒杯等很多看起来或是很合理或是很离谱的东西,具体组成成分因人而异。如果一个人从未发现过求死这种行为模式,那一般就不可能理解一个无法控制求死的人。即使是发现/经历过以后控制住了,也有很大一部分人无法理解其它人,甚至无法理解过去的自己。

\indicate{绝大多数人所拥有的求死是一种污染}。那些“活不下去”“死了就好了”的念头绝对不可能凭空出现,一定是先想到了“死亡有这种作用”的结论,再去将其当成解决方案。这个结论在人与人之间传来传去,最终失去了所有的背景,只剩下了脍炙人口的一句话。“活不下去”对某些人来说甚至已经降格成了平淡的抱怨,但对另一些人来说就是自身每个侧面最真实透彻的判断,是不敢直视而又充满了诱惑的终极答案。当你听到这些东西,于是思路总是会落到“活不下去了”,总是会绕回“死了就好了”时,你就受到了求死的污染。

这种想法可能来自某些对比。当身边的谁能逃离一种痛苦的环境时,不管是上岸还是远走还是死亡,都会因为“不再痛苦”而被羡慕。至于这个谁到底之后又经历了什么,或者曾经经历过什么,其实无人在意,只有“死了就会脱离这个环境,不再痛苦”的结论留了下来。这种脱离实际环境的认知,混杂着羡慕、期待、发泄、指责等多种感情被广泛传播,就成为了污染源。

这种想法也可能来自某些透彻的思考。人类的历史相当漫长,各路思想家、哲学家、文学家已经将死亡讨论了个遍,很多学者和书中角色基于\indicate{理性}的判断而选择了自杀。求死的行为模式对他们来说不是污染,他们的\indicate{最终决定}被单独取出,被评价为“透彻”“豁达”,并且广为流传。但他们的\indicate{思考过程}并没有跟着一起流传,于是那个被单独取出,被高度评价,被广为流传的最终决定就也成了污染源。

如果我们不用理性来思考,那么会遇到的最大问题是“自己还有其它行为模式”。这些行为模式可能有自己独立的目标和渴望,其中大部分会和“死亡”冲突。它们和求死之间没有什么关系,也不能相互理解,甚至不一定能意识到对方的存在,唯一共同点就是都被塞到了你的身体里,除此之外就和两个陌生人一样。求死劝不动其它的行为模式,你就不能专心自杀。

而这反过来也一样,其它的行为模式也劝不动求死,无法消解它们无法理解的求死。\indicate{其它行为模式对求死的劝告、限制、阻拦几乎都是污染},无论这个行为模式是自己的还是别人的。大部分人没有透彻地想过自己为什么要活下去,只是习惯于这么做,他们\indicate{求生的行为模式也是污染}。即使一个人A有控制自己的求死的能力,能让自己想开一点,开心一些,也很可能没法通过劝告、陪伴、介绍自己的心路历程等方式,让B也想开。这些举动在B看起来很可能无比苍白,完全是站着说话不腰疼。A和B的求死很可能需要用截然不同的方式来打断或消解。

如果你能够使用理性,对自己的所处情况和自身的每种行为模式展开全面深入的分析,并且最后得出“自杀是最好的选择”的结论,那么没有任何东西可以有效地劝住你。虽然大多数人获得了理性以后,实际上不会得到这个结论,但每一个拥有理性的人,在具体了解了你的处境以后,都会得出和你一样的结论,都会认为你的自杀确实是解脱,这个决定确实透彻豁达。你的决定也会得到本指南的尊重和祝福。

你如果靠意志力压制住了其它的行为模式,强忍着痛苦,以卓绝的精神韧性程度成功自杀,那本指南也会称赞你的求死是一种很坚强的行为模式,饱含勇气和决心。但这种称赞不会上升到意识层级,因为你的意识中混杂着污染。面对那些带有污染的行为模式,你只能控制,而不能将其消解。在对人这一层级的的看法上,本指南会惋惜你的离去。

本指南不教唆自杀,也不劝导求生,不对“人对自身生命的处置”持任何态度。本指南只聚焦于“理解现实,处理事务”。虽然这些方法看起来只有活人才能用(这也是本指南劝读者至少在看完全书以后再自杀的原因),但你也可以依靠它们来成功地选择死亡。清除了自己身上的污染以后,路怎么走是个人的事。
\begin{examples}
我把石头还给石头,让胜利的胜利,今夜青稞只属于他自己,一切都在生长。
\raggedleft 海子《日记》
\end{examples}
某种意义上,能让所有行为模式都满意的结果,是让求死带着会触发它的所有东西死去,想留的留下来。

\smalltopic{(2)关于外部干涉}

我估计上一段的讨论内容还是会让很多人觉得我就是在教唆自杀。这其实还挺合理的:这相当于有人指着你的鼻子骂你去死,或者有人指着你的孩子说这小东西该死。

但我们不妨换个视角来想一下:什么样的人看到了这段内容以后,会更想自杀呢?

看了这段内容以后,如果会愤怒或者反感,那肯定就不吃这一套;如果从头到尾都没什么感觉,看这些就和普通的社会调研报告似的,只是感到痛心、担忧,或是只是平静,那也不会有啥事。会出事的有两类人:一类是本来没有求死的行为模式,但看完以后就拥有了;一类是本来就有求死的行为模式,看完以后深感共鸣,觉得自己确实该死。

第一类人其实很少见。准确来说,“在自己的一生中,除了本指南以外,接触不到其它的求死污染”的人很少见。我们生活在一个信息爆炸的时代,各种关于人生意义的思考满天飞(由此我们可以将第一类人并入第二类人中一并讨论)。我们无法将某个人完全隔离在这些东西之外,唯一的可选项就是建立正确的认知。想要得到充分的理解,必然需要充分的分析。
\begin{examples}
大多数人一生都遇不到一次火灾,但消防宣传仍然是很必要的。展开消防宣传,并不意味着在咒你被火烧死。

对死亡的思考可比火灾常见多了,它差不多和诈骗一样常见。开展反诈宣传不意味着歧视你的智商,不意味着觉得你容易被骗。不要上升到人身攻击的高度。反诈是一种纯粹的技术,任何人都可以学习和掌握,也应该学习和掌握。应对“死亡”和“求死”也一样。
\end{examples}
面对第二类人的时候就需要注意一个问题:他们求死的行为模式是哪里来的?他们是因为“厌恶某些东西,恐惧某些东西”,从而才会将死亡视作解脱,还是“没什么原因,只是学别人说话”?后面那种确实可以当做“被别人带坏了”(于是我们需要防范),但前面那种应该怎么对待和处理?以及在此之前还有一个问题:我们应该怎么区分这两种情况?
\begin{explain}
\label{para:无效沟通}坏消息是,我们几乎总是会因为信息不足而无法区分。即使你去当面问,也不会得到有效的回答。原因有三:

一、求生和求死的行为模式在一个人身上可以共存。这两种行为模式无法相互理解,对对方来说相互是污染。一个人在求生行为模式下可以去回应那些关心,但这不能代表求死行为模式的态度。求死者切换到求死行为模式以后,甚至更会觉得大家花在自己身上的精力浪费了,负罪感更强,这反而会刺激求死欲。

二、“自己对自己求死行为模式的描述”不一定是有用的观察。它可能大幅度弱化和忽视了许多关键点,不是有效且正确的信息。一个人自身可能没有任何一种行为模式(无论是求死还是其它)能理解自身的求死。求死者如果不能客观全面地认识自己的状态,那么很可能发自内心地觉得自己没问题,并且对别人也这么说。

三、很多人会预判别人的态度和反应,会为了避免各种类型的麻烦(比如别人的担心、伤感、拯救欲、约束、教育、责罚)而选择撒谎,于是只能得到“我没事”的答复。大部分人无法给出消解求死的有效方案,实际的建议和要求要不然没有作用,要不然折腾人。面对这种情况,求死者选择隐瞒是相当合理的举动。
\end{explain}
如果不能直接观察一个人所拥有的求死的行为模式,就无法确认这个人到底为什么求死,也无从将其解决。方式不对,就会无效,或者起反面效果。发现并理解一个人的某种行为模式是相当困难而专业的事,我们绝对不应该草率地归因,绝不应该草率地断定“某个人身上不存在求死的行为模式”。绝不应该把“偶尔自杀”当成是“受刺激了想不开”处理,绝不应该把“念叨着想死”当成是“被带坏了”。

\subsection{再谈心理咨询}
运用第二章的概念,我们能够比\hyperref[sec:实操1]{之前}更详细且深入地讨论心理咨询了。我们将心理问题重新定义为\indicate{会带来负面效果的污染},而心理咨询的目的,就是清除污染。

我们将\indicate{能够发现、分析、理解、模拟、培养、消解别人的行为模式}的知识体系称为\indicate{心理咨询能力}\label{def:心理咨询能力},从而一名有充足能力的心理咨询师是一位\indicate{有心理咨询能力的人}。心理咨询这种服务,是\indicate{根据来访者的需要,帮助来访者掌握自己}。

由此衍生出了三方面问题:一方面是“如何拥有心理咨询能力”,这其实和上一小节讨论求死的时候提到的内容差不多,大体上来说,切忌先入为主,切忌偏听偏信,要时刻以客观谨慎的态度,从一个人的实际情况出发,展开全面且深入的分析,并且时刻比对自身的认知是否符合实际情况。具体方案参见第三章,此处不做过多讨论。

第二、三则是“如何识别一个人是否有心理咨询能力”和“心理咨询能力如何提供心理咨询服务”。1.3节中这两者均有提及。大多数人既不知道如何识别心理咨询师是否有充足能力,又不知道心理咨询服务如何起作用,从而使得心理咨询的失败率远超心理书籍中介绍的预期(毕竟里面介绍的成功案例居多),进而形成了“心理咨询都没用”的观点。

\smalltopic{(1)如何识别一个人是否有心理咨询能力}

识别一个人是否有心理咨询能力,是一件困难到几乎不可能实现的事情。非常概括地说,能识别出来,几乎一定就意味着自己也有心理咨询能力——但自己都有能力了,还需要找咨询师干什么呢?从别人那打听也差不多:如果你无法确认那个人有心理咨询能力,你就无法判断他的观点靠不靠谱。大部分关于咨询师的评价,靠的都是感觉和信任,而不是证据充足的判断。有这么几种经常会引起误判,\indicate{错认为“对方很靠谱”}的现象:

\indicate{善于倾听}:善于倾听是一种少见的行为模式,它会给人“能够沟通”“温柔亲和”的印象。有时候我们不一定会意识到一个人善于倾听,但也会因为对方易于亲近而乐于与对方接触。不止是心理医生,我们很多时候也会感觉自己身边的朋友有这种特质,能和他们聊得非常投机。

这种行为模式的形成有两种可能性,其中比较常见的一种,是“因为缺失其它行为模式”。如果一个人在某种环境下没有什么主动性,在听到别人倾诉的时候,既不会频繁打断发表自己的见解,也不会觉得厌烦从而起身离去,而只是一直听着,或许会随声附和,或许只是点头微笑,或许是过来蹭饭的,那么这看起来就像是“善于倾听”了。客观上来说,和这样的人相处,向这样的人倾诉,会让人感觉到安全、放松、信任。但这不代表这样的人能够理解你,不代表这样的人拥有心理咨询能力。他们可能根本没在思考,只是行为客观上可以提供陪伴。

另一种,是“明确意识到应该这么做”。而这么做的动机也多种多样:有的就是看中了倾听能起到的安慰效果,从而有意识地这么安抚别人。你身边的朋友和心理咨询师都有可能是这样的人,这样的人能起到的效果和上一种差不多。有的是有意识使用倾听来理解你,这就是咨询师的专业范围了。这种动机会让咨询师问你很多问题,这些问题有很明确的主动性,但同时不预设任何立场。\footnote{当然我们不一定能准确区分一个问题/一串问题是否有立场,同时也不一定能区分是否咨询师真的问了很多问题,这只能作为辅助信息来大致判断。如果你能准确判断,那你也就有心理咨询能力了。}这样的咨询师同时也会使用很多其它手段,总体来说倾听的频率可能不如专门用倾听来安慰的咨询师。一位很有能力的咨询师会在该倾听的时候倾听。

\indicate{善于指引}:相比于善于倾听,善于指引的行为模式会更常见一些,它会给人“聪明可靠”“充满智慧”的印象。我们比较容易意识到一个人善于指引,并且因为对方总是能给出解决方法而乐于与对方接触。除了心理医生,这种特质同样会在身边的朋友身上出现,他们总体来说会更成功一些。

善于指引需要区分不同的方面。这种行为模式的形成需要两个条件:一个是“愿意提供建议”,一个是“在对应方面有充足的能力”。如果以朋友/长辈的标准来评判,善于指引已经相当可靠了。能在你有事情要解决的时候过来帮忙,在你情绪低落的时候给你有效的安慰,在你遇到麻烦的时候帮你梳理思路,在你接触新东西的时候提供教学的人,基本上就是关系最好的人了。这种程度的关系对于很多人来说只存在于梦中。但这不代表这样的人能够理解你,不代表这样的人拥有心理咨询能力。他们可能根本没在思考,只是在响应你展露出来的诉求。

这对于咨询师来说是不够的。无论是对你的问题给出针对性方案,或者是长篇大论地和你讲人生的道理,或者是带你摆沙盘和画房树人,我们能感受到的都只有“咨询室内的氛围融洽”。而咨询室内的氛围如何,和咨询室外的情况无关。在咨询室外你可能仍然保持着旧有的行为模式,没有改变。这些做法有合理之处,一个有心理咨询能力的咨询师也会用,全面地改变旧有的行为模式也得依靠这些方法。但仅凭这些做法本身,仅凭“咨询师善于指引”的特点,是不足以判定咨询师有充足能力的。

\indicate{除此之外},还有一些主要依靠外在条件展露出来的特质,比如说室内布置、音乐、香氛、疗养条件等等。这些相比以上的两方面,更容易让人看出“和心理咨询能力没有直接关系”。

总的来说,虽然“会让人感到安心和可靠”的特质很少见,但不足以让我们判断一个人有充足的心理咨询能力。并且,由此产生的(可能自知,也可能不自知的)依赖和不切实际的希望,可能会对你所拥有的关系(不只是心理咨询师,还有友情、爱情、亲情等)造成持久的打击。你可能会因为“终于找到一个可以理解你的人了”而喜悦,之后又因为“觉得对方应该能理解你,但对方没有”而厌烦、难过、委屈。越是对对方的能力没有客观全面的评估,就越容易出现这样的问题。

\smalltopic{(2)心理咨询能力如何提供心理咨询服务}

本节中我们仅考虑具有充足能力的咨询师和咨询服务。如果不考虑经济等其它外部因素,心理咨询服务的最优水准,大概可以描述为“能够理解来访者的每一种行为模式,并且按需要培养和消解其中的一些,帮助来访者理解和控制其它的行为模式”。其中,对于一些行为模式来说,培养、消解、控制它们需要一些外部条件(比如说住院、疗养、搬家等方式以远离某种环境)。考虑经济条件的话,大多数心理咨询不具备这样的条件,所以在这段讨论中我们尽量避免涉及这一方面,只考虑那些免费或低价的资源,如书籍、视频或其他形式的知识,或是是app或其它形式的计划单。

在这样的条件下,普通心理咨询的最优水准,大概相当于“在来访者的身上增添理性”。我们将心理咨询增添的理性称为\indicate{咨询理性},将来访者本身可能拥有的理性称为\indicate{自身理性}。咨询理性对来访者来说实际上是一种污染,但它一般会比自身理性更加强大,主要起正面作用,能够发现更多的行为模式并做相应的处理。但同时缺点也很明显:\indicate{咨询理性仅能观察和处理咨询室内发生的事},不能跟随来访者,随时观察自身行为,指导自身行动。这会造成两方面问题:

\indicate{咨询理性难以认识某些行为模式}。这里的“认识”包含“发现、理解、模拟”。当然,这不是说“要完全信任咨询师,将自己的信息和盘托出”,这么做有严重的隐私问题。来访者在日常活动时很可能不会有意进行全面的自我评估,从而在观察、认知、表达三方面都有可能出现问题\footnote{读者可以注意到,“观察、认知、表达”其实就是事务处理的三要素在“使用理性”上这一事务上的具体体现。具体展开可以参考在2.4.2小节(2)中关于“\hyperref[para:无效沟通]{当面问也不会得到有效的回答}”的讨论。}。如果仅根据咨询室内的情况来分析和判断,就有可能忽略更重要的根本原因。

因此,咨询师会采用很多种不同的手段,来尽可能全面准确地得知来访者的完整状态。这些手段包括但不限于倾听、安慰、问卷调查、沙盘、情景模拟、催眠、环境布置......如果在某种环境中,心理咨询师能够观察到来访者在用某种行为模式来应对外部环境,那我们就称咨询师在和这种行为模式\indicate{对话}。

对话不是咨询师了解来访者的唯一方式,来访者也可以自己发现某些关键信息(比如“回忆起了某些关键事项”、“某些突发事件的反应和处理”、“应咨询师要求留意某些日常行为”等),并转述给咨询师。在转述时,来访者可能会重新进入当时的行为模式,从而咨询师可以与之对话,了解更多;也从头到尾只是转述,咨询师根据来访者的描述来理解和判断。

来访者要是因自身理性缺陷而无法观察到这些现象,或是观察到了却没有意识到重要性,或是意识到重要性却想不起来转述或羞于转述,咨询理性就没有发挥的空间。在日常生活和咨询时尽可能维持自身理性,会有效加快心理咨询的速度。

\indicate{咨询理性难以改变某些行为模式}。这里的“改变”包含“控制、培养、消解”。在咨询室内,来访者可以和咨询师展开深入全面的交流,对自己的行为有非常清晰透彻的理解,明白哪些事情是一厢情愿,哪些做法会事与愿违,哪些行为会带来痛苦。但一回到日常生活中,失去了咨询理性,就可能又恢复了原样。维持自身理性有助于改变行为模式,但培养自身理性本身也是改变,也很困难。

咨询理性仅在咨询室内存在(有些时候也在其它和咨询师有关的地方存在,比如说和咨询师远程沟通时,或者是回看自己记的笔记时)。在脱离咨询师的环境下,自身理性有可能相当脆弱,很容易因为各种刺激而遗失:有可能是“明明在咨询室已经明白了,但是真的搞砸了还是会忍不住地指责自己”,有可能是“明明知道要冷静,但是真的遇到事情还是会惊慌不知所措”,有可能是“没遇到什么事情,只是睡醒起来就忘了”。遗失了自身理性,就无法想起来要使用正确的方法面对,而只会重蹈覆辙。

因此,咨询师会采用很多不同的手段,来尽可能全面具体地培养来访者的自身理性。这些手段可能包括:一定程度的咨询时间外联络;带有仪式性质的固定行为训练;反复加深记忆的认知方法训练;对来访者行为模式深入的了解、分析、讲解;成体系的心理能力介绍......这些方法中有一部分可以视为污染,但污染在一些时候比理性更能实现有效的控制。

来访者也可以通过一些方法来维持自身理性。一部分方法通过借助外力的方式起作用,比如说使用定时闹钟来替代自律,使用笔记来替代记忆。这类方法省时省力,不需要通过训练就可以起效;但较为机械呆板,无法随机应变。一部分方法通过培养新习惯的方式起作用,比如说遇到麻烦环境时迅速脱离,在情绪不稳时使用吃喝、玩具等方式压惊。这类方法需要一定的培养成本,但面对常见情况时可以有效保持自己的理性,或是记录下来和咨询师分析,或是自己尝试解决和安抚。一部分方法通过维持分析能力的方式起作用,能达到这个水平已经不太需要心理咨询了。

\indicate{总的来说},想要让心理咨询充分发挥作用,自身理性是其中关键的一环,在很多事情上是不可或缺的。维持好自身理性,才能观察到自身更不明显,更不容易被记住的特点,才能系统性地梳理自身的长期行为表现。有了这些细致的认知以后,才能据此展开更详细深入的分析,从而更有效地自控。自身理性越是强大,这种良性循环的效率就越高;自身理性若是过于弱小,这种良性循环就无法建立,心理咨询也会失效。
\begin{explain}
一些人可以时刻维持自身理性,一些人可以在脱离之后迅速重新进入自身理性,一些人在某些环境内(独处、读书、和朋友交谈、心理咨询等)会进入自身理性,一些人可以制作环境(闹钟、纸条、背诵等)以激发自身理性。

在自身理性可以稳定存在的情况下,可以有意识地使用一些手段来增强自身理性的能力(比如心理咨询,再比如阅读书籍、写日记、规划目标、整理优缺点、自我鼓励),以实现更好的分析效果。自身理性不一定能理解这些手段的具体作用(于是它们严格来说属于污染),而是因信任而选择和使用。

理想情况下,随着接受心理咨询的时间逐渐累积,来访者的自身理性会逐渐成长,能够维持更长的时间,分析更深入的问题,还能在脱离之后重新进入。但实际情况中,自身理性在经历了一定的成长后,通常只能维持在较低的水平,有时还会因为遗忘、失落、不信任等原因而消解。只有在遇到某些关键契机以后,自身理性才会又开始一次新的成长。%这主要有三方面相互作用的原因:咨询时间较短(一周一两个小时,仅占清醒时间的1\%)、咨询师能力不足(参见之前的讨论)、自身理性能力不足(参见之前的讨论)。

这些关键契机的形式可能十分不同。有的是面对旧有问题时突然灵机一动,想起来了正确的方法并用上了;有的是在自己常规的方法都用尽的时候,才从记忆深处捞起来了理性;有的是和咨询师谈着谈着,突然想起了过去,获得了新认知;有的是面对重复的日常,突然决定不再继续维持,选择重新开始......

这些关键契机会给我们带来重要的新认知,从而明显影响我们旧有的行为模式。如果心理咨询师能明确识别并充分利用这些关键契机,就会对自身理性的培养起到显著的推动作用。
\end{explain}

\section{实操:原生家庭及其处理方式\label{sec:原生家庭}}
\hfill\begin{minipage}{0.65\textwidth}
\fontsize{8pt}{12pt}\selectfont\fontsize{8pt}{12pt}
\raggedright 我可爱的缺陷者,你的父母没能将你成功抚育。\footnote{原文为“私のかわいい欠落者、あなたの親は、あなたを育てるのに失敗した。”}

\raggedleft 石田翠《东京喰种:re》第52话 夏娃

\raggedright 继承了萨腾努斯的偏执,继承了威诺希的无知,可怜的孩子浑身都布满瑕疵。\footnote{\bilibili{av22271765};\\\indent \netease{554508692}。}

\raggedleft JUSF周存《十二号诛杀者》

\raggedright 一个人被神明剩下,祭奠枯萎的花。\footnote{\bilibili{av3360633}。\\}

\raggedleft 赛亚♂sya《被拯救者的物语》
\end{minipage}

%虽然我也不知道为什么,但近些年只要是心理相关的话题,都避免不了要提一提原生家庭。为了不让本指南显得过于不合群,还是展开来讲一讲吧。

原生家庭这个古老的概念在现代显得有些水土不服,主要原因是现代的孩子不完全生活在家里。一个孩子待在学校的时间比待在家里的时间多得多,在网上的沟通量比在家里的沟通量大得多。家庭仍然会给孩子显著影响,但是并不是所有的影响都是家庭带来的。据此,本指南仍然保留原生家庭这一名词,但在以下的讨论中,采用稍微不同的定义:\indicate{原生家庭}是\indicate{一个人会遇到的所有作为污染源的环境的统称}。
\begin{explain}
由这个定义,我们可以将讨论原生家庭化归为讨论污染。

注意此处对原生家庭的定义不包含其它形式的污染源。把“学校”和“圈子”称为原生家庭大概没什么问题,但是把“一句偶然看到的话”称为原生家庭就太怪了,把“自己得出的结论”称为原生家庭也看着不像人话。

在使用这种定义后,“工作环境”等一般属于成年人的环境也会被称为“原生家庭”。这听起来有点怪,毕竟“原生”这一前缀会给我们带来“出身处”的感觉,我们不太习惯把成年人当孩子看。但是考虑到工作场合确实也在散播“把公司当家”的污染,就其实还好。如果实在觉得奇怪,可以使用“原生学校”“原生公司”“原生圈子”等词汇代替。如果实在觉得“原生”这个词也很怪,可以将其视为心理方面为主的职业病。

读者如果喜欢,也可以把每种传播污染的环境分别单独称为原生家庭,这样一个人就会拥有(原始含义的)原生家庭、学校、公司、圈子等多个原生家庭。这种定义上的细微调整不会影响后续讨论。
\end{explain}
原生家庭的定义是中性的,但是实际应用时大多是贬义。以下篇幅不涉及“良好的原生家庭”的讨论。由于我们还未系统讨论环境对人产生影响的具体机制(相关讨论见\hyperref[sec:人的基本意识模型]{3.4节}),以下篇幅也将不涉及对原生家庭本身的分析,仅聚焦于“原生家庭带来的问题”及其处理方式。
\begin{examples}
由原生家庭带来的污染可以分为两类:一类是本身就没什么好处的;一类是本身在原生家庭之内有好处,但是在原生家庭之外会显得不适应环境的。而绝大多数污染,是这两者的混合。我们既因为污染的坏处而痛恨它,又因为污染的好处而继续保持。有利于家庭关系和睦、有利于自身成长、忍受不了别人的行为、看不下去别人的遭遇......因为这些理由,我们自我安慰,自我欺骗,从而在原生家庭的里外反复逡巡徘徊。
\end{examples}
如果你觉得上面这段话说得很有道理,让你深感共鸣,那说明你没有掌握本章的内容,没有理解本章介绍的分析框架。“好处/坏处”本身是认知,而认知依赖于思考回路,我们所感觉到的“既有好处又有坏处”,是不同的思考回路带来的。当你无法控制这些会给出矛盾结果的思考回路时,无论你根据哪一方做出决定,这个决定都不可能得到自己全心全意的赞同,也不可能被自己坚定地执行下去。如果你想清除原生家庭给你带来的污染,全面的考虑是不可或缺的。
\begin{explain}
清除污染的意思是理解并且控制。从实际效果来看,“控制至完全不触发”等状态也可视为清除污染。读者可根据自身需要调整定义。

清楚污染不是“和解”“接纳”“忍耐”等行为。这些可以作为手段之一,但不能是目标。
\end{explain}
不幸的是,只要你踏上清除污染的道路,就会立刻遇到地狱难度的开局。大部分思考回路没有理解和控制其它行为模式的能力,而如果一个思考回路是被污染而来的,那么想使用它们来理解和控制污染,更是几乎没有可能。
\begin{explain}
一些带有污染的思考回路可能对某些思考回路或者行为模式有强大的控制力。这些思考回路会高效产生坚定的错误认知,这些错误认知会使我们自我怀疑、自我否定、自我攻击、敏感、狂躁、偏执,从而甚至很难完成一些相当简单的事务(比如买水)。

很明显,这种带污染的思考回路才是最应该被控制或者消解的,但是无论是理解它还是控制它,都不是其它受污染的思考回路能做到的。
\end{explain}
另一方面,你所拥有的,也只是一个“想要解脱”“想要幸福”的愿望而已。这个愿望不会告诉你应该怎么达成它自己,不会告诉你怎么清除污染。更糟的是,每当你想起这个愿望的时候,就会产生“如果那样的话......”的幻想(这是一种作为认知的污染);每当你回归现实的时候,就会得出“果然我得不到......”的结论(这也是一种作为认知的污染)。这种恶性循环会掐灭每一次刚点燃的希望之火,你只能收获更多的痛苦。于是,很多人就这样放弃了挣扎,产生了“原生家庭不可逃脱”的认知。他们之中相当一部分完全没有意识到过“自己曾经挣扎过”。
\begin{examples}
网上和其它渠道有很多对于原生家庭的讨论。有些人是先放弃希望再看到讨论,有些人是先看到科普之后尝试并且失败。在放弃挣扎以后再看那些讨论,有可能觉得它们浅薄而苍白,有可能为自己感到浓郁的悲伤,有可能怨恨责怪自己周围的环境......这些不同的经历不在我们的讨论范围之内,这里不做展开。
\end{examples}
有些放弃了逃脱的人,可能还有“环境可以改善”的希望。一部分人会尝试沟通交流,一部分人会希望环境自己变好,一部分人会选择换个环境。但是,绝大多数人不会去关心环境实际上会是什么样。希望原有环境改变,但不去分辨和核实哪些东西可以改变,那些东西不能改变;希望来到新的环境,但不去了解和确认哪些东西确实不同,哪些东西其实一样。愿望能实现多少只能看运气,而大多数情况下是颗粒无收。
\begin{examples}
有些人可能会用身份、责任、感情等方式来劝别人做出改变。依照语境的不同,这可能有道德绑架、投射性认同、爱、操心、牢骚等多种称呼。但人们不是因为拥有某个身份就能履行某种责任的。如果缺失对应的能力,就无法履行责任。关于责任的讨论参见第三章。
\end{examples}
有些外部因素会调动你的愿望,带着你分析,得出一些结论。沿用上一小节的称呼,我们将你自身原有的理性称为\indicate{自身理性},将外部提供的分析能力称为\indicate{外部理性}。

在接下来的篇幅中,为了简便起见,当我们用到处于、进入、脱离等词时,会使用“\indicate{环境}”来统一指代“外部环境、行为模式、目标”;将“找到某个目标的解决方法”“改变某个环境”“维持某种状态”“消解一个带有污染的行为模式”等事务统称为“\indicate{应对}环境”,对应的处理方法称为\indicate{应对方法};将需要分析和应对的环境称为\indicate{目标环境}。

要想有效地清除自身的污染,首先需要维持自身理性的持久存在。这需要我们在有外部理性的情况下,尽可能将其内化为自我理性;在处于自我理性时尽量找资源来充实自身,同时尽量反思;在处于其它环境时尽可能维持理性,以分析和控制目标环境。
\begin{explain}
我们需要让自身理性维持以下这么几种特点:

\indicate{自身理性需要保持稳定。}我们需要尽量避免脱离自身理性。如果目标环境会使我们脱离自身理性,则需要事先想好应对的方式,并且在进入目标环境时能够使用。若我们处于目标环境时脱离了自身理性,应该尽快重新进入。这可以依靠一些外力的提醒,或者培养一些习惯。具体做法可以参考上一节关于自身理性的讨论,或是参考下方讨论。

\indicate{自身理性需要保持纯粹。}我们会同时处于很多种不同的思考回路之中。这些别的思考回路会干扰我们,产生带有污染的认知、情绪等失真的内容。相比于一般的行为模式,思考回路会更容易被触发,更难以控制,于是需要特别关注。除了控制思考回路本身以外,我们还需要将带有污染的认知和自身理性隔离开,避免干扰自己的思考和判断。我们需要坚定“清除污染”的目标,并以此来指导自身行动。

\indicate{自身理性需要保持记忆。}我们维持自身理性时,可以进行有效且深入的思考。这些思考会帮助我们应对目标环境。如果在面对目标环境时无法应用这些思考,或者是自身理性总是会思考同一件事,得出同样的认识,根本没意识到自己之前想过,或者是明明下定决心但是转头就忘了,那么自身理性就不起什么作用。我们可以使用一些辅助记忆的手段来避免这种情况。
\end{explain}
我们将“可以分析目标环境的应对方式,且不会失控地进入该目标环境”的情况称为该目标环境对应的\indicate{安全环境}。
\begin{examples}
依照具体目标环境的不同,安全环境所需要的条件从低到高可能是“处于该目标环境时”“处于某种知识体系中时”“处于自身理性时”“处于某种外部理性时”“不存在”。

一些特殊环境下可能有更复杂的情况。比如说在心理咨询时,咨询师有可能在来访者失控时有效地和目标行为模式对话,并且在来访者平复后将分析和结论反馈给来访者。后续的反馈环节对来访者是安全环境,但整体经历对来访者不是安全环境。安全环境根据人的不同(准确来说,根据自身理性的能力不同)而有所不同。
\end{examples}
一种目标环境的安全环境不一定是“远离该目标环境”。
\begin{examples}
远离目标环境不一定会使你可以分析这种环境,也不一定会使你不失控。“不在环境内,就想不起来对应的细节”的情况更多,于是你无法分析;“偶尔想起来的时候,会想回归这种环境”的情况更多,于是你因此失控。
\end{examples}
对于一般人来说,安全环境仅有“刚脱离目标环境,重新获得理性”时。只有在这种时候才能有鲜活不褪色的记忆和感受,只有在这种时候才能开始分析。我们将“在这种安全环境中分析”称为\indicate{立刻反思}。
\begin{explain}
并非所有行为模式都有脱离的机会,我们可能会一直处于一些行为模式中,尤其是那些可以进行自我评估的思考回路。

并非脱离了目标环境就能恢复理性。如果脱离环境以后去犒劳自己/散心/找人倾诉,那就错过了这次机会。

自身理性很强大的人可以做到在未脱离目标环境时就进入/保持理性,直接使用理性来分析自身,并且指挥自己的行动。但这难度较高。
\end{explain}
想要一脱离目标环境就立刻进入理性,对大部分人来说是很难的事。如果不能稳定地进入理性,就需要意志力来辅助。但之前身处目标环境中时,你可能已经耗尽了全部的力气,精神疲惫至极,没有劲再自己思考了。
\begin{explain}
此时如果有外部理性的帮助,就有可能可以展开有价值的分析,得到可以具体落实的改变。

在目标环境内,或者刚脱离目标环境时,如果能保持基本的理性以观察自身,并且记忆/记录下来具体的感受和行动,就会给后续分析起到相当大的帮助。这比立刻反思要容易一些,但仍然难度较高。
\end{explain}
如果无法立刻反思,那就需要使用一些手段,重新触发该行为模式,进而与之\indicate{对话}。这就需要一些专业技能来辅助完成。我们将“在这种安全环境中分析”称为\indicate{重新面对}。
\begin{explain}
这种操作有可能由心理咨询师主导,有可能看了一些心理书籍后习得了相应的方法,也有可能有些人无师自通。这总是需要一段连续的不受干扰的时间,如咨询、深夜、路上、开小差。

相比于立刻反思,重新面对是相对更常见的方法,我们可以不用卡时间,选择空闲平静的时间段再回顾。虽然分析的难度一般来说会更高,收效一般来说也会更差,但很多时候这是唯一选择。
\end{explain}
在安全环境下分析时,我们可以控制自己进入的目标环境,展开专注的分析。如果有某个环境会触发不可控的行为模式,那么我们需要将其\indicate{隔离}在外。
\begin{examples}
比如,在自己面对某个需要自己能力提升,同时有考核的任务时,有可能会产生“我果然什么都不行”的认知。它和生成它的思考回路会极大影响心情,消耗意志力。当没有考核的压力时,这种思考回路可能就不会触发。这使得我们可以分别处理“能力提升”和“我果然什么都不行”两个目标。

我们同一时间不一定只面对一种环境,不一定只处于一种行为模式之中。只有精准识别每一种行为模式,并且分别处理,才能真正解决问题。
\end{examples}
目标环境分为两种。一种是“缺乏应对的行为模式”,另一种是“会触发带有污染的行为模式”。
\begin{explain}
想应对“缺乏应对的行为模式”的环境,只需要拥有分析目标环境的能力,这种能力可以来自自身或者外界,只要该能力能够给出可行的应对方法,那就可以直接执行。注意,在此过程中需要隔离“我什么都不行”等思考回路,这属于带有污染的行为模式。
\end{explain}
想应对“会触发带有污染的行为模式”的环境,则还需要拥有控制或消解该行为模式的能力。这会困难得多。在安全环境下分析时,我们需要可控地触发或者模拟出我们需要消解的行为模式。这样才能随时打断自己的行为和思路,使用理性来思考“自己在当前情况下应该怎么做”。
\begin{explain}
如果不打断原有的行为模式,就会出现两方面问题:

原有的行为模式可能很占用精力,以至于无法完成思考。有些思考有时会很耗时间,完全不可能靠临场反应得出结论,只能提前想清楚才用得上;有些思考可以靠临场反应,但是如果被分了神就不行。

原有的行为模式可能会提供混乱的认知,以干扰判断。这有可能是因为认知本身错误或带有污染,有可能是因为不同行为模式下的目标不同(并且当前行为模式不知道如何实现该目标,也不知道自己的方法无效)。
\end{explain}
在安全环境下分析时,我们应该尽量根据已知的所有信息,充分考虑可能出现的每种情况,并尽可能想出全面可行的计划。
\begin{explain}
如果有必要,还可以加入一些模拟演练的环节,以增加熟练程度。一开始没有用上新方法没关系,在安全环境下多试几次即可。

一些情况下,确定自己的目标后,就可以据此确定哪些事情应该重点关注,哪些事情不需要管,并且据此来控制自身的行为。但这难度较高。
\end{explain}
分析不一定能给出有效的应对方法。这有两种可能:\indicate{信息缺失}或\indicate{能力不足}。
\begin{explain}
有时在分析时就能直接确定无法得出应对方法;有时在再次面对目标环境时才能发现应对方法行不通。

信息缺失有多种可能,如“实在想不起来具体情况”、“不知道某些做法会带来什么后果”等。这种情况下做的计划不保证能达成自己想要的结果,具体执行时可能出现意料之外的情况。

绝对不应该在信息不足的情况下,给出确定性的判断,期待某种特定结果会发生。无论是觉得“这么做肯定会顺利”还是觉得“这么做肯定会搞砸”都不应该。这种混乱的认知会干扰你的判断。信息不足的情况下,应该对每种可能出现的结果都做好充足的准备,但这难度较高。

如果你能够承受可能带来的结果(可能是经济、关系、情绪等多方面),就可以主动去尝试以获取信息,但这难度较高。无论成功还是失败,尝试都可以为我们提供信息,从而进一步分析、判断、计划。这种情况下,我们可以说\indicate{失败是成功之母}。另一方面,如果失败不能提供信息,或者失败会造成大量损失,那就不应主动失败。
\end{explain}
如果在安全环境下确实分析到位了,那么当你再次面对目标环境时,你应该能按照计划好的方式思考和行动,得到意料之中的结果,实现想要达成的目标。
\begin{explain}
有可能你的意志力会慢慢耗尽,从而你只能在开头的一小段时间内保持理性,而后又会陷入旧有的行为模式中。但这也能说明你使用的方法是正确的,只需要继续提高自己的熟练度,让新习惯覆盖老习惯即可。只有当持续时间总是没有提升,总是会因为相同的原因而失控时,才需要另行处理。
\end{explain}
如果总是无法按计划行事,总是无法达成目标,或者在分析阶段就无处下手,那么就基本可以确定是\indicate{能力不足}。
\begin{explain}
我们以为的安全环境只能做到不失控,而无法做到分析,它实际上不是安全环境。一般有两种情况:要不然是因观察不足、条件限制等原因,无法模拟目标环境(比如说不见到陌生人就没那么害怕),要不然是因能力不足,无法分析并得到结论。

如果有条件,我们可以借助外部环境的帮助,从而补齐展开分析的条件。可以通过他人、录像、日记、随笔等方式来辅助观察;可以通过心理咨询、自己学习等方式来提升能力。

特别需要注意的是,在明确了自身\indicate{能力不足}后,\indicate{不要草率地反复尝试}(除非你想要通过尝试来获取信息)。初期的尝试需要消耗意志力,而挫败感会极大损耗意志力,加深恐惧和无力感。面对一种棘手的环境,在自己怕得无法迈出步伐之前,我们\indicate{只有少数几次机会来尝试新方法,一定不要浪费}。
\end{explain}
如果以上所有方法都无效,那么我们“使用理性,立即化解污染”的尝试即可宣告失败。我们需要使用更加曲折的方式来解决:培养一个新的行为模式。
\begin{explain}
这个新行为模式大部分情况下是污染,这是一种使用污染来对抗污染的手段。
\end{explain}
实际操作起来没有听上去那么糟:我们需要接触一个新环境,在新环境中培养新的行为模式。新环境和目标环境之间有一定的相似之处,但它不会触发/不会频繁触发原有的行为模式。新的行为模式同样可以应对目标环境,其中的行为和行为链会逐渐代替旧有行为模式中的行为和行为链。当新行为模式足够熟练后,旧有行为模式或是无法再被触发,或是触发后无法形成行为链,于是就被消解了。
\begin{examples}
由于换个新环境是很常见的事情,这种方式会自觉或不自觉地被人使用。即使不理解原理,换了新环境后,也有可能自然地感觉到了自由的空气。

很多人上大学之前会觉得家里住着没什么问题,但一个学期以后,就只会感觉到发现家里待着很不舒服;农村的孩子来到城市见过世面以后,就很难再待在自己村子里。

能清晰地感觉到不舒服,就可以使用理性来解决后续问题了。此时也会有一些人选择远离原有的环境,本指南不对此做道德和责任方面的评价,也不将其简单概括为“由俭入奢易,由奢入俭难”等顺口溜。我们只关心最普遍的“因为行为模式的变更,态度和行为产生了变化”的现象。
\end{examples}
这种操作有两方面要求:

一方面是\indicate{新环境需要提供足够多的触发条件}。我们不应该自己思考“如何处理新环境”,不然思考深了就会回到旧有思考回路中(因此会培养出带有污染的新行为模式)。等新行为模式足够熟练后,才有深入思考的机会。

另一方面是\indicate{需要尽量避免触发旧行为模式}。我们应该尽量减少对比新旧环境、新旧行为模式的可能性。如果条件允许,那么直接远离旧环境;如果条件不允许,则应尽量减少自己对旧环境的反应。

达成这两方面条件本身就有一定的难度。即使这两方面都能达成,这种方法也有隐患:我们为自己增加了一个新的原生家庭,受到了新的污染。
\begin{explain}
如果事先调查或了解过,就可以清晰地认识某种环境,清楚环境会培养出什么样的行为模式。这种情况下可以不受什么污染地进入该环境,得到自己想要的。常见的环境可能包括“好学校”“好工作”“自己的房子”“独居”“同居”“心理咨询”之类。

但是,如果只是对环境有一厢情愿的幻想,就有可能会相当吃亏。
\end{explain}
在一些特定情况下,旧行为模式会被巧合性地控制住。在旧环境和旧行为模式的互动中,我们会培养出一些新的行为,这些行为有可能会避免旧行为模式的触发。我们将这种行为称为旧行为模式的\indicate{掩盖}。
\begin{examples}
我们可能在大哭一场后就想不起来之前都受过什么委屈了;我们可能面对某些事情时会害怕但是不知道自己为什么怕;我们可能自杀以后就丢掉了几个月以来的记忆......

如果读者喜欢,也可以将其称为可\indicate{大脑的防御机制/保护机制}。

\indicate{掩盖}的形式十分多样。有可能是刻意忽视了某些信号;有可能是熟练地经历了某个思考回路后,仅输出结果认知;有可能是强行转移注意力......基本上任何形式的行为都有可能成为掩盖。
\end{examples}
“会触发旧行为模式”和“不会触发旧行为模式”相比,“不会触发旧行为模式”相对更有利一些。但掩盖也是严重的污染,需要处理。
\begin{explain}
在时间经过充分久,行为模式改变充分大以后,旧行为模式有可能已经自然消解,此时就可以使用理性来重新发现旧行为模式,将旧行为模式和掩盖一并清除。

如果读者喜欢,也可以将其称为\indicate{看到内在孩童}、\indicate{觉知自己的潜意识}或其它。
\end{explain}
至此,我们已经基本上讨论完了所有清除污染的方式:要不然主动控制,要不然等它逐渐消解。因为篇幅所限,此处的讨论不够具体,更详细的展开参见之后的内容。

拥有成熟自我的标志,就是\indicate{不再受到新的污染}。只要做到这一点,已有的污染就可以慢慢解决,你也就会越来越自由。

\titlespacing{\chapter}{0cm}{5cm}{1cm}
\begin{savequote}[450pt]
    承认吧,我们什么都不知道。如今才明白孤独并非孤独的全貌。我们所仰仗的那株脆弱的稻草,甚至比不上作家笔下的火柴牢靠。\footnote{\bilibili{av42228720};\netease{1343548780}。}
    \qauthor{DELA\_P\&雨狸《我对孤独一无所知》}
    正常来说,概念这种东西是没办法直接影响周围环境,伤害敌人的,不是说写下“大质量、大引力、高温、高热、聚变”这些词语就可以制造类似的效果。
    \qauthor{爱潜水的乌贼《诡秘之主》第八卷\ 第三十一章\ 概念化}
    你说人一样一样都没区别,在一堆一样一样自成一类。一样的世界,一样的感觉,一样地生搬硬套成了生活的情节。\footnote{\bilibili{av4894016}。}
    \qauthor{小旭PRO\&绛舞乱丸《一样一样》}
\end{savequote}
\chapter{复杂现象与复杂概念的组成方式}
%\titleformat{\section}{\bfseries}{}{0em}{}

可能有一部分读者看完第二章以后信心满满,觉得自己已经掌握了解决心理问题的方法,可以立即让自己挣脱污染,走向光明、希望和解脱。但很可惜,你高兴得太早了。事情哪有那么简单。

如果读者读过其它一些心理健康书籍,就会发现,前两节的内容和其它心理健康书籍所谈论的话题相差不大,本指南介绍的方法其它书里面也有。相比之下,其他书可能还写得更好一些,语言更亲切,有更丰富的例子来辅助理解,而本指南堆了一大堆理论辨析,十分让人头疼。但即使已经有这么多书了,而且有些人也看了不少书,收获却有限。

我们会被某些东西刺痛,我们会被某一句话打动,但是过不了一周,生活就又回到了原有的轨道中。到底欠缺了什么东西?为什么明明什么都懂,却还是无法面对?为什么明明听过很多道理,却依然过不好这一生?知识的积累和内化到底哪里有差距?本章所要做的,就是尝试回答这些问题。

要一句话回答这些问题也可以,那就是“事情没你想得那么简单,现实世界比你的理解更\indicate{复杂}”。但这一句话无法加深你对现实世界的理解,无法指导你解决实际事务。我们还需要更详细的讨论,用一整套分析框架来面对这个问题。
\begin{explain}
如果只是粗略阅读,那么读者可能会觉得本章所讨论的内容像哲学。有这种既视感是正常的,哲学家也会讨论同一类型的话题。但是,本章的思路和哲学无关,涉及这些内容完全出于实用性。

本章内容不需要哲学基础。无论读者的哲学基础如何,无论读者是觉得本章内容过难还是过易,都请读者尽量避免过多的哲学联想。如果确实有明显的阅读障碍(无论是看不懂还是无聊),可以先阅读本章总结,或者是本章后的整体梳理,再根据需要来回头阅读必要的部分。
\end{explain}

\section{概念与特点\label{sec:概念与特点}}
\begin{explain}
再次提醒:本节内容不是哲学,不要以“玄而又玄”的视角来看待。本节内容所涉及范围不超过中学语文课。
\end{explain}
\subsection{指代}
\begin{explain}
本小节关于笔的话题来源于知名meme《什么是笔》。
\end{explain}
“笔”在目前通行的语言环境中,主要指\indicate{用于书写绘画的工具}。
\begin{explain}
“笔”的原意是“书写”,但这一义项在后续演化中已基本消失。当今“笔”的“书写”义项多为引申义。本小节不做深入的语言学研究,仅略微提及。

别的语言环境下可能有所不同,如英语中没有只作为总称使用的词汇(pen还有额外的“钢笔”义项)。本小节仅以汉语为例。
\end{explain}
如果我们仅使用定义来看待“笔”,那么很多问题,如“这支笔你是在哪买的”,是无法回答的。我们在日常生活中,从来不会真觉得这个问题无法回答,是因为此时有一次\indicate{指代}。和“购买”有关的信息来源于“这支”:它锁定了交谈者附近(可能是身边,可能是手指的地方,可能其它)唯一具有“笔”的特征的东西。
\begin{explain}
\indicate{指代}必须结合环境。脱离环境的“这支笔你是在哪买的”问句无法回答。在没有语境的情况下,听到有人突然问了这么一句,多半会反问一句“什么?”。

而基于概念本身的提问,如“笔是干什么的”,则无需结合环境即可回答。
\end{explain}
我们将\indicate{通过部分特点锁定环境中的具体对象,以谈论它的其它特点}的行为称为\indicate{指代}。\\指代是人类在正常交流和思考时,不可缺失的一项重要能力。
\begin{explain}
我们永远都无法完全描述一件物体。一支笔处于什么位置,是什么材质、形状、颜色,为什么能书写,属于谁,现在还能不能用......只要想关心,总会有无穷无尽的信息可以获取。

但我们永远也不需要完全描述一件物体。通常只需要很模糊的信息(如“在你附近”“蓝色的”等),我们就可以完全锁定具体在指哪只笔。

仅靠部分信息,就完全锁定交谈对象,会不可避免地使用到指代。
\end{explain}
在一些情况下,\indicate{指代}可以不用完全符合定义给出的特点,词语于是有了引申义。
\begin{examples}
在使用“笔”的语境中,有些只在意笔的“记录信息”功能,如软件中的画笔工具;有些词组引申出了“书写”行为,如“执笔”;有些不在意功能,只关注笔的外形或其它方面,如“触屏笔”、“录音笔”、“坏掉的笔”等。
\end{examples}
指代经常导致词语含义的演变。
\begin{explain}
使用“书写”来指代“书写工具”,指代多了,就会使笔的含义也变为“书写工具”;在某个较为独立的环境,如具体学科、家庭、学校、公司中,也经常会有各自的“黑话”。

本指南不过多探讨词语在社会中的整体语义流变,仅关注在具体小环境下的变化。
\end{explain}

\subsection{概念}
我们所使用的每一个\indicate{概念}都具有如下特征:概念中提供了一组\indicate{特点},我们通过判断某个具体对象是否符合这些\indicate{特点},来判断是否可以用这个概念来指代它。我们将符合定义的具体对象称为这个概念的一个\indicate{实例},将用于判断的\indicate{一组特点}称为这个概念的\indicate{定义}、\indicate{含义}或\indicate{判据},将判断过程称为\source{识别}、\indicate{锁定}或\source{特征提取},将指代也称为\indicate{套用}、\indicate{概括}或\indicate{称呼}。

如果一个词语有多种定义,那么将每种定义称为它的一个\indicate{义项},此时称这个词语有\source{歧义}。
\begin{explain}
此定义的\indicate{概念}的定义较为宽松,只要是一个实词,都可以算做概念。甚至名称(针对某一特定个体的概念)也算。

对于有一定语言学或哲学基础的读者,此处的“概念”有能指和所指的混用,笼统地描述整个意指作用;“字”“词语”是能指;“特点”“含义”“定义”等表述是所指。本指南不深入讨论这部分内容,做了相应简化。

使用统一的概念可以大幅度缩减思考和沟通的成本,可以将原本使用数句/数段的描述简化为少数几个字。使用概念是人类在正常交流和思考时,不可缺失的一项重要能力。

本指南使用“特点”这一概念或其实例时,默认其没有歧义。行文时会尽量避免使用事实上有歧义的特点实例。

歧义总体来说是贬义的,但也有“同一个词同时表示多个不同义项”之类的用法,这在文学作品或其它领域是正常的运用方式。我们在此仅关心“因为表达不充分或理解不充分而产生歧义,导致交流出现障碍”的现象。这毫无疑问是负面的。
\end{explain}
%\begin{examples}
%大多数情况下,我们不需要很精确地知道自己的判断依据。我们在确定“面前的人是谁”的时候,绝对不会在脑子里清晰地把自己的理由都过一遍。很多判断(如通过外貌识别出眼前的人是谁)是完全不可控的行为。
%\end{examples}
在特定环境下,对于同一个概念,可能有好几种不同的判据都会识别出同样的对象。此时我们经常通过最方便的那一种来识别。
\begin{examples}
严格来说,名字用于指代一个具体的人,而不是某种特定的外貌和行为。但没有谁是通过“每时每刻观察一个人,确定这个人一直稳定存在”才能确定这个人的名字。靠外貌和行为通常就能锁定一个人。

靠外貌和行为的判断方式在一些情况下会失效,如双胞胎,或者恰巧外貌和行为相似;很久不见一个人后,外貌和行为都有可能发生大变化。

以上的误判可以通过“一个人亲口说出自己的名字”等方式来解决。这种判据也会在一些情况下(如神志不清时)失效,此时可以使用一些其它方法(如检查身份证件、查验DNA)确认。

我们极少通过“一个稳定存在的人”的原始定义来识别这个人。这多数情况下既无实践可能也无使用价值,只在少数罕见情况下(如有替身,需要持续监视)有必要使用。
\end{examples}
我们不一定明确知道自己使用的概念是什么定义。这有两种情况:一种是有确定的判据,但自己不清楚;一种是多种判据混用。我们将后者视为\indicate{指代},而前者不视为指代。
\begin{explain}
如果不清楚定义,我们在使用同一个词语时,就有可能在根据它的不同定义分别谈论。这些谈论有可能偷换概念,错误地关联起一些本来无关的事情,得到的结论也有可能无效。

可以通过增加具体描述来避免义项混用,将讨论的概念限制在“xxx的外貌”“xxx的(某个具体)行为”等具体特征上。
\end{explain}

\subsection{能力边界\label{sec:能力边界}}
定义只判断事物是否符合对应的特点,而不涉及这些特点的来源。使用概念来概括实例,就仅能分析它的一部分特点,遗漏了。这会带来不可避免的\indicate{信息损失}。\\
\indicate{存在信息损失时,分析是单向的,不可能由果推因}。
\begin{examples}
这里的“果”指“概括得出的特点”,“因”指“对应的实例的其它特点”。熟悉逻辑学的读者可以看出,这里指的是“若A是B的充分条件,那么可以由A推B,但不能由B推A”。

笔的特点仅有“能书写”。不同的笔书写原理不同,铅笔、蜡笔等使用摩擦留下痕迹,其它笔多使用墨水或颜料,而具体的方式也有不同。这不影响它们统一被称为“笔”。在通用含义下,“笔为什么能书写”这一问题等价于“能书写的工具为什么能书写”,我们不可能脱离语境,给出统一的答案。
\end{examples}
每个思考回路都会使用很多概念。我们将\indicate{某次思考时使用过的所有概念的所有特点}统称为此次思考的\source{出发点}或\indicate{依据},并且将出发点称为此次思考和所得到结论的\indicate{适用范围}或\source{能力边界}。
\begin{explain}
我们不一定需要分别定义每一个概念。实际上,我们很多时候只关心一些概念之间的联系。这些联系也是出发点的一部分。
\end{explain}
我们将“在某一范围内,所有有关的所有现实现象”依照具体语境的不同,称为一种\source{环境}、\indicate{情况}或\indicate{现实情况}。
\begin{explain}
具体的范围需要在使用这一概念前具体指定。在本指南中,若无明确说明,默认的范围是“会使讨论对象(如行为或结果)出现差异的相关因素”。
\end{explain}
如果我们在思考时,仅根据出发点来使用概念,而不使用指代,那就将这次思考称为\indicate{完全理论性的}、\indicate{和环境/外部无关的}或是\indicate{就事论事的}思考。相反,如果使用了额外的信息(也即使用了指代),那就将这次思考称为\indicate{和环境/外部有关的}或是\indicate{依赖直观}的思考。
\begin{examples}
如果问题是“钢笔为什么能书写”,就可以结合物理知识,使用“因为墨水会随着毛细现象渗进纸张”来回答。钢笔包含“自身结构特征”的信息,这是完全理论性的分析。

如果问题是“这支笔为什么能书写(指向一支钢笔)”,那么同样的回答对“笔”这一概念来说,就是和环境有关的的分析。如果听到这句话的人(比如小孩子)不清楚钢笔的相关知识,就有可能误用这一结论来解释其它笔的书写原理。而另一方面,这个回答对“这支笔”这一概念来说,是完全理论性的分析。

在极少数情况下,一个思考回路可以不包含识别方法,巧合地每次都就事论事地分析。除此之外的绝大多数情况下,总是在能力边界内分析的思考回路都包含对应的识别方法,都是\indicate{知识体系}。\footnote{\rigorous 可以验证此处的定义和\hyperref[para:知识体系]{2.1.1小节}中的定义一致。判断一般的行为模式是否是知识体系,仅需验证其子思考回路是否包含对应的识别方法。}
\end{examples}
如果我们的认知来自某个/某类具体事件/现象,那么就将其称为该认知的\source{参考事件/现象}。如果我们通过分析参考事件/现象得到认知,那么也将其称为该分析的参考事件/现象。
\begin{explain}
这里所区分的是“该认知是出发点还是推理的结果”。如果该认知的形式是“有信号就会有相应影响”,那么“对该信号的预料”就有对应的参考事件;如果该认知的形式是“有信号说明这是某一类事件,而这类事件会有影响”,那么它就没有参考事件,而“这类事件的判断标准”和“这类事件的影响”需要分别再判断是否有参考事件。一个认知可以有多个不同的来源,此时我们应该分行为模式看待。

参考事件可以是自身经历,也可以是文艺作品、听人讲道理等其它类型的事件。参考事件不一定是自知的,识别信号和后续行为有可能完全是下意识的反应。这一过程会产生不全面和不真实的认知,此时这种认知是污染,参考事件是污染源。
\end{explain}
如果某次思考时,明确知晓自身的出发点,并且只做完全理论性的分析,那么就将这次思考称为\indicate{在能力边界内}思考,或是\indicate{清醒地}思考。
\begin{explain}
出发点中的一些认知可能有自己的分析过程。如果此次思考并未涉及到那些分析过程,就可以不向前追溯;出发点中的一些认知有可能是污染,但只要此次完全理论性分析时知道自己使用了该认知,那么就仍然认为是在能力边界内;不同的思考回路涉及同一个词的时候可能使用了不同的定义,如果每次思考都清楚这次使用的定义,并且清楚之前的结论用了什么定义,那么也应认为这是在能力边界内。

和环境有关的结论在脱离环境(且无法模拟环境)后就会变为污染。这一过程也会发生在同一个人的不同思考回路之中。前后两段思考的出发点可能不同,前一段思考所使用的概念可能无法迁移至后一段思考中,前一段思考所得到的结论可能无法与后一段思考出发点的其它部分相容。
\end{explain}
如果一个思考回路能完全理论性地思考某个概念或进行某段分析,那么就称这个思考回路可以\indicate{容纳}这个概念/这段分析。
\begin{explain}
注意这里的\indicate{容纳}不是“包容、忍耐、忍受”之类的含义。它和具体的价值判断与感情倾向独立。

思考回路无法容纳的认知,对于这一思考回路来说,就是污染。类似于模拟,我们可能也只有在特定环境之内才能容纳某个认知。

如果出现“不使用某个概念,你就无法把思路顺下去”的情况,那么基本就可以确定你实际上没有掌握这个概念,自身不足以容纳对应的分析。能容纳分析时,我们能将概念替换成相应的描述,仅使用描述中的特点也可以完成分析。概念应仅起到简化的作用。
\end{explain}
我们将\indicate{因为忽视了前提,导致分析和结论无效的现象}称为\indicate{劣化}或是\indicate{想当然}。一段思考中使用了越多的劣化想法,就称这段思考越\indicate{劣质}。
\begin{explain}
越是不可避免地需要借助环境来思考,对自身思路的整体把控能力就越差,就越是无法容纳完整的思路,就越是需要借助环境中新的信息来思考。随着这一恶性循环,劣化的想法就越来越多。如果一个思考回路包含了足够多的劣化想法,随着不自知的偷换概念,有效的思维链条会很短,就会产生内部的矛盾。但同时,因为该思考回路对自身整体把控很差,它无法将自身整体纳入考量,只会判断“每条想法是否有问题”,然后得到“没问题”的结论。
\end{explain}

\subsection{应用}
\label{def:知识体系}一个知识体系会明确自身的出发点,使其不因具体应用而改变。我们将其称为知识体系的\indicate{基本假设}。
\begin{examples}
这些基本假设包括但不限于数学中的公理,物理中的定律、理想模型,以及一些其它学科中“对现实现象的归纳总结”等前提条件。

在19世纪,数学发生过一次重大的认识论转变,公理不再被看作“不证自明的命题”,而是被看做“某个研究领域的基本假设”。我们通过检验“数学对象是否符合公理”来判断“是否可以使用该领域的结论”。公理和对应的领域不负责阐述“符合公理的数学对象从何而来”,只从基本假设出发继续推导性质。一个明显的例子是,数学家们找到了欧几里得的很多依赖直观的漏洞,并且使用希尔伯特公理体系将其严格化(这不是双曲几何的故事)。

20世纪时,这种观点转变影响到了物理学,尤其是难以直接观测的量子力学。物理学家们不再使用“定律”一词,而是将薛定谔方程等内容称为“基本假设”。理论增添了“适用范围”的判断标准(如经典力学仅适用于宏观低速情况,中学物理还会要求“可以看做质点”等),在应用前需要判断对象是否符合要求。但这并不影响我们做完全理论性的推导。

\trained 可能可以看出,这里在讨论\indicate{范式}的一些特点。本指南不深入认识论和科学哲学内容,对此仅做简单介绍,以供读者参考。
\end{examples}
对某一具体领域展开研究的学科,可能因为历史发展,内部包含多个不同的知识体系。更换基本假设有三种可能的情况:发现某基本假设多余,可以由其它基本假设得到;发现了等价的基本假设,该假设在一些情况下更方便(或有其它优势);找到了更深入的基本假设,能推出原有基本假设,并且能解释一些新现象。
\begin{examples}
第三种情况涉及到研究领域的改变,是重大的研究突破。不可能只靠完全理论性的分析,就从原知识体系出发,得到更深入的基本假设。原有基本假设不包含它的信息。只有获得了新信息(或者是调研,或者是直接猜),才能得到新的基本假设。
\end{examples}通过明确基本假设,一个知识体系得以只研究某种特定层次的现象,而不用无穷无尽地向前还原,不用解释每一种现象的原因。
\begin{examples}
基本假设可能来自对现实的直接观察,也可能来源于对某些现象的深入分析和提炼,也可能就是凭空创造。无论哪种情况,都可以做纯粹理论性的分析。只要对象符合基本假设,那么就可以使用分析得出的结论。

%读者可能会注意到,本指南中对概念下的定义都很简短。对概念的其它讨论主要有以下三方面用意:向读者演示如何识别概念对应的实例;明确表述,消除可能存在的歧义;从概念出发做完全理论性的分析。如果读者能力足够,可以仅凭定义,就独立完成这些讨论。
\end{examples}
一个知识体系通常体量很大,内部包含的分析相当繁杂,通常难以穿过冗长的逻辑链条,回归基本假设。更常见的情况是\indicate{以知识体系的某些结论作为出发点,展开分析}。我们将这种思考称为\source{应用}这一知识体系。
\begin{examples}
一些学科的基本假设来自领域中已有的深入研究,对初学者可能过于抽象复杂,无法准确识别和正确使用。此时可以通过一些覆盖面更小,但同时更容易上手的基本假设,来辅助初学者接受这些概念。
\end{examples}

\section{复杂系统\label{sec:复杂系统}}
\subsection{编织}
我们将\indicate{由一个事件产生的现象}称为这个事件的\indicate{影响}或“这个事件\indicate{影响}了现象”,将\indicate{多个事件相互影响,组成更大的事件}的现象称为\source{编织}或\indicate{编织过程}。我们将编织过程中的每个事件称为\indicate{具体事件}、\indicate{微观现象}或\indicate{简单现象},将组成的更大事件称为\indicate{宏观现象}或\indicate{复杂现象}。我们将每个微观现象称为宏观现象的一个\indicate{组成部分}。我们允许“微观现象编织成宏观现象”的表述。

我们也将“微观过程组成宏观过程”\label{def:复杂过程}称为编织(注意过程是指一类具有共性的事件);也将“微观认知组成宏观认知”称为编织(注意“使用认知”可以视为事件)。
\begin{examples}
这里所定义的编织是一个非常广泛的定义。相互影响可以是各种形式的接连触发,可以并行、串行,或更复杂。可以是时间上的前后顺序,也可以是逻辑上的前后顺序。

前文中也涉及到很多可以被视为编织的概念,如行为编织成行为链,进而编织成行为模式,行为模式编织成人,以及事务编织成更复杂的事务、事件编织成更复杂的事件、行为模式编织成更复杂的行为模式。

其它学科中也有对应于编织的概念或例子,如专门的复杂性理论、系统论、统计力学、神经网络、格式塔、回声室、由个人编织成组织、由组织编织成更复杂的组织等。本指南不对这些内容展开深入讨论,仅作提及。
\end{examples}
我们将\indicate{只能从宏观现象中概括的特点}称为\indicate{宏观特点}、\source{表面特点}或者\indicate{表面现象}。
\begin{examples}
表面特点的定义也很宽松。它有可能是“某种稳定的状态”,比如说人的性格;也有可能是“会导致某种结果”,比如从某种思考回路中产生了某种情绪、使用了某个概念。

“\indicate{只能}从宏观现象中概括”的意思是“无法将其归因于复杂现象的某个组成部分”。会体现同一表面特点的不同复杂现象,可能包含完全不同的具体事件。我们只能从所有(主要的)微观现象出发,通过分析它们之间的相互影响,来推理出表面特点。不可能反过来,仅从表面特点出发,反向推导出复杂现象的具体组成。
\end{examples}
当我们称某个事物为表面特点时,若无特别说明,提到的复杂现象总是指产生该表面特点的复杂现象。
\begin{explain}
在同一个复杂现象中,某个特定的现象不可能既是表面特点又是微观现象。但复杂现象A的表面特点可以产生影响,作为微观现象参与复杂现象B的构成。

事实上,我们身边会遇到的绝大多数事件,要不然可以直接视为复杂现象,要不然可以通过少数几步归因,归到某个表面特点上。纯粹由简单现象组成的事件极为少见,通常只能在主动设计的介绍、教程等环节中见到。
\end{explain}

\subsection{涌现}
如果在某一类事件中,存在一些事件能够编织出宏观现象,那么就称这一宏观现象从这一类事件中\indicate{涌现}出来,称这一类事件为\indicate{低层次事件}或\indicate{深层事件},称涌现出的事件为\indicate{高层次事件}或\indicate{浅层事件}。

我们同时也称该宏观现象对应的宏观特点从这一类事件中\indicate{涌现}出来。我们将一类低层次事件和其中能涌现出的所有高层次事件和其宏观特点统称为一个\source{复杂系统}。
\begin{explain}
能从一类事件中涌现出的宏观现象数量不定。有可能一个都没有,也有可能涌现出多个。这一类事件的特性和相互之间的触发关系决定了它们能涌现出什么。这也导致某一个事件可能有多种不同的影响,想要找全“一个事件的所有可能影响”通常是很困难的事。

此处关于复杂系统的定义仅涉及两个层次的事件。也可仿照该定义,定义包含更多层次的复杂系统。本指南仅关注涉及两个层次的复杂系统,故不下更广泛的定义。

对于一个宏观现象的所有组成部分来说,它们所涌现出的内容可能不止这一个宏观现象。其中的一部分事件可能可以组成别的宏观现象。

每一个行为模式都是从行为和环境中涌现出来的现象。行为也可视为“从当前所有信号中选择了一个响应”这一涌现过程的结果。
\end{explain}
与编织不同的是,涌现并不要求每一个事件均起到作用。一类事件中可能有大量和宏观现象无关的其它微观现象。
\begin{explain}
对于一个宏观特点来说,不一定其对应宏观现象的所有组成部分都起到了作用。每个宏观特点可能仅和一些组成部分有关,不同的宏观特点所依赖的组成部分可能不相同。

这使得我们在面对涌现现象时,无法从“出现了某一类事件”就断定“一定会出现相应的宏观特点”。会导致宏观现象的事件可能具有另外的特点,与这一类事件并不重合;另一方面,另外的特点可能无法被良好地总结归纳,只能具体问题具体分析。
\end{explain}
如果一个概念的定义包含表面特点,那么就称其为\indicate{大词}、\indicate{复杂概念}或\indicate{宏观概念}。
\begin{explain}
思考回路由想法编织而成。大多数思考属于复杂现象,而使用概念和获得结论都是思考的表面现象。\indicate{人所使用的绝大多数概念都是复杂概念}。

我们日常使用概念时,大多数情况下不会很严谨地完全明确自身的出发点,而是会根据某些特点来套用。实际的使用方式,会从“所有和该概念相关的认知”中涌现出来。根据当前所处思考回路的不同,我们根据的特点可能很不稳定,会给很多不同的东西冠以同一个名字,用同一种结论套在不同的东西上。具体讨论见\hyperref[sec:人的意识演化基本模型]{3.4节}。

因此,任何一个复杂概念,都会不可避免地具有很重的\indicate{歧义}。如果不加以分辨就使用这些概念分析,得到的结果就会完全不可信。这些歧义有可能会让人感觉这些概念很\indicate{深奥},但“有歧义”和“确实很复杂”都会使人迷惑,都需要深入思考,一定要区分这两种情况,而不能将其笼统地看做“很有哲理,揭示了某种本质”。
\end{explain}

\subsection{对复杂系统的分析\label{sec:对复杂系统的分析}}
我们将\indicate{分析某个复杂现象时,仅将微观现象视为原因}的行为称为\indicate{微观分析}或\indicate{深入}、\source{客观}、\indicate{符合现实/实际}的分析。
\begin{explain}
微观分析的参考事件可能是微观现象之间的因果关系,或是编织过程整体。后一种可以用于分析“表面特点如何产生”。

分析微观现象时,有可能使用这些微观现象的某些特点来指代微观现象本身。我们也将其视为有效的分析。

分析编织过程的难度一般较高,需要对复杂现象有整体把握才能得出有效结论。这在很多时候需要掌握相应的知识体系才能做到。

“深入”一词可以指代很多不同类别的东西,如“全面”“准确”“有预测性”等。我们这里采用的定义大体上符合这些表面特点,但不可因此而使用这些表面特点来理解“深入的分析”,必须从定义出发。
\end{explain}
我们将\indicate{在分析某个复杂现象时,将它的某个表面特点视为原因}的行为称为\indicate{宏观分析}、\source{表面分析}、或\indicate{草率}、\indicate{武断}、\indicate{脱离现实/实际}的分析。
\begin{explain}
两种微观分析都会因为忽略了结论的适用条件,而劣化为宏观分析:

将考察微观现象时的结论误用于其它现象,通常是因为我们使用特点来指代微观现象,同时将微观现象的性质误认为特点的性质。这种情况可能在总结经验时,或者是举例、类比时出现。

将考察编织过程时的结论误用于其它现象,通常是因为我们过度简化了分析的流程,将表面特点简单归因于某个具体现象或者另一个表面特点,而不归因于整体的复杂现象,忽略客观存在的编织过程。这里有一种常见的很有误导性的情况:某个表面特点确实总是会导致另一个表面特点,但不同的实例中有不同的中间过程。此时我们很容易将其中的某个中间过程视为唯一可能的原因。

如果不能从微观现象中分析出表面特点,就不应认为自己理解了该复杂现象。不应认为宏观分析是对复杂现象的有效理解。如果我们脱离了宏观概念后就无法再对复杂现象展开讨论,那么先前使用宏观概念做出的所有分析和判断,就全都是\indicate{正确的废话}。
\end{explain}
对同一个复杂现象中的两个特点,在没有展开微观分析前,我们仅应认为这两个特点具有\indicate{相关性},而不应该认为这两个特点具有\indicate{因果性}。
\begin{explain}
这类因果性结论包括但不限于“xxx会导致xxx”、“xxx都是xxx”等绝对性的判断。

在研究简单现象时,如果我们确实控制好了其它所有变量,那么当有当两个信号总是先后出现时,就确实可以得出“二者具有因果性”的结论。

但我们在现实生活中遇到信号时,无从判断这个信号是出自简单现象还是复杂现象。如果两个先后出现的信号都出自复杂现象,都只是表面特点,那么它们只具有相关性,而不具有因果性。现实生活中的绝大部分现象都是复杂现象,如果没有确切的“当前信号出自简单现象”的证据,应该始终将其按复杂现象处理。\indicate{没有调查就没有发言权}。

如果我们总结出“出现前信号之后会出现后信号”的相关性结论,那就可以加以利用:比如在见到前信号以后,则可以提前准备,以利用后信号,或规避后信号的不利影响。但这仅限于被动地接受并处理现实,我们无法主动改变这一切。

如果我们总结出“出现前信号会导致出现后信号”的因果性结论,并且试图通过控制前信号来控制后信号,那就会出问题。由于前后信号没有因果关系,现实不会按照预料来进展。这种错误的归因已经是污染,而为失败另找理由还会产生另外的污染。
\end{explain}

\subsection{预料与副作用\label{sec:预料与副作用}}
我们将\indicate{认知中一个事件的影响}称为一个人在该事件上的\source{预料}。
\begin{explain}
“预料”指的是“将要出现的现象”,而不是“希望出现的现象”。“影响”可以仅关心“所重视的方面”,而不关心“是否观察到了事情的全貌”;“预料”仅关心“有这种认知”,而不关心“预料是否符合实际情况”。

以不同视角看来,同一事件有多个不同方面的结果,产生不同的影响。对同一事件的结果有多个判断是常见情况。比如“这事好麻烦,但很快就会结束,忍忍就过去了,但真的好烦”就包含两种不同的判断,它们基于不同的出发点,因此会让人纠结和烦躁。

有时会出现“虽然每一思考回路下只能得到一个判断,但是在多种场景中能得出不同的判断”的情况。每个思考回路都会认为自己是清醒的,直到因某些情况而将它们放在一起比较。
\end{explain}
我们对预料的判断会影响对事件本身的判断,进而成为对事件本身的判断,从而参与我们在该事件相关行为上的分析和决策。
\begin{explain}
考虑可能出现的影响会使我们的分析更加客观和实际。预料仍然可能变成污染,是因为这一过程往往是自知但不可控的,有两个可能出现问题的方面:一方面是判断本身就不真实/不全面,迁移到事件上后问题仍然保留;另一方面是有可能逐渐忘记了自己的预料,看到事件就直接条件反射般地产生判断,判断也因此成为污染。

这两种情况都会导致无法沟通的现象,具体讨论见第四章。简单来说,我们会因为观察到“别人无法产生相同的判断”而感觉“别人不清醒”,进而将其错误地总结为“别人无法沟通”。
\end{explain}
对于由行动而产生的事件,我们将\indicate{影响作为预料参与(某思考回路的)决策,从而成为该行动的动机}的过程称为“(该思考回路)\indicate{在意}该影响”,反之则称为\indicate{不在意}该影响。\indicate{事件的实际影响中,不在意的部分}称为“和动机无关的作用/影响”或“该行动的\source{副作用}”。
\begin{explain}
副作用的定义看起来有点绕。我们来具体看一下不同情况下的副作用:
\begin{itemize}
\item 对于不可控的行为,由于没有决策环节,行动直接被某个信号触发,此时该事件的所有影响都是副作用。
\item 预料可能不真实/不全面,此时错误和没有考虑到的方面就是副作用。如果这种预料被用于决策,那么这种操作可能无法实现目标。
\item 在针对某一目标做计划时,可能能预料到某步操作有负面影响,但因为还有其它更主要的正面影响,故仍然选择这步操作。此时负面影响是自知的副作用。
\end{itemize}
副作用是依据行动而判断的。对于“行动A有负面影响,从而采取行动B以消除/规避该负面影响”的情况,则只有触发行动B的思考回路在意这一负面影响,负面影响只算作行动A的副作用。对于\rigorous,可将“或是有负面影响,或是需要采取行动B”整体作为A的副作用看待(毕竟这两项中仅有一项是实际影响)。

一个行动可能产生不计其数的影响,从而也有不计其数的副作用。我们实际上无法完全认识到一个行动的所有副作用,不过好在我们也不需要这么做,只需要在特定的方面关注特定的副作用即可。
\end{explain}

\section{价值观\label{sec:价值观}}
\subsection{对人的表面分析}
行为模式是由行为编织而成的,行为模式的一次触发在绝大多数情况下都是复杂现象。因此,\indicate{人的绝大多数行为,及其所产生的影响,是表面现象}。
\begin{examples}
如前所说,行为本身可以作为微观现象,参与其它复杂现象(如另一行为模式)的构成。这不影响“行为本身可以是其它(更偏向无意识的)行为模式的表面现象”的结论。
\end{examples}
如果我们观察到了一个人的举动(可能是行为/行为模式),同时观察到了与该举动的某个影响,就有可能将影响视为这个人的目的。将\indicate{分析出的原因}称为该举动在\indicate{此次分析得到的动机}。这种分析在绝大多数情况下是表面分析。我们将此\indicate{通过表面分析得到动机}称为对该举动的\indicate{表面归因}。
\begin{explain}
分析得到的动机不一定是真实的\note{动机},有可能是该行为的副作用。这种分析方式的出发点是“人会依照自己的目的行事”。但这种草率的归因是完全不可信的,有以下几方面问题,可以与副作用的情况一一对照(以下将做出举动的人简称为对方):
\begin{itemize}
\item 一件事的影响有很多方面,每一方面都有可能是目的。对方不一定将其中的某个影响当做了自己的目的,有可能只是副作用。
\item 对方可能有自己的目的,但自身的分析有问题,从而导致举动不能实现目的。此时的目的不是任何一种影响,不可能通过这种方法分析出来。
\item 对方可能没有目的,而只是感受到了某些信号,并且条件反射式地做出了举动(同样地,在得到更多的信息之前,也不应草率地认为就是条件反射)。
\item 对于行为模式层次的举动,有可能对方在具体的行为上有明确的目的(或是感受到了信号),但是没有整体的行为模式层次的考虑。行为模式客观上从行为中涌现出来,但对方对此没有自觉。
\end{itemize}
以上的所有内容,在分析“自己的另一种行为模式/思考回路”时也适用。同样不要草率地概括自己的动机,不要草率地觉得自己的行为出自某种特定的潜意识,这会掩盖真实的原因。
\end{explain}
\indicate{通过表面归因得到的动机,对改变现状毫无帮助。}
\begin{examples}
这里的现状指代任何一种行为模式。导致举动的另有其它因素。无论是想要培养,还是想要去除,通过表面归因都起不到什么帮助。\trained 可以将其称为“形式主义”“主观主义”。本指南不引入这些概念。

比如,通过参考一些例子,我们可能会得到“吃苦才能成功”的结论。“吃苦”作为一个复杂概念,具有很重的歧义,如“做自身抵触/缺乏意愿的事”和“投入资源做事”。真正起作用的特征是“做会有成效的事”,二者都离其有一定距离,成功的人实际上也不一定真吃了苦。正确的思路是“对于一个确定的目标,和会有成效的事,即使需要强逼自己和投入资源,也需要去做”。但如果只看到了“需要强逼”,那么这不仅无助于成功,甚至无法起到磨炼意志力的作用。
\end{examples}

\subsection{大道理}
我们将\indicate{通过表面归因得到,并且可以指引行动的结论}称为\source{大道理}。
\begin{examples}
每一条大道理都是一条\note{认知},在对应场景下会变为动机。表面归因不提供对复杂现象的理解,所以所有的大道理均可视为污染。大道理仅能指引一个人按简单逻辑做事,无法起到辅助思考的作用。

污染不一定起到负面作用,按简单逻辑行事有时便足以维持某种状态。如“爱护环境”“勤俭节约”“遵守交通规则”等大道理,在大多数情况下都是良好品质,有利于社会的和谐与稳定。

但同时我们也能举出一些过犹不及的例子来,如“因为不舍得剩饭剩菜而食物中毒”“因为要等红灯所以堵住了救护车的通行”等情况。如前所述,这是因为忽略了这些大道理的适用范围,无条件地依据这些大道理而行动。

站在客观的立场上,我们能够轻易指出这些极端情况的谬误之处。但是如果以遵守大道理为优先,那么这就是大道理指引的行动,这种选择就是正确的。
\end{examples}
如果在应用知识体系或其它结论时,忽略、遗失或过度简化了结论的适用范围,那么结论就会劣化为大道理。
\begin{examples}
很多情况都会导致劣化,如“因使用过于熟练而不再检验前提条件”、“与他人沟通时因为篇幅不足而只能使用结论”、“无法熟练应用知识体系,未意识到其适用条件,仅觉得某一句话很有道理”等。具体机制可以参考2.2.3小节的讨论,此处不再赘述。

结论可以起到记忆锚点的作用,可以提醒一个人“此时该用这种知识体系来分析”。一个深入的结论能够有效地指引人思考和行动。这种作用不来自这句话本身,而来自它背后的整体分析框架。当我们面对某个现象,想到某个结论时,如果能从此出发,遵循已有的分析框架,从而验证结论确实正确,那我们就\indicate{理解}了这个现象。如果想到结论时总能具体地思考,那我们就\indicate{掌握}了这个结论。

而如果不能调用相应分析框架,那么就无法达成上述效果,此时该结论就只是大道理。无论看上去多么理所应当,多么符合自己的实际体验,\indicate{任何一句单独的话都是大道理,都不可信}(这句话也仅起到提醒的作用,不要以此为信条去否定一切规训,这超出了它的适用范围;本指南中着重强调的其它句子也仅起提醒作用)。就算是1+1=2一类的话也是如此。当它脱离数学语境,被拿去做“1+1>2”之类的比喻时,就不再显然正确了。我们必须重新完整地判断相关结论是否可信。
\end{examples}

\subsection{价值观}
我们将\indicate{比较一个事件“发生了”和“没发生”孰优孰劣}的行为称为\source{价值判断},或简称为\indicate{价值}或\indicate{评价}。
\begin{explain}
显然价值判断是判断。同时,每个价值判断都可视为一个认知。以下的行文中将不区分这二者,读者应能自行分辨。

一个价值判断可能仅关注属于某类特定过程的事件,而不关心其它的过程。一般地,一个价值判断关注的事件越多,涉及到的概念就越宏观,其判断也就越脱离现实。

价值判断是很常见的行为。有很多常用词汇包含价值判断的因素,如“好/坏”、“善/恶”、“对/错”、“应该/不应该”、“期待/抗拒发生”、“正常/不正常”、“恰好/多余”、“有无作用/意义/价值”、“是否本质”等。这些词汇都是复杂概念,以下使用“好/坏”来代指这些词汇。如果不定义“什么是好”,那么价值判断就是大道理。

有的价值判断是基于事件的某些特征,有的价值判断是基于事件的影响,出发点千奇百怪。按照是否自知出发点,价值判断可以分为两类,其中不自知的那一类价值判断均为大道理。但在自知的那一类中,价值判断的出发点可能是另一个价值判断,甚至可能出现“好几个价值判断互为出发点,循环论证”的情况。

为了确定是否自知“好”的定义,我们有必要将价值判断分为“可以归因于大道理以外的东西(比如说“是否有助于实现某个具体的目标”)”和“不可以归因于大道理以外的东西”两种。此处我们将“价值判断自身就是大道理”也视为“将价值判断归因于大道理”。
\end{explain}
有些词具有实际的含义,如“公平”“正义”“热心”“亲密”等。但如果这些词语劣化成了宏观概念,那么与之相关的判断就也会劣化为价值判断。
\begin{explain}
用大词容易让人容易忽略“这是在价值判断”,比如“没有意义”“不应该这么做”之类。但绝大多数人在下这样的判断时(无论判断自己的行为还是其它事件),都不是根据某个很严密的普遍终极答案而得出的结论。此时不能默认这种评判是恰当的,一定要确认清楚出发点。真的回头一想,大部分出发点是“某个很简单的思路,想达成某个很直接的目的”和“不清楚自己的出发点,价值判断是个污染”两种情况之一,有严密判断的情况极少。过于草率的价值观所产生的判断没有实际参考价值。
\end{explain}
我们将\indicate{参照价值判断的结果,进一步判断其它事物的好坏}的思考回路称为\source{价值观}。
\begin{explain}
多数情况下,价值观由多个价值判断与一些其它的想法组成。这些价值判断和其它想法越草率,包含越多大道理,价值观也就会越劣质。一个人可能因为各种原因被草率的价值观污染,而后该价值观就会持续生产脱离现实的结论。

比如将“达到要求”认为是“好”,“没达到要求”认为是“坏”,并根据“自己没有做到最好”认为自己是坏的;或者反过来,根据“别人没有做到最好”认为别人是坏的。具体的心态、行为和用词可能更多样,比如说“责骂/被责骂”“差劲”“无能”“没有希望”“失败”“不配活着”等。
\end{explain}
不同的价值观的“好坏”定义不同,脱离具体价值观谈论“好坏”没有实际意义。
\begin{explain}
此处的“没有实际意义”指的是“会引入歧义,从而无法得到有效结论”。更一般地,本指南中不说明前提时,所有默认价值判断全部都是以“是否会有歧义”“思路是否连贯”“推理过程和结论是否有效”“是否有助于目标”等判断为标准。

特别地,使用一个价值观的“好坏”去评判另一个价值观是否“正确”,是无意义行为。“正误”只能用于评价分析和结论,而无法用于评价出发点。
\end{explain}

\subsection{责任与要求\label{sec:责任与要求}}
我们将\indicate{一个人应该完成某个目标}的认知称为\source{责任}。我们也将这个认知表述为“这个人有完成目标的责任”。我们将“按照责任行事”称为“承担责任”,将“驱使某个人承担责任”称为\indicate{要求}。
\begin{explain}
每一个责任都可以视为一个价值判断。在不同的价值判断下,一个人会拥有不同的责任。

责任的来源多种多样,可以是“自己觉得自己应该做”,也可以是“别人觉得自己应该做”。在行为模式层次的视角下,这二者没有本质区别。
\end{explain}
如果某个责任是因为“觉得某个目标好”这一价值判断而产生的,那么就可以将这个目标视为一个\indicate{事务},将责任视为\indicate{处理方法}。我们将这个价值判断称为\indicate{该责任依照的价值判断}。
\begin{explain}
责任也是价值判断,它和其对应的价值判断不同之处在于,责任中针对“一个人”这一行为主体做判断,而其依照的价值判断对于“目标”这一和人无直接关系的现象做判断。

责任中的“一个人应该完成某个目标”和价值判断“觉得某个目标好”中的目标不一定是同一个,其中可以有“觉得目标A好→觉得实现目标B可以实现目标A→觉得应该完成目标B”,或更复杂的思考过程。

这里可以讨论两种问题:一种是“责任是否正当”,这取决于判断“目标A是否正当”,本指南不引入特定的价值判断,这不在本指南的讨论范围内;另一种是“责任是否有效”,这取决于判断“目标B是否可以实现目标A”,这是本指南着眼之处。如前所述,有效的责任在本指南中被称为\indicate{解决方法}。
\end{explain}
责任不一定有助于其依照的价值判断。必须有解决事务的能力,才能承担责任。
\begin{explain}
如果对现实的认知和对事件的分析很草率,那么分解目标的方式也就会很草率。正因如此,我们需要调研能力以清楚地认知事物的复杂性,需要分析能力以理清事物之间相互影响的方式,需要计划能力以编织有效的操作,最后需要执行能力以实现这些操作。如果没有处理复杂事务的完整能力,就无法分解并达成目标。具体论述参见\hyperref[sec:现实事务处理能力]{1.2节}。

从价值判断中直接得出责任,并且据此来提出要求,经常会遇到被要求的人没有对应能力的情况。\indicate{宏观特点只能作为某些复杂现象的结果,而不能作为目的和要求。}
\end{explain}
无视一个人的能力情况,就对一个人提出要求,是不负责任的。
\begin{explain}
这里表示本指南态度的“责任”指的是“应当完成‘完成目标’的目标”,遵循本指南的默认价值判断。

这里的“要求”涵盖很多内容,如“应当坚强”“应当成长”“应当优秀”“应当可以熟练应用某知识”“应当听得懂意思/指令”“应当明白自己的需求”等。读者可能觉得这种描述有些像是投射。严格来说有些不同,因为这里还涉及自我污染。本指南不引入投射的概念,故对此不做展开。如果读者有这方面的讨论需求,可以参考\hyperref[sec:指挥]{4.5节}。
\end{explain}

\section{人的意识演化基本模型\label{sec:人的意识演化基本模型}}
\hfill\begin{minipage}{0.55\textwidth}
\fontsize{8pt}{12pt}\selectfont\fontsize{8pt}{12pt}
\raggedright 思想越过六千层腐烂的美梦,浇筑成一台链接神经的水泵。\\
我的星球还没学会自西向东,就要被迫融进这条神的河流。\footnote{\bilibili{av9206055}\another\netease{466403600}。}

\raggedleft JUSF周存《欲》

\raggedright 傀儡之身,空有懵懂灵魂。是非不分,渴望人类的身份。\\
越是天真,越是无法放弃认真。\footnote{\bilibili{av2858071}\another\netease{34380080}。\\}

\raggedleft 鸟爷ToriSama《告别诗》
\end{minipage}

在系统介绍了复杂性的相关概念后,我们得以从更深入的视角出发,刻画人的意识现象。概括地说,第二章中的分析考虑简单现象如何编织成复杂现象,而本节的分析则考虑简单现象都能涌现出什么复杂现象。我们将本节介绍的内容称为\source{人的意识演化基本模型},或是简称\source{演化模型}。
\begin{explain}
本节的部分内容是\hyperref[sec:原生家庭]{2.5节的内容}在本章用语体系下的重新叙述。读者可以对比本小节与2.5节,以获得更全面的理解。

我考虑过要不要用“动力学”的来代替“演化”为这一模型命名,但考虑到大部分读者会对“动力学”这个术语不熟悉,所以选择了“演化”。不过读者仍然不应望文生义,需要将\indicate{人的意识演化基本模型}当成一个整体概念理解,将本节的所有内容均当做其定义。
\end{explain}

\subsection{并行与竞争\label{sec:并行与竞争}}
我们将\indicate{同时触发多个行为链/行为模式/思路/思考回路}的现象称为这些行为链/行为模式/思路/思考回路的\indicate{并行}或\indicate{并存}。
\begin{explain}
这是由于外部有多种不同的刺激,触发了多种不同的行为模式。即使刺激消失,这些行为模式也可能可以自己触发自己,因而维持下去。
\end{explain}
我们在同一时刻接收到的所有刺激,可能会满足多个行为的触发条件,而我们可能无法执行所有的行动。
\begin{explain}
刺激有可能来自外部环境,有可能来自当前处于的行为模式。

无法执行有可能是因为外界条件不允许(比如说一个东西只有一个用途),有可能是因为自身条件不允许(比如同一时间只能做一件事),有可能是因为计划和决策(比如克制某些不利行为)。

为了准确描述这样的现象,我们修改一下关于触发的定义:我们将“接收到信号”称为\indicate{触发},将“产生实际行动”称为\indicate{执行}。在不需要严格区分的场合下,本指南有时依然使用“触发”来指代整个过程,请读者自行分辨。
\end{explain}
我们依刺激最强烈的那个信号而行事。我们将这一现象称为刺激或行为的\source{竞争},并称被触发的刺激/行为\indicate{参与}了竞争,最终依照的刺激/行为\indicate{赢得}了竞争,\source{覆盖}或\indicate{打断}了其它刺激/行为。
\begin{explain}
我们也可以依此定义行为模式的竞争和相关概念。此处的“打断”将“从触发到执行”的过程视作行为链,与\hyperref[def:打断]{前文}定义相同。

\trained 可以将“参与竞争的所有行为(以及对应的刺激强度)”视为先验分布,并以此引入贝叶斯分析。本指南不过度深入这方面细节,在此仅作提及。

注意这里仅是非常模糊的描述,本指南之前只定义过\note{刺激},从来没有定义过“刺激的强度”。实际上,不深入生理过程,就无法精确描述“刺激”的具体机制。引入“刺激的强度”这一概念是为了之后叙述方便。以下将会引入另外几个假设,本指南认为信号的刺激强度仅会因这几个原因而改变。
\end{explain}
没有其它改变时,同一个信号在不同时刻带来的刺激程度大致相同。
\begin{explain}
由此,在面对相同的环境时,人的第一反应是相同的。我们将由此触发的行为称为该信号的\indicate{应对方式}。

刺激强度不以人的主观意志为转移,人的主观意志可以视为名为“意志力”的另一种刺激。如果你发现自己身上总是有改不了的坏习惯,那么说明这一习惯的刺激要强于意志力的刺激。在想办法降低习惯刺激(比如远离诱惑源)、提升意志力刺激(比如增强理性)、或引入新的刺激(比如遭遇突发事件留下深刻印象)之前,每次尝试都几乎必然失败。
\end{explain}
在短时间内,同样的信号刺激频率越高,总刺激强度越高。
\begin{explain}
这里的短时间只“从察觉到信号到做出行为”尺度的事件。依照具体行为的不同,时间尺度可能在几毫秒至几分钟内不等。

当前信号的刺激会与之前信号的刺激相叠加,从而印象更深。这一假设能够解释2.2.1节中对“\hyperref[para:记忆]{有记忆是形成行为的一个必要条件}”的论述:经历过并记住的相同事件越多,就越容易联想到,并在脑内预演可能的结果。每次想到结果都是一次信号,想得越快,信号出现的频率就越高,刺激也就越强。这产生了一个正反馈循环:刺激越强越容易触发该行为,越触发记忆越深,记忆越深刺激越强。行为在这一过程中逐渐形成。
\end{explain}
简单的东西刺激更强。
\begin{explain}
这出自两方面原因。一方面是简单的东西过脑子的速度更快;另一方面是简单的东西会在更多场合内出现。

当一个行为逐渐形成以后,除了“之前经历过的相同事件”以外,我们的脑子里还会多出另一种记忆:“见到信号就做出行为”。这种只有首尾的记忆比包含过程的回顾更简单,回顾这种记忆比回顾事件要更快,从而会提供更强的刺激\footnote{实际上不会分两步,而是渐进式形成。这里为了讲解而做了简化。}。当这一过程再次熟练以后,我们就多了一个无意识的条件反射。
\end{explain}
如果一个行为A触发了另一个行为B,并且行为A和行为B竞争,最终行为B覆盖了行为A,那么就称行为A\indicate{转化}、\indicate{转变}或\indicate{演化}为了行为B。
\begin{explain}
同理,我们也可以定义行为链和行为模式的转变。对于\rigorous,转变指的是“某一次行动时的触发和覆盖”。

转变这一概念仅针对有因果关系的A和B,不是所有的“行为B打断了行为A”都是转变。

如果行为A总是会转化为行为B,那么我们就可以认为它们已经形成了一个新的行为(行为链)。如我们上面所讨论的“复杂的行为链逐渐演变为简单的行为”,就是这样的一种演化过程。在不引起歧义的情况下,我们也会使用“行为A转化/演化为了新的行为”的表达。
\end{explain}

\subsection{缺失与外溢\label{sec:缺失与外溢}}
我们将\indicate{一种行为模式一次触发中行动的数量}称为该行为模式本次触发的\indicate{精力},将\indicate{一个人行动的数量}称为这个人的\indicate{精力}。
\begin{explain}
类似于\note{意志力}的定义,我们难以精确地将一个人的精力定义为“总量”“频率”等,仅笼统地将其称为“数量”。不同行为模式的精力特性有不同之处。有些可能以固定的频率行动;有些可能一天只能有一定数量的行动;刺激较足的时候可能精力也会多些;也可能因为有别的行为模式触发导致精力变少。请读者根据实际情况,为不同行为模式分别选择合适的理解方式。

这些行动不都是我们决策后行动的,不属于意志力部分的精力就是不可控的部分。它们会按照自身的发展机制演化,上一小节所说的“依刺激最强烈的信号做事”就是一个例子。我们也仿照之前的定义,将这一现象称为“行为/刺激在\indicate{竞争}精力”。
\end{explain}
我们将\indicate{无法使用正确的方式应对信息}的现象称为\indicate{缺失}、\indicate{缺乏},或“\indicate{缺乏}有效手段”。我们将\indicate{没有应对方式的信息}称为\indicate{盲点}。
\begin{explain}
缺失和盲点都不依赖于自知,并非“只有发现的缺失才是缺失”,也并非“只有没发现的信息才算盲点”。这里的“正确”取决于具体情况下的判断。

盲点一定缺乏有效应对手段,但缺失却不一定是因为对应信息是盲点,应对方式错误也算作缺失。

盲点普遍存在,在环境切换时会大量出现。在熟悉的环境下,很多信息会起到“提示你什么都不用做”的作用,依实际情况不同,这可能是盲点、有效手段或污染。

我们有时能观察到在盲点处发生的竞争过程。依照情感色彩的不同,我们可能使用“醍醐灌顶”“灵光乍现”“一念之差”“脑子一抽”等不同的描述来称呼这一过程。
\end{explain}
当我们同时遇到盲点和能刺激其它行为的信号时,盲点会因此与该行为建立条件反射,逐渐成为新的信号。我们将“随着多次竞争,应对某个信息的行为逐渐固定”的过程称为该行为\source{关联}了该信息,或是信息与行为之间建立/产生了\indicate{关联}。如果该行为不是正确的应对方式,那就将其称为\indicate{外溢}的行为,并将这一过程称为行为的\source{外溢}。
\begin{examples}
关联过程就是条件反射的形成。

竞争和外溢的过程可视为“行为\indicate{污染}了盲点”,此时充当污染源的是这一行为。该行为不一定是自己之前做过的,也可能是记住了之前其它人的操作,并在此模仿。

外溢可以被阻止。在污染还未形成时最为方便,仅需抢先建立正确的反应即可。污染已经形成则会比较麻烦,参见后文的论述。

\indicate{关联}是一个中性词汇,它指代所有行为的形成。而\indicate{外溢}特指其中处于应用范围之外的部分。
\end{examples}
外溢行为产生的刺激会为我们处理事务产生\indicate{干扰}\footnote{一个行为干扰一个行为模式指“它们同时触发,并且行为会覆盖行为模式中的行为”。}。如果外溢行为的干扰过强,那么即使我们当前还有别的安排,也会转而去按照外溢行为而行动。我们将\indicate{某行为产生刺激,使得我们脱离当前的行为模式}的过程称为这一行为\indicate{打断}\footnote{这里的打断和\hyperref[def:打断]{先前}的定义一致,只不过在这里我们更关注主语。}了行为模式。
\begin{explain}
外溢行为高度简单,高度熟练,在面对很多情况时,都是反应速度最快的那个。这使得它可以迅速给出简单、详细、可以立刻执行的操作。这些操作的正确性不得而知(比如说焦虑了就去刷手机),它们的唯一特点是会给人最强的刺激,吸引人去选择。
\end{explain}
竞争和关联是很随机的过程,需要视为宏观事件。
\begin{examples}
从很抽象的意义上来说,竞争过程与自然选择类似。一个信息会被什么行为污染,后续又会有什么改变,高度不可预测,需要具体调研具体分析。

随机性来自两方面:一方面是前提的任意性,即同一个刺激对不同的人来说,参与竞争的行为本身就会有不同;另一方面是过程的不可预测性,即竞争过程经常受到当前环境的明确(经常是主导性的)影响,过程经常是不可复刻的。因此,不同的人所拥有的外溢行为可能完全不同。

读者不应将其总结为“缺失是一切问题的本质”之类的观点,这只是正确的废话。我们不可能仅靠这种表面分析就彻底分析透彻这一宏观事件,不可能仅靠讨论成因就获得“一切问题的答案”。请读者将本指南中的概念与其它语境下的概念加以区分。本小节讨论缺失的唯一原因是缺失涉及因素较少,可以解耦出来独立讨论。它讨论起来较为简单,并且有些结论能在之后有所应用。

\trained 可以将“能指与所指之间关联的任意性”与此处观点对照理解。本指南不深入语言学内容,此处仅作提及。
\end{examples}
% 如果一个思考回路没有发现信号,我们的行为就触发了,那么将这个行为称为该思考回路下的\indicate{潜意识/无意识/下意识行为}或简称为\indicate{潜意识}/\indicate{无意识}。
% \begin{examples}
% 这里的定义相比术语进行了一些简化,定义仅在本指南范围内生效。外溢是很随机的过程,这也使得潜意识高度混乱。

% 本指南中使用的“自知”“有明确认识”等词语不仅用于区分行为,而是可以区分任何类型的认知。“有明确的目的,但没有注意到行为的副作用”等很多情况下,用“无意识”不能无信息损失地描述现象。实际上,所有仅关注行为本身,而不关注其影响的特征都只是表面分析。关于无意识行为的讨论不是本指南的重点,本指南会尽量避免使用这些术语。
% \end{examples}

\subsection{全局行为模式\label{sec:全局行为模式}}
我们将“某一行为/行为模式关联的的所有信息”称为这个行为的\indicate{触发范围}。
\begin{examples}
对同一个行为来说,外溢得越多,触发范围就越大。但是触发范围大不代表行为外溢得严重,属于触发范围但不属于应用范围的部分才是外溢的部分。一个行为/行为模式外溢得越多,它劣化也就越严重。
\end{examples}
行为的外溢会使得它在更多方面参与竞争,并且更容易赢得竞争,这会使其更容易进一步外溢。
\begin{examples}
盲点可能是具体事物,也可能是很抽象的内容,比如说“恐慌”“焦虑”“不知所措”。这种机制会导致“越焦虑就越刷手机”等现象,因为“刷手机”这一行为已经严重外溢,导致它在很多情况下都是最常被触发的行为,这进一步导致了“刷手机”继续外溢。掌握的信息越少越偏,面对事件时就越不容易拥有恰当的反应方式,盲点就越多,于是就越有可能产生污染。

每一次行为外溢都可以视为形成了一条\indicate{价值判断}。这样形成的价值判断大多是不自知的,人会先感觉到“我想这么做”,然后再将其用语言总结成“应该这么做”。可以为自己的行为找很多很充分的理由,但它的最初形成是污染(找理由后,有可能确实因为理由而培养出新的行为,这应当算作另一条行为)。
\end{examples}
同理,认知越具有普遍性,就越难正确地总结出来,但也越容易外溢。
\begin{examples}
“因为指代而使词语概念发生改变”的现象也算是“概念被词语污染”,这种机制是我们会将具体事件劣化为宏观概念的主要原因;“根据自身见识给复杂现象的结果归因”会经常归到错误的原因上,使其和结果错误地关联,得到“这一定就是因为某个原因”的外溢认知。
\end{examples}
行为的外溢持续下去,会导致某些行为几乎在任何时候都会触发。我们称这样的行为称为\indicate{全局行为}/\indicate{默认行为}。
\begin{explain}
全局行为只关注触发范围,而不关注触发范围与适用范围的差别,不关注行为是否外溢。全局行为也会参与竞争,如果同时有更强的刺激,它也会被覆盖。全局行为不包括呼吸、心跳等生理活动(因为它们不受其它事情触发),但包括“时时刻刻都小心翼翼屏息”、“因为病痛折磨而无法集中精神”之类的情况。全局行为较为少见,但讨论起来较为简单,可以用于展示一些特点。这些特点对稍后介绍的全局行为模式也适用。

有一些比较简单的现象可以视为全局行为。例如,当“想一想其他事情和我有什么关系”成为全局行为时,就会觉得万事万物都和自己有关,于是就会产生名为“全能自恋”的现象;当“发现事情是别人干的”成为全局行为时,就会把一些其它的类意识行为(比如推销电话或app)视为有人在操控,进而可能发展成被害妄想等。本指南不过度使用精神分析语言,故不做过多探讨。

瘾也是一类比较常见的全局行为,如烟瘾、酒瘾、性瘾、毒瘾等。需要注意的是,并非所有通常意义下的瘾都是全局行为,但它们被称为瘾,至少属于外溢行为。瘾很难戒除,是因为其有过多的触发条件,这也就是通常所说的“身瘾好戒,心瘾难戒”。

当自身存在全局行为时,进入新环境后,全局行为很容易污染新环境中的信息。手里有把锤子,看什么都是钉子。自己会真心实意地相信自己认知的正确,全心全意地践行全局行为。如果一个人的全局行为中不包含自我审视和反思,那么就丝毫不会怀疑其它全局行为是否恰当。

因为行为的外溢高度不可预测,全局行为因人而异。不同的全局行为之间很可能互斥,拆解一个全局行为培养一个新的,难度很高。本指南的目标是让读者将“唤醒理性”培养为自己的全局行为,以做到时刻清醒地认识世界,有效地行事。这难度很高。
\end{explain}
和行为类似,行为模式也有“刺激越强越容易触发”的机制,这一部分以及有关的定义不再赘述。如果一个行为模式在几乎任何环境下都会触发,就将其称为\indicate{全局行为模式}/\indicate{默认行为模式}。
\begin{explain}
也可仿照外溢行为,定义“外溢行为模式”。本节中的讨论同时适用于普通的外溢行为模式与全局行为模式,仅讨论全局行为模式是为了方便。读者可以通过加入“行为模式会触发的环境”的条件,来使以下讨论同时适用于普通的外溢行为模式。如果同时还有更强的刺激,全局行为模式可能被其它行为模式覆盖。

行为外溢得越严重,就越容易被触发。外溢的行为越多,这些行为之间就越容易相互触发,从而越容易形成一种严重外溢的行为模式。全局行为模式可以通过这种机制,直接由一些外溢的行为产生。之后为了叙述方便,我们将全局行为模式的组成部分也称为全局行为,读者应能自行分辨。

由于外溢的行为大都是高熟练度的无意识行为,全局行为模式也因此容易成为无意识的行为模式。
\end{explain}
类似地,如果一个思考回路在几乎任何环境下都会触发,那就将其称为\indicate{全局思考回路}/\indicate{默认思考回路}。
\begin{explain}
全局思考回路都是全局行为模式。也可类比定义“外溢思考回路”。

全局思考回路几乎每个人都有,一个明显的例子是“语言”。注意,语言的触发范围不都属于语言的外溢,一些概念、分析、认知是正常行为(虽然在一些情况下很难分辨到底是否是外溢)。本指南不过度深入哲学和精神分析,不对这方面展开讨论。

全局思考回路中的价值判断组成了一个价值观,该价值观会作为刺激,触发全局行为模式;同时,它会展示全局行为模式每一部分的正确性。
\end{explain}
我们不应将全局思考回路拟人化。
\begin{explain}
“将行为模式拟人化”是一种常见的外溢认知。在我们分析全局思考回路时,它会持续施加干扰。必须时刻注意。

具体来说,我们不应默认全局思考回路具有“保守”“懒惰”“抗拒改变”“主动”“固执”“控制欲”“指导欲”“想吸引注意力”等特点,不应该为全局行为模式赋予“本心”“本色”“底色”“灵魂深处”“骨子里”“精神内核”等感情色彩。这些词语都是行为模式层次,或人的层次的宏观概念,使用它们会带来不必要的污染。我们需要且只应该站在行为的层次来看待全局行为模式,将其视为一种需要研究的客观现象对待。

只有站在行为的层次,充分调研并分析我们所拥有的所有外溢行为和它们之间相互触发的方式,我们才有能力施加影响,从而主动改变全局行为模式。如果将其视为一个整体,就无法执行这么精细的操作。劣质的全局行为模式从来不是一个整体,没有什么确定的人格和倾向,没有什么统一且坚定的看法和信念。它只是一堆容易互相触发的外溢行为,在某一件具体的事上看起来可能很统一,但整体是混乱无序的。
\end{explain}
全局行为模式越是劣质,就越会以高频率发生不可预测且不可控的改变。
\begin{explain}
行为会依照行为的演化机制发展,不断因为新出现的刺激和记忆而更新。行为的外溢不可控,于是由外溢行为组成的全局行为模式也不可控。

这也导致两个不同人的全局思考回路之间差距非常大,可能完全无法交流。这种现象也可能发生在现在的自己和以前的自己之间。全局思考回路越是劣质,就越难以理解、控制、改变别的行为模式(包括自身)。这会给人以“人不可能相互理解”“做人只能靠自己”之类的错误结论。在大多数环境下这种看法甚至没什么问题,但它们本身不正确。详细讨论见第五章。
\end{explain}

% \subsection{清除外溢\label{sec:清除外溢}}
% 我们将\indicate{每次遇到一个信息时,都按照固定的方式操作,直到形成习惯}的过程称为\indicate{培养}。面对未被污染的信息时,我们可以较为轻易地培养出正确的应对方式。
% \begin{explain}
% 仅需遵循行为的产生机制,坚持重复使用正确的方法,直至产生关联即可。

% \indicate{这一切的前提是我们有意培养}。我们需要有获取正确应对方法的途径(无论是靠自己或外部)。如果没有主动培养,行为就会遵循上述机制,逐渐将反应固定为刺激最强的那个。
% \end{explain}
% 我们难以处理已经被外溢行为污染的信息。草率的处理手段\indicate{注定失败},只会使情况更糟。
% \begin{explain}
% 这里的\indicate{注定失败}带有一定的修辞效果,不意味着“一定不能成功”,但其极低的成功率使得成功不可复刻。这不是科学可重复的处理方式。本指南其它地方出现的“注定失败”“没有成功的希望”“只能看运气”等表述也是这种意思。

% 打断的机制使得难以修改并清除污染。如果没有充足的准备,无法拿出另一套简单、详细、可以立即执行的操作来代替原有的外溢行为,那么就很容易在坚持了几天以后半途而废,反而让外溢行为又污染了想要改变的尝试,使得下一次更容易半途而废。

% 这是处理“未被污染”和“已被污染”的两种信息时的本质不同之处。必须有区分这二者的能力,才能分清是偶尔还是习惯,才能确定改正或处理的方式。
% \end{explain}
% 如果有持续且强烈的刺激,全局行为模式会因此改变。但如果仅有刺激,改变的方向不可控。
% \begin{explain}
% 当刺激长期存在时,全局行为模式会不断被触发。在经历了充分长的时间后,要不然全局行为模式污染了刺激(此时归入先前的情况),要不然我们会获得“全局行为模式处理不了这一刺激”的新信息(总体来说,越善于总结经验就能越早发现这一点,越不善于总结经验就会越慢,只能指望最基础的行为养成机制)。

% 这一信息会在每一次触发全局行为模式时都出现以给予我们刺激,打断全局行为模式,使我们放弃接下来的行动(此时不同的人会感受到不同的情绪,包括但不限于急躁、失落、自责等。这不是讨论的重点所以略去。同时,如果这些情绪属于全局行为模式的一部分,或是因此而成为了全局行为模式的一部分,那么应该被视为“刺激被污染”的情况),转而尝试新的行为。

% 此时的情况类似于面对未被污染的信息时的情况,但仍有主要不同:全局行为模式仍然在持续触发,它会频繁打断我们的思考,从而使得我们难以维持培养的过程。这使得主动修改自身的全局行为模式和全局思考回路极为困难。

% 如果我们理性的速度和系统性不够,无法追上全局行为模式,只能提供刺激却不能提供完整的路线,就无法有效地改变,只能从一个污染跳到另一个污染。具体讨论见第六章。
% \end{explain}
% 我们应当将理性培养为自身的全局行为模式。
% \begin{explain}
% 理性的适用范围是所有行为模式,其不存在外溢现象,我们只需要担心“无法保持理性”的问题(但是每一种具体的处理手法都有可能超出其适用范围,产生外溢,请读者注意区分)。

% 最优状态应该是“使理性成为自身唯一的全局行为模式”,但那样难度过高,理性能够监督自身的意识活动就足够了。甚至在实践中,这一要求还是有点高,我们只需要在处理事务时能够唤醒理性,就已经足够了。

% 需要说明的是,理性不和情感互斥,不会出现“因为理性所以变成冷血机器人”的情况,过于冷漠正说明不够理性,身上有不可控的“压抑情感”的全局行为模式。非要找个对应的性格形容词的话,应该使用“豁达”“超脱”等。

% 掌握一种能力(知识体系),等价于在该能力的所有应用场合,都不被外溢行为模式打断(不受干扰更好,但更难以实现)。只有当我们不被劣质的全局思考回路干扰,思路不被打断,我们才能得到整体观察事物的机会,进而才能培养处理复杂现象的能力。

% 我们需要使用理性来修正自身的行为外溢现象,并且打断行为继续外溢。我们需要在每个场合都全面地接收所有信息,之后选用正确的看待方法。如果不知道怎么才算正确,那么就去研究或学习,在找到方法之前,控制住自身任何下意识的反应,不作任何草率的行动。
% \end{explain}

\section{认知外溢与因果性破坏}
\subsection{想法的竞争与外溢\label{sec:想法的竞争与外溢}}
\noindent\note{想法}远比其它类型的行为更容易触发。
\begin{explain}
触发难度上,想法可能只需要上一个想法就可以触发,而其它行为需要外部刺激,外部刺激通常没那么多;持续时间上,一个念头可能只会持续0.1秒,而大部分完整的行动需要数秒或更长;复杂程度上,我们的记忆可以使多个彼此差异很大的想法同时存在并且快速切换,而外界信息变化幅度通常较为缓慢,且关联性通常更高。
\end{explain}
想法会提供远比其它类型的行为更多的刺激。
\begin{explain}
大多数信号仅能触发一次行为,但其只要进入短期记忆,就会触发多次想法。即使是持续存在,能多次触发行为的信号,想法的触发频率也要远高于其它类型的行为。

这使得认知的形成可以非常快速,可能仅需一次接触就足以形成稳定的认知。在一些语境下,这被称为“第一印象”。同时,认知会刺激其它有关的认知,从而形成思路;如果能够循环触发,则会极大地加强这些认知之间的关联,有可能一次循环触发便能形成一个新的思考回路。之后每次触发这个思考回路,都会强化其中的关联。对于知识体系来说这叫融会贯通,对于脱离应用范围的认知来说就是加重污染。
\end{explain}
我们将“随着多次竞争,逐渐接收到信息就产生某个想法/联想到某个认知/触发某个思考回路”的过程称为该想法/认知/思考回路\indicate{赢得}了\indicate{竞争},\indicate{关联}了信息,\indicate{覆盖}了其它想法/认知/思考回路。如果该认知不正确/想法和思考回路与信息无关,那就将其称为\indicate{外溢}的想法/认知/思考回路,并将这一过程称为想法/认知/思考回路的\indicate{外溢}\label{def:外溢认知}。
\begin{examples}
盲点(和其它信号一起)出现时,我们会将受到刺激最强的认知套在盲点上。这一机制会在盲点和认知之间建立虚假的相关性。像不意味着真的有关系,除了“会先后出现”以外,它们不保证有任何共通之处。认知污染了盲点以后,就会妨碍对盲点本身信息的正确收集、理解、应用。
\end{examples}
我们思考时,会受到相关外溢思考回路的显著干扰。思考链越长,外溢思考回路的干扰就越大。如果某个认知会被其它想法覆盖,那么我们就称在此认知上存在\indicate{障碍}。
\begin{explain}
外溢思考回路会提供大量熟练且缺乏系统性和逻辑的想法。如果思考链被这些内容结论打断了,此次分析就会无效。思考链越长就越容易被打断。换句话说,只有在外溢思考回路不产生明显干扰的情况下,我们才能进行长时间、大规模的思考。

并且,我们很难去验证外溢思考回路给出的答案。外溢思考回路会非常熟练地证明自身每一个结论的正确性。如果没有其它得出结论的手段,便会又回归外溢思考回路的判断。除了少数情况有“无法辩驳的事实直接摆在眼前,让人印象深刻无法忘记”以外,其它刺激都会在某一次注意力被转移以后,再也想不起来。
\end{explain}
如果关于某个事件的认知/行为/...不能对事件的实际结果产生影响,那么就将这一判断称为该事件的\source{无效认知}/\indicate{行为}/...。我们也将无效认知称为\indicate{无关认知}或是\indicate{废话}。
\begin{explain}
读者可以自行验证,第一章中的\note{无效操作}与此处的无效含义一致。本指南中还会用到“无效努力”“无效精力”等概念,请读者自行套用。

如果认知是正确的,但也不产生影响,那么这就是\indicate{正确的废话}。

认知本身当然不能直接影响事件。这里指的是其指引的行为对结果产生的影响,例如“得到了有效的方法”就“按照方法达成目标”、“判断出一定没希望”就“放弃目标更改计划”等。而像是“感觉自己受摆布”“觉得自己运气很好”这种认知,如果没有进一步的行为,那么就不可能产生影响。我们无法仅从认知中判断它是否是废话,必须结合现实情况。

这里的“产生实际影响”的比较基准是“自身没有参与”。如“这事情气到我了,骂两句舒服多了”的“舒服多了”不算做影响了原事件,但是可以算影响了“气到我了”这一事件。由于我们总是将分析视为“自身在模拟另一个行为模式”,我们总可以将事件视为外部的,不会出现“自己没法不参与心理事件”的问题。

劣化或外溢的认知基本上都是废话。脱离应用范围的大道理是废话。大多数\note{预料}是废话。这不只是因为“预料会劣化和外溢”,还有另一种原因:我们不一定具有“参照自身的预料,做出有效的举动,影响事件的实际结果”的能力。
\end{explain}

\subsection{认知的内容}
我们将一个现象本身称为它自己的\indicate{内容}。我们将\indicate{获得某个认知时,相关的所有事物的内容中,符合这一认知适用范围的部分}\footnote{对于\rigorous,这里指的是“所有内容取并集”后再和适用范围取交集。}称为这一认知的\source{内容},将认知称为\indicate{关于}这一内容的认知。
\begin{explain}
由于“旧认知生成新认知”是一个递归的过程,我们这里也遵循同样的递归流程,来定义认知的内容。

如果一个认知的内容仅有唯一的具体事件/现象,那么这一具体事件/现象一定是该认知的\note{参考事件/现象}。如果的参考事件/现象在该认知的适用范围内,那么它就上该认知的内容。

一个认知(或者其它定义了内容的事物,如后文提到的\note{资源}和\note{转述})也可以视为现象,这导致它具有两种不同的内容:一个是它本身(作为现象),另一个是它的内容。比如“有人和我说楼下出了车祸”就同时有“有人和我说话”和“楼下出了车祸”两种内容。这两种内容需要分别处理,如果不做提及,本指南中我们默认不将它本身视为它的内容。

内容和拥有该认知的思考回路相关,内容有很多不同的类型。某个具体物体、某个实际过程、某种概念、某种特点,都有可能是某个认知的内容。如果认识得不全面,认知的内容会小于它的适用范围;对于能明确知道适用范围的认知来说,它的内容就是适用范围。

一个认知的内容有可能是另一个认知,比如“xx认知的适用范围”这类认知的内容就是“xx认知”(当判断这一认知是否适用于目标情况时)或“适用范围”(当判断目标情况都能使用哪些认知时)。这使得认知可以通过内容组成复杂的递归嵌套。区分认知本身和认知的内容有助于我们讨论这类嵌套。

如果在后续的发展中,一个认知产生了变化(如“突然意识到它的范围可以更大”),那么我们将其视为另一个新的认知,并且对其重新定义它的内容。更一般地,如果一个认知的内容要多于它的参考现象,我们就将其称为认知的\indicate{泛化}或者\indicate{抽象化}。
\end{explain}
和\note{理解途径}处的处理方法相同,我们在确定内容的时候,会忽视掉一些不重要的差别。
\begin{explain}
差别可以是大家都知道的省略,比如说一些情况下的“某种特点”和“拥有这种特点的所有事物”;“一种事物”和“这个事物可能直接参与的所有过程”等。也有可能是某些指代上的差别,比如“用词不同”,这包括“错误理解了一个概念,然后用错误的理解去描述现象,但如果将描述中的概念替换为对应的理解,那么描述的现象没有问题”一类的现象。展开的讨论见\hyperref[sec:解读]{4.4节}。

但是如果内容脱离了适用范围(如“总是用某个特定的参考事件去理解一类事物”),则差别不可忽视。不同人拥有的认知可能表面上看起来一致,仅从用词和描述中看不出差别。但如果内容有不可忽视的区别,仍然不应将二者视为同一认知。

这也是定义中要加入“符合应用范围”这一限制的原因。如果去掉这一限制,剩下的定义将会类似于“能联想到这一认知的其它内容”。我们仍然可以据此分析“这一认知指代的现象是什么”,但这与本小节的讨论无关。
\end{explain}
因污染而形成的认知有可能没有任何内容。
\begin{explain}
污染可能来自外部,是对某些行为的单纯模仿,如“学说话”、“过家家”、“背诵名言警句”等;也可能来自内部,是自身认知外溢的结果,如“因为熟练而省略中间步骤”等。具体机制在前文已有展开说明,此处不再重复。
\end{explain}

\subsection{内容顺序与触发顺序}
如果认知A会成为触发认知B的信号,那么我们就将其称为认知A和认知B之间的一个\indicate{触发顺序},或是简称为“A\indicate{可以触发}B”。
\begin{explain}
触发顺序具有方向性。两个认知之间的触发顺序一共有4种:相互不可以触发、只有A可以触发B、只有B可以触发A、A和B可以相互触发。

“A可以触发B”可以作为一个独立的认知,但也可以是不自知的。这种认知也可能是外溢(比如在“这种触发顺序实际不存在”时产生)。

A触发B后,B有可能参与到和其它行为/认知的竞争中,最终可能无法执行。我们在定义中用“可以触发”来强调这一区别。

如前所述,当A与B同时被触发时,它们之间有可能产生\note{关联},从而相互触发。这样形成的触发顺序是随机的,四种情况(如果只算存在触发顺序的情况,那就是三种)都有可能出现。除此之外,A还可以通过其它认知/行为,和B产生关联。如果这样的行为链有顺序,那么就会形成“只有A可以触发B”的情况。
\end{explain}
如果认知A的内容会导致认知B的内容,那么我们就将其称为认知A和认知B的一个\indicate{内容顺序},或是简称为“A\indicate{可以导致}B”。
\begin{explain}
虽然这里采用了“A可以导致B”的措辞,但实际上A和B都是认知,它们对应的实际内容不以主观意志为转移。此处的措辞仅起到简化表达的作用。

“A可以导致B”可以作为一个独立的认知,但也可以是不自知的。这种认知也可能因外溢,在这种内容顺序实际不存在时产生。

和触发顺序一样,内容顺序也有方向性。两个认知之间的内容顺序一共有4种:相互不可以导致,只有A可以导致B,只有B可以导致A,A和B可以相互导致。

但触发顺序和内容顺序也有明显不同,会相互导致的内容远少于会相互触发的认知。中思考回路内,认知的关联会新增触发顺序,但不会新增内容顺序。
\end{explain}
在两个认知之间,如果\indicate{一个触发顺序不对应实际的内容顺序},那么我们就将这种现象称为触发顺序对内容顺序的\source{因果性破坏}。
\begin{explain}
触发顺序是认知之间的因果性,内容顺序是内容之间的因果性,所以我们如此命名这一现象。

因果性破坏在一定情况下是好事。当我们需要决策的时候,经常会有“因为要完成某个目标,所以要达成某个前提条件/避开某种不利影响”之类的思路。如果我们只会顺着内容的顺序思考,而没有逆向思维,就无法做到(或是无法思路清晰地做到)很多事情。因果性破坏在一定情况下是坏,比如认知的外溢总是会导致因果性破坏。

两认知内容的相关性会导致认知触发的因果性,但认知触发顺序的形成不一定与其内容相关。如果将认知触发的因果性/相关性错误地当成了认知内容的因果性/相关性,就有可能得出“是结果导致了原因”“两个(实际上只是相关的)现象有因果关系”之类的结论。这是由于因果性破坏而得到的劣化认知\footnote{对于\rigorous,这里的具体细节如下:我们产生了“A和B内容具有因果性/相关性”的认知C,它可能是“A和B触发具有因果性/相关性”的认知D的外溢。但更有可能的是,我们并未得出认知D,而是直接从这一现象中错误总结出了C。}。
\end{explain} 

\section{总结与讨论}
\subsection{本章总结}
本章内容围绕\indicate{复杂系统}及其看待、处理方式展开。相比第二章,本章的讨论内容较为集中,除了复杂性的概念介绍,就是使用系统性的视角,来分析各个具有复杂性的事物。正确运用系统性的视角分析复杂问题,是我们处理现实事务的必备\indicate{能力}。

正式介绍复杂系统之前,我们需要先做一些准备,对我们认知世界的方式建模。这一部分虽然也属于人的意识活动,但与第二章的内容关系不大,与第三章联系更紧密,故将其放在第三章。我们引入了\indicate{概念}和\indicate{特点}作为基础概念,并且引入了\indicate{识别}和\indicate{指代}两种基础操作。据此,我们得以引入\indicate{出发点}和\indicate{能力范围},以界定在思考时是否产生\indicate{歧义},以使得思路\indicate{劣化}。

劣化是思考的主要危害,而对复杂性的认识不足,思考得过于概括粗疏,导致\indicate{信息损失},则是产生劣化现象的主要原因。在充分讨论劣化的相关内容后,我们得以在引入复杂系统的概念后,立即讨论其处理方法。

我们介绍了复杂系统中会出现的两种现象:更偏向从某一具体的\indicate{高层次事件}的视角看的\indicate{编织},与更偏向从全体\indicate{低层次事件}视角看的\indicate{涌现}。同时,我们还展开讨论了会导致思考劣化的\indicate{宏观分析}现象,仔细分析复杂系统为何会使得浮于表面的思考脱离实际。

讨论完复杂系统以及宏观分析的一般理论,就可以将这些理论应用于实际事物上。我们首先深入考察了事务处理方面的内容。如果对事务的认识不足,无法将它的本质困难纳入考量,只是因为觉得“应该这么做”就将其定位了目标,就会产生很多没有能力承担的\indicate{责任}。如果任由这种草率的\indicate{价值判断}组成\indicate{价值观},就会产生很多无意义的内耗。此处我们还声明了本指南所持有的价值观:解决现实事务。

最后,我们使用复杂性的视角,重新梳理了一遍人的基本模型。我们考察了\indicate{刺激强度}对\indicate{行为形成}的影响这一复杂系统,通过分析\indicate{缺失}的现象如何产生行为的\indicate{外溢},部分解释了不可控行为的产生原理,同时回顾了处理方法。我们提出了与\indicate{全局行为模式}的概念,并分析了它的复杂形成过程,从而确认了将其清除的困难程度。需要注意的是,此处的模型仍然是高度简化的版本,仅供本指南展开后续理论使用,有很多细节缺失。读者需要注意其应用范围。

作为基本模型的一个应用,我们分析了\indicate{认知的外溢},主要考察了其中的\indicate{因果性破坏}现象,展示了外溢机制如何在现象之间产生虚假的因果性,并因此使认知劣化,无法包含任何\indicate{内容}。

除此之外,本章中提及的一些概念,在之后的一些篇幅中也会频繁使用,一定程度上方便了行文与理解。

\subsection{澄清与叠甲}
大家在阅读本章的过程中,或许会产生一些联想。可能有些读者会觉得本章的内容和一些其它理论很像,也可能有些读者觉得本章的内容可以应用到一些其它领域。读者可能会疑惑,为何本章没有对这些内容展开更深入的讨论。这是本指南有意的安排,目的是尽量少地设计无关领域,将关注点聚焦在本指南的主题上,尽量避免因发散过多,讨论不充分,而产生污染。

\smalltopic{(1)关于应用范围}

读者在阅读本章内容时,尤其是介绍复杂系统时,会很容易将其套到社会现象上。我们似乎可以很轻易地就得出一大波有用的分析和洞见:社会中的组织由人编织而成,但起作用的不一定是人,组织可能只需要人的行为模式,这样就产生了异化;污染在人与人之间传播,于是形成了模因;不同人的行为可以在舆论场中相互激发,从而形成长久存在的行为链,形成回声室;这种纯粹的涌现现象,从表现上与行为模式无异,很容易被错认为是有人在背后布局;但实际上并不是这样,世界是个草台班子,大家都不知道它会变成什么样;大部分人根本就没有考虑长期问题的能力,都是乌合之众,都是巨婴;而使得大部分人的能力差成这样的社会更是万恶之源,是恶之花绽放的土壤......

以上所有这些,都是不负责任的暴论。本指南介绍的理论,不足以容纳全部的论述过程,也无法判断这些内容的正误和适用范围。任何从本指南出发的推导,思路中必然包含大量不客观内容。

“期望给事情一个简单的归因”是一个严重外溢的行为,必须警惕它的影响。社会是一个复杂系统,它的复杂程度远超人的意识现象,必须谨慎对待。读者可以衡量一下本指南介绍“理性的分析框架”所用的篇幅,以及将该分析框架熟练运用在自己身上的难度。这有助于破除“似乎掌握了社会运行的本质”的错误认知。在充分掌握理性之前,几乎不可能做到谨慎地采用科学方法从微观视角客观研究社会运行的具体规律,任何使用宏观概念的概括都会劣化成大话空话,任何自己得出的结论都会不可避免地带有偏见。

本指南所介绍的理论,仅用于分析每个具体的人的意识现象,不涉及更复杂的内容。这已经够难了。

\smalltopic{(2)关于其它理论}

阅读量比较多的读者可能会觉得本指南的论述在某些地方见过。我猜测这样的既视感应该集中于哲学和心理学领域,还有少量可能属于文学、社会学等领域。这种现象很正常,本指南没有介绍什么新东西。但如果将本指南中的内容按其它理论理解,可能会有害。

我们在评价理论的有效性时,经常会用到“现实情况符合理论预测”与“理论可以推出现实情况”两种表达。这样的表达同样可以用在理论上,但语感会有所区别。“理论A符合理论B”和“理论A推出理论B”是一个意思,它们都指“理论A的特点满足理论B的基本假设”,但语感上会借用之前表达中的“(普遍的)理论比(一个)现实情况更本质”的感觉,认为前者中理论B更本质,后者中理论A更本质。

给理论之间排“本质性”非常危险,本质是个宏观概念,同时还是个价值判断,不存在一个稳定,能经受住所有质疑的“本质”定义。理解一套理论仅能从其自身出发(如果自身没能严密到建立完整框架,那再另做处理,可以根据具体情况选择忽视或者是要求作者重新修补,不在当前话题讨论范围之中),不应使用别的理论的概念来代替对其本身的理解。这种行为会带来污染。

本指南仅介绍了一套适用范围较广的概念体系,它可以和其它领域的观点和论述相容,但不意味着本指南要更“本质”。读者可能看到概念部分的论述,感觉像是在讲结构主义或现象学;看到关于复杂系统的实事求是处理方法,感觉像是在讲辩证唯物论;看到价值判断的讨论,感觉像是在讲虚无主义;看到人的基本模型,感觉像是在讲精神分析与自由意志;或者看到一些别的地方,又有其它的既视感。如果要问“这么看对不对”的问题,答案是“这些东西确实算是这些领域的话题,相应的书籍中都有详细论述”;但如果要问“这么看有什么用”的问题,那大概没什么用。无法从这样的判断中,找出正确的废话以外的东西。你无法从中获得确切的知识,只能得到“我好像又学到和发现了东西”的幻觉。

% 本章所介绍的模型,乃至本指南所介绍的理论,是关于“因关联性而演化的复杂系统”的理论。它和具体生理现象无关,仅讨论意识现象本身\footnote{对于\rigorous,这里指的是“我们仅需要验证‘生理现象能涌现出关联性’即可”}。

\vspace{10pt}

对于那些真正深入理解这些话题的读者来说,本指南中所提到的这一点篇幅,又会显得过于简化,精华尽失。这是为了可读性而做的妥协。本指南是一本应用性的书籍,需要在保证完整介绍分析框架的前提下,尽量减少提及不必要的内容。完整阅读这一分析框架已经很困难,那些过于细节的论述会进一步增大读者的理解压力。我们选择将这些领域的观点一并作为本指南的出发点介绍给读者。

如果读者对其他领域有所研究,本指南欢迎读者在此基础上理解并评价。但请读者尽量避免概念的跨领域应用,尽量避免使用自身无法掌握的概念和结论。这会造成本可避免的麻烦。

\section{实操:情绪、感受与其分析与处理方法\label{sec:情绪、感受与其分析与处理方法}}
\hfill\begin{minipage}{0.7\textwidth}
\fontsize{8pt}{12pt}\selectfont\fontsize{8pt}{12pt}

\raggedright 把我的心都切碎了标上价码,卖掉它就能得到幸福吗?\footnote{原文为“心ってモノを量って切って、売れば幸せですか?”。\\\indent \bilibili{BV1r4411A77J}。}

\raggedleft 田中姬铃木雏《ヒバリ》(姬雏鸟)

\raggedright 大人说笑一笑解千愁,可是为什么,为什么,心中有火?
是不是我还太小,境界还不够?\footnote{\bilibili{av6107357}\another\netease{1377097873}。}

\raggedleft 绛舞乱丸《寿星街小结巴》

\raggedright 当我恻隐于哀恸,就听到心魔在教唆。\footnote{\bilibili{BV1yD4y117jU}。\\}

\raggedleft lemon夹子《行刑者赞歌与诡丽的戴罪人》

\end{minipage}

感情是一种我们极易感知到,但是又极难讨论和理解的意识现象。常见的几种定义都会有很多问题:
% \begin{explain}
\begin{itemize}
\item 我们可以从生理现象出发定义感情,比如说“哭泣了就是悲伤”“分泌肾上腺素就是恐惧/兴奋”。这样做最明显的问题就是“这么定义没多大作用”。生理现象一方面不直接能够代替心理现象(它很多时候作为表面特点而出现),一方面指标较为模糊,日常能够关注并分析的信号较少,难以刻画细致、微妙的感情。
\item 我们可以从“模因”出发来定义感情,这种处理方式的合理性主要来自“感情总是作为词语被我们认知,然后我们会把自己往上套,觉得自己会悲伤就真的会悲伤,觉得自己会愤怒就真的会愤怒,形成了自我实现的预言”。这样做的主要问题在于无法涵盖本人无法识别的情感(或者至少是周围人都没有识别出来的情感)。
\item 我们可以从“某种特定的心理机制”出发定义感情,比如说“恐惧和愤怒都是因为自身受到了外界的侵犯”“一切问题都能从潜意识中找到答案”之类。这样做虽然比从生理出发要更好观测,也可以直接用于分析一些简单的情况,但仍然不够精细。我们很难用这种方式定义每一种情感,通常只能定义几种主要的,然后声称“自己已经讨论了所有本质的情感,其它的都是变体”。但感情现象的关键正在于“感情怎么变”上。
\item 我们可以从“某种类型的具体经历”出发来定义感情,比如说带有正面情感色彩的“人生阅历”“境界”,或是带有负面感情色彩的“创伤”“原生家庭”。我们确实可以任意详细任意真实地某一个特定的人身上特定的一种情感,但这样做最大的问题在于我们讨论得不全面。在我们明白了某一个情感的生成原理后,就容易把故事套到其它人和其它情感身上,从而产生误导。
\end{itemize}
这些定义每个都有可取之处,每个都能解释一部分问题,甚至每个都真的能被人用来分析所有的感情现象。但这样得出的结论主要应该归功于个人的洞察力,仅靠定义本身出发就只能是盲人摸象。

我们能从中确定的是,感情是一类歧义很重的概念。每一种感情都能用于指代很多种不同的东西,而我们又能比较顺畅地在每个具体情境中理解具体的感情指什么。这看起来就像是“我们在根据关联性来理解和应用感情”。于是,我们不妨将关联和外溢的理论套用到感情上,看看可以分析出什么来。

\divider

\noindent 我们用以下的一个事例\footnote{这一融合了校园霸凌、家暴、ptsd的事例不代表现实情况。我们提高了冲突的激烈度,以便可以在更短的篇幅内包含所有要点。}来展示这一思路:
\begin{explain}
朱自清被选入语文课本的《背影》中有“父亲为作者翻过火车站台买橘子”的桥段。A自从初中学到这篇课文后,相关的骚扰\footnote{此处重点在于整体的流程,关于霸凌的具体分析见后文。我们这里采用的假设是“本来就有人在霸凌A,并且这些人学到《背影》以后就开始用橘子了”。}就充满了整个中学生涯。这样的事件深刻地参与了A的人格构成,使得A自此以后对橘子应激。

多年以后,A成家并生下了B。在B的朋友看来,B有个小毛病:不吃橘子。B自己也不知道为什么,但总之自己很讨厌橘子的外形和气味,碰到就赶紧离得远远的。B学到《背影》的时候,始终无法共情朱自清。
\end{explain}
\trained 可以一眼看出,这是一个“创伤在代际之间的传播”的小例子。B对橘子莫名其妙的讨厌,绝对是因为A在B小的时候的做法(无论是自己对橘子应激还是限制B都能产生同样的效果)导致的。接下来我们就需要做两方面,一共四件事情:1、此时我们应该怎么得知整体的真相?2、B应该如何解决问题?3、我们应该怎么理解这类现象?4、我们应该怎么处理这类现象?

一些方法只做表面功夫,比如说“直接将B的情况命名为‘柑橘恐惧症’,并且开始脱敏疗法”。这么做只关注问题2,无视了其它三件。而且,这么做对问题2也不一定有效。我们暂且把它们放在一边。

正经的心理治疗流派此时都会先试着开始找原因。考虑到这只是一个简单的情景,所有流派应该都能在短时间内让B发现“自己的讨厌来自童年创伤”。对于一些流派来说,只要得到了这个信息,就可以直接让B打开心结,尝试自由地吃橘子(当然,如果操作不当,可能会出现“B从此更加怨恨原生家庭(同时也仍然讨厌橘子)”等失败情况,相关讨论见下文)。到此为止,我们通过问题1的部分答案,就解决了问题2。

如果B仍然没法完全想通的话,我们还能在这个基础上做得更好一些。如果一切顺利,我们可以见到“B回家和A谈心,A回忆起了往事,(可能在一大堆跑偏之后)聊到了自己被霸凌的经历,从而家庭关系得到了修复、滋润和建设”的情节。这种成功经验自然是值得借鉴的,于是我们开始自然地尝试理解“为什么会成功”\footnote{此处可以视为对这个新的情节询问问题3。}。我们可以很容易地总结出一些B做对了的地方,比如说“使用沟通代替冲突”“使用包容代替对抗”“善于倾听从而共情”之类。但这么做会有两个问题:

其一,这些操作不保证成功。当B去和A沟通时,实际还可能出现另一些失败的情况,包括但不限于“A无法正常沟通(可能是语言或肢体上的暴力行为,可能是打断或回避)”、“A只记得霸凌相关内容,而不记得最初的源头是课文”等。我们不能保证“B一定有能力处理这些事情”,因为在这个事例中,A所具有的创伤要远比B所具有的创伤更复杂,需要更强的专业性才能应对。让B直接去和A沟通,性质上和“不自量力的见义勇为”差不多,很容易造成“除了让B受到伤害以外没有任何其它影响”的后果。“包容、理解、倾听、沟通、共情”的方法论中不包含“处理更复杂的创伤”的内容,B如果只学到了这些就很容易纯粹拖累自己。

其二,这些操作很难使用理论来概括。以上的一整套流程如果概括来说,都可以称为“觉知潜意识”。细分来说,有两次觉知:一次是“B发现自己讨厌橘子的原因”,一次是“A发现自己讨厌橘子的原因”。对于B来说,我们可以很确凿地下“只要觉知了潜意识,心理问题就可以解决”的结论;但是对于A则不行。A对橘子的厌恶只是“对霸凌的厌恶”的一小部分,而霸凌创伤远不是一次对话就可以解决的(如果产生了这样的事情,只能说明A本身已经想通/不在意了,当B问起来的时候顺便说一下。这种情况下,可能还同时有“A给B道歉”等环节)。如果A没想通,B去问的时候,A就会觉知出一大堆散乱不成体系的潜意识,无助于问题的解决。

即使我们不管A,只看B,仍然会有一些问题,比如说:我们的理论貌似过于丰富了。上面已经讨论了问题2的三种有效解决方案:通过脱敏治疗、通过觉知潜意识疗愈、通过亲密关系疗愈。而只要我们想,还可以在这个案例上给出至少十多种思路和操作都有所不同的解决方法,从“阉割”到“改变核心信念”一应俱全。

综合来看,这些方法总是会在一些方面上惊人地有效,而在另一些方面(据介绍别的方法的书的描述)毫无效果。那么这是为什么?它们为什么会成功,又为什么会失败?

\divider

首先我们可以注意到一个显然事实:所有的成功都有某种改变,而“解决原有问题”是这种改变的结果。改变可能是直接的脱敏,也可能是通过某些信息让B放下。但同时也不是所有改变都能带来成功,比如“B从此开始怨恨原生家庭”这种改变就不会起到什么作用。于是我们需要讨论两个问题:“怎么样才能有改变”和“什么样的改变才能有用”。

后一个问题有一个非常简单,完全从结果出发的回答:只要能解开B的心结,那么就是有用的。这句话看上去像是废话,它主要的作用是提醒我们“该关心具体情况”了。比如说,当B发现“讨厌橘子是童年创伤”以后,会有两种可能:一种是“发现这也没什么大不了,于是开始吃橘子”;另一种是“觉得原生家庭果然太黑暗了,于是更加讨厌橘子”。这直接说明“通过觉知潜意识来解决问题,不保证成功”。其它的任何一种方法也都能构造类似的例子以说明“它们不是通用方法”。

一些读者可能会认为“这是在钻牛角尖,极端个例不能否认整体的有效性”。这个观点本身过于概括,结论也过于草率。我们不需要借助“这种方法整体是有效的”的经验性结论,而是能够很轻松地说出“前一种情况为什么成功,后一种情况为什么失败”:因为前一种情况下,B只关心“讨厌橘子的原因”本身,而后一种情况下,B还关心“原生家庭”,注意力被转移走了。

“B的注意力从‘讨厌橘子’转移到‘讨厌橘子的原因’”,和“B的注意力从‘讨厌橘子的原因’转移到‘原生家庭’”,是两个性质相同的事件,只有感情色彩的区别。如果我们只关注“有用的前一半为什么有用”,那么就会很轻松地发现很多有效的方法,然后因为各种奇奇怪怪的原因,换个人就不灵了。

再将“转移注意力”抽象一些,我们就得到了“B在讨厌橘子时,激发了某个新的行为,这个新的行为覆盖了讨厌橘子的感受”的叙述。这一论述可以普遍地套用在每一个有作用的方法上。它不负责解释现象,只负责描述现象。
\begin{explain}
至于到底什么样的行为才可以让B覆盖讨厌橘子的感受,这是因人而异的:
\begin{itemize}
\item B有可能在某个C(通常是B很亲密的人)的影响下,潜移默化地开始吃橘子。最开始可能只是“C让B帮忙剥橘子”之类的,后来可能“打打闹闹就吃进去了”,然后彻底放开。或者也有其它很多过程不同的情节,喜欢看情景喜剧或是小甜文的读者应该可以举出很多例子。C可能有意帮B克服,也可能只是性格开朗,但这个相对不重要。
\item B可能记住了“有一个爱吃橘子的C”,但是并没有潜移默化地开始吃橘子。直到某个和C强相关的时刻(比如说分开很久了想C,或者见到橘子了想到C),然后或是在激烈的思想斗争下,或是在浓烈的情绪下,开始吃橘子。这种情况下C不一定是真实人物,也可能是文艺作品中的角色,但这个相对不重要。
\item B可能受到了某些观念的影响,决定要克服自己的缺点;或是因为有一些目标要完成(比如说在果园工作、成为吃播、演戏或其它)而有接触橘子的必要,从而在坚定的勇气和顽强的毅力下不再讨厌橘子。这种成长小故事会多少被人夸赞,从而夸赞代替了讨厌,成为新的感受。
\end{itemize}
\end{explain}
以上这些小剧情想列多少就列多少,但它们在其它理论下其实不好叙述,从而也就不太能当例子出现,只能作为心灵鸡汤喝一喝。它们在文艺作品中出现的频率要大大高于在心理书籍中的频率,只有现象而没有理论分析。而如果我们问“都有什么样的行为能覆盖讨厌橘子的感受”,就天然地需要考虑所有的情况,这样我们至少能够在理论上描述这些情况,不至于一上来就完全漏掉。

如上面所说,同样是基于“可能会被覆盖”,以上这些情况都也有对应的失败可能。比如“B有可能因为C的持续骚扰而讨厌C并产生冲突”、“B可能因为C喜欢橘子而直接远离C”、“B可能讨厌宏观叙事和规训”等。如果想细分,我们可以将其分为“完全没意识到”“因为反感而不去接触”“了解并且厌恶”等多种情况,但这在我们接下来的讨论中不起关键作用,我们仍然将所有情况都统称为“覆盖”(它们也确实都符合本书中\note{覆盖}的定义,区别只在于参与的竞争过程不同)。实际上,对于任何一种行为,我们都可以找到能覆盖它的其它行为\footnote{对于\trained,关联与覆盖的机制本身是图灵完备的。于是根据Rice定理,我们不可能得到它的任何非平凡性质。}。这使得任何一种方法都不能保证在所有情况下都奏效。

于是,我们对于“方法是否有效”的判断,变成了“该方法是否会被覆盖”。如果从抽象分析的角度来说,这和“方法有效的时候方法有效”没什么差别,但具体到人身上就很有价值了。最为明显的是,这提供了一个明确的判据,让我们得以识别“目前的方法无效,应该换方法了”。新方法可以是“解决那个会覆盖原方法的问题”,也可以是“完全另起炉灶”,它们都有可能生效。但我们需要注意,新方法是针对这一覆盖的对策,不应将其和原方法视为“同一方法的两个步骤”,或者是“原方法的深化”之类的东西。如果对另一个人一上来就使用新方法,反而有可能出另外的问题。

比如说,我们如果就此认为“所有的话聊全是为了制造出一个能覆盖的新方法”,并且据此设计出“构筑幻想以解决问题”的方法论,那也是不符合实际情况的。我们上面讨论的只是“因为某些因素而形成了行为后”的处理方法,而没有涉及“形成行为前”的处理方法。我们对后者的归因在很多时候被视作“复盘”,主要聚焦于“为什么这一系列行为会获得不如人意的结果”。这一系列行为不一定都属于潜意识,甚至可能每个都是决策,都有充足的理由。但因为没有全局视野,不会整体规划,才导致事情最终走向出人意料的不利方向。此时归因就真的能起到“在原因处更改决策,避免后续行为”的效果,而不是“在发现结果以后,通过新的思路避免后续影响”。

% 采用这种视角,还需要回答另一个问题:为什么某些方法(比如说觉知潜意识)普遍地对很多人有效。一个比较不负责任的回答是“因为他们在某个方面很相似”。这里,有一些分析方式会将相似视作人的某些普遍特征,比如说“俄狄浦斯情节”“投射性认同”“白骑士人格”之类的,归因于一大堆心理学名词;而另一些分析方式则会将相似视作相同的外部环境造成的影响,比如说“意识形态”“规训与惩罚”“模因污染”之类的,归因于一大堆社会学名词。但如上文的论述,这种单一归因很草率,每种情况都能构造出正例和反例。我们只能用“因为它不容易被覆盖”来回答这个问题,而这基本相当于什么都没说。

\divider

让我们把注意力放回到问题1上。当我们在溯因的时候,总是需要考虑这样一个问题:我们怎么才能确定我们掌握了真实的原因?

想要回答这个问题,其实相当困难。我们不总是有“判断这个思路是对的”的方法,也不总是有“判断这个思路是错的”的方法。在“回忆过去”这一方面,这两项都不是很稳定。如果有某个特别的记忆点和我们的分析一致,那么就可以确定分析对了;如果不一致,那么就可以确定分析错了。而如果已经啥都记不起来了,那就无法确认。

此时,我们的判断方法就变成了“这看起来对不对劲”。具体每个人的“对不对劲”都有区别:有些人注重“这是否合乎逻辑”;有些人依照某些通用的理论来刻画自己的行为;有些人给别人讲故事,以别人的标准来选择性地描述情节。这种情况下,即使我们问“是什么”,也只会得到“信什么”的回答。

这会为找寻原因带来相当多的隐患。比如说,A被霸凌的过程可能有多种不同的情况:
\begin{explain}
以下我们将霸凌者称为D。我们此处不在意D是一个人还是一群人,毕竟我们只关心D的霸凌行为,不关注其它方面。一个人或一群人对以下的分类不会产生什么影响。
\begin{itemize}
\item A可能自己发现了这个伦理桥段。D在注意到了A的某些表现(比如说“A自己往外说”、“A听到就一激灵”等)后,就开始专门根据这一点来刺激A。
\item D可能自己发现了这个伦理桥段,并且在A的面前有意模仿以刺激A。随着时间,D可能逐渐获得了一些新的主意。
\item D可能在发现以后,直接想到了一些新的主意,并且对A单方面取乐(比如说“硬把橘子塞A嘴里”(为了符合人设,读者可以认为橘子是烂的)之类的操作)。此时A不一定清楚D的动机。
\end{itemize}
\end{explain}
这样的分类有其现实意义。如果干预得够早,对于第一种情况来说,“让A收敛一些,看开一些”的方法可行,确实可以预防之后的霸凌行为(因其它原因而产生的霸凌行为另算);而对于后两种情况,则无法仅要求A是无效的,此时不可以使用“一个巴掌拍不响”的论述来转移责任。

如果A记得够清楚,并且情况够清晰,那么对前两种情况的反应大概会是“对没错就是这样”,对第三种情况的反应则会是“我就说D为什么会干这种事”。但这些情况本身可以复合,比如说最开始是情况1,D发现了以后无异于情况2,又发展成无异于情况3。A可能也自己想过“到底是为什么”,从而实际是情况3,也会被A归因成情况1。在混乱的记忆下试图归因,就不保证能得到正确的结果了。

如果我们仅关注问题1和问题2(准确来说是A在问题2的对应版本“A应该怎么处理自己的心理阴影”),那么归因的错误不会产生影响。任何一种A信服的解释都可以作为疗愈的素材。但对于理论分析来说,这还远远不够。我们必须要搞清楚,出现这种“无法溯因”的状况的原理是什么。

以上三种情况中,有一类值得注意的现象:\indicate{同一个认知演化的过程,可以在不同的人身上发生。}\footnote{此处的“不同的人”也可以代指“同一个人身上的不同行为模式”。我们需要微调后续的论述以符合这种情况,但同一个人的不同行为模式之间也可以产生下述的跳跃性演化现象。}比如说,“从《背影》中的情节联想到伦理桥段”可以是A想出来的,也可能是D想出来的;而“从叫爸爸到塞橘子”可能是D在霸凌A时想到的,也可能是D独自想到的,甚至有可能是A自己想到的(比如说看到橘子就犯怵,然后又被D发现)。如果我们只看认知和行为,那么以上几种情况分不出区别来,从而这也很容易形成完善的逻辑,进而变成无懈可击的分析。

但是如果我们具体到“演化发生在谁身上”上,这就会产生明显的区别。在第三种情况下,A如果完全不知道D的动机,那么他的恐惧就只能归因到“D的霸凌行为”上,绝对不能下“潜意识里,A害怕的是朱自清的橘子”的判断。我们在观察B的时候,也能看到同样的现象。B的创伤直接来自于A,而这和《背影》没有直接关系。同理,B在后续也可能会得到救赎或者更深的创伤(见前文讨论),而这些也都可以发生在A身上。

“只将某个单独的行为传递给另外的人,而不同时传递它的形成过程”的过程是可能并且相当常见的。不管前一个人中间有多少,多复杂的过程,后一个人都总是可以只学到起因和结果。至于中间的过程,可以以任何形式和触发顺序作为“这个行为相关联的其它东西”,而具体得到了多少,又破坏了多少因果性,是不可预知的。新行为的内容不再是老行为的内容,而就是老行为本身。认知和行为经过这种跳跃性的视角转换,得以跳跃性地演化。

再叠加上关联的任意性,就使得每一个通过这种方式传播的概念,都会不可避免地产生严重的歧义。不同的理解途径广泛而均匀地散布着,一个人的认知可能从任何几个碎片中拼凑出来。
\begin{explain}
\trained 可以把语言的演化当成一个例子来理解。比如,随着时间的推移,一些词语(和一些其它类型的表达)会不断产生引申义,但我们在学习语言的时候并不是从它的原始含义一路学过来,而是直接接触它当前的含义。它的演变过程可以作为趣味性或是专业性的内容,来加深理解和强化记忆,但直接接触引申义也是可行并且十分常见的操作。

再比如,当我们只知道“词语当前的义项”时,通常无法推断出什么是它的本义,又有哪些是引申义。虽然整体上可以总结出“由具体到抽象”等规律,但总是会有反例。我们总得借助一些其它的东西(比如说古时的预料,再比如说“该词语分地区演化出了不同引申义”、“如果一个东西在古时不存在,那么这个义项一定是引申义”等现象)才能判断。
\end{explain}
而此时,如果我们把“认知的某一种完整演化过程”讲给“只有部分演化过程的人”,那么这个人对该认知的看法就很容易被覆盖。这里的“讲”不一定是“由一个人给另一个人说”,也可能是“自己分析出一些结论”;“完整演化过程”也不一定就是真实发生的过程,标准可以低到“没有其它证据来反驳”(见上文)。接受这一思路的一瞬间,就形成了对自己的污染,就会觉得“这事情一直是这样的”。这使得我们无法通过这种方法区分“是一直有某种认知”还是“可以接受某种认知”。
\begin{explain}
    \begin{itemize}
        \item 这样的思路如果扩张一些,就会变成“人无法想象自己没见过的东西”之类的判断。如果我们仔细理解的话(比如允许“通过某些部分的认知造出整体,然后再去认识整体得到新认知”的过程),这句话还可以称得上是正确。
        \item 再扩张一些,就会变成“人在共情的时候都是在代入自身经历”之类的判断。如果我们强行让“自身经历”也包括“从别处听来的故事”之类的内容,那么这也能算是正确——只有能模拟才能理解,不理解就共情只是自我感动。
        \item 再扩张一些,就会变成“接受一个观点,是因为自己内心深处本来就认同”之类的观点。这就无论如何也圆不回来了,只要我们把注意力从人际交往方面放到教育方面,就能立刻发现反例。当然这仍然在很多方面有解释力和说服力,比如说“寻找自我”“别人不理解”的时候。
        \item 再扩张一些,就会变成“人会有某些行为,是因为人具有某些特质”。不管我们把特质叫“集体潜意识原型”、“天赋”、“性格类型”、“即时奖励”还是什么,试图这么给所有人共同归因,最后只能使心理学名词越来越多,“人的本质”也越来越多。
    \end{itemize}
    \noindent 以上这些,发散得越远,适用范围就越窄。发散得足够远以后,除了“可以指代一些现象”以外,已经无法以此为基础展开有效的分析了。每一步推理都会产生无穷无尽的反例,每一步动机都会在另一些人身上导致截然相反的行为。
\end{explain}

\divider

我们再回过头来,看看最初的那个问题:什么是感情?由于感情也在这样跳跃性地演化,导致我们“从内容角度对感情下通用定义”的尝试必然失败。我们不可能把所有感情都还原到生理反应上,因为确实有很多感情完全不涉及生理反应\footnote{对于\rigorous,这里省略了“将想法排除出生理反应”的讨论。这样的措辞应该不至于引起歧义,请读者自行补充。};我们也不可能把所有感情都还原到思想上,因为任何一种思路都可以被截断,最终只形成跳跃性的条件反射。

想要完全不出错,那只有一种合理的定义方法:把感情完全定义为主观认知,即“感情是所有‘会被人称为感情’的现象的总称”。这个定义天然地蕴含了感情的主观性,不同的人用同一种感情来指代不同的东西变得相当合理。它们之间并非完全不可交流,无论是经历、举动,甚至是词语本身,都可能触发某些人的行为和思路。我们甚至可以见到“看见悲伤这个字眼就会流泪”“见到有人描述空虚就会抗拒”“别人问‘这能忍得住’就冲上去了”之类的情况。这些情况既可能是完全浮于表面的为赋新词强说愁,也可能是深切厚重的回忆和创伤。在大多数情况下,我们就这么互动,也这么共鸣。

在一些情况下,我们其实不需要去理解感情。无论是没时间深入交流,还是有其它的事不需要顾及感情,还是因为各种原因交流无法进行下去,还是之前已经沟通过了,都不会涉及“理解感情”的部分。我们此时对于感情的思考、讨论,重点集中在“会有什么影响”上。避免触及他人的霉头可以更和谐,识别自己的心不在焉可以及时调整,这同样很有作用。此时因为有了明确的指代,我们在结合现实情况的同时思考,所以相对来说更靠谱一些;但是如果把“其他人情绪的影响”直接套用到当前事件上,就仍然有可能出问题。我们仍然需要从实际出发来分析和判断。

但是这样的交流并不能为我们带来理解。互动不是理解,共鸣也不是理解。想要确保理解,那就只能\indicate{从一开始就考虑所有情况,并且一直讨论所有情况,直到完全讨论清楚。}处理某一个具体的情况可能很简单,但处理任意一种不定的情况却相当困难。在此过程中要排除任何先入为主的思路,排除任何分析带来的污染,避免真相被淹没在合理的分析或者是真切的共情之中。

当然,我们也不是每次都要从这么根本的视角出发。在绝大多数情况下,我们分析“某个具体的感受是从哪来的”的时候,都不需要完整地走一遍流程。在具体分析时,上述论述的主要作用是“提醒我们不要过早地下结论”。我们应该首先尽可能全面地收集所有相关的信息(包括但不限于知识和记忆),并且尽量只从事实出发开始分析。必要的时候可以提出猜测并验证,但不要抱有任何一种偏向,以免对不符合推测的信息视而不见。只有这样,我们才能尽可能准确地逼近真相。


\setcounter{secnumdepth}{-1}
\titlespacing{\chapter}{0cm}{2cm}{2cm}
\chapter{理论部分总结\label{chap:midterm}}

不知道有多少读者读完了前三章的内容。可能有百分之一?千分之一?感谢你们能读下来,接受这种信息密度的枯燥文本不是一件简单的事。辛苦了。

还有一部分从各种地方跳转至此的读者,感谢你们没有弃书。在接下来的总结中,我将假设读者都是看完前言直接跳转过来的,会尽量清晰易懂地介绍前三章的整体思路,以便读者查阅相关的部分。

% \vspace{10pt}
% \hrule
% \vspace{10pt}
\divider

本指南的目标是尝试解决\indicate{如何用过于简单的自我意识去面对和处理过于复杂的现实世界}的问题。简单来说,解决方法是\indicate{提升自身思维的系统性,逐渐使自己能够容纳对复杂现象的完整思考}。

理解这个目标需要以下几步:
\vspace{10pt}

\indicate{(1)理解每个概念都在说什么。}这句话中有两个必须解释的概念:“自我意识”和“简单\&复杂”。二者分别是第二章和第三章的基础概念。

\indicate{自我意识}:本指南为自我意识构建了一个三个层次的模型,最主要的部分见\hyperref[ref:意识的层次]{2.2节}。我们将\inote{行为}视为意识活动的最基础单位,并将多个行为组成的,在一段时间之内稳定存在,具有固定特点的现象称为\inote{行为模式}。不同的行为模式决定了人在面对外部环境时的具体行为。我们可能使用“能力”“性格”“态度”等词语来称呼行为模式。本指南认为\hyperref[def:意识现象]{\indicate{人的意识现象}}由多个行为模式组成,不同的行为模式(以及散在的行为)依照环境而触发,使人与外界交互。

\indicate{简单\&复杂}:本指南构建了\inote{复杂系统}的模型来解释简单和复杂,具体介绍见\hyperref[sec:复杂系统]{3.2节}。我们引入了\indicate{微观现象}和\indicate{宏观现象}这一对概念,将微观现象之间通过相互影响,组成的更大的现象称为宏观现象,将这个过程称为\indicate{编织}。行为会编织出行为模式,而行为模式编织出人的意识现象。我们将“现实世界”拆分成一个个更具体的宏观现象来考察。
\vspace{10pt}

\indicate{(2)理解采用这种模型的原因。}这需要解释两个方面:“为什么不采用更通用的心理学术语”和“为什么要单独为复杂性建立模型”。

\indicate{为什么不采用更通用的心理学术语}:这出于两个原因。一方面是这些术语在长期使用中,已经出现了很多歧义,会为讨论带来障碍。即使本指南明确给出定义,也难免会受到其它义项的影响。另一方面是一些术语本身较为模糊。将一切都归因于“情绪”“内耗”“自我意识”的方式无法精细描述客观存在的意识现象,必须使用其它方法来补足细节。

\indicate{为什么要单独为复杂性建立模型}:本指南需要刻画以下两方面的复杂性:一方面是成因复杂,由大量微观现象编织而成;另一方面是解决复杂,难以使用简单且固定的方法应对,具体讨论见\hyperref[sec:对复杂系统的分析]{3.2.3小节}。满足这两方面,且会频繁出现,持续对我们的生活造成影响的现象,便是我们需要面对和处理的复杂现象。
\vspace{10pt}

\indicate{(3)理解这个问题为什么会存在。}这需要解释三个方面:“简单的自我意识有什么特征”、“我们需要面对和处理哪些复杂现象”、“用简单的自我意识处理复杂现象会出什么问题”。

\indicate{简单的自我意识有什么特征}:本指南的标准是“所有的行为和行为模式都是从环境中被动习得,没有成功主动改变过自身的行为”。具体从环境中习得行为的机制见\hyperref[sec:并行与竞争]{3.4.1小节}。本指南的所有方法介绍均以这种水平的自我管理能力为基础,更加有能力的读者可以依照自身情况参考书中需要的部分。我们不区分“没有改变的意识”“尝试过改变但失败”等情况,认为只要有阅读的能力,就有可能理解并掌握本指南中介绍的内容。

\indicate{我们需要面对和处理哪些复杂现象}:现代媒体的高速发展使得任何一个人都会接触海量的信息。这些信息中有一大部分都和复杂现象相关,而其中又有一部分问题是我们无法有效处理,只能无助恐慌焦虑愤怒绝望。同时,我们需要对自己的人生负责。脱离能力而空谈目标没有意义,具体论述见\hyperref[sec:责任与要求]{3.3.4小节}。必须具备处理现实事务的能力,才能处理这些客观存在,必然面对的东西。我们只能选择是先面对才知道自己要面对(可能意识到的时候已经是二十年后了),还是先发现再面对。

\indicate{用简单的自我意识处理复杂现象会出什么问题}:最大的问题就是无法正确认识世界,也无法达成目标。我们在\hyperref[sec:污染]{2.2.4小节}中将这种既不理解也不可控的东西称为\indicate{污染},并在第三章中具体介绍了污染的形成机制。简单的自我意识容易过度简化地概括事务之间的联系,草率地触碰\inote{能力边界}之外的东西,无法也不去分辨自己是否真的知道具体情况,能力是否足够处理这些事。这种现象在面对复杂系统时尤为明显,甚至越熟悉某个复杂系统,情况就会越糟。

如果读者的自身情况不符合以上三个基本条件,那么本指南的内容对您可能是无用的。如果对您来说还有其它方面的作用(如推荐给其他人),请按照自身实际情况便宜行事。
\vspace{10pt}

\indicate{(4)理解这个问题会有什么难以解决的难点。}难点集中在“简单的自我意识能力过弱”上,具体有三方面:

\indicate{无法有效理解世界}:简单的自我意识相当混乱,会包含大量不受控的行为和带有污染的认知。这些\inote{大道理}会主导我们的行为,具体见\hyperref[sec:全局行为模式]{3.4.2小节}。我们只能不停地从大道理出发开始思考,最终得到新的大道理。这些大道理都脱离现实,要不然充满了认识错误,要不然是正确的废话。我们绝不可能在短时间内就充分认识一个具体的复杂现象。

\indicate{无法有效达成目标}:简单的自我意识生产的那些大道理会使我们得到错误的认知。我们会无法理解那些复杂的事情到底困难在哪里,觉得自己能成功但实际情况会失败,于是一边盲目乐观一边盲目悲观。过于简单的自我意识无法支撑我们完成复杂的工作,这种情况下是否能完成目标,完全不取决于我们的努力。解决那些深层次的问题靠的绝对不可能只是态度、热情、冲劲、一次努力,而必须是充足的认知、完善的判断、周密的计划、实在的执行。

\indicate{无法有效控制自身}:\hyperref[def:意识现象]{人的意识现象}是一个复杂系统。简单的自我意识普遍地缺乏处理复杂现象的能力,无法有效控制自身意识现象的其它部分。每当我们想做一些长期的事情,总会被它们打断。具体讨论见\hyperref[sec:缺失与外溢]{3.4.2小节}。需要注意的是,控制自身是一种目标,具体需要做什么按照具体情况而定,这里的控制指的不是“完善地知道每一个细节”。关于什么是“控制自身意识”,参见\hyperref[sec:自控]{2.3.3小节}。
\vspace{10pt}

\indicate{(5)明确解决问题的方法。}本指南将足够复杂的自我意识称为\inote{理性},将拥有理性所对应的能力称为\hyperref[sec:现实事务处理能力]{\indicate{现实事务处理能力}}。这一能力有以下四个方面:

\indicate{分析能力}:正确认识自身的能力边界,总是使用恰当的方式分析问题;

\indicate{调研能力}:明确并获取处理事务所需要的所有信息和知识;

\indicate{计划能力}:将复杂事务按照执行能力拆分成可以完成的简单事务;

\indicate{执行能力}:运用相应的知识体系完成目标中的简单事务。

除了上述提到的内容外,前三章中还有大量内容用于补充细节。本指南力求搭建一套完整的分析框架,使每位读者都能尽量顺畅地应用。这些内容可以补全读者在具体细节方面的问题。

\divider

解决某一个具体的问题或许较为容易,但这一问题的解决方法如果会加重其它问题或是产生新问题,则解决它未必是一件好事。这要求我们不能将每个问题孤立看待,必须整体考虑所有可能发生的情况。我们不止需要处理每一个单独的心理问题。它们虽然成因复杂,但大多数可以在短时间内(如每10个小时的心理咨询以内解决一个的速度)理清并解决。本指南的主要着眼点是那些全局性的,参与了性格主要组成,难以总结出,并且即使发现也没能力解决的问题。

但现实情况是,我们通常既缺乏发现问题的视角,又缺乏处理问题的能力。但绝不应该因此就觉得“遇不到什么很复杂的问题,可以不管”。以解决表面的简单问题为终点,只会失去对深层次复杂问题的认知。越是缺乏能力的人就越是会独自面对复杂问题,同时越是无法发现问题整体,只能发现“不断有新的小问题冒出来”。

读者可能会注意到,本指南在行文过程中着重描写“什么时候会处理不了问题,为什么会失败”,关于失败的解释和分析通常比关于成功的更全面细致。这有四方面原因:1、本指南的理论框架本身就是对待现实的有效思维模式,应该也算入对成功的讨论篇幅;2、如果成功了,就不需要再用到本指南了,只有当前的理论无效时,才需要继续阅读;3、我们需要正确认识这些问题的复杂性和困难程度,避免因为认识过于简单而采取无效甚至会起反面效果的操作;4、越简单的问题越好处理,处理不了的问题往往更复杂,需要更系统的分析视角。这需要更多讨论篇幅,而这些篇幅被安排在了更后面的位置。

对于状态较好的读者来说,前三章的内容应该已经足够用来构建理性。但状态不好的读者会受到自身意识的强烈干扰,这些抽象的理论不足以用于解决现实中的问题。再完美的理论落不了地也没用,本指南的后续内容会偏向于具体应用,希望可以对这部分读者有所帮助。

虽然也可以选择从此处继续向后阅读,但若读者有确切的自我改变需求,我推荐至少通读一遍前三章。阅读顺序没有严格的限制,读者可以从上面介绍过的概念中挑选自己感兴趣的开始阅读,但最好确认自己能理解相关段落的用词。如果最终的目标是熟练应用,那么应该不止会翻一遍这本书。在熟练掌握之前,本指南仍然可以起到辅助记忆、整理思路的作用。

我姑且认为那些走投无路的读者更有可能读完本指南,其它人不一定坚持得下来,会有其它东西分心;走投无路的读者也不一定是真的坚持下来了,大多数应该也只是没别的事好干,然后慢慢磨完了。如果您凭借渊博的知识、持久的习惯或坚韧的毅力读完了本指南,请接受我的敬意。

前三章的总体可读性较差,仅能起到大致的观点介绍作用,在具体行文和整体安排框架上均有考虑不周之处。这是我经验不足所致,由此造成的阅读障碍我深表抱歉。其中的观点可能有考虑不周之处,甚至可能有严重错误,希望读者不吝指出。

\newpage

\chapter{应用部分前言\label{sec:应用部分前言}}
为了方便各位读者,我们在进入应用部分之前,先简单罗列一下后文会频繁使用到的概念与分析思路:

\begin{itemize}
\item 本指南主要关注\inote{行为}与\inote{行为模式}层次的问题。人的意识活动是主观的,但“人有意识活动”是客观现象,可以观察和分析。由“主观”提供的复杂性在于“在同一情况下的行为因人而异”。我们如果想要\inote{客观}地分析意识现象,就必须且仅需确保自己从现实出发,对分析对象有充分了解,足以\inote{模拟}分析对象。
\item 本指南认为,我们总是以某个一个行为模式为出发点,来观察与分析。这些行为模式可能有不同的特点,因而会对同一现象产生不同的看法,具体讨论请见\hyperref[sec:解读]{4.4节}。为了方便起见,当我们分析另一个行为模式时,会将其简称为“人”,使用“对方”等称呼。仅在少数需要特别强调之处(如分析多个行为模式相互之间的影响时),我们会使用“对方的这一行为模式”这种更长的称呼。读者应时刻留意,切勿以偏概全。
\item 理论部分提出了\inote{人的意识演化基本模型},将“人的行为模式会随经历缓慢变化”这一现象的主要的机制归结为\inote{外溢}。外溢是指“在信号和行为之间错误地建立了关联”的现象。应用部分将会和大家一起分析“什么样的关联是不真实的”,并且参考这一机制,得到培养正确行为的方法。
\begin{explain}
需要注意的是,识别“什么样的关联是不真实的”不需要回顾这一关联形成的过程,改变相应认知也不一定需要。我们只需要让其\inote{可控}即可。但是培养新行为时,一定要充分\hyperref[def:理解行为]{\indicate{理解}}这种机制。具体讨论见第六章。
\end{explain}
\item 本指南推荐各位读者在阅读时,不带有任何先入为主的\inote{价值观},不要代入任何角色和身份,不要为例子和论述中的观点强加任何\indicate{责任}、\indicate{意义}和\indicate{作用},不要在充分了解之前就草率地下判断。我们应该优先关注能力方面的问题,即“当事人是否有能力解决所面对的事务”与“读者是否有能力客观地分析对应的情况”。我们唯一应该使用的价值观是\indicate{就事论事}。在应用部分中,本指南会将不符合这一标准的现象,如行为外溢、\inote{表面分析}等,称为“坏的”“不应该的”。请读者注意,这不是指“某一种具体行为不应该”,而是指“脱离了该行为的应用范围”这一现象不应该。
\item 本指南给出的定义中包含“是xxx的yyy”“xxx的yyy”“对xxx的yyy”“关于xxx的yyy”等表述,其中xxx是概念yyy的前提条件。在使用本指南中的概念时,请读者务必留意其适用范围。如果忽视前提条件,随意使用这些概念,就可能使结论外溢,从而失去意义。文中有时为了行文方便,会省略前提,读者应结合上下文以自行补全。
\item 本指南会有“将xxx视作yyy”一类的表达。这种表达的意思是“yyy所具有的性质xxx都有,所以讨论yyy时得到的结论在xxx处也可以用”。xxx的其它方面则不参与讨论。这并不是说“我们总可以把xxx当作yyy,xxx的其它方面不重要”,只是为了当前段落的简便。我们仍然需要在其他地方单独处理xxx的其它方面。
\item 本指南所介绍内容中,最重要的部分是\indicate{概念在分析框架中的应用}。分析框架是一个完整的系统,不应该将其中的某些内容单独拿出理解。读者如果觉得“某一句/段话很有道理”,请核实自己是否可以从本指南所介绍分析框架中,独立总结出相关观点。本指南不建议读者将特别看重某一句话。一句印象深刻的话只应起到提醒的作用,使得我们可以想起并使用分析问题的框架。
\item 很多概念,特别是和意识强相关的概念,因使用者的不同,所指代的具体内容可能有极大差别。为这些概念下一个客观定义是不可取的行为,无论怎么下都无法普遍地覆盖所有人的感受,总会有人觉得“不是这么回事”。于是,我们转而用“每个人自身的感受”来定义这些概念,并将这种定义方式称为\indicate{主观定义}。这些概念具有某些共同的特点,从而我们可以分析它们的影响。但在涉及到“主观定义的形成”等环节时,不应继续使用主观定义来分析问题。
\begin{explain}
\trained 可能更习惯\indicate{泛性质定义}、\indicate{现象学定义}等其它术语。本指南出于贴近日常用语的考虑,选择“主观”一词。
\end{explain}
\item 本指南会特别关注“一个人是否形成了某种认知”,只有明确的认知才能产生后续影响。在对行为归因时,本指南会尽量避免使用“其实是想被关注”“其实是想被回应”这一类用语。我们按照对应的方式来与其互动确实可以解决问题,但如果没有形成明确认知,这只应视为对应行为模式整体所体现出的特点,而不应该被视为原因。我们有必要更深入具体地研究这些行为模式,而不只是草率地表面归因。更详细的展开讨论请见\hyperref[sec:对竞争过程的解读]{4.4.3小节}。
% \item 为隐私考虑,在本章以及后续的所有内容中,凡是涉及案例分析的环节,事件主角的名称均使用\source{泠珞}替代。如果案例中会出现咨询师,则名称均使用\source{零羽}替代。如果案例中出现其它人物,会使用龙吟、辰柯、墨默等名称替代\footnote{这些名字都是《妄想症》系列的角色,\moegirl{妄想症}。}。读者请勿在此产生有关性别、身份或其它方面的刻板印象。
\end{itemize}
\setcounter{secnumdepth}{2}

\titlespacing{\chapter}{0cm}{5cm}{1cm}
\begin{savequote}[450pt] %指定引言宽度
    \fontsize{8pt}{8pt} %两个参数分别指定字号和行间距
    音乐嘛,只有量身定做的才能打动人心啊!\footnote{原漫画因有妖气停止服务已下架。\moegirl{你什么都没看见}。}
    \qauthor{笛子Ocarina《你什么都没看见》第47话 塞壬}
\end{savequote}

\chapter{认知形成与信息交流}

\addtocounter{section}{-1}
\section{基础讨论与用语}
我们将“某个现象作为刺激,触发了某个行为模式,并形成了认知”的过程称为\indicate{信息传递},并将其描述为“信息从某个现象\indicate{输出},\indicate{输入}到某个行为模式中”或是“某个行为模式\indicate{接收}或\indicate{提取}了这一信息”,将形成的认知称为对该现象/信息的\source{解读}。
\begin{explain}
输出信息的现象不一定是另一个行为模式,也不一定是一个人。由于我们可以从任意现象中受到刺激(不过本章的讨论仍然会以人为主),这里的“输出”指的就是现象本身。一个现象可能分时间和方面输出多个信息。

而输入环节,因为本指南仅关心人的意识现象,故此处不讨论其它的信息传递途径,仅考虑输入到行为模式。因为外溢现象的存在,一个现象可能刺激出任何一种行为,形成任意一种认知。

如果没有形成认知,那么这个现象在刺激对应的行动结束后,便不再对该行为模式有直接影响。这不是本章关注的情况。形成的认知不一定可以使用语言来描述,记忆、情景、条件反射等其它形式也计算在内。

在一些情况下,本章所讨论的内容也可以套用到行为上,此时我们可以将信息传递的定义扩充为“形成了行为/认知”。读者可以自行替换相应段落。
\end{explain}
以具体语境的不同,我们将\indicate{一个信息刺激了某个行为模式,输出新的信息}的过程称为该行为模式\indicate{回应}了该信息;将\indicate{一个信息输入到某个行为模式,产生新的认知,并输出新的信息}的过程称为该行为模式\indicate{传达}/\indicate{传递}/\indicate{描述}/\indicate{表述}了该信息。我们将“行为模式A传递的信息触发了行为模式B”称为“A向B传递信息”。
\begin{examples}
所有的传递都是回应。回应和传递的区别在于\indicate{回应不要求产生新认知}。

一个信息可能会触发不止一个行为(比如说总是提起印象深刻的事情),此时我们将每个行为单独看做一次传递。同时,我们不将“行为模式内部的相互触发”视为回应,只将“产生了外部影响的行为”视为传递。

传递有很多种类。从形式上来说,新信息可能和原信息几乎一模一样(比如复述一句话),可能转换成了另一种形式(比如说向别人讲某个事件)。

以上两种传递都保持了认知的\note{内容}不变(后者可能有一些细节损失),我们将其称为\indicate{准确}\footnote{准确这一概念的正式定义请见\hyperref[def:准确]{4.4.1小节}。}的传递。但传递也有可能劣化。比如,当我们\note{表面分析}的时候,经常会得到错误的认知。此外,传递的输出不一定是认知。任何形式的行为/行为链,无论是自知的还是不自知的,只要被这个信息触发,都可以是传递。此时,传递信息的行为可能完全脱离了原意。

在本章讨论范围内,如果B总是可以准确传递A,那么我们就将B和A视为同一个行为模式,将“B(向C)传递A的行为”也称为“A(向C)传递”。
\end{examples}
如果输出信息的现象是某个行为模式,那么我们就将对应的行为称为该信息传递对应的\indicate{表达}。两个(或多个)行为模式的一次接触中可能有多次表达,我们将\indicate{一次}\inote{接触}{中的所有表达}统称为\indicate{沟通}或\indicate{交流}。
\begin{explain}
所有的行为都可视作表达,如语言、动作、神态等。使用“表达”这一称呼,表明我们仅关注两个行为模式的接触,不关注其它方面的影响。我们将对应的信息传递过程称为“输出者向/对输入者表达”。在一次沟通中,输出方和输入方的身份可能有多次变化。如果我们想只考察输出和输入的现象,就必须在表达的层次上分析。

由于信息输入会形成认知,认知会长期留存,并持续触发其它的行为,我们其实很难定义“一次接触”。本章的讨论中,沟通可能包含新形成认知的后续影响(这种沟通通常会在短时间内结束),或不包含新形成认知的后续影响(这种沟通可能持续很久),后者主要指“某个中断又恢复的话题”。读者应能自行区分这两者。

在第二、三章中,我们已经详细讨论了一类沟通现象:发生在同一个人身上,在不同行为模式之间的信息传递。这种交流现象会使行为模式之间相互触发,此处不再赘述。

大多数信息输出是不可控且不自知的。实际上,因为我们什么都可以解读,任何行为都可以视为输出了某些(只能激发某些特定行为模式的)信息。只不过大部分行为不会产生什么实际的影响。

人类传递自身认知的方法不是很多,大多数通过语言/文字叙述,动作、情景等情况占少数。本章中的讨论也会因此偏向语言沟通。
\end{explain}
我们将“一个信息完全不触发行为”的现象称为“行为模式未接收到信息”或是\indicate{无效表达};如果某次沟通/表达是实现某目标的一步操作(目标可以是“传递某个信息”“布置某个任务”或其它),但是该操作没有取得预料中的效果,就将这次沟通称为\indicate{关于该目标的无效沟通/表达},在不引起歧义的情况下也简称为\indicate{无效沟通/表达}。
\begin{explain}
我们有时将无效表达称为“行为模式没有留神/注意/关注该信息”。

信息的输入需要形成认知。在此定义下,很多日常语境下的交流(比如说相谈甚欢的闲聊)会被判定为无效沟通。对无效沟通的具体展开讨论见后文。
\end{explain}
% 如果有一个稳定的过程,可以给某个行为模式输入信息,那么我们就将这个过程称为一个\indicate{话题},并称该行为模式在这一话题上\indicate{可沟通/可交流}。
% 如果有一个稳定的过程,可以给某个行为模式输入信息,那么我们就将这个过程称为该行为模式的\indicate{沟通渠道}% ;如果有一个稳定的过程,可以接收某个行为模式的信息,那么我们就将这个过程称为该行为模式的\indicate{解读渠道}。
% \begin{explain}
% 注意,沟通渠道仅对于一个行为模式而言,仅针对其输入的
% \end{explain}
一个人的\note{想法}只能由自己的行为模式回应。
\begin{examples}
这句话听起来像是什么“做不被定义的自己”“只有自己才能决定自己的未来”之类的意思,但实际上不同。它所表达的仅是“想法不具有任何外部影响,所以只能刺激自身的行为”。想法只有作为动机,被表达为其它行为(可以是语言、表情、肢体动作或其它)时,才能被间接地注意到。这不应该算作是“想法直接刺激了外部”。

本指南仅考虑“行为模式之间的交流”,这是一种较为方便的安排。我们不关心是“自己的行为模式之间的交流”还是“两个不同的人在交流”,因为这两种情况只有“是否能察觉到想法”的区别。其它的任何不同,如“是否熟悉”“是否有盲点”“表达是否准确”之类的,都和“交流的是谁”无关,都可以归因到行为模式上。本指南将人视为一种\note{共识环境},这样就可以将采用统一的分析框架分析。关于共识环境的讨论见4.3.2小节。

和\hyperref[para:不研究人]{之前}一样,我们会使用“人”来简化叙述。本章中使用“人”“对方”“xx者”等概念时,有可能指人或行为模式。读者应默认指行为模式,除非出现“人拥有的行为模式”等概念。
\end{examples}
在行为模式A和行为模式B的某段沟通中,如果B没有得到任何输入,那么就称该沟通为A对B的\indicate{倾听}、\indicate{观察}或\indicate{观看}。如果同时A也没有得到对B的任何理解,那么就称其为\indicate{无效倾听/观察}。
\begin{explain}
B没有得到输入有很多种情况,如“A的所有回应都是想法,没有外在表现”“A仅采用‘嗯’‘点头’等方式回应”“A实际上刺激了另一个行为模式C”等。

此处定义的倾听不要求“A有倾听的意愿”,其语感类似于“B单方面表达”。同时,倾听不意味着增进理解,“双方各执一词,沟通毫无进展”的情况在本定义下算作“双方同时倾听对方”。
\end{explain}
某个思考回路如果\hyperref[def:理解行为模式]{理解}了某个行为,那么它就能准确传递这个行为;某个思考回路如果能\note{模拟}某个行为模式,那么就总是能准确地传递这一行为模式的行为。

本章的讨论重点就在于此。什么样的行为模式总是能准确传递另一个行为模式?什么样的思考回路能够理解另一个行为模式?如果不能,又是被什么原因干扰了?是产生了错误的认知,还是根本没接收到信息?

准确传递一个行为模式,是理解和改变的前提。而如果对这个行为模式的认知本身就是不准确的,那么进一步的理解和改变的尝试只能做无用功。

% 本节内容应该是本指南全书中和其它心理书籍最像的一部分,使用的方法论高度类似。毕竟这确实经过大量实践检验,相当行之有效。

% \begin{examples}
% 以下是一个本章会讨论的情景:一个人具有思考回路A,通过语言等行为模式B将其表达出来,尝试和另一个人的行为模式C交流。由于是不同的人,A无法直接和C表达。在此过程中,B的表达能力和C的理解能力都有可能不足,从而导致无法准确传达关于A的信息。
% \end{examples}

\section{作为副作用的沟通}
\begin{explain}
\note{副作用}是指“不会参与决策,成为行动的动机”的影响。本节中,我们将\indicate{拥有其它动机的行为模式}简称为\indicate{自身},将\indicate{受到行为影响的行为模式}简称为\indicate{其他人}。
\end{explain}
自身出于沟通以外的动机而行动时,“别人接收到信息”和“沟通”的现象就都是副作用。这种情况下,自身不将本次行动视为沟通。
\begin{explain}
这可能是因为“根本没意识到会对其他人有影响”,也可能是因为“能认识到会对其他人有影响,但是不在意”。
\end{explain}
不将行动视为沟通,就会有两个特点:没有“将信息有效输出给其它行为模式”的目标,也没有“接收其它人的回应”的能力。
\begin{explain}
特别地,如果一个行为模式被激发时,总是有沟通以外的其它动机,那么我们无法找到稳定可行的方式,给该行为模式输入信息。这使得除了倾听外,不存在其它与该行为模式的沟通方式。

% 两个有其它动机的行为模式相遇时,虽然看起来有可能很热闹,或是相谈甚欢或是要吵起来,但是实际上两人完全在各说各话。
\end{explain}
本节内容讨论各种不同的类型以及它们的成因。

\subsection{零散表达}
\noindent 如果一个表达不是任何思考回路的结果,我们就将其称为\indicate{零散表达}。
\begin{explain}
零散表达不经过思考回路,而是直接在刺激和表达之间建立了\note{关联},可以视为纯粹的条件反射。关联可能有很多种不同的成因:

最常见的一类是“模仿”。当一个人长期接触某个环境,环境中的其它人都以固定的方式对待固定的刺激时,自己就也慢慢形成了相同行为。这类情况包括但不限于“背诵/唱歌/跳舞”、“进球/听到高音就欢呼”、“重复某个词语/句子/称呼”\footnote{熟悉相关梗的读者应该知道我在指什么。为了避免写出来就过时,这里就不举例子了。}等。这种现象以及对应的行为应当视为污染。

另一类常见情况是“熟练”。当一个人长期使用某个行为去应对某个刺激时,也会在二者之间产生关联。可能在关联形成之前有完整的思路,但是关联形成之后的触发就和思路无关了。这类情况包括但不限于“说话风格/小动作/口头禅”、“各种情绪”、“看见某个东西就开始夸/骂”等。这种现象是行为的外溢(当然也是污染)。

两类情况的主要区别在于“污染源来自行为模式外部还是行为模式内部”。在一些情况下,如果污染源来自内部,那么自身会对这一行为的形成过程有明确认知(即“是否\hyperref[def:理解行为]{理解}这一行为”),从而方便对该行为的后续处理。不过实际上,“模仿”也可能知道这种方式整体的发展过程(可能伴随一些社会学研究),“熟练”也可能无法完整回想自己的思路。这使得这一区别变得不重要。

如果我们持有“只有思考回路以及更上层的意识结构才有意识”的看法,那么零散表达是无意识的表达,自身的行为不代表自身的意愿。虽然说了一句话,但这句话本身是背诵出来的,话是什么意思完全不知道。如果它是因熟练而无意识化的,那么它仅代表过去的意愿;如果它是因模仿而无意识化的,那么它完全不代表任何意愿。
\end{explain}
零散表达也可以编织成沟通,但这不代表其中有任何有效的信息传递。
\begin{explain}
在合适的环境下,零散表达会被反复触发。每一次触发都是一次独立事件,触发原因可能是环境中的刺激,也可能是上一个表达。我们必须对每个问题单独分析,不应因为“一个人在连续表达”的现象就将所有表达视为一个整体,草率地认为“因为上一次表达触发了下一次表达”。

另外,两个行为模式的表达如果分别能触发对方的零散表达,那么就会编织成一次沟通。这种沟通的总体氛围可以很融洽,或是很尖锐,沟通参与的双方或许也能得到“聊得很开心”“吵得很激烈”之类的认知。但这个认知是此次沟通本身的宏观特点,该信息由沟通输出,不可视为行为模式之间的信息传递。
\end{explain}

\subsection{转述} % 复述?
在行为模式的一次触发中,如果其中的每个行为都是对另一个事件/现象的表述,那么我们就称这次触发为对此事件/现象的\source{转述}/\indicate{复述}/\indicate{复现}/\indicate{模仿},将事件/现象称为这次转述的\indicate{内容},将行为模式称为此现象的\indicate{转述者}。
\begin{explain}
如果我们将此次触发视为“使用对应的认知”,那么被转述的事件/现象是此次触发的\note{参考事件/现象},转述的内容属于认知的\note{内容}。

一个行为模式有可能不总是在转述,但我们总可以将其中的转述视为一个子行为模式,并将这个行为模式的其它部分视为另外的行为模式。这种处理会为之后的讨论带来便利。

转述和零散表达不是互斥的两种表达类型,转述本身可能也与思考回路无关。另一方面,转述也可以完全可控,比如对知识体系的\note{应用}。参考现象可以有以下几类:

最常见的一类是“自己之前的思路”(这里的“自己”指一个人的意识整体)。当自己对某个现象思考过以后,再次遇到这一现象时,就有可能回想起之前的思路。这不只限于当前思考模式,只要是能想得起来的,都会产生影响(读者如果觉得“转述自己的思路”这种用词较为奇怪,请注意,思路和转述者是不同的行为模式)。在一些情况下,这被称为“第一印象”。

另一种是“之前了解过的现象”。具体来源多种多样,可能有“亲身经历”、“别人的一句话/一段论述”、“一段剧情中的表达”等。如果自身对此印象颇深,那么就会(可能无意识可能有意识地)反复观看/回忆对应段落,从而逐渐内化为自己的思路。如果现象本身没有明确记忆点,那么在一定时间后,我们可能忘掉其具体来源,将其完全转化为自己之前的思路。
\end{explain}
一个行为模式在转述一个现象时,能够对相关的刺激内容做出\indicate{回应},但这不代表它接收到了任何信息。
\begin{explain}
在遇到“询问”“情景再现”等一些刺激时,转述者看起来能够顺畅地交流。依照具体情况的不同,我们可能会用“来者不拒”“外向开朗”“积极地介绍自己的爱好”“亲切友善地回应”“打开了话匣子,滔滔不绝地讲述”“对这个问题很有见解”等来描述这种场景。如果当前事务是“倾听转述者,从它那里获取信息”“让转述者应用已有的行为模式解决问题”一类,那么进展就会很顺利。

但转述者无法接收信息。参考现象是已经固定的现象,就如同转述者无法更改故事的过程和结局一样,转述者也无法更改一个成型的思考回路。当看到某个信息时,转述者会做的事情是“检索参考现象,并在其中找到与刺激最接近的内容,并且将其转述出来”。这看起来像是在讨论这个信息”,但实际上说的还是参考现象。一些情况下,这可以被称为\indicate{路径依赖}、\indicate{自说自话}、\indicate{代入自己(过多)}等。这会导致如“无论怎么做都无法令转述者满意”一类的现象,因为转述者的行为模式中有可能只包含“表达需求/情绪/...”的行为,而不包含真实的需求,不包含满足的条件,不包含后续行动,不包含对其内容的任何理解。

我们同样也可以考察两个行为模式都在自己身上的情况。如果我们在面对某个现象时,同时触发了两个转述者(且只触发了这两个思考回路),它们具有不统一的想法,并且在价值判断上一个正面一个负面。它们会不停讲述自己的理由,而我们由于缺乏其它有能力的思考回路,无法将它们给出的信息梳理清楚。此时我们就会陷入纠结,还可能伴随有迷茫、烦躁、痛苦等情绪。

为转述寻找例子是很麻烦的事情,任何例子都会导致一部分读者深感赞同,另一部分读者感到愤怒恼火/窒息绝望。这会干扰读者对这一部分讨论的理解。确实有需要的读者可以参考龙应台《目送》中“我不爱吃鱼”的段落,或其它内容。
\end{explain}
\indicate{转述者是大多数人沟通时最常见的一类行为模式}。
\begin{explain}
这一结论基于两个假设。一个是“熟练的行为更容易触发,容易触发的行为更熟练”;另一个是“想法远比其它类型的行为更容易触发”。相应的讨论参考\hyperref[sec:想法的竞争与外溢]{3.5.1小节}。

外溢的参考现象和思考回路会大幅度干扰我们的思路,打断我们原有的行动。脑子里只有一种念头的时候,按着这种念头行事是顺理成章的。转述是纯粹的习惯堆积,不抱有任何明确的目的,也不体现原有的意愿。我们不应将其称为“享受自我表达”,也不应将其称为“渴望他人认同”。它自身会指引我们接下来的行动,一些场合下可以将其视为“新的意愿”。

触发转述时,如果原有思路(和意愿)能不被打断,那么它有可能依然可以接受信息,可以解读转述者的表达。但这不应该理解为“转述者体现原有意愿”,它和原有思路应当被看做两个行为模式。我们不可能时时刻刻清醒地反思自己的每一个行动和念头,转述者经常会成为唯一被触发的行为模式。

因此,“转述者之间的沟通”经常被人当做是沟通的参考样板。我们确实有的时候能从转述者那里接收到信息,但总体来说这是一种低效的方式。对此的展开讨论参见后文。
\end{explain}
转述者可能会以任意逻辑和次序表述参考现象。
\begin{explain}
这使得我们即使想接收转述者输出的信息,也不总是能成功。关于接收的具体讨论参见后文。

一方面,这是因为来自外界的刺激本来就随机;一方面,接收到刺激后,转述者可能在产生了数个想法后,才对外表达一次,具体表达哪个想法,很大程度上是随机的,即使内部思路很连贯,在外部也会体现出跳跃性;一方面,转述者之前可能进行过类似的表达,一些认知之间已经建立了额外的关联,这使得内部思路也有可能是跳跃的。

在合适的情景下,转述者有可能条理清晰地完整表述某一段参考现象。这种行为有可能再重复中不断固化,从而形成“转述者似乎很条理清楚”的表象。但实际上这只是处于行为这一层次的条理清楚,不代表转述者整体的特点。

如果转述者拥有“想到什么就立刻说出来”的特点,那么我们在外部也能观察到连贯的思路。但连贯的思路不等于顺畅的逻辑,转述有可能仍然充满了额外的关联和突然的跳跃。
\end{explain}

\subsection{拒绝沟通}
我们将\indicate{发现并不理会某个现象的刺激}\footnote{“不理会某个信息”指的是“不因其而产生行为”。对于\rigorous,“发现”一定会产生认知,和“不理会”矛盾。此处的处理方法是将发现视为另一个行为模式的行为。}的决策称为与该现象\indicate{拒绝接触},或是\indicate{拒绝}与该现象接触;如果该现象是行为模式,那么就对应地称为\indicate{拒绝沟通}。
\begin{explain}
拒绝接触一个现象时,自身一定对“接触该现象”有预料和相应的负面价值判断,如“接触本身有害(比如横穿马路)”、“对方本身有恶意(比如霸凌)”、“对方行为冒犯(各种不文明举止)”、“对方性格如此(还有别的方面可以接触)”、“对方和我不和(可能有宿怨)”、“对方不可沟通(对方是转述者的话确实)”等。

拒绝和自身的其它行为模式沟通也是常见的现象,我们经常会做出“想不明白,不想了”“不理会脑子里嘈杂的声音”“不要为这件事坏了我的好心情”“过去的事情就过去吧”“已经太迟了没办法了”等决定。

这些预料、价值判断不一定代表事实,只是自身的主观判断。自身可能只是转述了之前的记忆,而没有从客观出发分析。对应的决策也不一定在事实上有利的。
\end{explain}
拒绝接触这一行为本身和相关认知都有可能产生外溢,从而污染我们对接触和沟通的认知。
\begin{explain}
对于一些现象,不予理会是正常的选择,能够有效避免我们遭受影响/邯郸学步/裹足不前。对于沟通来说,这也可以使我们不被尖酸刻薄/不怀好意/煽风点火/嚼舌根传闲话/......的行为模式影响。

输出方面,沟通时最常遇到的行为模式是转述者。在面对转述者时,所有输出都没有作用,和它争论只是浪费时间,只会看到一些之前不知道见过多少次的重复行为。转述者压倒性的出现频率会让人很容易产生“所有的沟通都没有用”的全局认知。这个全局认知在一些例外情况下会被覆盖,比如说“和好朋友交流”等。

输入方面,大部分表达是零散表达或者转述,不包含什么有价值的认知。这使得“不理睬别人的风凉话/瞎操心/乱提意见”在大部分情况下是正确的决策,进而会让人很容易产生“不需要受别人影响,只要自信地走自己的路就好”的全局认知。这个全局认知在一些例外情况下会被覆盖,比如说“听某些老师讲课”等。

在思考沟通及其相关现象的时候,我们难以将全局的“沟通无效”认知和例外的“沟通有效”认知统一起来,大部分情况下是各论各的。在遇到新的沟通场景时,“沟通没有用”的认知和拒绝沟通的行为更容易触发和外溢。

这样的认知会干扰我们真正深入透彻地看待沟通这一客观现象,使我们既难以具体定位“无效的沟通为什么会无效”,也无法系统讨论“有效的沟通都做对了什么”。只有从中提取出原理,才有可能运用到其它方面,带来沟通能力的实际提升。
\end{explain}

\section{有效沟通的必要篇幅}
我们将\indicate{一次沟通中,内容相同的所有表达}称为一个\indicate{话题},将这些表达的\note{内容}也称为这个话题的\indicate{内容}。我们将\indicate{其中一个行为模式在此次话题中,所有(不重复的)认知}称为该行为模式在这个话题上的\indicate{篇幅}、\indicate{表达篇幅}或\indicate{讨论篇幅}。% 我们将\indicate{属于所有行为模式的篇幅的认知}称为这些行为模式在本话题中的\indicate{共识}。
\begin{explain}
篇幅和\note{出发点}是两个不同的概念,出发点的范围要更大。篇幅仅包含表达中涉及的认知,而不包含没有表达出的内容。我们只能从对方的表达篇幅中接收信息,所以有这种区分。

同一个话题中,不同行为模式的表达篇幅可能差别很大,两者可能完全是在自说自话,除了“使用同一门语言”以外找不出什么别的共识。

% 我们也可以对于某一些行为模式单独定义处于它们之间的共识,不过这对本节的讨论没什么帮助,故不额外定义。
\end{explain}
本节内容讨论因表达篇幅不足而产生的无效沟通,以及对应的处理方法。

\subsection{认知的形成与维持}
\begin{explain}
本小节中讨论的内容不仅限于沟通之中。出于简便的考虑,本小节会使用“一个认知”来指代一个/一组认知,读者应能自行补全。
\end{explain}
我们将\indicate{一个思考回路拥有且不被覆盖的认知}称为这个思考回路能\indicate{维持}的认知,将“之前维持,之后不维持”称为这个思考回路\indicate{丢失}了这个认知。
\begin{explain}
这里的“覆盖”指的是“在想起一个认知时,总会想起另一个认知,并且另一个认知覆盖了原有认知”。

虽然“维持”听起来像是个能持续时间很长的特点,但“思考回路能维持什么认知”必须在每一时刻分别判断。它经常处于快速变动之中,我们很多时候无法对于“某个沟通过程”或者其它的一长段时间来定义“一直维持的认知”。“花很大力气讲明白了一件事,然后转头就忘了”的事情时有发生。

对于一个理解途径,如果我们能维持它的每一个想法,那么在想过一遍理解途径后,我们就会临时组成一个“理解途径的转述者”的思考回路,认知也就能够维持,并且(在注意力转移之前)会越来越熟练。如果我们无法维持它的每一个(重要)想法,那么就总是无法理解这个认知。“一部分理解途径的转述者”可能会得出某个其它的认知或原认知的劣化版本。

维持认知不一定需要用原始的理解途径,一些认知完全可以由高度熟练的想法直接触发和维持(这个想法因此成为高度个人化的,无法让别人也使用的,高度外溢的理解途径)。如果在触发后有“验证它是否符合当前情况”的步骤,我们仍然认为它是可控的。
\end{explain}
我们将\indicate{某个认知的一个}\inote{理解途径}\indicate{的出发点}称为这个认知的一个\indicate{理解门槛}、\indicate{理解前提}、\indicate{理解条件}或\indicate{背景知识}、\indicate{背景信息},将理解门槛中的每一个认知称为这个认知的一个\indicate{前置认知}/\indicate{前置信息}。
\begin{explain}
如果某个思路同时得到了多个认知,那么也可以将其称为这些认知共有的理解途径,并以此来定义共有的理解门槛和前置认知。多认知共有的理解途径可能比其中一个认知的理解途径长,前置认知可能更多。

不是所有认知的形成都需要理解门槛,不由思考回路得到的认知(比如简单的模仿)就无法定义其理解门槛。

前置认知的范围很广泛,包括用词、指代对象等等基础而容易被忽略的东西,具体讨论参见\note{出发点}的相关内容。

在一个理解途径中,除了前置认知和目标认知以外,可能还有一些中间认知。如果这些中间认知可以在过程中自然地获得并维持,那么我们一般将其忽略;如果这些中间认知有可能无法维持(可能因为没有引起重视、比较复杂等情况),那么就应当将其也视为前置认知。如果有必要,可以将理解途径按照这些中间认知分为多段,将中间认知变为“上一段的目标认知”和“下一段的前置认知”,然后分别处理。在接下来的篇幅中,我们总是这样切分理解途径。

拥有背景知识不意味着就能够获取这个认知,还需要实际触发才行。有背景知识但没有对应认知,有可能只是纯粹的“从来没有想过”,也有可能是“有别的思考回路覆盖了对应的思考”。

为了方便后文的叙述,我们可以适当扩大定义:如果一个思路实际上没有得到某个认知,我们仍然可以将其视为理解途径,并且将这一认知本身(或是某些额外的前置认知)加入到理解门槛中,以使得添加后的思路称为该认知的理解途径。
\end{explain}
对于一个思考回路,我们将\indicate{一个认知的理解门槛中,该思考回路不能维持的部分}称为该思考回路(通过这个理解门槛)理解这个认知的\indicate{必要铺垫}。
\begin{explain}
必要铺垫实际上和三个东西有关:认知、认知的一个理解门槛(或是理解途径)、思考回路(能维持的认知)。

一个认知的理解途径可能不止一个,它们除了“都能得到认知”以外可能没有任何共同点,对应的理解门槛也可能没有共同点。这使得我们必须对于每个理解门槛去单独定义“必要铺垫”的概念,而无法一个认知定义统一的必要铺垫。不过在后文中为了简便起见,我们经常省略对理解门槛的强调,请读者自行补全。

必要铺垫中可能会包含“某些外溢的认知为什么不对”“是这种解读不是那种解读”等内容,这部分内容因人而异,有可能某个认知对大多数人来说不需要任何必要铺垫即可理解,但对于少数人来说完全不可理解。

除此之外,必要铺垫中的每个认知,都有可能还需要它自己的必要铺垫才能理解和维持。这种递归式的条件可能会带来无法克服的沟通困难,会使我们被迫在多个话题之间切换,并且经常忘了之前讲的是什么。
\end{explain}
如果\indicate{有一种方法,能使一个思考回路在一段时间内维持某个认知(的某个思考途径)的必要铺垫}\footnote{对于\rigorous,“维持必要铺垫”的意思是“维持这个必要铺垫内的所有认知”。如果写全会导致“认知”这个概念复用,有可能使后面的定义出现歧义,所以省略以避免这一点。},那么就称这个思考回路在这段时间内\indicate{可以理解}这个认知,或是有\indicate{理解(这个认知的)能力}\label{def:行为模式的理解能力},或是在这个认知上\indicate{可沟通};将这个过程称为对这个认知/理解途径的\indicate{铺垫},将方法称为这个认知的一个\indicate{铺垫方法}。
\begin{explain}
这里的“在一段时间内”依照具体语境确定,比如说“某次沟通结束前”。

不同于必要铺垫,我们确实可以对一个认知统一地定义理解能力,而不用对每种理解门槛分别定义。但将理解能力视为和理解门槛(以及理解途径)相关的概念会方便后续的分析,所以我们仍然这么做。同时,铺垫方法仍然是和理解门槛(以及理解途径)相关的概念。

而由于我们的定义是对于某个特定思考回路而言的,统一的定义不会使这个定义变成“大家最终总是能互相理解”的鸡汤式论断。“先用老思考回路接收信息,形成一个新的思考回路,然后再使用新思考回路来理解”的操作不算做老思考回路有理解能力。这使得我们可以放心地将转述者一类的行为模式从“可沟通的对象”中排除出去,因为转述者无法接收任何信息,无法形成任何认知。

如果铺垫方法基于沟通另一方的行动,那么可以称对方有\indicate{铺垫能力},不过我们在后文中不会经常用到这个概念。% 铺垫方法不一定要由对方来解释。如“观察”“试验”“实操”等操作,就完全不需要对方的行动。对方的作用只限于“提了一个建议”。

% 有理解能力不意味着在本次沟通中自身就可以理解这个认知,甚至不意味着在本次沟通中自身可以理解必要铺垫中的认知。大多数铺垫方法会超出对方的能力,并且沟通中也不一定触发了铺垫方法,并且沟通双方可能都未意识到“需要理解某个认知”。即使自身具备了理解能力,如果对方无法完整叙述对应的理解途径,那么也无法通过此次沟通而理解这一认知。

定义理解能力的主要目的是区分“因为没有理解能力而导致的无效沟通”和“因为没有使用铺垫方法而导致的无效沟通”。如果我们不从思考回路的层次来研究,而是将人视为一个整体,那么这两者就是不可区分的。拒绝向不可沟通的行为模式沟通传达信息,按铺垫方法和可沟通的行为模式沟通,才是正确的沟通方式。
\end{explain}

\subsection{资源}
如果思考回路A希望思考回路B通过“接触某个/类”事物的方式获得某个/类认知,那么我们就将这个/类事物称为(思考回路A视角下)这个/类认知的一个\source{资源}/\indicate{理解资源},将这一过程称为思考回路A对思考回路B\indicate{使用}了这个资源,将这个/类认知称为A这次使用资源的\indicate{内容}。
\begin{explain}
这里的思考回路A和思考回路B有可能是不同人的思考回路,也可能是同一个思考回路,或是同一个人身上的不同思考回路。

在不会引起歧义的情况下,我们也会使用“这个资源的内容”的表达,请读者自行补全。

理解资源可能有很多种形式,如“课程”“实验”“沟通”“对某理解途径的转述”等。一个资源包含的内容可能不止一个认知(比如课堂或书籍)。对于人创造的理解资源而言,资源的体量越大,一般来说包含的认知也会越多。

资源的内容完全是A的主观认识,“资源能使B获得认知”的认知可能有“A的亲身经历”“别人的经验”“B的亲身经历(此时可以视作A尝试让B回忆起来)”等多种来源。“A使用资源的内容”不一定包含于“A对这份资源的认知的内容”中,因为它没有对适用范围的限制。这里使用“内容”一词,是将“A的希望”视作客观现象,从而其内容就是它本身。相比起B实际因接触而形成的认知,内容既可能过多(B不具有某些认知的理解门槛/A本身理解错误或存在外溢),也可能过少(A对这份资源或是B没有充足认识)。

思考回路B不一定因接触资源而触发对应的理解途径。如果B无法维持所有的前置认知,那么有可能对其毫无感觉,并且有可能触发其它的行动。触发了理解途径后也可能因各种原因中断,无法完整经历理解途径,获得对应的认知。
\end{explain}
我们将“思考回路A对思考回路B\indicate{使用}了这个资源”也称为思考回路A(向B)\indicate{传播}了这个资源,B(从A)\indicate{接受}了这个资源。我们将A和B分别称为这个资源的\indicate{传播者}和\indicate{接受者}。
\begin{explain}
A使用的资源可以是由A制造的,比如“A自己给B讲”、“A写了一份材料”等方式;也可以使用已有的资源,比如“让B去别的地方学”、“用别人做好的教材”等方式。这也使得对于一次接受来说,传播者可能不唯一。

同样,对于一次传播来说,接受者也可能不唯一(比如讲课、视频、写书等),但这种情况通常可以视为多个独立的资源使用过程。
\end{explain}
我们将\indicate{一次接受中,所有传播者的内容中,B实际获得的认知}\footnote{对于\rigorous,这里和定义认知的内容时类似,也是先将所有传播者的内容取并集,然后再与B获得的认知取交集。}称为这次接受资源的\indicate{收获}。
\begin{explain}
一些传播者会在资源中加入一些额外信息(比如明确说明需要什么前置认知),以增加有效沟通的可能。这些额外信息也会成为内容的一部分。

B获得的认知不一定包含于任何传播者的内容中,因此不一定属于我们这里定义的收获。额外获得的那些认知有可能出于B对资源的深入学习和研究(以使B发现了前人没有发现的东西),也有可能出自B结合自身经历的体验(比如“一千个读者就有一千个哈姆雷特”),也有可能出于B自身已有的外溢(这样获得的认知可能没有实际内容)。

内容不一定是具体的认知,也可能是“关于某个特定领域的知识”这种模糊的判断。这让我们允许考察“在自身不掌握对应认知时使用资源”(比如自学)的现象,和其对应的收获。
\end{explain}

\subsection{话题的并行与切换}
如果某个事件会让某个思考回路/人丢失一些认知,那么我们就称这个事件\indicate{打断}或是\indicate{覆盖}了这些认知。
\begin{explain}
定义中的“人”指意识整体。此处的\indicate{打断}将“维持一个认知”视为一个独立的思考回路,与\hyperref[def:打断]{先前}的含义一致。这里的覆盖适当扩充了\hyperref[def:覆盖]{先前}的定义。

大体来说,覆盖分为两种情况:一种是事件本身产生了新认知,然后新认知覆盖了旧认知;另一种是触发了新的行为模式(或是增强对某个已有行为模式的刺激),从而产生行为模式层面的切换。本小节主要关注后面这一种。

能切换行为模式,覆盖认知的事件有很多种,如“自己突然想到了什么事情”、“突然有别人找”、“切换话题”等,不一定需要在沟通中,不一定源于沟通的双方。
\end{explain}
我们将\indicate{从一个信息中解读出不同的认知}的现象称为\indicate{分歧},也称其中一个认知是另一个认知的\indicate{分歧}。
\begin{explain}
在3.1.2小节中提及的\note{歧义}可以视作解读“词语”这种表达时产生的分歧。

分歧是很普遍的现象,它不局限于某次沟通之中。同一个行为模式随着时间的推移可能对某件事产生不同的看法;不同的人可能对同一个东西有很不同的反应和感悟。解读信息的方式和每个行为模式的出发点、认知的关联方式高度相关。

% 因为关联机制和\note{因果性破坏}的普遍性,任何会引起解读分歧的行为,其成因理论上都是不定的,分歧中的任何一种解读都有可能是成因。实际上的成因往往有某些特点,某些特定种类的事物会更容易比其它种类的事物成为成因,但不可因此就草率归因。
\end{explain}
一个思考回路已有的认知外溢会影响我们的解读方式,从而产生分歧。
\begin{explain}
“表达感谢”有可能被认为是正向情感交流,也有可能被认为是“敷衍了事,只停留在口头上”,也有可能被认为是“太有礼貌了让人很膈应”;“提供帮助”有可能让人十分感动,但也有人会认为自己被看不起了;“夸人坚强”时有可能会让人十分受用,也有可能让人想起自己孤立无援,觉得这是在说风凉话......以上的这些解读,在特定的场合下都是合理且唯一正确的解读,但是脱离了相应场合后,就只剩下了“这人咋这么别扭”。

个人过去的亲身经历、文艺作品中的剧情、他人对同一话题的探讨都有可能外溢并影响认知,这使得一个人对信息可能有任意的解读方式。因此,\indicate{任何表达形式都不能保证达成共识}。
\end{explain}
切换思考回路会覆盖一些认知。因此,如果在沟通中切换了思考回路,我们\indicate{应该将不同思考回路的表达视为不同的话题}。
\begin{explain}
这可能是因为“新思考回路不具备关于某现象的认知”,也可能是因为“新旧思考回路之间有分歧”。

如果有一些事件会触发新思考回路,进而覆盖认知,那么我们就也称这个事件覆盖了这些认知。

旧思考回路不一定就此停止,可能在一段时间后还能切换回来,那些认知也会恢复。但处于新思考回路的这段时间内,那些认知确实丢失了。更极端一些的情况下,两个(或更多)思考回路可以快速切换或并行,但是每个表达只出自某一个思考回路。这会使得交流变得非常混乱。
\end{explain}
在一次沟通中新获得的认知更有可能因话题切换而丢失。
\begin{explain}
新认知可能需要更复杂的刺激(当前思路)才能维持,并且这可能是得到新认知的唯一方式。相比之下,旧认知会和思考回路中的更多想法产生关联,更容易被想起,于是更容易维持。
\end{explain}
我们将\indicate{因沟通中的某个表达而触发的,内容不同的话题}称为当前话题的\indicate{子话题}或\indicate{衍生话题},将当前话题称为子话题的\indicate{母话题}或\indicate{原话题},将这种现象称为话题的\indicate{分化}或是\indicate{分散}。
\begin{explain}
我们已经见过很多衍生话题的例子,如“对话题中某一认知的前置认知的讨论”“某个表达触发了转述者的转述行为”等。

虽然对应的内容不同,但这两个话题通常不是无关的,甚至某个话题算不算子话题,需要依据“我们将什么视为内容相同”而定。对前置认知的讨论可以视为原话题的子话题。它们会共用很多东西,比如说讨论对象和用词。但同时,二者可能在词语的指代、侧重点等方面有不可忽视的差别。这使得原话题和子话题中经常存在很多分歧。如果讨论双方对“现在处于哪一个话题”有统一的认识,那么这种分歧不会造成影响。但如果讨论双方并不统一,或是两个话题在混杂着并行讨论,那么这种分歧就有可能造成更大的误解。
\end{explain}

\subsection{讲解}
\begin{explain}
本小节中,我们将\indicate{需要输出信息的思考回路}简称为\indicate{自身}、\indicate{我们}或\indicate{思考回路A},将\indicate{思考回路A觉得需要输入信息的思考回路}简称为\indicate{对方}或\indicate{思考回路B}。同时,本节会研究“自身观察对方以确定对方是否形成了认知”的过程,所以自身也会接受信息。读者可以使用“有意识维持有效沟通的一方”来代替“自身”的定义。
\end{explain}
% 一个认知仅在维持时才能触发其它行为。
% \begin{explain}
% 这是一个很显而易见的事实。如果不拥有这个认知,那么它也就不会产生任何刺激;如果这个认知被覆盖,那么它产生的刺激只会触发覆盖它的那个认知。

% 之所以强调这一点,是因为我们有的时候会混淆相关性和因果性,错误地将某些行为的触发归因于特定的认知。常见的情况包括:
% \begin{itemize}
% %\setlength{\itemsep}{0pt}
% \item 对方同时产生了认知和行为,并且都被我们(可能是这个思考回路本身)观察到。如果没有观察到对方的其它行为,那么就有可能认为是认知导致了行为。
% \item 对方产生了行为,同时我们对对方的思考产生了认知。对方未必真的具有这个认知,不具有的时候当然无法因此而产生行为。
% \item 我们在之前对其它行为模式的观察中,得到了“某认知会触发某行为”的结论,并且在观察到对方具有某行为后,认为对方也具有该认知,并且认知触发了该行为。
% \end{itemize}
% 我们在通过行为反推认知时,一定要谨慎,不要想当然。
% \end{explain}
如果\indicate{有一种方法,能使思考回路A判断思考回路B是否维持着某个认知},那么就将这个过程称为思考回路A\source{鉴别}了(思考回路B的)该认知,称思考回路A\indicate{可以鉴别}这个认知,或是有\indicate{鉴别(这个认知的)能力};将这个方法称为这个认知的一个\indicate{鉴别方法}。
\begin{explain}
鉴别方法和思考回路A有关,对相关领域更熟悉的思考回路A通常拥有更简单的鉴别方法;和认知有关,不同认知的鉴别方法复杂程度不同,有的认知仅需要观察已有行为或是提一个问题,而有的认知则必须通过数天或更长时间的高强度互动才能得出结论;和思考回路B有关,同一个认知在不同的思考回路下可能有不同的表现和叙述方式。

鉴别是一种客观过程。如果A有可能判断错,那么就不将这个方法称为鉴别,或是称为\indicate{无效}的鉴别方法。使用无效的鉴别方法仍然可以判断B是否具有认知,但是判断结果可能不反映客观现实。

由于“B具有某认知”/“B不具有某认知”是A的认知,鉴别过程中一定有从B到A的信息传递(如果没有信息传递,那么可以确定这一定不是一个鉴别过程)。

A在鉴别某个认知时,不需要自己也具有这个认知。但是当A不具有某个认知时,一般很难做到鉴别这个认知,很可能混淆不同认知的特点。
\end{explain}
对于一个认知和它的一个理解途径,如果思考回路A\indicate{可以鉴别思考回路B是否维持着所有前置认知,并且在B维持每一个认知的情况下,可以对B使用对应的资源,触发对应的理解途径},我们就称思考回路A\indicate{可以讲解}这个认知,或是有\indicate{讲解(这个认知的)能力};将鉴别和转述合称为这个认知的一个\indicate{讲解方法},将这两个步骤分别称为\indicate{鉴别环节}和\indicate{教学环节}。
\begin{explain}
讲解方法是一个和认知、理解途径、思考回路A、思考回路B都相关的概念,具体需要注意的点参见前文,此处不再赘述。

讲解方法不是让B获得对应认知的唯一方式,B也可以以其它方法接触资源得到认知。我们这里定义的出发点是“A已经为这个资源做好了所有可以做的准备,尽最大可能提高了成功率”。

鉴别环节是从B到A的信息传递,教学环节是从A到B的信息传递。两个环节不一定有严格的先后顺序,有可能交叉进行。缺失了其中一个环节的行为全部不可视作讲解方法。实际情况下,缺失鉴别步骤要更常见一些,像是“转述者的叙述”“公开的教学资源”等均不是完整的讲解方法(但它们仍然可以视为相应认知的理解资源)。

鉴别环节不一定需要当场进行,也可根据过往的接触来判断。如果讲解方法发现了“B没有维持某些前置认知”的情况,则应该转而使用对应的铺垫方法。铺垫方法里可能包含前置认知的铺垫方法和讲解方法,如果产生无限递归,或是这些方法打断了其它前置认知,则无法讲解成功。
\end{explain}
我们将\indicate{一个理解途径对应的铺垫方法和讲解方法篇幅之和}称为这个理解途径的\indicate{必要篇幅},将\indicate{思考回路A有讲解能力,并且思考回路B有理解能力的所有理解途径的必要篇幅中,最短的那一个}称为A向B传递这一认知的\indicate{必要篇幅}。
\begin{explain}
我们将铺垫方法、讲解方法、鉴别环节、教学环节也分别称作必要篇幅的铺垫环节、讲解环节、鉴别环节、教学环节。

对于同一认知的不同理解途径,“一个理解途径的所有前置认知”、“一个资源”、“讲解所有的前置认知(并将它们维持)”三者的复杂程度一般是此消彼长的,且都和这一认知本身的复杂程度有关。

在培训领域和游戏领域,存在“(某个技能/技巧的)学习成本/认知成本”等概念,这种概念只计算投入的时间/精力,不直接包含其它类型的成本(如经济开销等)。这些概念与必要篇幅类似,接触过这些词语的读者可以用于对比。

虽然这里也定义了统一的必要篇幅,但是其实际意义不大,A不总是能每次都精确选中必要篇幅最短的理解途径来讲解。并且,“最短”是一个较为模糊的概念,根据具体需求可以指代“认知数量”“持续时间”“所需成本”等多种内容。以后的运用中,还是以分理解途径定义的必要篇幅为主。我们在行文中也会省略必要篇幅所对应的理解途径,请读者自行补全。

“篇幅”仅指在一个话题上的不重复认知,而“传递一个前置认知”等行为被我们视为另一个衍生话题,其中如果出现了和原话题或另一个衍生话题重复的认知则需单独重复计算。由于确实会出现“因不太重视/记不住而丢失某些认知,而必须重新交流”的现象,这种规定是合理的。这种事情可能在“B丢失前置认知”“A丢失了关于‘B掌握多少认知”的认知”等情况下发生。

必要篇幅无限长代表无法让B通过这个资源获得该认知。在实际应用中还要更宽松一些,我们经常因为“必要篇幅充分长”(具体有多长根据情况决定,可能是“两句话说不清楚”“一节课讲不明白”“一辈子也听不懂”等)而放弃对当前资源,选择另一资源或是另一理解路径。

真实的交流中,经常会有其它因素干扰,比如“对同一现象的其它讨论”等,由此造成的打断和重新维持的过程不算在必要篇幅之内,由此而形成并维持的前置认知则需要算在必要篇幅之内(由此我们可以看出必要篇幅的“最短”含义)。如果读者有相关的讨论需求,可以将其它话题视为“同时触发的另一个行为模式”,并对新的行为模式B重新定义相关的必要篇幅。
\end{explain}

\section{巧合有效的沟通}
\begin{explain}
本节主要讨论未经过鉴别理解门槛就产生的认知,以及对应的沟通。
\end{explain}
\subsection{认知的巧合形成}
% \begin{explain}
% 本小节沿用上一小节的记号,将\indicate{需要输出信息的思考回路}简称为\indicate{思考回路A},将\indicate{思考回路A觉得需要输入信息的思考回路}简称为{思考回路B}。
% \end{explain}
如果\indicate{思考回路B形成了一个认知,而思考回路A没有鉴别B是否拥有理解门槛},那么就将“B形成认知”这一过程称为(对A来说的)\indicate{巧合}形成,或是“\indicate{巧合}的认知形成”。而如果该认知因必要篇幅而形成,那么就将其称为(对A来说的)\indicate{完整}形成,或是“\indicate{完整}的认知形成”。
\begin{explain}
“巧合”一词的意思是“不因A具有讲解能力而达成”。以下几种情况都算是巧合形成:
\begin{itemize}
\item A完全没有参与这一形成过程,B(关于A)完全独立地获得了该认知。
\item A知道形成过程本身存在,但没有直接参与,如“A让C给B讲”、“A知道B身处某个环境”等。
\item A直接参与,有可能观察到形成过程也有可能没有,有可能自身的行动对认知形成有影响有可能没有,但有其它主要因素。
\item 认知在A和B的沟通中形成,但A没有“让B形成该认知”的意愿。
\item A有意愿让B形成某认知,但B形成了另一认知。
\item A有意愿让B形成该认知,B也确实形成了,但过程中没有鉴别环节。
\item A鉴别了B拥有某理解路径的理解门槛,但B从另一条理解路径获得了这个认知。
\end{itemize}
一次沟通中,可能巧合地形成多个认知,可能巧合地形成多个不同方面的认知。

一个行为模式的巧合认知形成需要对另一个行为模式而言,这是不可省略的。B形成认知时,可能“对A是巧合,对B不是”,可能“对A不是巧合,对B是”,可能“对A和B都是巧合,但是对C不是”,这些差别在后续分析中十分重要。

如果一份资源不面向特定的接受者(如果使用下一小节中的概念,即“不面向某个共识环境”),而是面对基础不一的群体,那么资源的传播者实际上不可能鉴别每个接受者是否具有理解门槛,所有的“接受者获得认知”的现象对传播者来说都是巧合形成的。为了规避这种现象,传播者可以在资源中加入对前置认知的讲解,以方便接受者或另外的传播者鉴别。
\end{explain}
巧合形成是最常见的信息传递方式。
\begin{explain}
这主要有三方面原因:一方面是“完整展开某个认知的必要篇幅,需要消耗的时间精力过高”,一方面是“多数人不具备完整的讲解能力”,一方面是“一些情况下,沟通只是副作用”。

这一事实容易外溢成“所有的信息传递都是巧合形成”的认知。由此产生的“某个巧合形成的沟通过程是有效/无效的沟通方式”也都是劣质的认知。这个认知有很多不同的表述,比如说“只有少数几个人/没有人懂我”、“孩子/老人/客户怎么这么笨/迟钝/固执,教/说了多少遍都不会/不听”等。

这一事实的三方面原因相辅相成。比如,A如果长期缺乏完整的讲解能力,就会在“某个认知很重要,值得花费相应的时间精力给B讲明白”的时候,无法将其讲解清楚,只能选择其它方法替代。这会让A进一步加深“B总是不听我的”的认知,更加没有动力去提升自身的讲解能力。
\end{explain}
巧合形成的认知可以用于其它认知的完整形成。
\begin{explain}
我们将“其它认知”称为目标认知。

鉴别环节中,思考回路A通过B的某些表现,得到“B维持着某个认知”的鉴别认知,这一过程在多数情况下不包含B的鉴别环节(即“B确定A是否维持着鉴别认知的理解门槛”),而仅是“A对B的单方面观察”“A询问B回答”等简单流程。此时,“A形成鉴别认知”对B是巧合。

铺垫环节中,思考回路A需要向B传递某个前置认知,这在一些情况下可以简化为巧合形成,比如“A问B‘你知不知道这个’,然后发现B知道”、“A跟B刚讲了一个开头,B就想起来了”之类的情况。此时,“B形成前置认知”对A是巧合。

如果以上两环节中的所有认知都必须要完整形成,且不论这些完整形成的篇幅中还会引入新的认知,仅考虑这些篇幅本身,就通常是完全不可行的操作了。从外部因素来看,这会大量消耗时间精力,绝大多数情况下不值得这么做,并且也很容易被其它行动打断;从内部因素来看,这会引入多个衍生话题,从而使得某些前置认知无法维持,使得B不具备理解能力,反而不利于目标认知的形成。
\end{explain}

\subsection{共识环境}
我们将一个/一组认知/行为定义为一个\indicate{共识环境}。如果\indicate{在某环境中的所有行为模式都能维持这个/这些认知},我们也将这一环境称为这一个/一组认知的一个\source{共识环境},并且将这个/这组认知/行为称为该环境\indicate{维持}的一个/一组\indicate{共识}。
\begin{explain}
由于我们可以任意地定义环境,我们也可以(几乎)任意地定义共识环境:
\begin{itemize}
\item 可以是“一种语言/一种文化”。在本指南内,我们默认这是所有行为模式的共识环境。
\item 可以是“一门学科”。对此的具体讨论参见\hyperref[sec:知识与信息]{2.1.1小节}。
\item 可以是“某个现实场所/场景”。一些直接的所见所闻,比如说地理位置、人员构成、重要/规律性事件等信息,可以看做是该环境中的共识。但需要注意的是,“某一行为模式对这一环境的认识”不都属于共识,不同行为模式的侧重点和盲点经常有区别。
\item 可以是“某种特定的措辞/语境”。在一个稳定的讨论环境中,经常会出现这种所有人都熟悉的表达方式。这种表达方式的成因不固定,概括来说,由一些有意或无意的指代而逐渐形成。
\item 可以是“某几个行为模式及其相互接触”。这可以规避“有新行为模式进入环境后,该环境丢失了一些共识”的现象。这种定义下,上述情景可以被描述为“新行为模式和这个环境接触”。这在研究一些小圈子时很方便,此时小圈子经常既是触发环境又是共识环境。
\item 可以是“(某环境下)所有维持某个/某组认知的行为模式及其相互接触”。这种定义看起来完全是废话,但它是一种很方便的理论分析工具,以至于它经常外溢——很多人会有“处于同一环境中的所有行为模式维持着对应的共识”的认知。这会为交流带来一些麻烦。
\item 可以是“某次沟通”,此时参与的所有行为模式都在从这次沟通中获得认知。不过这些新获得的认知不一定是共识,有些行为模式可能不重视一些地方而没有获得,有些行为模式可能被覆盖或形成了别的认知。在使用这种定义时,一定要谨慎,不应草率地将某个行为模式的认知当成所有行为模式的共识。
\item 可以是“一个行为模式”,此时环境中的所有共识就是这个行为模式能维持的所有认知。不过这种环境的讨论限制较大,我们将其并入下一条一起讨论。
\item 可以是“一个人的意识活动”,此时环境中的所有共识是这个人同时触发的所有行为模式维持的所有认知。这是一种很重要的环境,因为其他人无法直接接触这一环境,会少获得很多信息。我们将其称为一个人的\indicate{意识环境}或\indicate{脑内环境}。
\end{itemize}
对前一种定义来说,“共识环境中的行为模式”定义为“维持这些认知/行为的行为模式”。我们在后文中不会明确指出使用的定义是哪一种,但读者应该总是能自行分辨。

共识环境不一定就是这些行为模式的触发环境。不过不同的行为模式之间越像(可能除了“属于不同的个人”以外没有其它区别),触发环境也就越像,相互之间的共识也就越多,触发环境就越有可能也是共识环境。
\end{explain}
我们将\indicate{一个人(因获得了这些共识而)从不属于这一共识环境变为属于这一共识环境}称为这个人\indicate{进入}了这个共识环境,将\indicate{一个人(因丢失了这些共识而)从不属于这一共识环境变为属于这一共识环境}称为这个人\indicate{脱离}了这个共识环境。
\begin{explain}
这里的术语和\note{行为模式}处的术语一致。

关于共识环境的讨论其实和对“很多行为模式”的讨论没有本质差别,在一些情况下,我们也可以将共识环境内的所有行为模式视为同一个行为模式,并且可以据此定义每个共识环境中的集体意识、集体潜意识、立场(也即出于这一共识环境的\note{价值判断})等概念。此处提出这个概念,主要是因为“环境”容易被我们(有意识或无意识地)当成一种独立的对象,判断“某个环境是不是共识环境”“某个行为模式是否属于某个共识环境”会比较方便。

只要一个环境会被人用来思考“属于这个环境的人都有什么特征,会去做什么”,我们就将其称为“将环境视为共识环境使用”,并且在这个意义上将其视为共识环境。由此,我们将“具有一些共同特征的集体”也视为共识环境。这么思考的人可能不清楚共识环境的概念,但是使用“某种身份”等措辞也会起到同样的效果。
\end{explain}
对于一个共识环境和一个认知的理解途径,如果有一种方法,能够使共识环境中的每一个行为模式都维持这个认知的所有前置认知,并且共识环境可以理解每个前置认知,那么就称共识环境\indicate{可以理解}这个认知,或是有\indicate{理解(这个认知的)能力}\label{def:共识环境的理解能力},或是在这个认知上\indicate{可沟通}。
\begin{explain}
这里的定义几乎是照搬“思考回路的理解能力”。该定义实际要求“共识环境中的所有行为模式使用同样的理解途径”,定义中出现的“理解前置认知”的嵌套也是出于此要求\footnote{这个定义不会导致“因为定义得过于严格,递归条件过强,导致共识环境没有任何理解能力”的情况。一个人在初次接触共识环境时,当然会基于自身情况产生认知。不同人的自身情况不同,这一阶段不会有共识。但随着接触逐渐加深,一些重要认知会显著更为熟练,最终成为不需要任何理解路径的条件反射。此时我们就可以以这些认知为起始点来递归定义理解能力了。}。

对于一个共识环境可以理解的认知,有一个行为模式使用某个资源理解了这一认知,只要对应的理解途径不是太难懂,那么这一环境中所有的行为模式都能使用同样的资源理解这一认知。这一现象很容易让这些行为模式产生“只要这么讲就能给人讲懂”的认知,将其视为有效的信息传递方式——但它仅在共识环境中保证有效。

共识环境中的行为模式也可以从资源中理解出其它认知,不过我们应该将其视为另一件事。如果这个行为模式的新认知覆盖了一部分共识,那么就应视为其已经脱离了这一共识环境。

% 如单独的行为模式一样,共识环境也有可能从不同的理解途径中获得同一认知,这使得“某一共识的一种理解途径”有可能不是共识的一部分,不同的行为模式会使用不同的理解途径。这有可能是源于对同一理解途径做了不同程度的简化/补充,也有可能是本身理解途径就不同。
\end{explain}
不属于共识环境中的行为模式接触资源时,不保证能得到相同的认知。
\begin{explain}
单看这一句话像是废话。在实际应用中,这个结论最容易出问题的地方是它的前提条件:我们在给资源的时候,经常不去区分行为模式是否属于共识环境。主要的疏忽可以归结为两种:
\begin{itemize}
\item 没有发现对应的共识环境,没有意识到资源需要在相应的共识环境内使用,将某一共识环境中有效的信息传递方式视为无条件下传递这个信息的有效方式。我们将“将某种其它类型的环境误认为共识环境,并让环境中的行为模式接触资源”也算进这一类中。
\item 发现了对应的共识环境,但将某一不属于该共识环境的行为模式误认为属于该共识环境,并使其接触对应的资源。这种情况例子很多,如“学习成绩差就是因为不专心听讲/做作业”“我说过的话别人就要听”等。
\end{itemize}
无论是哪种情况,一个行为模式A直接认为“某资源对B有效”,并让共识环境外的行为模式B接触资源,B如果形成对应认知的过程对A一定是巧合的。

在一些情况下,我们可以将共识环境视为资源/讲解方法的适用范围。共识环境一般比真正的适用范围(有理解能力的所有行为模式)要小,但“判断一个行为模式是否属于某个共识环境”在很多时候要比“判断一个行为模式是否能维持全部前置认知”要容易,要更为实用。
\end{explain}
我们可以得到一个\indicate{不可能三角}:不存在对\indicate{所有人}都\indicate{有效}的\indicate{简短}讲解方法。
\begin{explain}
如果不面向所有人,只在共识环境中讲解,那可以做到简短而有效;如果不求听众一定能获得该认知,那可以使用非常精炼高深的讲法;如果不求在短时间内讲完,那可以根据每个人已有的前置认知来因材施教。

在实际使用中,我们经常仅关注“在共识环境中的有效讲解”,并且从中得到认知和行为,比如“听不懂就是没认真听,态度不端正”等。但如果没意识到相应的共识环境,这就只是巧合的认知形成,这一方法本身也会在共识环境外失效,不是正确可推广的认知。
\end{explain}

\subsection{巧合行动}
对一个共识环境和一个不属于它的行为模式,如果它们会因同一个刺激而触发同样的\note{行动},那么就将这种现象称为行为模式关于共识环境的/行为模式和共识环境之间的\indicate{巧合行动}或是\indicate{巧合相同}的行动。
\begin{explain}
在实际使用时,我们经常先选定一个行为模式A和一个和A相关(可能是包含A,可能是A能模拟)的共识环境,讨论这一共识环境和另一行为模式B之间的巧合行动。在后文中,为了简便,我们会在不引起歧义时省略共识环境,而是用“两个行为模式之间的巧合行动”的措辞,请读者自行补齐。

两个行为模式的巧合行动一定是由不同行为导致的。虽然它们有同样的刺激和同样的结果,但触发的行为链不一样。定义中包含共识环境,就是为了刻画出两行为模式中,触发的行为链的差别。如果我们不关注具体的行为链,只关注刺激和行动,那么这就可以视作同一个行为。

认知的巧合形成不都可以视作巧合行动,有可能恰巧使用同一份资源触发了同一种理解途径获得了对应的认知。这里复用“巧合”一词,是因为巧合行动是巧合认知形成的重要原因。
\end{explain}
“A因为发现了B有巧合行动,就认为B属于共识环境”的现象是一种认知外溢。
\begin{explain}
单独的“B属于共识环境”的整体认知可能没什么影响。我们这里关注的是“认为B拥有共识环境中的行为和理解能力”的这种更具体的,精确到行为和认知层面的认知。这才是真正影响沟通的现象。为了叙述方便,我们暂且将“巧合行动”和“共识环境中的其它共识”分别称为“标志行动”和“其它行为”(这么命名是因为A会根据标志行动来识别共识环境)。

同一个客观存在的共识环境,会对A造成很多种不同的影响:
\begin{itemize}
\item A明确认识到并理解共识环境,并且有有效的方法判断B是否属于共识环境;
\item A错误地认为B属于某个共识环境(这一步可能以很多不同的形式实现,比如“认为B具有某种身份”),并且按照共识环境认为B拥有其它行为;
\item A从共识环境中得到了“有标志行动就有其它行为”的认知(此时A大都对共识环境没有清楚认识,会将这一认知视为普遍规律),并且根据这一认知认为B有其它行为;
\item A自身属于共识环境,并且自身会因标志行动而触发其它行为,并且将“拥有其它行为”的认知外溢到了B身上。
\end{itemize}
后三种情况都会造成“A认为B拥有其它行为”的结果。事实上,A在没有明确认识到这一共识环境时,因为缺少了验证认知的适用范围的步骤,反而更容易产生“B拥有其它行为”的外溢。
\end{explain}
如果B被误认属于的共识环境包含某个认知,那么这就会成为A关于此认知的一次无效的鉴别。
\begin{explain}
我们暂时将这一认知称为“目标认知”。会被误认的巧合行动包括但不限于以下这些:
\begin{itemize}
\item 触发行为的刺激实际是“在和对方沟通”的现象,而不是“对方话中的内容”。这可能包括“积极地响应和附和(但附和中不包含信息)”;“被问‘知不知道/听没听懂’时,不根据自身情况判断,而是直接回答‘知道/听懂了’”等情况。
\item 目标认知会触发一些行为,但B通过另一种不包含目标认知的行为链巧合触发了同一行为。这可能包括“A能确定B有某个认知,于是觉得B有某个前置认知”、“A发现B表达了该认知,实际上B只是在转述别人的观点”等情况。
\item B在某些环境下可以维持这一认知,但在另一些环境下不可以。环境可能包括“某个话题中”、“另一个共识环境”、“某种特定的措辞”等。
\end{itemize}
在这些鉴别方法失效时,有一些更细致的方法可以用来进一步验证B的情况,比如“让B自己完整地叙述”,但具体什么方法有效,还需要根据实际情况判断。

如果A在讲解时产生了这样的误认,那么鉴别环节就会失效,对应的认知也就成为了对A巧合的认知。
\end{explain}

\section{解读\label{sec:解读}}
按照\hyperref[def:解读]{4.0节中的定义},如果某现象作为刺激,触发了思考回路,并形成了认知,那么我们就将\indicate{由此形成的认知}称为对这一现象(信息)的\indicate{解读},或是从这一现象中\indicate{获得}的\indicate{解读/认知/信息}。
\begin{explain}
我们有可能从同一现象中获得很多种不同种类的解读:
\begin{itemize}
\item 解读可以是对现象本身的\note{特征提取}。提取的特征有可能被用于继续思考这一现象,产生其它解读。提取出的特点不一定正确,不一定全面,特点相关的认知也不一定适用。
\item 解读可以是对现象本身的\note{理解}。我们需要找到导致这一现象的过程(自身找到的过程不保证是真实的过程)。如果现象是人的行动,那么对现象的理解就变成了寻找这一行为的动机。对方有可能是有意为之,也有可能只是没多想,无意识地触发了某个行为。这一理解过程很容易产生认知外溢,给别人安上不真实的动机。
\item 如果现象具有\note{内容}(比如说“对某认知的转述”,或是“接受某种资源”),那么解读可以是对其内容的理解。这和上一条有不同之处,比如对于“老师讲课”这个现象来说,对现象本身的理解可能是“老师想教会学生”,对其内容的理解则是“知识是什么内容”。对具有内容的现象的解读总是会同时有这两种理解方向,从而会产生天然的分歧。
\item 解读可以是对现象本身的\note{预料}。这样的解读会影响我们对待这一现象的方式,指引我们的后续行动。预料不一定准确,后续行动也不一定是恰当的,具体讨论参照前文。
\item 解读可以是与该现象有关联的其它认知。这些认知(或是它们的内容)可能与该现象具有实际的关联,也有可能这种关联完全是由外溢产生(此时会产生\note{因果性破坏})。
\end{itemize}
在本节中,我们主要关注“理解现象本身”和“理解现象的内容”两方面。
\end{explain}

\subsection{对资源的解读}
\begin{explain}
本小节考察“思考回路A接触资源时,是否能获得某个/组特定认知”的情况。我们将这个/组认知称为思考回路A接触资源的\indicate{目标认知}\footnote{注意思考回路A不一定有要获取认知的主动性(所以目标认知不一定是A接触资源的内容)。目标认知也可能是其它思考回路的目标,或者是任何思考回路都没有目标,我们仅考察“A是否获得了认知”的客观事实。},并只关注“有助于获得该认知”的解读。如果还有其它的目标认知,那么我们将其视为另外的资源使用过程。

本小节主要关注对资源的内容的解读。如果一个资源本身有方便/不方便解读内容的地方,我们也会因此而解读资源本身的创造和设计。除此之外的部分不在本小节讨论范围之内。
\end{explain}
如果A可以在某些条件下接触同一份资源直到获得目标认知,那么就称这份资源在这些条件下对A是\indicate{稳定}的或是\indicate{可以重复使用}的。如果没有A可以达成的条件,那么就称这份资源对A是\indicate{不稳定}的。
\begin{explain}
根据资源的类型不同,这里的“某些条件”可能是“另一个人在场/有空(可以讲解)”、“自己在场/有空(听别人讲/自己练习)”、“有网(可以看公开课/下载资源)”、“付钱(各种类型的课程)”、“带在身边(书籍、电脑等资源)”、“场地和设备(实验或是收集数据)”等。

A无法获取的资源对A当然是不稳定的。除此之外,另一种常见情况是“A只有一次(少数几次)机会接触资源”,如课堂或是演讲。若A无法在这几次机会中稳定获得认知,这种资源对A就是不稳定的。我们也有将不稳定资源转化为稳定资源的方法,比如做笔记或者录像;而“找别的课/资料学习”应当被视为另一种独立的资源。

A即使一直无法维持所有的前置认知,一直无法获得目标认知,但如果可以一直接触资源,那么资源也算是对A稳定的。

资源应当能够(在满足理解门槛时)触发目标认知的某一理解途径(如果理解途径也属于目标认知,那么理解途径也应固定;否则可以在不同的接触时触发不同的理解途径)。如“虽然每天都在讲课,但每天讲的都是新内容”的情况不能被算作稳定资源;“定期复习”可以算作稳定资源(但它可能有其它问题,比如在A能够维持认知的时候复习,就会浪费时间)。

稳定的资源最大的优势在于可以多次使用,并且每次使用时获得不同的认知。这使得我们可以有“第一遍先学个大概,知道整体方向,第二遍再去关注细节”、“遇到不会的/不熟的先停下来,去找别的资源/往前翻/练习/复习到掌握后再继续”之类的方法,这可以在很大程度上保证我们稳定地获得目标认知。
\end{explain}
一部分学习方法实际上是在制作(稳定的)资源而非使用资源。
\begin{explain}
制作资源有很多种常见的方式,比如:
\begin{itemize}
\item\indicate{复制},如“背诵/记忆”、“抄写/拍照/复印/录制”、“笔记/日记/随笔”等。这些方式要不然直接保留了资源本身,要不然保留了资源的内容。
\item\indicate{整理},如“制作表格”、“编写教材”等。这些方式保留了资源的内容,但是更换了理解途径,使其更适配于某些使用者(比如“自己”“初学者”等)。
\item\indicate{索引},如“制作档案”、“大概留一个印象”、“知道怎么找专业人士”等。这些方式不直接保留认知的内容,但也提供了接触资源的途径。
\item\indicate{训练},如“刷题”、“按流程实操”等。这些方式让人可以维持更多的前置认知,以便“把思路顺下来”,达到其它更深入认知的理解门槛。
\end{itemize}
一些制作资源的方法同时也有获得认知的效果,比如说“训练同样可以巩固认知,查漏补缺”、“背诵本身一定记住了对应的内容”、“整理时或许可以促进整体理解”等。这些资源可以起到“辅助获取目标认知”的效果,所以获得的这些认知可以视为前置认知。但“通过制作资源从而获得目标认知”不是可以稳定成功的方法。“能否获取目标认知”由“是否达到了理解门槛”决定,制作资源只是一种巧合的方法。

同时,还有另一种“将制作资源获得的前置认知就当做教育的目标”的思路,比如“走入社会才知道语文书上写的是什么”“为人父母才知道父母的不易和爱”之类。虽然客观上确实有一些“缺乏社会经验/生活经历(对应的前置认知)从而无法理解”的障碍,但因此就将其视为教育所能做到的极限,是舍本逐末的行为。事实上,大多数这样的行为中,这样制作的资源对理解目标认知没有任何作用,当事人完全是根据另外的理解途径(比如自身经历)来获得目标认知的。将其作为教育的核心传承下去反而会起到反效果,在不相互理解的情况下强行相互包容,掩盖了真实存在的矛盾。具体讨论请见第六章。
\end{explain}
A接触资源时,如果对于目标认知的每个前置认知,都要不然可以获得前置认知本身,要不然可以获得前置认知的鉴别方法,那么我们就称这个资源是A获取目标认知的一份\indicate{铺垫资源},或是这份资源对A提供了\indicate{够用}/\indicate{充足}/\indicate{充分}的铺垫。
\begin{explain}
定义中的前置认知与理解途径相关,为了使叙述不过于臃肿,省略了这一点。

定义中的“获得”只关注“在接触资源后拥有”,而不关注“接触资源之前是否拥有”。这里定义的“充足”在语感上接近于“A接触资源后可以自行补足前置认知”。

为了方便后续讨论,我们一般在讨论“资源是否充分”时,会假设该资源对A是稳定的,以排除“因接触时间过短而产生的解读失败”情况。

我们也可以分别定义“资源可以让A获得所有前置认知”或是“资源可以让A获得所有前置认知的鉴别方法”,但这两种情况都比较少见到,最常见的还是二者的混合。我们既会经常遇到“有小细节看不懂得去查一查”的情况,需要用到其它的资源;也不太需要每一步都脚踏实地地鉴别,经常是遇到问题再去修正。

如果一份资源是由人精心制作的,质量很高,那么它通常既包含前置认知本身,又包含其鉴别方法。同时,资源中还可能包括多种不同理解途径的前置认知。当然,如果前置认知本身不属于创作者所关注的重点(这可能是进阶的材料),那么这份资源可能无法用于获取前置认知。我们还需要使用其它资源(一些更基础的内容)来获取。

但大多数资源还是需要A自己来分辨前置认知是什么。对于复杂的自然现象,需要事先掌握某些基础的自然现象;对于一段完整的工序,需要事先掌握每个单独的步骤;对于某段论述,需要事先掌握其中的用词指代和逻辑......只有当A自身的调研能力能够满足“获取这份资源的前置认知”的需求时,这份资源对A才是铺垫充足的。而调研能力不强的人就无法从资源中获取目标认知,如果强行接触资源,可能会获取其它认知。
\end{explain}
如果A获得的认知属于该资源的内容,那么就称A获取了(相对于该内容的传播者)\source{准确}的认知。
\begin{explain}
资源的内容是主观的,依据传播者的意图而定。一种常见且符合直觉的情况是,资源是由人创造的,或包含某些特定的知识。这种情况下A只有获取这些知识时,才是准确的。

另一种情况是“A自身作为传播者,要对自己使用某个资源”。这种情况下,资源的内容由A任意指定。
\begin{itemize}
\item 如果A有“获得任何认知都行”的想法,那么任何认知对A都是准确的;
\item 如果A完全没有此类想法,只是巧合地获得了认知,那么任何认知对A都是不准确的;
\item 如果A有“要得到准确的理解,获得真实的原因”之类的意图,那么只有符合实际情况的认知才是准确的。
\end{itemize}
如果没有特别说明,后文的篇幅总是认为A在对自己使用资源时,有“要得到真实的理解”的目标,并且据此来判断准确。
\end{explain}

\subsection{对人的解读}
\begin{explain}
本小节和下一小节考察“思考回路A接触行为模式B时,对人B的某行为/行为模式的理解”的情况。我们将B拥有的行为/行为模式称为思考回路A的\indicate{目标行为/行为模式},并只关注“有助于获得对该行为/行为模式的理解”的解读。B身上如果还有其它行为/行为模式,那么我们将其视为另外的解读过程。

本小节和下一小节主要关注行为/行为模式的形成和触发,不直接包含对行为/行为模式内容的解读。如果其内容有助于理解行为/行为模式的形成和触发,那么我们也会关注。除此之外的部分不在本小节和下一小节的讨论范围之内。
    
如果使用上一小节的概念,我们这里将行为/行为模式本身(或是其转述)视为“内容是行为/行为模式的资源”,并且A接触该资源的目标认知是“行为/行为模式的形成和触发”。本小节主要讨论稳定性,而下一小节主要讨论充足铺垫。
\end{explain}
如果目标行为/行为模式对A是解读的稳定资源,那么就将它们称为对A\indicate{稳定}/\source{可见}的行为/行为模式。
\begin{explain}
对于不同的行为/行为模式,稳定需要满足的条件也不同。如果是动作为主的,那么条件可能是“A在场旁观”、“录像”等;如果是口头表达为主的,那么“录音”也算;如果是技巧或是想法为主的,那么就需要B通过转述来表达自己的思路,此时可能需要A有一些询问技巧(这是直接对人的调研能力)。在接下来的篇幅中,为了叙述方便,在不引起歧义的情况下,我们总是忽略如“录像”“录音”等需要额外信息载体的间接解读方式。这些间接方式也符合相关论述,请读者自行补全。

并非B的所有行为和行为模式都对A可见。对于每一种特定的刺激,B在此刺激下确实能触发某些行为/行为模式(也有可能完全不触发),但A不一定能观察到所有的刺激类型。具体原因可以分为以下几类:
\begin{itemize}
\item A没有信息输入,即A没有观测到某一类刺激/行为的条件。比如“A不在场”“A缺乏前置知识”等;
\item B没有信息输出,比如“刺激和行动中间有由想法组成的行为链,A无法观测”“B的表达能力不足,没法说清楚自己的情况”等;
\item A产生了竞争和覆盖,即A对B的行为解读被自身的外溢认知干扰。这类外溢认知包括但不限于“我很了解B”“我很了解这个领域”“这个现象就是这个原因/就会这么发展”的情况。
\item B产生了竞争和覆盖,即A自身的行动(或是间接的信息记录)会更改刺激类型。“旁观”“询问”“指挥”“记录”“重复(比如总是教不会)”等行为有可能触发B的其它行为/行为模式,导致原本需要接触的行为/行为模式被覆盖。
\end{itemize}
如果A不具有良好的询问和分析能力,那么通常只能接触B的有限几个行为模式,也只有这么几个稳定的行为模式可以解读。
\end{explain}
\indicate{将对行为模式的解读视为对人的解读,是一种外溢。}
\begin{explain}
这是本指南一直坚持并多次强调的观点。

在接触环境固定的情况下,“将行为模式视为人”这一行为本身不会出现什么问题(这里不计“A对行为模式的解读有误/不足”的情况)。我们同样能对“这个人在这种环境下的所有行动”得到完全的理解,同样能充分地预测。本指南中也多次将行为模式称为人,以顺应读者的习惯性思路。

将行为模式视为人,最重要的问题是无法分析竞争过程。当A预测B在另一个环境中的行动时,如果只知道B的一个(或少量)行为模式,就只会在这些行为模式之中分析B最可能的行为(这个分析过程不一定是有意识的思考,也可能是无意识的联想)。这种对竞争过程的推断很可能不符合现实,从而会得出错误的结论。
\end{explain}

\subsection{对竞争过程的解读\label{sec:对竞争过程的解读}}
对于某个行动,我们将所有参与\note{竞争}的行为/认知称为这个竞争的\indicate{范围}或是这个行动的\indicate{竞争范围},将范围中的行为称为这次竞争/这次行动的\indicate{潜在行为}。如果范围有一些共同的特点,就将其称为此次竞争的一个\indicate{限制}。我们也称这个竞争是“在这个\indicate{范围}中的竞争”或是“\indicate{限制}在这个特点/这些行为/认知中的竞争”。
\begin{explain}
此处为了方便讨论,我们将没有参与竞争而直接触发的行为也视为参与了竞争,并且该竞争的范围中只有这一个行为。同时,我们仍然沿用“没有参与竞争”的表达。

限制可以采用一些比较宽泛的特点,比如说“符合格式和韵律”、“用不超过四个字指代一种事物”、“某些特定的符号和表达”等,都可以作为某些竞争的限制。在电影、舞蹈、绘画、音乐、游戏、诗歌等一些特定的文化领域,还会在特定的限制上发展出“服化道”、“镜头语言”、“肢体语言”、“空间构图”、“和弦走向”、“技能叙事”、“隐喻比兴”等多种不同的设计语言(解读时也可能出现“有很多种备选的意思,无法确定”“怎么解释好像都有道理”“过度解读”等情况,此处不展开讨论)。

以上这些例子中,竞争过程都是有意的设计,都是从某些特定的行动中,挑选出最能符合自己所想表达的内容的那一个。但这个过程同样也可以由“只是学到了这种特定的表达”的方式产生,此时它们来源于不同的竞争过程(后者竞争的范围中只有一个行为),不应视为同一种竞争过程,不应因“使用了高深的表达”就认为“有复杂的思想”,也不应因“使用了高深的表达”就认为“是在当名词党”。针对竞争过程的评价必须依据具体的竞争过程来判断。

一个人意识活动中的竞争有一个通用的限制:它只能在这个人已有的行为/行为模式中产生。同时,不同于上述例子中有意的设计,意识活动中大多数竞争过程是不自知的,特别是在“触发了某个范围很广的行为模式,以至于很多其它行为被覆盖”的情况下。只有在少数关键决策时,我们才能发现“自己在面对一个竞争过程”,但即使不计算只有单一行为参与的竞争过程,实际发生的竞争过程数量也要远多于我们能察觉的。
\end{explain}
理解一个竞争过程,需要的前置信息是“这一竞争的范围”和“范围中行为之间的覆盖关系”。我们将其称为这一行为的\indicate{逻辑}或简称为\indicate{行为逻辑}。
\begin{explain}
理解一个行动需要找到它的成因,而它的成因总可以视为一个竞争过程。因此,这也是理解一个行动(进而行为/行为模式/人)的前置信息。

在A试图理解B的行动时,这些前置信息不都是对A可见的。A有可能无法直接观测到一些行为,只能依赖“B的转述”等一些间接方法。甚至B在分析自己的行为时,也经常会遇到自身无法察觉的行为/想法。为此,我们可以将条件做一些弱化:如果某些行为总是会被覆盖,那么我们可以将其视为不参与竞争。我们只去观察那些有明显表现的行为即可。

有明显表现的行为有可能看起来和触发条件没有直接关系,或者是容易被视为性格或类似因素(如紧张、恐惧、自负、小动作等)。这些行为同样有对应的竞争过程。如果我们的目标是控制原行动,但理解了其竞争过程无助于控制,那么我们进一步对这些行为展开分析(或许还能产生递归的分析)可能会有所帮助。但仍需注意,原行动的竞争过程和“潜在行为的形成”的竞争过程是两个独立的竞争过程。即使理解后者对理解前者经常有帮助,对后者的理解也不可替代对前者的理解,同时对后者的理解也不属于对前者理解的一部分,它们是不同的认知。

熟悉博弈论、微观经济学或其它相关领域的读者可能会尝试在此引入专业的分析,通过相应学科的语言来分析出占优策略。这确实是可行的方式,这套分析框架理论上可以用于任何竞争、博弈和决策过程。上述的两个要求,在相应术语下变为“需要将哪些策略纳入考虑,将哪些策略排除”和“这些策略之间到底谁占优”。在分析时不一定能引入统一的策略收益(即使我们已经认为价值是主观的)有可能会出现“三种策略相互占优”,如剪刀石头布一样之类的情况;也不能认为不同的场合下(即使环境基本上没变,只是时间不同),同一个人的策略集一定不变。我们可以将这两种缺陷视为“人的非理性”的具体刻画。对于非理性的研究在对应学科已有论述,本指南在这里不展开。
\end{explain}
\indicate{对于任何行动,我们在尝试理解时,总是需要先得知它的竞争过程,再根据竞争过程来模拟。}任何不经过此步骤的理解都最多只能得到(对自己而言)巧合正确的认识。
\begin{explain}
最终的分析结果有可能是“仅有一种行为参与竞争,并且完全下意识地想做就做了”,有可能是“在两难中踌躇徘徊”,有可能是“经过了深思熟虑的决策”,所有的这些都有可能通过“猜测”“预感”“共情”“默契”“共鸣”等对不同人采信度不同的,不包含分析竞争过程的其它方式得到,它们也可能在某些共识环境下确实正确,但不可视为这种行动的统一原因。

对于结果相同的竞争,不同的竞争范围会产生不同的覆盖关系和不同的覆盖方法。其中一个比较好理解的例子是:因理解途径和必要篇幅不同,维持一个前置认知所使用的方法也不同。当我们需要排除某些行为(经常是负面且不太可控的,比如说偷懒、畏惧、片面、冲动)时,我们就需要用很多理由来说服自己。但如果另一个人没有这些缺点,那么就不需要克服这些缺点,自然也不会有意培养一个说服的行为来覆盖。而有时却恰好相反,我们使用的说服方法能说服我们自己,却无法说服对方。这两种情况下,“要求对方培养相应的说服能力”的行为不仅没有必要,而且浪费时间,并且分散话题,同时对方也可能无法理解,但它们的成因截然不同,解决方法也截然相反。

% 另一种例子是,同一种行为可以有很多种不同的特点,其中每种特点都可以作为决策的动机。有些特点十分宏观和抽象,比如“维护自身团体的利益”“象征着某种被压抑的需求”;有些特点只针对具体情况,比如“这么做就可以实现目标”“也不知道为什么,反正我就是喜欢这样”(这些可能是外溢行为/认知,但此处不关注来源)。

另一种类似例子是,“根本不存在某个行为”和“存在某个行为但是被覆盖”这两种情况会产生同样的影响,都会使我们产生“另一个人和我的做法不一样”的认知。我们在感觉上更容易将前者视作“无意的”,而把后者视作“有意的”,并且将覆盖视作主动决策。在一些语境下,我们会将前者称为“蠢”,而将后者称为“坏”。两种方向的误读均有可能发生:我们既可能在另一个人不存在行为(比如完全不知情,有另一套行为逻辑)的时候,认为对方存心对着干;又有可能在另一个人有明确认识和动机的情况下,用“还小”“不熟”“粗心”等方式理解。这些解读都既有可能对又有可能错。如果想教育对方,那么前者应该培养这一行为,后者应该消除覆盖的行为。方法错配就会起到反效果:对前者讲解“如何消除”会不知所云,而对后者讲解“应该怎么做”会加重抗拒的强度。
\end{explain}

\section{指挥\label{sec:指挥}}
\begin{explain}
虽然看起来不是很像,但是本节中的行为模式A、行为模式B、旁观者可以指代同一个人身上的不同行为模式,理论分析全部通用。我们可以用本节的理论来分析和处理“眼高手低”“半途而废”“无法自控地走神”“突然心就乱了”“知道了很多道理但仍然过不好这一生”等情况。自我指挥和解读的部分在下文中不会单独讨论,读者可以自行套用。
\end{explain}
\subsection{指挥}
我们将\indicate{行为模式A触发行为模式B的某个行动}的过程称为行为模式A\indicate{指挥}行为模式B(做这个行动)。如果可以确定具体的刺激(比如A的某个行为),我们也称A使用这一刺激\indicate{指挥}B,或是简称为这一刺激\indicate{指挥}B。
\begin{explain}
这里的指挥和第二章中定义的\hyperref[def:控制人]{控制}的适用范围不同。控制针对一个人具有的行为/行为模式,而指挥所针对的行动不一定对应着某个实际存在的行为/行为模式。多次指挥同一种行动,可能每次的刺激和行为链都不相同。

这里采用这种定义,是因为在很多情况下,我们不关心行动的刺激和成因,需要的仅是行动本身或其影响,如“完成某项任务”“获取某个认知”等。

\indicate{指挥是一种客观现象},不需要带有A的主观意图。因此,它的指代范围比日常语境下的指挥要更大,包含了很多A或B没有意识到的行动(很多情况下A和B双方都没有意识到)。它在这里的实际语义比较贴合“所有可能被认为是A指挥B的现象”。此处对指挥的定义不以“A和B是否有主观认识”来区分,仅关注二者的触发,以便建立更普遍的理论。和主观认识有关的部分在稍后单独定义并分析。

A指挥B不一定需要A行动,“A在场”等现象本身也可以提供刺激(比如说“老师维持纪律”“粉丝见面”“嫉妒别人”“见前恋人尴尬”等)。

A和B都分别可能对这类指挥现象有从一无所知到一手策划的不同程度的认识,有从积极迎合到无关紧要再到饱受其苦的不同程度的感受。本节内我们不关注这些主观评价。
\end{explain}
如果某个过程\indicate{需要B的某个行动才能进行},那么我们称这个过程具有指挥B这一行动的\indicate{需求},或简称为(对B的)\indicate{指挥需求};如果A\indicate{认为B需要执行某个行动},那么我们就称A有指挥B这一行动的\indicate{意图},或简称为(对B的)\indicate{指挥意图}。
\begin{explain}
\indicate{指挥需求是一种客观现象},完全根据过程的性质而定。过程可能是某个与意识无关的现象,比如操作机器;也有可能是“某个行为”或是“事务的某个步骤”这类虽然和主观意识有关,但可以独立分析的过程;也有可能是“B这么做能触发/避免A的某个行为”这类看起来完全是人际关系互动,既可以看成是“B指挥A”也可以看成是“B疲于应付A”的过程。

指挥现象与指挥需求的区别在于,指挥是“某个刺激导致了B的行动”,B的行动是果;指挥需求是“B的行动导致了某个过程”,B的行动是因。

\indicate{指挥意图是一种主观现象},完全是A的认知。该认知可能来自某些合理的思考与决策,比如说“因为目标有指挥需求所以要去指挥”;也可能完全是污染,只是“自己不知道为什么,但感觉一定得这样”。A一定自知自身的指挥意图,而B不一定对A的指挥意图有认识;B可以在没有发现指挥意图的情况下配合这一指挥意图。

A有指挥意图的情况下,“完成这一意图”的过程就有了指挥需求,但是其它的相关过程(比如“维持和A的关系”等)不一定;A有指挥意图不代表A有对应的行动,也不代表A的行动提供的刺激可以用于指挥B。
\end{explain}
指挥、指挥需求、指挥意图这三者可以在某些事件中同时存在,这也使得我们很容易出现“这三者是密不可分整体”的外溢认知。
\begin{explain}
比如说,经常会有人将这三者笼统地称为投射\footnote{这里指的不是投射的原始学术定义,而是其在社区讨论时因过度泛化而产生的外溢。投射被用来指代过多了的现象,已经产生了严重的歧义。}。这会为我们的分析带来明确的障碍,经常会让我们解读出某些实际上不存在的动机。所以,本指南全盘避免了引入投射性认同及其相关术语。

在A和B的接触中,对于B的某个行动及其产生的影响来说,是否具有以上三者,以上三者是否具有因果性,有数百种%\footnote{我们按照“二者之间有4中不同的因果性可能”来计算,则总共有$2^3\times 4^3=512$种情况。注意其中也有“因为不存在指挥所以产生了指挥意图”等情况。其中不少情况在实际中很难出现,但总能构造出符合条件的极端性例子。}
不同的可能。其中既有“因为发现了有指挥意图所以产生了指挥需求然后去指挥”的情况,也有“有指挥意图但是不会指挥”的情况,也有“因为无法指挥所以反而产生了指挥意图”的情况,也有“只根据‘B会被指挥’的预料而做了计划,B不配合就无法进入下一步”的情况......以上种种可能都既可能存在又可能不存在,在分析时必须依据实际情况判断。
\end{explain}

\subsection{响应指挥}
如果B执行了指挥/指挥需求/指挥意图对应的行动,那么那么就称B\indicate{响应}/\indicate{听从}/\indicate{服从}了这一指挥/指挥需求/指挥意图;反之则称B\indicate{没有}/\indicate{无法响应}(\indicate{听从}/\indicate{服从})这一指挥/指挥需求/指挥意图。
\begin{explain}
指挥只关注“B有行动”这一现象。B只要响应了指挥需求/指挥意图,那么就产生了一个指挥现象,B也就响应了这一指挥。这是三者关联性(以及关联性外溢)的一种重要产生途径。

如果我们严格按照定义,那么就会发现“B无法响应指挥”这种情况不对应任何实际现象,因为此时不存在指挥。我们在这里将“B无法响应指挥”重新定义为“某个刺激无法触发B的行动”的现象。比如在“可以指挥其它行为模式的刺激对B无效”,或是“存在指挥需求/指挥意图,但B没有响应”的情况下,使用“无法指挥”就是合适的措辞。
\end{explain}
如果B\note{可控}地响应了某指挥/指挥需求/指挥意图,那么我们就将这种响应也称为B\indicate{配合}了或是\indicate{服务}于这一指挥/指挥需求/指挥意图。
\begin{explain}
这里的“可控”如之前的定义,指“B经过自己的决策,决定是否响应”。

从表面上来看,“响应指挥”有可能是A发现了B的某个条件反射,然后在B不自知的情况下加以利用;而“响应指挥需求”和“响应指挥意图”却都需要B先发现这二者,看起来像是必须自知的决策。但指挥需求和指挥意图本身也可以通过竞争和关联机制而成为B行为的触发条件,从而变成不自知也不可控的潜意识行为。

分别根据A和B的主观意图,我们可以将指挥现象分为4类:
\begin{itemize}
\item A没有指挥意图,行动对B不可控。这种情况下,指挥其实可以视为一种完全自动,不出于任何人主观意图的现象。如果这样的指挥稳定存在,那么就可能会形成一个共识环境。A和B可能分别对这一共识环境有任何形式的看法和态度。
\item A有指挥意图,行动对B不可控。这就可以视作A对B的\hyperref[def:控制人]{控制}。这有“父母让孩子远离危险”“有经验的从业者传授技巧”等正面例子,也有“过度干涉”“限制自由”“剥夺隐私”“测试服从性”“支配”等例子。
\item A没有指挥意图,行为对B可控。此时我们就需要关注B的真实意图:可能是“完全不在意”,可能是“讨好/畏惧/迎合A”,可能是“对A有所图”,可能是“有备无患”,可能是“有自己的信念和追求”。不同的原因会对应着不同的评价和对待方法。
\item A有指挥意图,行为对B可控。此时除了上述对A和B行为的评价以外,还需考虑A和B之间的博弈。依据A和B水平的不同,博弈可能处于从完全没有策略到无限层嵌套之间的任意复杂程度。
\end{itemize}
如果B没有响应指挥,那么也可以按以上标准分为四类,并且展开对应的讨论。此处略去这一部分篇幅。

“只关注到指挥现象”或是“只发现了指挥现象(或是对应的刺激)”的现象,无论对旁观者还是A/B来说,都相当常见。如果我们只从指挥现象出发思考,而不关注其它交流,那么在A视角中不能直接获得“B是否有控制能力”和“B对行动的态度和意愿”的信息,第一、三种情况不可区分,第二、四种情况不可区分;在B视角中不能直接获得“A是否有指挥意图”和“A对行动的态度和意愿”的信息,第一、二种情况不可区分,第三、四种情况不可区分。而旁观者则四种情况都不可区分,只能笼统地分析。

不可区分的现象之间容易产生理解和归因的外溢,这种外溢还有可能影响对指挥现象和其它现象的解读。这会造成一些不利的后果,如“本来可以区分‘对方是否有意’的信息被覆盖”、“影响A或B对该现象的性质判断和后续的行动,激化冲突”等。
\end{explain}

\subsection{交流和讲解时的指挥需求\label{sec:交流和讲解时的指挥需求}}
对于“行为模式B获得某个认知”的过程,我们将该认知的\indicate{一个理解途径中,对B的所有指挥需求}称为这个理解途径关于B的\indicate{自理部分},将每个指挥需求称为一个\indicate{自理需求}。如果B能响应某个自理需求,那么就称B在这一自理需求上有自理能力;如果B能够响应所有的自理需求,那么我们就称B对这个理解途径有\indicate{自理能力}。
\begin{explain}
自理部分有很多容易理解的例子,比如说“形成/维持前置认知”“知道某句话的重点”“练习以巩固”“接触资源”等。之前已经讨论过,这里不再赘述。如果B有自理能力,同时响应了所有的自理需求,那么B就获得了这一认知。

自理能力的定义不需要有讲解,但本小节主要还是关注“A给B讲解”这一过程及其可能遇到的障碍。除了“让B获得目标认知”的指挥意图外,这一过程经常还会包含很多其它的指挥现象。这里将指挥需求重命名为自理能力,是为了防止在后续讨论中,“指挥”一词出现过多,进而产生不必要的理解障碍。
\end{explain}
如果A可以识别B在某一自理需求上是否有自理能力,那么就称A可以\indicate{鉴别}这一自理需求。
\begin{explain}
此处的鉴别比之前定义的\note{鉴别}涵盖范围要大,之前的鉴别只包括对“维持前置认知”这种特定的自理需求的考察。前文的讨论同样可以直接迁移到此处。

A的鉴别能力会在两种方面出问题:一种是“A认为B需要做的行为不属于真实的自理需求”,另一种是“A缺乏判断自理能力的方式”。这两种都会导致A产生额外的,和理解途径无关的指挥意图,并且成为B的负担。

前者一般发生在“A没有正确归因‘如何获得认知’”时。A可能通过某种表面归因获得了一些经验(比如说“要努力”之类,很多时候可以算作正确的废话),也可能通过无序的关联而得到了外溢的认知(比如“将结果当成原因”之类),也可能A处于某个共识环境内,B缺乏对应前置。

后者一般发生在“A和B的交流不畅”时。由于B没有输出、A没有输入,或是信息被覆盖,B的自理能力对A不\note{可见}。如果A完全无法/不去判断B是否具有能力,在A觉得B需要获得能力时,即使B已经获得了对应的能力,A仍然会一遍一遍地指挥B获得能力;在A觉得B已经掌握能力时,也会指挥B做一些需要该能力作为前置的行动,从而导致B无法完成。

此时,B经常出现“为了完成A的指挥而行动”或是“按照A评判的标准来行动”的现象。B本身没有掌握对应的内容,这对A完全是巧合行动。

这两种情况经常同时出现,此时会产生“A一见到B,就会说一些只有A自己会看重的话”(可以是在任意事情上任意程度的看重,比如说“某种现象的特定处理方法”,或是“A数十年来的人生经验”)的现象,此时A的讲解行为退化为纯粹的转述。A所说的话都会刺激A联想到目标认知,但这只是A自身的关联,对B则不是有效的理解途径。于是会产生“A觉得自己已经从多角度全方位地讲了很多遍,而B就是不听”的现象,而此现象在B眼中是“A又在唠叨了,(如果知道A在说什么)自己只要搭理就会惹上更大的麻烦/(如果不知道A在说什么)完全听不懂只是浪费时间还限制自由”。双方都会因此而产生不满和负面评价。

B有可能从另外的途径(比如亲身经历)获取和A相同的认知,这对A是巧合的。B此时可能会同时出现“感受到了A的爱”“辜负了A的良苦用心”等感受,但B如果只感到了这些,而没有想到或是覆盖了“A的讲解无效”的认知的话,就也会做出和A相同的举动,重复相同的无效沟通。这是不可取且无意义的。
\end{explain}
在实际情况中,我们经常能观察到这样一种情况:一些人(使用C指代)可以鉴别自身的自理能力,并且因此而获得认知。
\begin{explain}
根据语境不同,我们会将这种能力称为“学习能力”、“自学能力”、“察言观色”等,并将其评价为“聪明”“会学习”“开窍”“机灵”“圆滑”“八面玲珑”等。为了方便理解和方便叙述,以下例子将以学校作为背景,不过理论可以通用地迁移。

在所有有助于信息传递的能力中,学习能力是较为常见的一种。它是唯一不明确需求“解读他人的能力”这一前置,就能掌握的能力(即使对于“C很擅长社交”的情况也是如此,C可以在“只掌握了某种特定的社交方式”的情况下就能处理所有自己会接触到的社交内容)。相比于“需要明确鉴别对方是否具有前置认知”的讲解能力,学习能力更容易自然习得:C可以更简易地获得自己的认知情况,从而有针对性地调整。

习得了学习能力就可以保证“获取认知”的目标达成,于是就可以承担对应的“获得认知”的责任。而其它没有学习能力的人(使用B指代)无法承担这样的责任。对于不明就里的人(使用A指代)而言,很容易将其总结成“勤奋”“用心”等特点,并且得到“只要勤奋就能学好”之类的认知。

A如果对认知形成的过程没有充分的认识,并以“只要用心就能学好”的观点要求B,那么B就需要被迫同时处理“时间分配”“学习顺序”“应付A并接受惩罚”等多项超出自身能力范围之外的自理需求。除了少数B能够巧合地拥有学习能力,少数B能够巧合地按部就班补起来以外,这种要求对绝大部分B来说,只会浪费时间而没有其它收益。

我们如果想有效地教会B,甚至是给B培养出学习能力,就必须具体地分析B的行为逻辑,而不能笼统地推卸责任,将所有事情都称为“态度”。

当ABC其中的两者是同一人X的不同行为模式时,还会产生一些更为复杂的变化(此处仅讨论A的看法和行为):
\begin{itemize}
\item 如果B和C都属于X,那么A很容易据此认为X的偏科实际上是出于态度不端正的偏心/不专心,而因此对X展开道德和关系上的指责。部分A可能会产生“人总有长处和短处”之类的认知,并因此同时接纳X的优缺点,但这样的认知也有可能外溢,缺失判断标准地无条件地包容一些不应被包容,有明显实际害处的事情。

\item 如果A和C都属于X,那么X很容易据此认为学习没什么难的,无法观察到B的难处并提供有效的帮助,从而有可能产生有意识或无意识的歧视。即使X很热心也很有耐心,经常给B讲解,B也很容易什么都学不到(此时X可能同时是B关系最亲密,最依赖和信任的人,但是此处不讨论人际关系问题)。

\item 如果A和B都属于X,那么X获得了“C很努力刻苦地学习”的认知后,可能会吸取一些表面经验并以此来要求自身,但不得其法的学习很容易除了假努力外什么都没有。同时,X如果认为“刻苦努力”之类的认知有效,就可能会拿去指挥别人;也可能因为“自己努力了没有成功”,就演变成X自我攻击的素材。
\end{itemize}
\end{explain}

\subsection{指挥现象的一个实例解读\label{sec:指挥现象的一个实例解读}}
\indicate{为了加深对指挥现象复杂性的理解,我们重点考察这样一种情况作为例子:A原本有某个行为a,在B做了某个行为b之后,A的行为a消失。}
\begin{explain}
这里的行为a可以有很多种不同的类型:既可能有“闲逛”、“娱乐”、“分享”、“邀请”、“演示”等带有正面评价的行为,也可能有“唠叨”、“盘问”、“辱骂”、“责罚”、“威胁”等带有负面评价的行为。

我们对这两种情况的分析有时会不同:当行为b打断前者时,我们一般会将其称为“扫兴”,将其纯粹地视为打断;而行为b打断后者时,却有一种完全不同的分析:将行为b(或其影响)视为A的一种需求,将行为a视为A表达该需求的方式(有可能会使用“压抑在深层潜意识中的需求”之类的表达),将“行为a消失”视为A满足了自身的需求,将整个过程视为A对B的一次指挥(或是控制)。下方的讨论仅关注这种负面(并且B希望制止)的行为。
\end{explain}
\indicate{但这种分析仅代表了多种情况中的一种,可能的真实情况包括但不限于:}
\begin{explain}
\begin{itemize}
\item A完全没有指挥行为b的意图,只是被行为b转移了注意力,覆盖了之前的行为。此时严格来说,应该将这一过程视为B对A的指挥。A没有思考过“指挥B”的原因,B对A的理解不来自A自身,而是来自其它地方。%如果有必要,我们还可以考察B除了“制止A的行为”外还是否有其它的动机(动机可以很正当,比如说学习),此处不做展开。
\item A完全没有指挥行为b的意图,但是会为自己的行为找原因。这些原因包括但不限于“我只是希望B好”“我只是很难受/害怕”“我大意了/没想到”等。在A有相应表达(可能是主动说出,或是在别人询问时给出)时,在B(或者其它的旁观者)看来,这和“A说出了自己的指挥意图”不可区分。

但是A的“讲述原因”这一行为可能是由“被问到”“看到B”“行为a”等原因触发的结果,而A的“总结原因”这一行为也可能是由“被问到”“听到别人说”“自己想”等原因触发的结果。A所给出的原因可能并不代表行为a的动机(同时并非有意欺骗),而只是A因缺乏分析自身行为的能力而产生的外溢认知。如果行为a可以由其它刺激触发,那么基于A给出的原因进行的后续分析和处理就无法达到“制止A的行为”的效果。
\item A有指挥行为b的意图,但是未经决策,而是由指挥意图直接触发了行为a。指挥意图和行为a之间的关联有可能来自A之前的思考,也有可能来自某些其它人的讲解,也有可能是在纯粹模仿其它人的行为。如果A还记得完整的思考过程(无论是自己的还是别人的),那么A有可能有相应表达,此时在B(或者其它的旁观者)看来,这和“A说出了自己的决策过程”不可区分。但与上一种情况类似,和A讨论思路无助于打断“由指挥意图直接触发行为a”的过程。
\item A有指挥行为b的意图和明确的决策过程,并且据此行事。此时可以和A讨论这一过程是否合理,是否有助于完成目标。但A的决策未必可信,相关的出发点未必可讨论,这也不保证成功。
\end{itemize}
以上的分析中,为了简化考虑没有涉及“A因为B的劝说而触发其它行为”的现象(原因有很多,比如说“不想合作”、“感觉到背叛/变心”、“(在有指挥意图时)觉得B不听话/想偷懒/没眼力”、“只是单纯地应激”、“话题发散到讨论别的东西”等)。如果产生这种现象,则应将其视为一个单独的事项,独立分析和解决。

从我们对上述四种情况的讨论中也可以看出,B对后一种情况的认知会覆盖前一种。如果B去询问A的动机,那么就有可能从第一种变成第二种;如果B去询问A的决策过程,那么就有可能从第二种变成第三种。B得到的回答,可能只是触发了A的一个无关的转述。再加上“B视角下,无法区分A的某些行为出于相邻的哪种情况”的现象,最后很容易使B产生“B最终发现A这么振振有词,果然是第四种”的认知(同时A也会有对应的“B怎么这么难以交流”的认知)。真实的原因因此被掩盖。
\end{explain}
\indicate{无论我们的目标是“理解行为a”还是“制止行为a”,“A无意识地指挥B”的分析都不够用。}
\begin{explain}
对于“理解行为a”的目标来说,仅有第三、四种情况可以适用“A有压抑在深层潜意识中的需求”的说法,才有可能出现“A觉知了自己的深层潜意识”的现象,进而才有“接受自己”“悦纳自己”之类的进展。这不因我们将其称为“深层潜意识”“核心信念”“自证预言”还是什么别的名字而有所改变。

前两种情况和后两种情况的主要区别在于,后两种情况的“指挥B”作为A的指挥意图,是明确的目标;而前两种情况的“指挥B”只是A行为的影响。不加分辨地将影响直接视为A潜意识的一环是不妥当的行为,它会覆盖我们对A真实行为逻辑的分析,使我们无法找到触发行为a的真实原因和对应的竞争过程。

对于“制止行为a”的目标来说,如果“A无意识地指挥B”的分析过程能够强有力地说服A,让A每次触发行为a的时候都想起这个论述过程,并且因此决定不再执行行为a,那么无论是上述情况中的哪一种,这一目标都能实现。而如果没有这个在触发行为a之后的竞争和覆盖过程,而只能在触发之前加入竞争的话,如上述讨论,这四种情况都无法保证目标达成。它们的失败方式也各不相同:
\begin{itemize}
\item 对第四种情况来说,A的决策过程中可能包含一些B不认同的认知。在双方都无法改变认知的情况下,就会以观念的冲突(可以是搁置、吵起来等不同形式)结束一场无成果的讨论。
\item 对于第三种情况来说,说服的尝试仍然可能如上变为观念之争,但无论争论无论成功还是失败,都是无效的。即使成功,A的“由指挥意图直接触发行为a”的行为依然存在。如果A还记得相应的论述并且愿意为此而改变的话(这在一些比如“为了B好”的意图中很常见),就会遇到一个问题:A不一定有充足的自理能力,不一定知道怎样改才是B可以接受的。由此出现可能“A十分别扭,不知道怎么和B相处”,也可能“在A看起来改了很多,在B看起来没啥变化”等多种情况。在B的“你从来没听过我”和A的“你还要我怎么样”中,冲突不但没有解决,反而扩大了。
\item 对于第二种情况来说,说服也可能变成观念之争,也可能成功了但无效。但除了“A主动改变行为但不知道怎么改”以外还会遇到另一种情况。A的动机和指挥意图相关,但并不完全相同。它在大多数时候会更具体(比如说“成绩应该再高一点”“想找个人聊聊天”),从而A不能直接从这里联想到自己提出的动机。于是,A可能虽然同意改也准备改,但是找不到任何应该改的地方,从而没有任何改变。
\item 对第一种情况来说,除了上述三种情况外,还有另一种可能:A有另外的行为逻辑,不认为自己的行为和B有关,根本不认为它们会对B有影响(主要指负面影响)。这很容易让A产生对B的负面认知,错误归因B的动机,认为B就是过于敏感,从而无视B的反馈。双方虽然形式上有沟通,但没有任何的信息传递。
\end{itemize}
以上的四种情况中,均只关注A对行为a的处理。实际情况中,A不仅获得了行为a,而且还得到了其它很多行为(有可能能形成完整的行为模式)。这些行为可能只有在B眼中能体现某个统一特征,而对A来说每个都有不同的动机。一个个处理时有可能按下葫芦起了瓢,远比单个行为更加复杂和棘手。

以上的讨论中不包括A和B行为逻辑的动态变化。如果考虑这一点,无论是与外界交互还是自身产生的新关联。都会极大增加处理的复杂程度,收集到的有关行为逻辑的信息可能在一段时间后完全失效。
\end{explain}

\section{总结与讨论}
\subsection{本章总结}
本章内容围绕\indicate{信息传递}的过程而展开,讨论“形成了认知的沟通过程”以及这一类过程的具体原理。

本章中定义的有效沟通需要满足“有一方形成了认知”的条件,这使得它比我们日常生活中使用的“沟通”概念涵盖范围更小。为了防止读者在理解上出现分歧,我们首先讨论了几种不会被我们视作沟通的情况:\indicate{零散表达}、\indicate{转述}和\indicate{拒绝沟通}。我们仍然能从这些现象中获得认知,但这只是单方面的。有这种行为的行为模式只有输出而没有输入,即使自身有沟通意愿,也会被覆盖。在此情况下,沟通现象完全是单方面的副作用。其中,\indicate{转述}同时作为一种好用的理论分析工具和反例,在本章后续内容中被多次提及。

在此之后,我们便得以专心于讨论本章的主要内容。我们首先讨论了“认知的形成过程”本身。虽然相关内容在第二、三章中已有涉及,但是当时仅讨论了“已经形成的认知有什么特点”,而不涉及“形成认知需要什么条件”这一方面。本章从此处出发,将获得认知的过程抽象为“维持\indicate{前置认知}并触发\indicate{理解途径}”的过程。我们因此得以更加细致地描述认知形成,本章的很多重要概念都依据前置认知和理解途径(而不是“获得认知”这一整体)而定义。我们将可以触发理解途径的事物称为\indicate{资源},并着重强调了“同样的资源在不同人眼中不同”的客观事实。

从而,我们得以将两个行为模式的沟通视为获取认知的资源来处理。在沟通中,话题的\indicate{分散}不可避免,而由此产生的解读\indicate{分歧}会覆盖前置认知,打断理解途径,影响沟通。这是我们在所有沟通中所必须注意的事情。除此之外,认知的输出者是否对另一方有足够的了解,也决定了输出者能否有效调整\indicate{讲解方式}并最终成功地输入信息。我们引入了\indicate{鉴别}这一概念来描述“输出者确认另一方是否维持前置认知”的过程,并且以此定义了输出者的\indicate{讲解能力}。

在初步讨论完“有效的信息传递所需的要求”之后,我们需要来研究“虽然存在有效的信息传递,但并不通过稳定的方法实现”的情况。我们用\indicate{巧合}来指代这一类信息传递现象。如果我们从巧合现象中提取认知,总结规律,就有可能产生错误的判断。每一种讲解,每一个资源都有它的适用范围,而认知的适用范围就是“前置认知和理解途径”,我们也据此定义了\indicate{共识环境}的概念。资源只在对应的共识环境内有效。

在给出“信息传递的方法在什么时候会失效”的刻画后,我们下一步需要考察的是“如何有效地信息传递”。这需要再次深入研究获取认知的过程。我们从解读资源的过程中提出了\indicate{稳定}(可以一直接触资源)和\indicate{充分}(可以补齐前置认知)的概念。

本指南的向读者介绍的核心能力之一便是有效地分析人的意识现象这一复杂系统。另外,不管是“听从讲解”还是“鉴别对方是否具有前置认知”,我们都离不开对人的解读。故此处我们用了一些篇幅来详细展示“分析一个人所需的做法”。本指南认为,除了单独观察并总结人的每个行动/行为(行为模式)外,理解行为还需要一个必要步骤。竞争过程导致了行动/行为,而\indicate{准确认识行动/行为,需要找到真实的竞争过程}。

最后,我们具体分析了一种经常与沟通与讲解同时出现,并且原理十分相似的现象:\indicate{指挥}。当信息输出者自身的讲解能力不足时,经常会需要另一方在一定程度上\indicate{自理}来补齐讲解的短板,从而产生了指挥的需求。除此之外,我们解读指挥现象的时候也经常会得到不准确的认知,因此我们将指挥细分为\indicate{指挥}、\indicate{指挥需求}和\indicate{指挥意图}三种不同的概念,以澄清可能出现的误解。

本章的一、三、五节主要围绕“沟通为什么会失败”,而二、四节主要围绕“沟通为什么会成功”,从两方面出发,尽可能全面地讨论了信息传递和认知形成的现象。

\subsection{本指南的写作考虑}
读者容易注意到本指南在写作风格上很有特点。其中一些特点在\hyperref[sec:应用部分前言]{应用部分前言}中已有提及。

比如说,本指南会花很大力气去定义很多概念,并且花很大篇幅去辨析这一概念的适用范围。这样做的主要目的,是尽可能地使更多读者可以准确地理解这些概念,以方便之后的正确应用。这些概念大多较为贴近日常,本指南认为读者应该对其中的大多数至少有大体的认知。但贴近日常同时也会导致它们的歧义很重,这些歧义难免会干扰本指南后续应用这些概念时的信息传达。所以本指南将相当多一部分的篇幅放在了“排除这些歧义认知”上,甚至在某些小节会见到“除了定义和概念辨析以外什么都没有”的现象。对于任何一位的读者来说,其中大部分排除歧义的篇幅都是不必要的,但它们应该总是会对一些读者起作用。本指南尽可能地全面讨论了所有可能产生(有明显影响的)歧义的部分,希望没有太大的遗漏。

同时,本指南假设读者对于“精细操作概念”的做法不是很熟悉。由于我们客观上有“区分含义不同但用词相同的概念”的需求,我们需要做的是“区分每个不同的义项”(本指南使用“给每个义项分别起名字然后对比讨论”的方式)和“确定概念的适用范围”(如应用部分前言中提到的“对xxx的yyy”)两方面。前者只要起名字就可以解决,而后者则需要一定的训练。如果读者无法为概念配上适用条件,并从此处开始思考,那么在实际操作中就又会回归“带有歧义地讨论”的情况。为此,在概念的讨论中也包括了这些适用条件的具体用法,包括“什么时候可以省略”“什么时候应该注意”等内容,尽可能方便读者自学和自我鉴别。这一部分内容对于已经习惯于精细操作概念的读者是不必要的,而对于完全没有意识到必要性的读者来说,也无异于文字游戏。

以上两方面会不可避免地影响本指南的可读性。不过为了完整地展开分析,同时能准确地向读者传达分析思路,这些概念作为前置认知,是必不可少的。我个人能力有限,无法兼顾两方面,只能优先选择准确传达。本指南为因此而造成的阅读障碍表示由衷的歉意。

另外一个会影响可读性的方面是本书缺乏实际案例。读者经常可以在其它的书里见到“咨询师遇见一位来访者,通过交谈、分析和计划,逐步发现来访者的心理问题,并使用有效的方法改变和解决”的小故事,这些故事能使读者更容易理解咨询师所使用的方法论,容易理解所关注的重点和对应的解决思路。

——表面看起来是这样。但本指南认为,案例在实际上起到的作用,主要是“污染了读者,给读者灌输了不全面的认知”。这主要有两种可能的原因:一方面是“一些读者见什么信什么”,导致他们只学习到了“在这种特定案例下的结论”,而“通用的分析方法”被覆盖了(即使作者着重强调),进而他们会把特定结论草率地套到别处;另一方面是“读者没见过其它可能的成因”,而自身又没有充分掌握通用的分析方法,导致特定结论不可控地和其它东西产生关联\footnote{这两种原因也同样是“从效果来看,被覆盖和缺失不可区分”的一个例子。}。

为了避免先入为主,我们有必要展示完整的分析过程,涵盖所有可能的情况,也即“理解行为要从对应的竞争过程出发”(见\hyperref[sec:对竞争过程的解读]{4.4.3小节})。这使得我们在分析任何一个现象时,都需要细碎冗长地分类讨论。\hyperref[sec:情绪、感受与其分析与处理方法]{3.7节}和\hyperref[sec:指挥现象的一个实例解读]{4.5.4小节}中,我们分别按此方法讨论了一种情况。我们即使排除了很多相对关系不大的分支,但仍然不可避免地需要分很多类分别讨论。其中每一类都是可能真实发生的情况,都可以视为一种案例。相比之下,我们在\hyperref[sec:交流和讲解时的指挥需求]{4.5.3小节}中对学习能力的讨论就相当不完善,遗漏了很多种可能的情况。这种不可压缩的必要篇幅使得我们不可能举很多这样的例子,只能挑重点来集中演示。

读者如果确实需要一些例子来辅助理解,可以参考以下两方面内容:如果读者具有编程(尤其是运维)相关基础,可以使用找bug的思路来理解。在不考虑“重写整个业务逻辑模块”这一举动的情况下(毕竟我们没法对人这么做),我们总是要定位到具体的代码执行层面,才可以确定问题所在。问题可能是单个模块的设计漏洞,也可能是多个模块间的逻辑冲突,代码整体的复杂性产生了不可预知性,导致我们不可能保证在遇到问题时就直接定位。如果读者没有编程基础,那么可以参考(大学以下的)教育领域。教育领域是最为集中的信息传递环境,非常全面系统地梳理了所有常见的理解顺序。其中需要特别关注的是“总是有办法把差生带起来的老师”。他们需要处理各种不同学生的前置认知/自理能力缺失的问题,每个人都具有一套行之有效的定位问题的方式。这些方式可以用于参考。相比于心理咨询方面的资源,这二者的最大优势在于资源充足,量又大,处理的问题又集中,可以基本实现“覆盖所有可能的原因”的要求,避免出现先入为主的现象。

除此之外,本指南在写作过程中经常出现“想到什么写什么”的情况。这在一定程度上影响了逻辑的连贯,进而会打断思路,影响可读性。读者可能会发现,本指南经常有“废了好大劲,兜兜转转又绕回之前讨论过的话题”之类的情况。虽然这在一定程度上也有助于理解,但比起“认识的不断深化”之类的评价,本指南更愿意将其单纯地视为“对结构没有充分的掌握,导致无法充分地解耦”的现象。类似的现象也包括“前面定义了一个概念,后面又定义了没什么区别的另一个概念”等。本指南为因此而造成的理解障碍深表歉意。

\section{实操:日常沟通与社交}
\hfill\begin{minipage}{0.6\textwidth}
\fontsize{8pt}{12pt}\selectfont\fontsize{8pt}{12pt}

\raggedright 你是沉默的,只对黑夜倾诉更多。总是藏起话不说,要把自我吞没。\footnote{\bilibili{BV1r4411A77J}\another\netease{1865377022}。}

\raggedleft 闹闹丶\&果汁凉菜《无题》

\raggedright 满载思考的脑袋,偏爱沉默。盛不住心事的我,倾囊而出,不怕干涸。\footnote{\bilibili{BV1Hf4y1L7MF}\another\netease{1979007507}。}

\raggedleft vsinger团队\&果汁凉菜《夏虫》

\raggedright 谁人能够听到些微我的歌声吗?\footnote{\bilibili{av2711298}\another\sing{2871399}。\\}

\raggedleft COPY《回音》

\end{minipage}

可能一些读者从本章一开始就在想“怎么还没讲到这个话题”了。本章所讨论的沟通一直以“产生新认知”为前提,这看上去充满了目的性。而日常生活中的交流,大多不会有“一定要说服谁”“无止境的勾心斗角”之类的情况。如果不加限制地在所有沟通的场合下都这么厚黑地思考,机心也太重了。

对每次沟通都这么分析,是一种“不能充分掌握和熟练运用本章内容”的体现。在那些“明确可以判断出参与沟通的人都属于同一个共识环境”的情况下,该玩就玩,该乐就乐,完全不需要考虑这些。我们所需要特别关注的,只有一件事:什么情况下,参与沟通的人不属于同一共识环境。
% \begin{explain}
% 如果我们有一个标准的共识环境作为参考,那么可以将情况分为“自己不属于这一共识环境”(比如“觉得自己无法融入某个集体”、“觉得别人不应该这么做”之类的)和“别人不属于这一共识环境”(比如“(虽然自己推荐,但是对方仍然)对某一方面不感兴趣”、“”)
% \end{explain}
\begin{explain}
上一次还是兴高采烈地一起去玩,下一次就一个人都约不到,所有人都说自己没时间;某些人一直以来都是很热心且可靠,但在某一次却突然发飙说“我忍你很久了什么事都往我身上推”;自己明明是一片好心,却不知怎么地就碰到了别人的雷点,被人揪着细节一顿数落......

这些例子想举多少就能举多少。我们当然可以将这些现象统一称为“人心难测”,但到底有多难测呢?我们还有没有别的视角,来认识这些事情,进而解决这些事情?
\end{explain}
为了方便后续讨论,我们先大体上将所有日常接触分成两类:在一个话题中,如果\indicate{分歧越来越多},那么我们就将其称为\indicate{沟通不畅}、\indicate{不良沟通}或\indicate{不良话题},否则就将其称为\indicate{(话题/沟通的)良性维持}、\indicate{良性沟通}或\indicate{良性话题}。
\begin{explain}
这个概念既可以放在“两个人之间”,重点考察这一沟通过程;也可以放在“一个人和一个共识环境”中间,重点考察这个人对某一共识环境的理解和接受程度。

此处“良性/不良”的价值判断仅针对“话题维持”本身,若还有其它的目标(如“考验”、“隐瞒”等),则可以不关注这一价值判断。

这里的“分歧越来越多”的定义比较模糊,我们考虑的主要是“分歧会为沟通(和其它方面的相处)带来什么不利影响”。对于“会形成长期认知的分歧”,如果一个分歧通过后续的沟通得以消除,那么我们就将其视为“分歧减少了”,此时比较的是“产生分歧的速度和达成共识的速度哪个快”;对于“不会形成长期认知的分歧”(我们关注这种情况一般是因为“这一分歧产生了某些影响”,比如说“实际上没有必要的防备”“越解释越乱”之类的),如果在它被遗忘之前(可能是这个话题结束之前,或是一些类似的其它情况)没有被纠正,那么我们也将其视作永久性的“分歧增多”。

定义中的“在一个话题中”的前提条件不可省略。我们经常能遇到如“其它方面都聊得挺好的,就只有某个特定的话题极为固执,怎么说都说不通”“和另一个人只有某一方面的交互,其它方面没什么牵扯”的情况。在两个人(甚至是两个行为模式)的接触中,“某些话题是良性沟通,某些话题是不良沟通”的情况很常见,必须按话题分开来各自处理。相对地,“因为某个话题沟通不畅就认为这个人不好相处”、“因为能在某个话题上良性互动就十分信任这个人”之类的认知都属于外溢。
\end{explain}
日常沟通中,虽然我们不需要很明确的“一定要形成某个认知”的意识,但是随着交流,客观上确实也有一些认知(和行为)在潜移默化地形成。这些巧合形成的认知不一定有助于良性维持某一话题。本节的主要关注点即为“如何识别和处理因共识不足而导致的沟通不畅”问题。这可以大致分为三方面:
\begin{itemize}
\item 识别与进入共识环境:参与某些互动需要熟悉相关的对象,包括但不限于“某个文艺作品”、“某个操作流程”、“某人的内心活动”等。此处的共识环境主要指的是“在话题开始前就已经存在”的部分,不涉及话题中的演化。
\item 识别分歧:依靠某些现象,确定“参与话题的另一方和自己有分歧”,包括但不限于“自己/对方忽略了什么”、“自己/对方理解错了什么”、“自己和对方的重点不同”等。我们在这里重点关注话题中自然产生的分歧。
\item 消除分歧:在发现分歧后,通过一定的方式调整自己或对方的认知,以消除分歧。这些方式包括但不限于“分析和思考”、“解释”、“转移话题”等。
\end{itemize}
\begin{explain}
这样的行为可以有很多种不同类型的负面评价,包括但不限于“每天都在勾心斗角很累”、“这样做就是为了讨好别人,令人看不起”、“城府越深越不可信任”等。本指南持有以下观点(读者也会在后文中看到相关讨论):这种评价的一种主要成因恰恰是“不具有充足的有效沟通与达成共识的能力,从而才会疲于应付”。

以下篇幅将着重关注技术问题,不会专门讨论态度和动机方面。本指南不持“大家一定要在每个话题上都好好沟通”、“识别了分歧以后一定要加强沟通消除分歧”之类的建议,这部分决策由读者个人决定,如“识别了分歧以后发现不可沟通于是远离”等,也是可行(且在一定情况下推荐)的操作。此处列出这三方面,仅因为它们是重要的观察与分析角度。同时,有效地使用这些技术,也可以使我们尽可能少地将沟通不畅归因于不可改变的“态度和性格”因素,而是尽可能多地有识别和解决的方法。
\end{explain}

\divider

“识别与进入共识环境”这一方面的内容在\hyperref[sec:解读]{4.4节}中已有提及。此处会从另一个视角出发,讨论一些之前没有覆盖到的地方。

如果我们只关注某一个具体的事务,那么就可以直接评估它的难度。但是在处理覆盖面很广的一大类事务时,我们如果“以某个具体事务为例,以这一事务的难度来估计这一类事务的难度”,就容易把将这一类事务估计得过难或者过于简单,从而不然盲目乐观不然盲目悲观。为了全面客观地认识和讨论这一类事务,我们首先需要问的问题是:“它最简单可以有多简单?最困难又可以有多困难”?

对于“识别与进入共识环境”来说,最简单的情况当然是“什么都不需要做,本来就在共识环境里,一切都很融洽”。这没什么值得讨论的,所以我们略过它,直接考察第二简单的情况:顺畅地识别和进入了共识环境。
\begin{explain}
对于识别和进入,我们都能举出很多足够简单的例子。

我们在很多情况下可以自然地发现“我好像有什么不知道的”,比如说“前面好像有一群人聚在一块,不知道在干什么”、“最近好像有一个梗很火,到处都在用”、“到了一个没去过的地方,不知道卫生间在哪”、“对方和我想的不一样,有自己的意见”等等。虽然我们不一定能在第一时间就发现“这个共识环境的全貌”,但此时识别“有一个自己不在的共识环境”还是较为容易的事。

我们在很多情况下也可以很简单地进入一个共识环境,比如说“观察别人的反应看都在关注什么事情”、“去查一查有没有成系统的介绍/教程”、“直接找别人搭话问发生了什么/卫生间在哪”、“问一问对方是怎么想的”等等。如果仅靠这一次没有难度的接触他人或是接触资源,就可以完全掌握所有所需的信息,那么这就是足够简单的进入方式。

识别容易不代表进入容易,我们可以轻易举出“很容易就能发现自己能力不足,但是想要补起来得花数百个小时好好学好好练才行”之类的例子;进入容易不代表识别容易,我们也可以举出“之前一直觉得没什么,直到某件事以后才知道自己一直想错了”之类的例子。

这里的所说的“简单”基于每个人的主观认识,严格定义相对困难。我们甚至没办法使用“必要篇幅”来定义,因为有“一直都不知道,但是在合适的契机下马上就懂了”这种,从“理解信息”角度来看没什么难度(甚至成功经验可以复制),但从“实现目标”角度来看很有难度(不保证能遇到机会)的例子。区分简单和困难的主要目的是提醒读者保持全局视角,所以此处不多做讨论。
\end{explain}
而最困难的情况是“所有人都没能力做到,超出了人类目前的认知水平极限”。无论是“死语言”一类的共识环境,或是其它世界未解之谜,再或者是人类甚至都没有发现的一无所知的领域,都同样也没什么值得讨论的。所以我们也略过它,来关注第二困难的情况:只有少部分人可以,但大多数人不行。
\begin{explain}
当然,这里说的“少部分人可以”指的不是那种推动了人类进步和学科发展的天才或大师级别的成就,那种事例的性质更类似于“超出人类极限”的情况,也没什么好讨论的。我们这里仅讨论那些在日常交流中,只有少数人拥有的“善解人意”特征。

不可否认的是,有一些善解人意的特征完全是巧合。一个人A在进入某个共识环境(可能由于自身经历、文艺作品或其它途径)后,每当发现另一个人B(可以是自己)具有同样的行为,就会开始代入。这很容易让共情变成自我感动,会阻碍我们发现共识环境,从而使后续的分析与理解变得不可能。我们在\hyperref[sec:情绪、感受与其分析与处理方法]{第三章实操}中已经详细讨论了这种情况,此处不再赘述。“对对方的行为有自己的理解和处理,导致信号被覆盖”和“根本没注意到信号”是识别共识环境的两大主要障碍。

上面这种情况想要修正起来也比较简单:保持谨慎,不要脑补过度即可。一个经常有效的方法,是“询问对方B的动机”(如果是在分析自己,那么就回忆)。如果对方的回答和这种共识环境明显对不上,就可以确定自己想的不对,从而避免一次错误的代入了。

这种做法的难点在于,直接询问不一定能得到真实的结果。B可能无法准确概括自己的思路,或是B受到了另一个共识环境的污染,从而说出来的东西不代表真实情况。不过这不意味着原因就对A不\note{可见},A仍然可以通过别的方法(比如多次接触、旁观、换话题等)来收集信息以做出判断,进而进入对应的共识环境。这在之前也已经详细讨论过,此处不再赘述。这可能需要长时间观察,大量分析和排除,才能获得准确的结论。
\end{explain}
我们可以观察到,在以上“什么简单什么困难”的讨论中,我们不可避免地需要分析“为什么简单,又为什么困难”。尽管上面的分析也举了很多事例,对这些事例的分析也并非遵循统一标准,但这仍然可以带来一些很重要的信息,即“我们为什么会成功,又为什么会失败”。

用某一个事例的判断标准套用到其它事例上,如果能得到符合预期的结果(有的时候论述可能很简单,是“只要看到就能”),那么就说明这个标准比较好用;如果发现套不上去,或是即使能勉强套上去,理论也过于牵强,那么就需要回头反思是哪里出了问题,这个判断标准和对应的思路是否过于草率。

尽可能全面地选择与考察事例,并且把自己的每个思路都在所有事例上都过一遍,就能知道哪些思路比较好用,哪些思路用不成,每个思路分别适用于什么情况了。在面对一个一无所知的复杂系统时,这种分析方法比较高效,又尽力确保了准确。它不保证能得到所有有用的知识,不保证能排除所有的失误,不保证分析适用于所有可能,但至少在处理常见情况(常见情况会在想事例的时候想到)时基本够用。

\divider

遵循着这个思路,我们也对“识别分歧”这一方面提出同一个问题:它最简单可以有多简单?最困难又可以有多困难?

和上面一样,讨论“能有多简单”时,我们也忽略掉什么都不需要做的“没有分歧”的情况。并且,也不重复讨论那些“可以视为不在同一共识环境”的情况,只关心那些在话题中产生的分歧。
\begin{explain}
这样的分歧可以分为两类:一类“和当前话题没有因果关系”的,比如说“走神”等;另一类是是“和当前话题有因果关系”\footnote{此处的“因果关系”指的不一定是“一个是因,一个是果”,还有可能是“二者有共同的原因”这类更复杂的关系(但不可以是“二者共同导致了结果”)。}的,比如说“侧重点不同”等。我们将前者暂时称为行为分歧,将后者暂时称为话题分歧。

二者都可以举出很多发现起来很简单的例子。对于行为分歧来说,“走神”、“没有回应(问题、请求或者其它)”、“同时在干其他的事”、“离开”等行为可以让人清晰地分辨“对方心思不在这里”。对于话题分歧来说,也有“对方主动表示听不懂/听不明白”、“回答或分析错误”、“有明确的其它重点”等。很多这样的分歧只需要识别出来,就可以通过“提个醒”等简便的方式直接解决。
\end{explain}
而讨论“能有多困难”时,也需要忽略掉“无计可施”的情况。并且,也不重复讨论那些“因为信息缺失从而无法解读,无法识别分歧”(比如说因为对方没什么反应/自己不关注对方的反应,所以不知道对方是否走神)的情况。除了这些单独的分歧以外,我们还会面对一种更复杂,更系统的分歧,表现为“从一个话题中,源源不断地产生分歧”。
\begin{explain}
这些不断产生的分歧有很多不同的形式,有可能是“从一个话题开始,不断延伸和发散”,也有可能是“多次涉及这个话题,每次说完之后下次又和没说一样”。这些分歧可能有任意程度的解决,可能是“双方大吵一架,什么都没沟通到位”,可能是“每个问题说了一半就转到下一个问题了(有可能是当前问题的前置、另一个相关话题、或无关内容)”,可能是“每次都能良好地沟通,都能达成共识(但是下一次又忘了)”。

这种情况下,“根据当前情况(主要指已有的分歧)来调整自身的行为模式”的能力需求,超过了参与话题双方(或多方)的自理能力。真正的分歧来自某些更深入\footnote{这里的“深入”指“作为成因”,不包含“这来自潜意识”的意思。}的思考回路,而不是表面暴露出的问题。需要注意的是,这仅在某些特定目标下才是需要处理的问题,而在另一些目标下(比如“只是随便聊聊,聊到哪算哪”等)不需要理会。

发现这种分歧是极为困难的事情,这需要参与话题的双方能够完整复盘所有相关内容(复盘可以仅由一人完成,也可以由双方合作,如“我们来好好聊聊吧”地完成;复盘不一定带有目的性,可能在某次无目的/其它目的的回忆时巧合地注意到)。这会有以下几方面的困难(以下的篇幅中有时使用“我们”和“对方”分别指代主动方和被动方):
\begin{itemize}
\item 根据频率的不同,我们可能需要定位昨天/一周前/两个月前/......的某次发言。一般聊天中很少能清楚地记下所有细节,甚至很少能清楚记下“有这么回事”。
\item 而相对地,“模糊地有一个印象”的可能性要高上不少。我们可能不记得具体过程,而只是留有“总是会吵架”“总是不听”“看起来答应了但是转头就忘”的印象,或是虽然没有自知,但已经在长期的接触中形成了某种行为习惯。这些模糊的判断和对策会干扰其它方面,比如我们可能将这解释成“对方的态度和喜好问题”。
\item 我们会向对方反馈这个印象。我们可能明确知道(并且可能持有包括但不限于“想好好谈谈”、“求求你了”和“你必须给我一个交代”的任意态度),也可能是通过不自知,但外部可观察的行为(比如说回避)。但如果我们的分析不到位,那么仅靠这些反馈想要定位问题,难度和不给这些信息不会有差别,对方仍然需要使用和以前没什么区别的自理能力,去分析和以前没什么区别的情况。并且,如果这些反馈没有明确指定问题(比如说只是笼统地说“你从来就没在意过我说的话”),那么还可能产生新的理解障碍。
\item 这种不包含有效信息的反馈,最好的结果也只可能是“双方浪费了一些时间,一无所获”,大多数情况下总是会伤感情\footnote{这里将“一方单方面付出和改变”也视为伤感情,因为这样忍耐和改变大概率改不到点子上,还是会在很多相似的地方产生同样的问题。我们不应该将偶尔的巧合成功视为常态和目标。}。由于以上的现象实在是太常见了,所以导致另一种现象也会很常见:对方会抗拒这样的反馈。根据具体情况的不同,这些反馈可能会有“翻旧账”、“情绪输出”、“道德绑架”、“投射压迫”等很多种不同的称呼,也有可能对方没有明确认知,只是单纯莫名反感。
\end{itemize}
以上这些因素导致我们在绝大多数时候都会将这种分歧视为某种不可改变的状态,将其称为“性格问题”“两人不合拍”等等。虽然在事实上,相当多的此类问题只源于某个很容易解决的分歧(比如说“只是不知道某个东西有什么用/某个词是什么意思”),但定位问题的复杂程度已经高到几乎不可能靠某个巧合操作就试出来。在没有充足能力的情况下,即使双方都很友善,都愿意复盘,复盘流程也很容易被其它事情(比如相互道歉和相互原谅)打断,从而得不到真实原因,大概率也只能是相互包容相互谅解了事。这无法增进理解,不是最优的结果。

有效的消除分歧的方法,如之前所说,是“观察并解读所涉及的行为模式和竞争过程”。这也是我们需要完整复盘的原因之一——过去的事例会带来一些有效信息。当然,信息不一定足够,有可能还会需要一些询问以补充。这种询问如果操作不当,也容易被对方认为是“借机找茬”,毕竟在对方眼中可能这是完全不同的两件事。在调研的过程中,我们要尽量避免所有可能打断这一过程\footnote{我们此处只关注会影响分析过程,使我们无法得出结论的打断,像是“今天没时间了明天继续”这种中断不计算在内。}的内容,包括但不限于“提前得出结论”、“关注点转移”、“不耐烦”等等。

如果打断过于频繁,那么可能无法分析透彻这个分歧,此时我们应当转而关注会打断分析过程的这些行为。如果反复转换关注对象也不行,那么基本可以下结论“无法定位这个分歧”,此时根据具体环境的不同,我们可能有“绕开相关内容”、“远离这个人”等多种不同的处理思路。
\end{explain}

\divider

同样,我们也对“消除分歧”这一方面提出同一个问题:它最简单可以有多简单?最困难又可以有多困难?

在准确地识别了分歧之后,消除分歧的目标就变成了改变认知。如果需要改变认知的是自己,那么根据本章所介绍的方式处理即可,此处不再赘述。以下篇幅主要关注“需要改变对方的认知”的情况。
\begin{explain}
在一些简单情况下,这种改变可以通过短期接触(比如说直接提意见)、指挥(比如说给对方推荐某个资源)、甚至是对方完全自理(自己什么都没做,对方直接发现并改正)的方式达成。具体的例子和前两方面列举出的大同小异,我们在这里略过,直接开始讨论可能遇到的困难。

此处的省略不代表“所有这类问题都是困难问题”或是“我们应该对每件事都很正式地分析”之类的观点。在想起这些观点的时候,总该同时想到“简单易处理的情况也存在”,并且使用所有的分析方法,对每种情况具体问题具体分析。
\end{explain}
较为困难的情况,是对方不具有相应的理解能力。此时若没有合适的资源,就需要我们主动来铺垫和讲解。讲解的整体思路是:选定一个双方都在,并且可以理解该认知的共识环境,并且从这里开始讲。
\begin{explain}
以上的描述包含“对方虽然不在共识环境中,但可以顺利地进入共识环境”一类的“需要连续维持多个共识环境”情况。常见的例子比如“我们给对方讲一个故事/一段经历”,所需的共识环境仅为“语言相通”“有近似的价值观”等,与当前话题不直接相关。如果每次进入共识环境都很顺利,那么我们也可以将其视为消除分歧的简单情况。

而相对地,失败和困难也很清晰:没有进入共识环境。无论是“没有判断共识环境就直接开始讲解”还是“找不到可以进入的共识环境”,都会导致失败。如果我们严格按照有效沟通的步骤,此时应该转而去识别分歧。但在实际交流中,大多数情况下我们不需要这么麻烦,也可以继续维持沟通。我们有更简便的方法,可以同时做到识别分歧和消除分歧——直接继续。

对于同一个刺激(比如说提问或者指挥),是否处于相应的共识环境,本身就会导致不同的反应,而这就可以用来识别分歧。同时,大部分分歧易于识别,并且可以通过很方便的方法(比如当场问)直接解决。此时按部就班地做事是没有必要的。

但面对一些更复杂,以至于不能当场消除的分歧时,如果我们习惯了以上的操作,就很容易路径依赖。当我们发现“对方好像听不懂”的时候,我们会再尝试解释或者指挥一遍。但理解这些思路或者行动同样需要共识环境。我们属于这一共识环境,所以它们可以让我们获得对应的认知;而对方一直不属于这一共识环境,就一直无法理解。常见的情况比如“给基础不好的学生讲题”,或者是“自己情绪很激动的时候问别人意见”。这种情况下,对方在这一话题上能做的事情只有附和(对方本身可能自知或不自知),比如“装作自己听懂了”或者“同意以提供认同感”,而无法达成消除分歧的目标\footnote{本指南不推荐读者选择“熬过去就好了”、“提供情绪价值”等浮于表面的目标。我们可以做得更好,至少可以试试看。}。

同时,这也会使对方容易习惯这样的相处方式,进而使对方无法进入某些原先可以进入的共识环境。我们经常可以观察到“为了不让对方操心,选择不和对方说”等情况,这是其主要成因之一。对方甚至可以和陌生人谈这些,陌生人也真的有可能冷静理智地分析,但是对某几个特定的人就是做不到,因为相处方式已经固定了下来,而其中不包含交心的内容。
\end{explain}
在发现了这种分歧后,我们应该做的,是重新选择一个内容更少的共识环境,并尝试从此开始。
\begin{explain}
产生分歧的双方在消除分歧时,都可能涉及“放弃之前的认知(如‘对对方的评估’)”、“改变之前的行为”一类的操作。这些操作在一些语境下会被称为(逻辑和思路上的)后退。这和态度与立场上的“妥协与退让”是两种不同的事情。虽然有些操作(比如说“放弃某个要求”)同时符合这两点,但不可将这两点混为一谈,也不可以认为“妥协和退让一定能换来理解和认同”,这是两个独立的方面。

根据对方实际情况的不同,我们后退的幅度也会有所不同。比如说,对于基础不好的高中生,直接从这节课的知识点开始讲是没有用的,我们可能需要后退到初中才行,由此可能产生极高的沟通成本(无论是自己处理还是找其它资源都极高)。

不根据对方实际情况判断,而是直接认为“后退到某个程度一定有效”,是草率的结论。如果后退得过多,那么就会产生一大段无效沟通(我们经常可以在“唠叨”“聊天吹牛”等场景见到这样的后退);如果后退得过少,对方仍然不属于这一共识环境,那么也会变成无效沟通。为了确保有效,我们需要使用一些方法(比如提问和观察)来确定对方的基础,并且由此判断应该后退多少。我们可能在同一个话题内多次后退,以寻求有效的理解和共识。这也要求我们对目标认知和其理解过程有明确的认知。
\end{explain}

\divider

读者可能会注意到,以上的讨论中没有涉及“礼仪”方面的内容,比如“什么时候应该道歉”等。这是因为,所有的礼仪都只在某个共识环境内有效。不同的共识环境对于相同的需求(包括但不限于求助、道谢、致歉等),有可能有不同的习惯表达,从而在一个共识环境内得体的表达,在另一个共识环境内可能反而是失礼的。

在不是很熟的时候,可以使用一些较为广泛的共识环境中的简单表达。此时,无论是有礼还是失礼不应该被过度重视,不应将其视为“需要遵循的全部准则”。要产生合适的默契,总是需要先在充分了解对方的惯用思路和表达方式——这本身是因人而异的。面对同样的事情,有些人会在意,有些人不会在意,有些人只接受特定的表达方式,有些人会非常善解人意地“只要你有心就行”。这些都是在充分了解后才能掌握的分寸,不存在“可以普遍地对所有人都适用”的统一社交方式。

% \titlespacing{\chapter}{0cm}{5cm}{1cm}
% \include{chapter05}


% 结语相关:写作动机、结构缺陷与改正、选歌的用意和推荐的对待态度、关于例子少的说明

\end{sloppypar}
\end{document}