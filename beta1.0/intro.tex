\chapter{前言}                   
%\addcontentsline{toc}{chapter}{前言} % 添加到目录


\noindent 本指南主要讲述在现代社会中高质量地活下去的方法。书中的内容可能主要对以下三类人群有所帮助:
\begin{itemize}
    \item 感到痛苦、迷茫、不被周围人理解,同时缺乏长期目标的人。这类人是本指南的主要服务对象。本指南将会带领你细致地拆解生存中会遇到的问题,并对此给出详尽可行的处理方案。指南的主要篇幅将直接面对你。若无特殊指明,文中的第二人称指这一类人。
    \item 身边有第一类人,并且有动力改变他们的人。你是否觉得你的孩子性格越来越封闭,情绪越来越低沉?你是否觉得朋友需要和你报团取暖,以面对社会的不公?你是否在群里遇到了经常情绪崩溃,不停诉苦的群友?最重要的是,你的关心和安慰是否没有起到你预期的效果,对方并没有如你所希望的那样好起来,反而只是将情绪压在了心底,或者把你的话当耳旁风?
    
    本指南会为你深入剖析他们的思想状态,并且向你展示切实可行的帮助方法。这不是个例,而是一种广泛存在的症结:这\indicate{不是态度问题,而是能力问题}。对方并非如你所想“不愿积极地面对生活”,而是缺乏积极面对生活的能力。而作为帮助者,你也缺乏帮助他们处理问题的能力。如果只凭“想要好起来”而行动,那你的举动大概率对现状毫无帮助,只会使对方更加敏感脆弱。奉劝各位\indicate{不要在充分了解实际情况之前,草率地做出决策}。
    \item 社会领域的研究学者。我相信您的研究水准,故我仅在此声明我的核心观点:本指南中提到的关于一类人群状态的刻画,不应只被看做广泛存在的心理疾病与精神疾病,也不应被看做青少年不成熟的想法。这是一种稳定的意识形态,暂且称为\indicate{失格主义}\footnote{我尚未确定使用何种称呼来命名该意识形态,目前还有“排除主义”“放逐主义”“局外主义”等备选项。deepseek建议我使用“离栖主义”来称呼,但我觉得这也不贴切。}。
    
    这一意识形态的出现有两个条件,其一为发达的互联网建设与极大丰富的信息资源,其二为大量缺乏“能带来长期影响的短期收益”的人。失格主义在学生与青少年群体中十分流行,即使断绝了个体与互联网的接触,身边环境对个体的影响仍会使个体高频率接触并认同失格主义。它在社会中广泛存在,具有坚实的社会基础,绝不可能仅靠隔离而消除影响。
    
    失格主义的主要特征为“认为自己的生命和努力没有意义,在社会上很多余”,带有明显的虚无主义色彩,但与虚无主义有明显不同。其未从根本上否定意义的存在性,而是带有强烈的现实色彩,觉得自己无法融入社会,被社会排斥,或者融入社会也无法做出任何形式的贡献。
    
    客观来说这种认知有其合理性,随着科技的进步,能推动领域进步的人综合素质要求越来越高,而一般岗位所能创造的价值越来越多地被信息化所取代,无法再提供自我实现需要。经济条件无助于解决失格主义,认知方面的提升是刚需。
\end{itemize}

可能会有读者注意到,上面的三类人中不包含心理工作从业者。这有两方面原因:一方面是本指南中所展示的内容在心理学领域内并不新鲜,相当篇幅是人尽皆知的结论;另一方面是本指南采取的思路较为激进,有很强的实验性,很少在其它方面的心理书籍中看到。相关的讨论更多集中在文艺领域,而大量文艺作品充分参与了失格主义的形成。读者可能会从本指南中找到精神分析和哲学的影子,但本指南会尽量避免使用相关领域的术语,也不宣称自己是心理领域的书籍。请读者切勿自行诊断精神疾病,无论是对自己还是对身边的人。如果确实觉得自身或身边的人患有精神疾病,请前往医院寻求专业诊断。若未成年、经济、时间等条件限制而无法稳定就医,本指南在后文提供了一定的替代方案。因自行诊断而产生的一切问题,本指南概不负责。

本指南所要介绍的仅是现代社会中高质量地活下去的方法,指引读者识别、思考、处理生活中遇到的各式各样的问题。正文中的楷体字均为解释说明,包含概念辨析、解释说明、分类讨论等细节内容。若读者能力充足,或仅想大致了解某部分内容,可以跳过所有楷体部分,仅关注宋体段落。

本指南的第一部分包括第一章“事务处理的三要素”、第二章“认知的分类与从外界获取认知的过程”、第三章“复杂现象与复杂概念的组成方式”。第一部分会介绍本指南的基本分析框架,理论介绍较多,内容较为枯燥。如果对于这些内容不感兴趣,\indicate{可以先看每章的实操部分或第三章后的}\hyperref[chap:midterm]{\indicate{第一部分总结}},之后再去补感兴趣的部分。三章各自的内容相对独立,不过后文会使用前文介绍的一些概念。其中一部分概念可以望文生义,但建议最好参照前文的解释说明。

从第四章开始的部分,则是对前三章分析框架的具体应用。本指南会带领各位分析身边发生的事,【等写完后面的以后才能回来补的逐章节介绍】

在此希望各位读者,无论你想做什么事情,无论是想自杀还是想举报,我恳请各位在读完整本书以后再做。恳请各位\indicate{不要在充分了解实际情况之前,草率地做出决策}。想自杀的读者,虽然我充分尊重你的选择,但为了本指南能让更多人看到,请尽量控制一下自己的冲动。想举报的读者,虽然我充分理解您对孩子的担忧,但仍然希望您可以在看完本指南之后再决定是否举报。对于想将本指南分享给其它人的读者,请仔细判断对方的性格,尽量避免本指南会因此次分享而下架。

% \vspace{2cm}                    % 留出签名区域
\hfill
\begin{minipage}{0.3\textwidth}
    \begin{flushright}
        \authorname \\
        %于清华大学 \\
        2025年3月27日%\today
    \end{flushright}
\end{minipage}