\begin{savequote}[250pt]
    \fontsize{8pt}{12pt}\selectfont\fontsize{8pt}{12pt}
    他们说要好好听话,又说乖小孩没想法。\footnote{\bilibili{av6636086};\\\indent 
    \netease{467394170}。}
    \qauthor{哦漏QAQ《他们说》}
    我所描绘的一切或许并非真实,然而我的感受绝无虚假。\footnote{\moegirl{妄想症Paranoia系列}}	
    \qauthor{雨狸《妄想症Paranoia》}
\end{savequote}
\chapter{获取认知的过程和意识现象的组成} %标题不能过长,超过一行会无法换页,具体原因不明

\section{知识体系与理解}

我们在第一章介绍“眼”这一要素的时候,将其定义为了\indicate{从外界获取能力提升}的所有方式的总称。我们能从外界获取的东西有很多:概念、思路、方法、工具、规则......
\begin{examples}
我们有很多种不同的视角来看待这些东西:比如说按很经典的“是什么/为什么/怎么做”划分;再比如按照第一章的观点,将它们完全视作事务,可以分为“有利于调研的/有利于分析的/有利于计划的/有利于执行的”;再比如根据具体应用范围,按学科划分,或者分为“理论层面/实操层面”......
\end{examples}

\subsection{知识与信息\label{sec:知识与信息}}
本指南从\indicate{系统性}角度出发,给出以下两个称呼\footnote{类似一开始划分简单事务/复杂事务的时候,我们这里给出的定义也是不严格的:既没有定义“系统”,也没有定义“从外界获取的东西”。提前区分能简化本章的行文,方便读者理解。严格定义会在稍后补上。}:我们所有从外界获取的东西统称为\indicate{信息},而将其中\indicate{包含于一个完整系统的信息}称为\indicate{知识}。当我们将信息和知识对立时,信息会偏向强调“不包含于任何完整系统”的意思。
\begin{examples}
一条信息是否可以视为知识,因人而异。比如“任何两个有质量的物体之间都存在万有引力”,如果一个人只是知道孤立的一条,那它只就是信息;如果这个人得到了充足的物理学基础教育,知道质量、运动、力等基本概念的的基础性质,能独立地使用它们来分析涉及万有引力的现象,那他就拥有了“高中物理(力学板块)”的完整系统,这一条就可以算作知识。
\end{examples}
“获取信息”是简单的调研事务,而“获取知识”则是复杂的综合事务:它由“获取一大堆相互关联的信息”的调研事务和“分析这些信息直到综合成一个完整系统”的分析事务组成。
\begin{examples}
从短科普中了解到的概念基本上都是信息,除非你对这方面有深入研究;听到的热点新闻基本上都是信息,除非你持续跟进知道来龙去脉;考前突击学到的基本上都是信息,除非你十分有天赋能看一遍就融会贯通。
\end{examples}
依照语境的不同,我们将一个\indicate{由信息形成的完整系统}称为一个\indicate{知识体系}、\indicate{体系}、\indicate{系统}、\indicate{能力}、\indicate{技能}或\indicate{分析框架}。

\begin{examples}
一个人可以拥有多个不同的知识体系;同一信息可以属于不同的体系。“任何两个有质量的物体之间都存在万有引力”可以属于“高中物理”体系,或是体量更大的“大学物理”体系;同时,它也可以属于另一些更不一样的体系:它可以属于“牛顿的生平”体系,也可以属于“物理学发展史”体系。但如果你在写什么励志文学,“牛顿发现了万有引力”被当成一个例子来用,就不认为它属于“写作”这个体系。这时它只作为孤立的事实而存在,发挥和信息一样的作用。
\end{examples}
“知识体系”的概念很宽泛,很多看上去很不同,具有相反特征的信息集合,都可以被称为知识体系。
\begin{examples}
不同的知识体系之间可能有很悬殊的大小区别,一些体系可以完全包含于更大的体系;较小的体系也可以通过学习和思考逐步扩大,多个小体系可以结合,一个大体系可以分解;一个知识体系可以只有深度,也可以只有广度;一些知识体系可能完全由命题组成(比如数学),一些知识体系可能完全由行为组成(比如跑步),一些知识体系可能完全由事实组成(比如新闻);一些知识体系每个人掌握的都差不多(比如交通规则),而另一些知识体系不同人掌握的千差万别(比如做饭)......
\end{examples}
这些例子表明,以上所提到的都不是知识体系区别于“一大堆散碎信息”的主要特点。作为一个完整的系统,而非一大堆散碎的信息,知识体系的唯一特点是:它可以区分\indicate{哪些事务它可以解决,应该使用什么方法解决。}\footnote{实际上本指南使用这个特点来定义“知识体系”\label{para:知识体系}。读者可以自行对照,确认前文的“事务处理能力”等概念确实可以视为“能力”。}。有很多不同的表述也在说同一个意思,比如“有自知之明”“知道自己的能力边界”“谨慎”等等\footnote{具体的定义参考\hyperref[def:能力边界]{3.1.3小节}。虽然这些表述通常被用来形容一个人的特征,但这种特征是由人拥有的知识体系决定的。}。
\begin{examples}
“一个体系可以解决的事务”来源于对这个体系的有效运用,比如说“认识客观现象”、“实施最优操作”、“对比已知信息”之类。“一个体系不能解决的事务”相对来说特征更为难以刻画,主要有一下这么几类:1、和体系完全无关的事务,比如跑步能力不能用于写歌;2、与体系有关,但缺少解决方法的事务,比如数学中的猜想;3、体系预测与现实情况相冲突的事务,比如说考试中的错题。
\end{examples}
一些知识体系有清晰明确的\indicate{标准}。\indicate{标准}也由相互关联的知识组成,个人可将自己的知识体系与之对比。如果存在大量“标准要求解决,但个人能力无法解决”的事务,则将这种情况称为(能力/知识体系/分析框架)\indicate{不足}/\indicate{缺失}/\indicate{缺陷}。有些时候我们将标准也称为\indicate{能力}或\indicate{知识体系}\footnote{注意此种用法的“能力”不因人的不同而有不同,它是一个固定的标准,具体展开见\hyperref[def:知识体系]{3.1.4小节}。因人而异的部分是“不同人能力缺失程度不同”。}。在结合上下文的情况下,读者应该能够区分“能力”的两种用法。

\begin{examples}
标准可以有多种来源。其可以使某门学科内部的共识,比如说生物学、经济学、建筑学;可以是某份公开且公认的文件,比如成文法律、规章制度、考试大纲;可以是客观事实本身,比如跑步速度、社会热点话题的整体时间线、案件侦破中的现场实际情况;可以是来自个人(或机构)的界定,如一本教材,或本指南对于“事务处理能力”的定义。该定义及其衍生的“调研能力缺失”等概念仅在本指南讨论范围内生效,不保证在其它讨论环境下仍然是清晰明确的标准。其它讨论环境下,同一词汇可能有不同含义。\footnote{更一般地,对于所有不来源于公认文件或客观事实的标准来说,脱离它的讨论环境时,都有可能产生歧义。具体展开参见\hyperref[sec:能力边界]{3.1.3小节}。}
\end{examples}

每有一个知识体系,或者一个标准,就会有它可以解决的\indicate{一类事务}\footnote{这里仅做描述,以方便读者理解。关于如何识别一类事务,请见下一小节}。这是本章的核心视角:对一类事务的统一研究和处理。

我们将“处理某一类事务的具有共性的一系列操作”称为\indicate{方法}或\indicate{流程}。如果这个方法能够解决这一类事务,则称为(这类事务的)\indicate{解决方法};如果需要强调“这个方法不一定能解决这一类事务,有明确的失败可能性”,则称为(这类事务的)\indicate{处理方法}。

方法不负责“判断某个事务能否属于自身的处理范围”。这种判断可以被视为另一种事务(调研事务或者分析事务)。我们将“判断某一个事务是否属于这一类事务”的解决方法称为这一类事务的\indicate{识别方法}。

\begin{explain}
一类事务的识别方法和解决方法可以组成一个完整的分析框架(知识体系)。如果缺乏解决方法,识别方法和“自身能力不足以处理此类事务”的判断也可以组成一个完整的分析框架。这两种搭配是最简单的分析框架,大部分分析框架具有更为复杂的结构,难以分解为清晰明确的“识别方法”与“解决方法”两部分\footnote{读者可以参考“调研事务”和“计划事务”的嵌套。识别方法与解决方法也有同样的嵌套。}。
\end{explain}

\subsection{识别方法与判断\label{sec:识别方法与判断}}

任何一个识别方法都由一些判断组成。我们将一个\indicate{一般疑问句}称为一个\indicate{判断}。
\iffalse 或\indicate{命题}\footnote{\rigorous 可能会要求命题可以判断正误,但这和个人具体拥有的知识体系相关,不同的人面对同一条判断时,有的人可以判断有的人不能判断,故在此不做这方面要求。}
\fi
判断同时可以作为动词,意为“回答这个一般疑问句”,其结果仅可能有三种:“符合该一般疑问句的描述”、“不符合该一般疑问句的描述”、“无法判断是否符合描述”。我们分别使用“真”“假”“不可判断”来称呼。
\begin{examples}
判断有很多种不同的类型。有一些涉及事务的某方面特征;有一些涉及事务的成因;有一些预测未来的发展......多条判断可以结合,以得到更详细的结论:比如“一种分类方式”就由好几条判断组成,其中恰有一条为真。
\end{examples}
人在判断时得到的结果,不一定就是真实情况。“不可判断”的结果必须由人主动给出,我们将这种“判断结果与真实情况不符”的情况不视为“不可判断”,而是称为\indicate{判断失误};而将“判断结果与真实情况相符”的情况称为\indicate{判断准确}。一个人在某个具体问题上是否判断准确,有些时候是不可得到的信息\footnote{可以在这里附加一个“判断自己的判断是否准确”的识别方法来套娃,但这样的套娃无穷无尽,套娃本身没什么特别的意义。}。
\begin{explain}
如果拥有足够的推理能力,识别方法足够强大,一个人可以尽量避免判断失误,并且修正自己的判断失误。常见的修正如“一段文字看到后面才意识到自己有个字看错了”。数个判断结合起来也可以修正一些判断失误,比如说“作业本在学校”和“作业本在家”不可能同时成立,一定有一个是错的。结合判断以修正判断/得到新判断的能力是分析能力的一部分,这可以通过一些手段培养,比如说接受逻辑学教育、阅读推理小说、观看剧情解析。
\end{explain}
解决一件事务的定义为“从当前情况出发,执行一系列操作,以达成目标”\footnote{这和之前的定义仅有用语上的差别,完全是一回事。}。据此,我们将\indicate{一类事务}定义为\indicate{当前情况具有共性,目标也具有共性的很多个事务组成的集合}。这里的\indicate{共性}由识别方法来判断。

不同的判断之间可以相互结合,通过分析以得到进一步信息,这些信息有助于下一步判断。对于“某个事务是否属于这一类事务”这个判断而言,如果存在一些其它的判断,将它们结合的信息足以使得在“判断为真”和“判断为假”时都不出现失误,那么我们就称这些判断为这一类事务的\indicate{识别方法}\footnote{对于\rigorous,该定义的严格表述为:如果对于某一类事务A,存在一个判断组成的集合S,使得对于任何事务a,“a是否属于A”这个判断都属于S,且在“判断为真”和“判断为假”时都不出现失误,那么就将S称为A的一个识别方法。}。识别方法是分析能力的重要组成部分,“使用识别方法判断”可以视作一种调研事务。
\begin{explain}
读者不难验证,这个定义和之前对于识别方法的定义是一致的。这个定义不允许“把不属于的事务判断成属于”、“把属于的事务判断成不属于”、“在无法确定的情况下以为自己能判断”,但允许谨慎地将很多事务分为“无法判断”,甚至允许将所有事务都分为“无法判断”。准确判断为“属于”或者“不属于”的事务越多,说明该识别方法的识别能力越强。

一些读者可能觉得识别方法很像是在玩\indicate{海龟汤}——还真没错,除了“没有主持人,你得自问自答,很多时候也没有汤底揭晓环节”以外,整个流程和海龟汤完全一致。
\end{explain}
很明显,对于“某个事务是否属于某一类事务”,当我们判断为“属于”的时候,我们就会用对应的解决方法来解决这个事务;当我们判断为“不属于”的时候,我们就不会用对应的解决方法来解决这个事务。而当我们判断为“不确定”的时候,我们也可以使用对应的解决方法,但此时这就只是一次\indicate{尝试}。尝试不保证成功。
\begin{explain}
    其它的文章可能有不同的表述。比如说很多人会有“我知道没希望,但我就是不甘心,想试一试”的表达,这种情况在本指南中被视为“不确定事务是否可以被某方法解决,并且尝试”。
\end{explain}
\begin{explain}
    同时,本指南不将“以前从来没成功过”视为“自己不具有解决的能力”的有效判断依据,而很多“我知道没希望”的说法其实表达的意思是“我以前从来没成功过”。在阅读本指南过程中,请优先使用本指南中给出的定义去理解涉及的概念。
\end{explain}
尝试的价值有两方面:一方面是可能会成功解决事务;另一方面,尝试很有可能带来有用的信息(这表明尝试也是一种有效的调研方法),从而增强下一次的识别能力。
\begin{explain}
    无论尝试成功还是失败,都有可能带来有用的信息。不能带来信息的情况有两种:一种是事务和方法过于复杂,而分析能力和已知信息不足,以至于无法有效处理;另一种是没有分析的意识,不主动去总结反思。大多数人的“不会总结经验教训”其实是前一种,但经常会被当成是后一种。
\end{explain}

\subsection{方法与过程\label{sec:方法与过程}}
我们将\indicate{一个原因和其对应的结果}统称为一个\indicate{事件}。我们也将原因称为\indicate{起因}、\indicate{前提}、\indicate{条件}、\indicate{前提条件}或\indicate{触发条件}。我们将\indicate{当前情况改变,满足相应原因,一个事件发生}称为\indicate{触发}这个事件。
\begin{explain}
和事务类似,事件也有复杂的递归和嵌套。几个简单的事件,如果前者的果是后者的因,就可以串联拼接成更复杂的事件;如果好几个果共同组成了另一个事件的因,就可以并联拼接成更复杂的事件。因此,在后文中,我们不区分“一个大事件”和“组成它的很多小事件”。我们将所有小事件前提条件的总和视为大事件的前提条件,把所有小事件结果的总和视为大事件的结果。
\end{explain}
我们将\indicate{一类原因和结果都具有共性的事件}称为\indicate{过程}\footnote{前文中提到的所有“过程”都是这个含义。},将“这一类事件中的一个”称为该过程的一个\indicate{实例},将\indicate{共性在实例上的具体体现}称为\indicate{特点},将\indicate{人能察觉到的特点}称为\indicate{信号},并将此称为“信号\indicate{刺激}了人”或“信号给人提供了\indicate{刺激}”。

我们称前提条件或者过程\indicate{导致}了结果,并将前提条件或者过程称为结果的\indicate{来源}。
\begin{examples}
在识别方法中,一步操作是“通过当前已有信息,得到判断的结果”;我们之前曾经将子事务称为母事务的一步操作,此处的“操作”可以视为“人主动选择了一个过程,触发相应的事件”,因为“解决子事务”可以“固定地实现子目标”;更一般地,“解决事务”本身也可视为过程。
\end{examples}
过程的唯一要求是“确定性的因果关系”\footnote{本指南不讨论“因果关系是否存在”的哲学话题,默认因果关系客观存在,但人认知到的不一定正确。对于\rigorous,本指南将所有“因果关系可能不存在”的情况全部视为“没有正确判断特点,从而无法在完全理论的环境下分析”,具体展开参见\hyperref[sec:概念与特点]{3.1节}。换句话说,本指南承认因果关系存在;人可以猜因果关系但不保证猜中;人可以交流自己猜的因果关系但不保证对面能听懂。}。如果具有某种共性的原因,总是会导致具有某种共性的结果,那这就可以视为一个过程。
\begin{examples}
    与事务不同的地方是,过程不要求“主动解决”,而是可以自然地发生。以下提到的都可以视为过程:完全的自然现象,如“水往低处流”;机械的运动规律,如“车燃烧汽油获得动力”;计算机的运行结果,如“应用程序的文字识别功能”;条件反射或肌肉记忆,如“流水线工人的熟练操作”;人/AI的事务处理。以上这些例子含有不同程度的智能和主动性。
\end{examples}
一些学科尽可能广泛考察并深入研究了具有某种特点的所有现象,这样总结出的过程足以被称为\indicate{规律}。但日常生活中的事特点太多,难以对每种共性都展开广泛深刻的分析。这时,“判断特点并总结共性,以认识过程”的过程,就只能由每个人自己来完成了。
\begin{explain}
    在提炼共性时,将一个现实事件按其特点分解成多个方面,每个方面分别分析,经常是一种有效的方法——控制变量法。仅考虑一个方面的分析通常会更简单;在一个方面得到的结果,也经常可以作为信息,辅助其它方面的分析。
\end{explain}
我们将\indicate{触发该过程的某一个实例}称为\indicate{触发}该过程。如果某个事务的当前情况符合某个过程的前提条件,同时这个过程的结果也是这个事务的目标,我们就能主动触发这个过程,以解决该事务。我们将“可以主动触发的过程”称为这个事务的一个\indicate{力所能及}的步骤/操作。

不管一个人有没有认识到某个过程,过程都客观存在。但是在处理某类事务的方法中,如果要主动选择一个过程作为操作,肯定需要先认识到这个过程。如果认识到的过程并不准确,那么这一步操作就有可能出现预料之外的结果,从而整体的方法就称不上解决方法,而只能是处理方法。
\begin{explain}
即使认识不够准确,也可能每次都能成功处理事务。这种情况应该被视为运气好,而不能视为掌握了解决方法。一个人总是成功处理某类事务,是因为运气好还是因为掌握了解决方法,不总能判断。但当一个人处理某事务失败时,可以判断这个人在当时没有掌握解决方法。一个人如果处理某事务总是失败,可以判断这个人没有掌握解决方法。
\end{explain}
解决方法可以使我们稳定地利用已知的过程,并且组合这些过程,来解决眼前的事务。触发已知的过程是执行能力的核心,组合这些过程是计划能力的核心。
\begin{examples}
掌握了某种解决方法,不一定就意味着有机会用它来解决实际事务,因为有的时候前提条件不具备。巧妇难为无米之炊。在沙漠中心不可能去某家便利店买水,虽然这在城里是可行的解决“口渴”的方法;在大海上没有手机信号不能打电话,虽然这在陆地上是可行的解决“思念”的方法;离开了某个岗位就不能调用某些资源,虽然这在担任这个岗位的时候是可行的解决“任务”的方法;一个小时后就得用的工具不可能网购,虽然这在时间充足时是可行的解决“需求”的方法。
\end{examples}
主动选择的过程仅需要确认前提条件和结果即可,不需要了解中间的具体步骤。这在某些语境下会被叫做\indicate{黑箱}。
\begin{examples}
可以将某个环节完全委托给另一个人做,常见的比如说约稿、外包项目、雇佣;可以借助某些机器的功能,使用冰箱的时候只需要知道它可以提供低温就行,不需要会制冷;可以只靠等待让事情发展到下一阶段,比如排队、加工、鹬蚌相争;可以只执行步骤而不懂每一步的具体原理,比如说按照教师要求练习、按操作使用机器;可以下意识地使用练习过的技巧,如游泳、绘画、歌唱。
\end{examples}
解决方法的子操作中,可以包含\indicate{尝试}。如果对于尝试的每种可能结果,都有下一步处理方法,都能最终实现目标,那么这当然也解决了事务。
\begin{examples}
去某个机构办理业务的时候,如果人生地不熟就有可能跑错,但一般也能在跑错的地方打听到正确的地点;考试没有通过可以明年再考、转换方向或者尝试就业;普通票抢不到的话可以考虑加钱。尝试有可能是因为信息不足,有可能是因为目标不单一:某种不保证成功的尝试有更大的收益,比如说更节约时间、更省钱、前途更好等等。
\end{examples}

\subsection{理解}
对于一个现象,如果一个人能\indicate{找到一个符合现实情况的过程,导致了该现象},那么就称这个人\indicate{理解}、\indicate{了解}或\indicate{归因}\footnote{有些读者可能会担心无限归因的问题。在本指南的定义中,对原因再进一步归因不算做对原现象的理解。理解仅需考虑当前现象即可。}了这个现象,将过程或者这个过程的原因,称为这个现象的\indicate{解释}或\indicate{归因}。如果\indicate{发现}/\indicate{知道}某个现象(掌握了对应信息)但没有理解,那只能称为\indicate{承认}或者\indicate{接受}。
\begin{examples}
并不是所有现象都能被理解。一些学科中所使用的基础概念,比如说经典物理学中的“质量”,就没有什么来源可言,我们只是承认了“物体都具有质量”“牛顿第二定律”等现象,并且加以应用。

相对地,物理学史中“物理学家定义了质量”“物理学家区分了质量和重量”等现象,有着清晰明确的来源:早期物理学家对日常现象和实验结果进行了总结归纳,从而提出了这些概念。

本指南中的“理解”不直接包含很常用的“理解一个概念”的含义。该含义在本指南中对应着“确认概念可以应用的边界”,请参考2.2.3节中关于理解和控制的定义。
\end{examples}
如果一种前提条件有可能导致多种结果,而某个现象只是这些结果中的一种,那么“找到这种前提条件”不能算了解了“这个现象”。但如果将“多种可能的结果”视为一种更大的现象,那么“找到这种前提条件”就可以算了解了这个更大的现象。
\begin{examples}
理解“涵盖所有可能结果的现象”有时候没什么用处,经常给人感觉像是废话一样,比如说“考试有可能考好也有可能考差”“目标有可能实现也有可能无法实现”之类的。但是在一些情况下,“只知道一个结果,而不知道其它结果”会造成重大的理解障碍,如“复习了就一定能考好”“努力了就一定能实现目标”。这样的错误理解会使得在制定计划时产生重大差错,而自己却无法发现具体错误。

“某个原因所有可能的结果”是一个基础但重要的信息,理解“所有可能的结果”比理解“其中一种结果”要容易。具体的结果需要更多前提条件才能理解。“什么样的附加前提条件才能导致一种确定的结果”很多时候是不可得到的信息。
\end{examples}
即使某些成对的现象总是先后发生,也不一定就能确定它们之间就有确定性的因果关系。之前的现象可能不足以导致之后的现象。但如果归因中包含背景,那么“之前的现象+背景”就可能足以导致“之后的现象”。
\begin{examples}
“种下橘子树,就能结出饱满鲜甜的橘子”不算完整的归因,还需要附加“在南方”的背景条件;“吹了空调,就会肚子疼”不算完整的归因,还需要附加“在自己身上”的背景条件。

将背景视为归因的一部分,相对来说不够深入,但得到的理解确保准确。当遇到超出背景范围的情况时,总是应该谨慎地重新审视新情况。“完整的背景”很多时候是不可得到的信息。
\end{examples}
只要理解了某个现象,那么就拥有了改变的可能。通过改变前提,可以使对自己不利的现象不再发生,也可以使凑出对自己有利的现象。当然,这一切建立在“自身有能力改变前提”的基础之上。不同的归因会使得需要更改的前提不同,从而难度和需要的能力也不同。
\begin{examples}
归因到环境上时,很多时候无法主动选择脱离环境来解决不利现象。在沙漠里口渴,也只能出沙漠再说;在大海上没手机信号,也只能靠岸了再说;北方种不出橘子,但土地是搬不走的;体弱不能吹空调,但身体是没法换的。
\end{examples}
为了有效地改变,或者更一般地说,为了保证能解决一类事务,很多时候我们只关注那些力所能及的归因。换句话说,所有的解决方法都会认定某个方面,并且只归因到这些方面。即便是有完善体系和公认标准的学科也是如此。因此,在特定语境下,“理解”也仅指“可以将现象归因到特定方面”。
\begin{examples}
相比于真实世界的复杂,物理学只研究有限的几个物理量,而更具体的部分被分给了各个不同的工程学。如果在中学物理中问“为什么桌子可以稳稳地放着”,回答“因为木工师傅的手艺很好”就不是有效理解。

相比于人类的复杂行动,经济学只研究经济的衍生现象,而其它更具体的假设衍生出了很多不同的经济学分支。在古典经济学中不总是能问“一个人为什么想要某件东西”,古典经济学不负责给“需求”归因。
\end{examples}
在本指南中,\indicate{理解}仅指\indicate{将现象归因到了力所能及的原因上}。根据各位读者的学识和能力不同,不同人力所能及的范围会很不一样。但对于意识活动来说,情况比较确定:我们不会将意识归因至生理活动上。你无法控制自己分泌多少激素,也无法使自己的心跳停跳。大多数人无法控制自己的情绪产生,最多只能控制自己情绪的影响。我们也不会将意识活动归因至环境上,大部分事情无法靠改变环境来解决,你也无法确定自己脱离了\indicate{一个}环境后就不会再遇到\indicate{同类}环境了。
\begin{examples}
我们没办法一个念头就增加肌肉、减少脂肪。想要做到这些,只能通过锻炼。身体如此,意识也一模一样。
\end{examples}
% 本指南将人的一切主动行为全部视作事务处理,这是我们的意识本身能掌控的边界。
意识活动的具体组成将在下节中讨论,我们将逐项研究意识中可控的部分,据此具体刻画出我们力所能及的范围。

\section{意识的层次\label{ref:意识的层次}}
我们的身上时时刻刻都在发生很多\indicate{过程}。有一些过程是自动的,我们无法主动控制,比如心跳、消化食物、免疫病原体;有一些过程也可以虽然自动,但想控制的时候可以控制,比如呼吸、眨眼、咽口水\footnote{如果这导致读者切到了手动挡,本指南深表抱歉。};有一些过程虽然一开始是有意识控制的,但熟练了就变成自动的了,比如打字、打游戏、流水线工作;有一些过程一直得有意识地做,比如做题、写文书、谈判。

\subsection{行为}
依照语境和具体限制的不同,我们将\indicate{一个人身上的过程}称为这个人的\indicate{行为}\label{def:行为}、\indicate{反射}、\indicate{习惯}、\indicate{瘾}、\indicate{条件反射}、\indicate{肌肉记忆}或者\indicate{下意识的}行为/举动/.....。%\indicate{自动发生}指的是\indicate{在过程中没有额外的主动触发}。
\begin{explain}
这种用词可能比较奇怪,毕竟我们确实不常见到“一个人有呼吸心跳的习惯”之类的说法,这听上去不像人话。“习惯”“条件反射”“肌肉记忆”看起来有明确的“从外界习得”的意思。

本指南不使用“从外界习得”来定义“行为”,是因为我们很多时候分不清“这个行为到底是从外界习得还是天生就有”,经常会产生错判。我们能确定的只有“某个过程会稳定地发生”。这词就凑合用一用吧。

脱口而出的一句话,突然涌上心头的情绪,看到问题就想到的思路,这些都被我们算作行为。行为的定义仅有“是一个过程”,偏向生理和偏向心理的都算。
\end{explain}
绝大多数行为都是后天习得的。其中有些是主动培养的,有些是被动受环境塑造的。本指南有时会倾向于将后天习得的行为称为\indicate{习惯}。
\begin{examples}
本指南不做过多的认知科学讨论,不从很深入的神经可塑性角度出发论证这一观点。如果读者对此感兴趣,请参考《刻意练习:如何从新手到大师》或更深入的书籍;也不很深入地研究具体案例,如斯特拉顿的翻转视觉实验,或@SmarterEveryDay的自行车握把翻转实验\footnote{原视频链接:\url{https://www.youtube.com/watch?v=MFzDaBzBlL0};中文翻译:\url{https://www.bilibili.com/video/BV1zx411m7cx}。}。本指南仅概括性地引用结论。
\end{examples}
\label{para:记忆}概括来说,我们能培养出行为,是因为人有\indicate{记忆}。当我们重复经历某一过程的时候,之前的经历会作为额外信息,辅助我们计划和决策。这使得我们的思考越来越快,越来越熟练,直到最终,只要察觉到某个信号,就会完全自动地作出一系列操作,决策也就变成了习惯。
\begin{examples}
相反地,如果在重复中无法参考之前的经历,那就无法培养出行为。这可能有很多原因,比如无法理解或者错误地理解了之前的经历;比如注意力在别的方面,没意识到这是相同的现象/没想起来之前的经历;比如没有找到可以主动控制的环节,过程自动地结束了;比如空闲时间没有思考分析,实际发生的时候来不及现场判断......
\end{examples}
如果在一个行为触发前,我们进行了分析\footnote{分析的定义见\hyperref[def:分析]{2.2.3小节}。},并在这之后决定触发这个行为,那么就称这个过程为\indicate{(主动)控制}或者\indicate{过脑子},反之则称为\indicate{自动发生}。我们将“一个人主动控制的(无论自身还是外部的)过程”称为这个人的\indicate{决策}\footnote{对于\rigorous,“一个决策”指的是“主动触发一次行为”而不是“一个可控的行为”。}。
\begin{explain}
我们察觉到的信号不一定是完整的归因。环境变化后,做同样的操作不一定能得到同样的结果。是否触发一个行为很多时候由“是否察觉到信号”决定,而是否触发决策一定经过了“是否能得到结果”的判断。
\end{explain}
行为有可能对我们有益,也有可能对我们有害。它自动发生的时候是不可控的。当环境变化的时候,原本的益处可能变成害处,原本的害处可能变成益处。
\begin{examples}
熟悉了一款游戏的键位,换到另一款游戏上的时候就会出问题,尤其是那些不太经常使用的功能快捷键或者连招,会更容易按错;熟悉了和一些人的交往模式,换到另一些人身上就很可能触霉头,有可能把别人惹到了,自己却还一无所知;甚至那些纯粹的生理过程也会在某些环境下带来麻烦,比如受伤的时候心跳反而加快从而失血增多;细胞因子风暴的致命性可能远大于病原体本身。
\end{examples}
一些行为可以作为知识体系的一部分,几乎每一次触发都是自动发生。
\begin{examples}
打(动作)游戏的时候,除了新手期,不可能每按一个键都过脑子。要是按键不靠肌肉记忆的话,你根本跟不上游戏的节奏。阅读的时候,你肯定不会每个字都想上好几秒才能知道意思。刚开始学一门语言的时候可能这样,但这样根本没法读下来一整本书。走路的时候,你完全不是在随时控制每根肌肉何时收缩,走路这个动作在婴幼儿时期就已经练会了。
\end{examples}
我们之前提过,\indicate{理解}一个现象,是指找到一个真实的过程,导致了该现象。对于行为来说,有两种现象需要理解:一种是“为什么会养成这个行为”,一种是“每一次行为为什么会被触发”。\\
理解“自己养成了一个行为”,也就是找到自己过去养成行为时,所追求的目标。
\begin{explain}
由于我们不总能知道自己的行为是否是后天习得的,这个问题不一定能找到答案。但对每个行为都思考一下“目标是什么”总是有帮助的,至少总不会比“从来没想过”更无知。
\end{explain}
理解“行为的每一次触发”,也就是需要给行为归因。按照本指南的约定,我们需要将其归因到事务处理上:只有能够主动控制能否触发,才算是理解了行为。如果只是找到了无法避免的原因,那只能算是\indicate{发现}/\indicate{知道}了这个行为(此时我们称这个行为为\indicate{自知的}行为),而不能算是理解。
\begin{explain}
有些行为不太好界定是否可控,比如说“看到了某个现象以后想到需要做的事”,这里前半段的“看到”不是决策,而后半段的”想到“是决策。此时我们应该将其拆成两部分分别看待。

如果一个行为总是不会触发,我们也认为自己理解了它的每一次触发,认为它是可控的。这在后续的分析中会带来便利。
\end{explain}
我们将“理解行为的触发”改称为(行为)\indicate{可控},而仍然将“理解行为的养成”称为\indicate{理解}。如果既理解又可控,那么就称为\indicate{掌握}了这个行为。
\begin{examples}
我们总是掌握\indicate{决策}所对应的行为,至少在决策后的短时间内如此(直到忘掉自己的思考过程)。我们不掌握绝大多数生理反应,理解生理反应需要相应的生物学/医学/认知科学知识,大多数生理反应不可控制。

注意,“理解”和“可控”都不包含“知道某个行为有什么用”的部分。这是因为行为具体有什么用,是好是坏,取决于具体环境。我们应该使用知识体系来判断,然后进行决策。%即使是很简单很初等的过程,也会在不同情况下有不同的作用。比如“心跳”,正常情况下其可以“为了维持血液流动,给器官提供物质和能量”,但有大伤口的情况下,反而会使人失血过多,加速人的死亡。
\end{examples}
即使某些行为既无法理解也不可控,也不代表它一定有害。
\begin{examples}
我们认识成千上万的字,每看到一个字,“想起它的含义”都可以当成一个行为。你肯定不记得自己是怎么学会每一个字的(只是可能会对少数几个字很印象深刻),最多有统一但模糊的“被父母老师教”的印象;每次阅读的时候,也肯定不是“先决定自己要看懂一篇文章,然后再一个字一个字想含义”(最多对少数生僻字会回忆一下)。当看到一段文字的时候,你就已经开始了阅读,触发毫无疑问来自外界,不受自己控制。

但会识字比不会识字要好太多了。“识字和阅读”是一个知识体系,这些不可控的底层行为为我们提供了可控的“读书”“理解句意”等完全可控的行为作为能力。真遇到少数情况,我们也能快速地纠正自己的误用。

而相反,当你初学另一门语言,掌握的词汇和语法还不足以构成完整的知识体系的时候,尽管你更会去有意识地想每个词的含义,但整体看来这不可控,会使你无法正确表达自己的意思,可能造成严重的沟通障碍。
\end{examples}

\subsection{行为模式\label{sec:行为模式}}
\noindent 我们将一段“连续触发的行为”称为一个\indicate{行为链}。%之前定义中提到的\indicate{稳定}是指“会有很多相关的行为和决策连续触发”。
\begin{examples}
这里的“连续发生”有可能是因为外部环境会持续地提供刺激,比如“嘈杂的周边环境”“不断到来的工作”“连续的办事流程”“无所事事”之类;也有可能是外部的刺激引起了自身强烈的反应,比如“做完事情以后的成就感”“突然遇到crush的一见倾心”“被恐怖电影吓得一晚上睡不着”之类;也可能和外界没什么关系,纯粹是自身内部引起的,比如“焦虑感”“抑郁感”“死亡冲动”之类。在“一个高度重复的环境”等极端情况下,一个固定的行为被反复触发,也可能组成一个行为模式。
\end{examples}
每一种环境中都有很多种特点,这些特点中的一些会被我们作为信号察觉到,从而触发行为,引起分析和决策。我们将\indicate{一个人在某个环境下所有可能发生的行为链}称为这个人在该环境下的\indicate{行为模式}\label{def:行为模式}、\indicate{状态}或\indicate{侧面}。
\begin{explain}
不同环境下的行为模式,会有相同的部分和不同的部分。在所有环境或大多数环境下都不变的行为和决策,很多时候会被叫做“性格”或“情绪”(但是按本指南的定义,会出现“心跳是性格的一部分”之类如果不是在青春浪漫文学中就没在说人话的句子。所以本指南没有引入这个概念)。
\end{explain}
我们将“对应的行为链正在触发”称为一个人\indicate{处于}这个行为模式中。我们将“之前不处于,之后处于”称为\indicate{触发}或\indicate{进入}这个行为模式,将“之前处于,之后不处于”称为\indicate{打断}\label{def:打断}或\indicate{脱离}这个行为模式,将“脱离一个行为模式,进入另一个”称为\indicate{切换}行为模式。
\begin{explain}
进入行为模式有可能通过触发某个行为达成,也有可能没有明确的行为触发;脱离行为模式有可能是因为触发了新的行为模式、触发某个行为/决策从而打断了当前行为链、所有行为链都自然结束等多种情况。
\end{explain}
我们可以轻易在自己身上找到成百上千个行为,但行为模式却没那么多。一个人拥有的独立行为模式,通常只有几个或者十几个。
\begin{examples}
虽然我们会遇到很多不同的环境,但这并不意味着每种环境下的行为模式不同。例如,对很多人来说,无论是在听课,还是在等车,还是晚上躺在床上,还是一段闲暇的午后,所面对的环境都和“时间空闲且有手机”没什么不同,只会在一些细节上(比如说上地铁需要过安检,而其它场合没这回事)有少量的区别。

一个行为模式内部通常包含多种行为链,根据具体环境的不同,触发的行为链也会不同。
\end{examples}
行为模式也具有和事务、过程类似的嵌套与递归。如果算得很仔细的话,行为模式也可以有成百上千个。但我们一般不去研究不同行为模式之间的细微区别。
\begin{examples}
我们可以将“做题”算作一种统一的行为模式,也可以将每一科的题目单独算作一种行为模式;面对自己无法处理的事务的时候,除了“尝试”或“放弃”以外,同时还有可能有“焦虑”“迷茫”“好奇”等多种子行为模式;“休闲”是一种统一的行为模式,每个游戏、软件或者其它项目都有对应的子行为模式。
\end{examples}
一个行为模式可以完全由知识体系组成,所有的行为全部是清晰的决策;也可以完全由不自知的行为组成,所有的行为全部是下意识反应。但更多情况下,行为模式是这两方面的混合。
\begin{examples}
在尝试处理某件事务的时候,越是能意识到眼前事务的专业性和困难性,行为模式就一般越偏向于决策(一个人在这种情况下会越认真),比如“考试”这个环境下。而偏向于行为的可能性有很多:比如眼前的事务不困难,自身已经能熟练解决;又比如缺少对应的知识体系,无法决策,只能慌张。通过决策来拆分事务,使得低层次的步骤可以熟练解决,是混合组成行为模式的常见方式。
\end{examples}
% 不同于习惯或决策,过程或事务,行为模式不一定有“明确的一类结果或目标”。习惯的结果和决策的目标不一定能上升到行为模式层次。
% \begin{explain}
% 行为模式可能没有什么目标(只有很多习惯和决策在互相触发),可能有很多个不统一的目标(这些目标有可能是不同方面的,相互之间可能相容,也可能有冲突),也可能确实存在一类明确的目标(比如纯粹由知识体系组成的行为模式,一切以“解决某一类特定的问题”作为目标)。
% \end{explain}
类似于行为,对于行为模式来说,也有两种现象需要理解:一种是“为什么会形成这个行为模式”,一种是“每一次为什么会进入某种行为模式”。类似于行为,我们将前者称为\indicate{理解},将后者称为\indicate{可控}。

\indicate{理解}一种行为模式(无论是自身的还是他人的),需要发现其所包含的所有行为(不要求理解每一个行为),并且分析清楚这些行为和决策如何组合和接续,以形成行为链。
\begin{explain}
一般来说,理解一个行为模式的难度要远高于理解一个行为。尽管我们不需要在理解了每一个行为之后才去理解行为模式,但仅仅是发现所有相关的行为,就往往已经比透彻地理解一个行为更复杂,况且之后还有“搞清楚行为和决策之间的组合”的步骤。
\end{explain}
一种行为模式是\indicate{可控}的,是指“其包含的每一种行为链中,都有一个可控的行为”。注意这并不需要发现所有的行为链才能做到。我们将“行为链发生时,控制那个可控的行为”称为\indicate{控制}这个行为模式。
\begin{explain}
一个可控的行为在行为模式中的价值,取决于“它还会触发多少个行为”。如果可控的行为模式是行为链中的最后一环,那控制它有可能没什么用。

不采用“控制每个行为链的第一个行为”的定义,是因为人很难完全确定自己的行为链都包含哪些行为,由哪个行为开始。这种定义没有实际价值。

不采用“在重要的事件之前控制一个行为”的定义,是因为这样还得额外定义“重要的事件”,而这个因人而异。如果读者想要更有实际价值的定义,可以根据自己的实际情况确定重要的事件,并且据此定义“可控的行为模式”。这种定义替换在本指南中不会产生任何论述差异。
\end{explain}
如果只是总结出了“自己在某些环境下的状态”“自己的性格特点”之类的事情,则不算理解了行为模式,只算\indicate{发现}/\indicate{知道}了这个行为模式(此时我们称这个行为模式为\indicate{自知的}行为模式)。
\begin{explain}
理解一种行为模式需要自知作为前提条件,但控制一种行为模式则不需要。我们可以巧合地控制住一种行为模式的每一种行为链。这与行为的情况略有不同,理解和控制一种行为都需要知道这种行为。
\end{explain}
如果既理解又可控,那么就称为\indicate{掌握}了这个行为模式。
\begin{explain}
我们总是掌握\indicate{知识体系}。知识体系的要求比“可理解且可控的行为模式”更高,在知识体系中,我们要求其中每一个行为要不然是决策,要不然在触发后能发现做得是否合适。
\end{explain}
即使某些行为模式既无法理解也不可控,也不代表它一定有害。
\begin{examples}
当你初次接触某一门课程时,你一定不知道每一个概念、每一步操作在这项技能中到底有什么作用(尤其是一些应用类课程)。你可能在学到一定阶段的时候突然开窍,但也有可能直到结课的时候依然只会死记硬背。即使是只会死记硬背,一般也能通过考试,老师不会出什么真正超纲的内容;并且有时候也能顺利完成工作,只要不总是被分配到超出能力范围的事情。

当外界有其它人、机构或系统掌握了一个行为模式时,就可以有意识地培养一些拥有这一行为模式的人。在可控的环境下,这不会有什么问题。即使没有掌握后的人在有意识培养,如果运气够好,外界环境仍然有可能正好适配这一行为模式。

而相反,当外界环境变化时,这种行为模式就有可能出问题。当面对更复杂的环境,或者更深入的知识的时候,就有可能完全解决不了眼前的问题,并且也完全找不到原因。
\end{examples}

\subsection{思考回路、认知、决定}
我们将起因和结果都与外界环境无关的行为称为\indicate{想法},并将由想法组成的行为链称为\indicate{思路}。
\begin{explain}
如之前所说,如果一种行为涉及到了外界,那么可以通过将其拆成几个组成部分的方式,从中分离出想法来。那些不属于想法的行为,总会包含“通过感官从外界获取信息”或者是“根据自己的决定去行动”的环节。将它们分离出去以后,就得到了只在意识内部发生的想法。

我们不一定能意识到自己都思考了些什么,尤其是当思考已经高度熟练的时候。福尔摩斯能轻松判断出一个人的身份,但华生问他为什么的时候,反而得思考好半天。我们的任何一种情绪背后都有可能有相当复杂的思路。
\end{explain}
我们将每一行为模式中\indicate{由想法组成的子行为模式}称为其所对应的\indicate{思考回路},将触发一次思考回路称为一次\indicate{思考}。
\begin{explain}
不是所有的想法都能组成思考回路,很多想法只短暂存在于一套行为链之中。不是所有的行为模式都有对应的思考回路。运动、电子游戏、体力工作等行为模式可以没有思考回路。对应地,也有一些行为模式可以完全或几乎完全就是思考回路,如一些知识体系、所有识别方法,或者一些情感。
\end{explain}
我们将思考回路的得出的结果称为\indicate{认知}或\indicate{认识}。
\begin{explain}
认知是一种\indicate{信息},而应用认知来分析是一种\indicate{行为}(可能是决策)。当我们谈论认知是否可理解/可控时,我们相当于在谈论“应用认知”这个行为是否可理解/可控。一个认知\indicate{可理解},是说自己明确知道这个认知的形成过程;一个认知\indicate{可控},是说自己明确知道这个认知的应用范围,并且总是根据应用范围来决定是否使用该认知。

信息不都是认知,有些信息不依赖思考回路得到,比如眼睛直接看到的影像。但像是认知那样,我们也能定义信息是否可理解/可控。可控的信息是知识。
\end{explain}
我们将\indicate{使用知识体系得到认知}的过程称为\indicate{分析}\label{def:分析}。
\begin{explain}
分析得到的认知都是\indicate{知识}。它天然地是当前知识体系的一部分(也可能同时属于其它知识体系)。分析不是获得知识的唯一途径。经过他人或自己有目的的培养,一些行为可以整体结合成一个知识体系。一些行为也可以巧合地组成一个知识体系。

分析中可能包含识别方法,此时的思考回路就是知识体系。
\end{explain}
在先前的讨论中,我们将可控笼统地定义为了“经过分析后,决定是否触发行为”。而现在我们可以看得更深入些:\indicate{决定依赖于思考回路},不同的思考回路的分析和得到的结果不同,决定也会因此而不同。
\begin{explain}
当我们做\indicate{决策}的时候,一定有一个\indicate{目标}(虽然我们不一定清楚自己的目标;目标可能依思考回路的不同而不同)。思考回路需要判断“触发这一行为,得到的结果是否有助于实现目标”。思考回路中可能包含错误的认知,于是可能判断错误;思考回路有可能自身不可控,控制行为完全是下意识举动。但这些都不会影响“这种行为是可控的”,这是思考回路的问题而不是行为的问题。
\end{explain}

\subsection{污染\label{sec:污染}}
\hfill\begin{minipage}{0.55\textwidth}
\fontsize{8pt}{12pt}\selectfont\fontsize{8pt}{12pt} %/selectfont使得英文字体也调整字号
像飞龙掠过我,转眼间规则都颠破。\footnote{\bilibili{BV1vW4y1H7kH}。}

\raggedleft lemon夹子《放生十万个地球》

\raggedright 如果不曾向天空仰望,也不会发狂。\footnote{\bilibili{av7772529};\\\indent \netease{450387538}。}

\raggedleft DELA\_P\&雨狸《海星人鱼》

\raggedright 我握紧这份痛,演出偏执的兽,被困在畸形的悲剧段落。\footnote{\bilibilid{av9214515};\\\indent \moegirl{九重现实}\\}

\raggedleft DELA\_P\&雨狸《九重现实》
\end{minipage}

\label{def:污染}我们将自身所拥有的\indicate{不理解或不可控的行为、行为模式、认知}称为\indicate{带有污染的}行为、行为模式、认知,或简称为\indicate{污染}。如果一些东西(如外部环境、自身经历、记忆、信息、知识、行为等)会\indicate{使我们形成污染},我们就将其称为\indicate{污染源},并将这个过程称为污染源\indicate{带来}、\indicate{造成}或\indicate{传播污染},或是污染源\indicate{污染}我们,我们\indicate{被污染}或\indicate{受到污染}。
\begin{explain}
“污染”带有明确的负面含义,而它的定义从表面看起来并不是那么负面。这是因为,虽然可以直接用“污染”一词来指代“有害的行为和行为模式”,但“是否有害”取决于具体的环境和判断标准,因人而异,不利于展开分析。采用了更为客观简单的定义后,我们就可以从外部判断行为和行为模式是否属于污染。略微扩大定义范围,使得污染有了更加统一的应对和解决方法。

如果强调“有益”,则可改称为\indicate{熏陶}、\indicate{耳濡目染}、\indicate{信条}等褒义词汇。我们在前几小节中已经见到了污染可能有益的情况,这里不再赘述。在其它语境下,污染也会被称为\indicate{混乱}、\indicate{盲目}、\indicate{规训}等。

“污染”带有文学色彩,它是从克苏鲁体系中借来的词汇。作为近年的流行词,它无论是从感情色彩还是从实际内涵上,都相当贴切。
\end{explain}
污染的收益随外部条件而变化。如果任由污染自动触发,我们就会承受损失。如果我们能掌握,那么就可以避免这些损失。
\begin{explain}
可控和理解分别可以减少这些损失。

如果某个污染是可控的,那么它所带来的损失也是可控的。可控的行为(决策)压根不会带来损失;可控的行为模式有可能带来损失,我们不一定能在行为链开始时就切断。比如说“哭半个小时就停”,也会有半个小时的煎熬。但不管如何,可控的行为模式不会带来无穷无尽的损失。

如果我们理解了某个污染,我们就能得知“哪些损失可以避免,哪些损失不能避免”。得到这个重要的信息后,就可以省下很多无效操作,加强可控性,并且可以针对不可避免的损失提前做好准备。比如说“知道了自己昏迷是因为有心脏病”,就不会像常人一样参加剧烈的体育运动来强身健体,而是转为康复锻炼,同时随身携带速效救心丸。
\end{explain}
以上是关于“污染的收益”的讨论。接下来的内容关注“污染的形成”。\\
\indicate{接受了不完整的教育,就会受到污染。}
\begin{explain}
如前所述,在接受教育的初期,几乎一定会受到污染。一套完整的教育可以教会学生完整的知识体系,但如果缺少环节(特别是缺少了识别方法),学生就无法获得完整的知识体系,只能获得带有污染的行为模式。不完整的教育有很多种情况,其中比较常见的有以下两种:

\indicate{得到了某种完整教育的一部分},如“只有教科书但没有听课”“三门前置课程少上了一门”“只上了一半课就不上了”之类。这种情况下,没有外部力量保障一个人接受完整教育,尤其是教育的直接提供者(比如老师,甚至是书)不知道完整教育都包含什么的时候。这或者是因为教育提供方没有特别关注某个学生(甚至完全没注意到有这个学生,比如说买书自己看的),或者是因为学生可以控制自己停止/放弃继续学习。

\indicate{得到的教育本身已经残缺},如“老一辈人坚守的信条”“一场激动人心的演讲”“没办法讲清楚自己为什么优秀的高手”之类。这种情况下,即使教育提供方可能可以维持学生持续学习,但本身教育体系的残缺会使学生永远无法掌握。同时,教育提供方往往没有意识到自身提供的教育有所残缺。学生或许可以从别的途径学会,或许可以依靠自身强大的调研能力学会(这一般被称为“天才”)。但这与残缺的教育体系无关。

有些时候我们无法区分具体是哪种情况,比如说“因为看见了书里的一句话而深感认同”,有时候不能确定书是不是完整教育的一部分,但这句话已经种下了带有污染的认知。
\end{explain}
\indicate{遗忘会使原本能掌握的东西变成污染。}
\begin{explain}
遗忘有很多种不同的机制。概括来说,遗忘发生在“脱离了之前曾进入过的思考回路”时。

为了简便考虑,我们重点关注行为和认知的形成。随着遗忘产生的污染,既有可能是我们自知的认知,也有可能是我们不自知的行为。限于篇幅,这里的两种情况讨论不是很严谨,仅供读者参考。

当我们处于某种思考回路时,我们会从一些特定的思路中获取一些认知。当我们脱离这一思考回路时,我们可能仍然记得这个认知,也会继续根据它来决策,但却无法再回想起当时的思路了。我们忘掉了这个\indicate{认知}对应的前提条件后,它就成为了污染。

新接触一件事情的时候,我们起初会认真思考和决策。随着我们在这件事上慢慢地熟练,它的决策用时和步骤会越来越短,最终完全不出现在我们的意识中出现。我们培养出的\indicate{行为}可能是不自知的,自然也谈不上理解和可控。此时,自己身上就增添了一个新的污染。

不只有“无法再次进入的思考回路”才会产生污染,只要脱离了之前的思考回路就会。当思考回路切换时,很多认知也会在掌握和污染之间切换。详细讨论参见\hyperref[sec:意识与自我]{2.3节}。
\end{explain}
\indicate{长期被迫响应某一环境的刺激,就会受到污染。}
\begin{explain}
如我们之前所说,处理同一件事逐渐熟练,就会形成不自知的行为。

如果自己的\indicate{能力足够应对那种情况},形成的行为就会偏向行动,如熟练的手艺、不假思索地寻求帮助、脱口而出的观点表达、对他人理所当然地要求等等。如果涉及到和他人的互动,那么这种行为就可以视作残缺的教育,也就会使他人受到污染。

如果自己的\indicate{能力不足以解决那种情况},形成的行为就会偏向思考,如不断担心是否会出问题、焦虑地寻找解决方法、习得性无助、完美主义等等。如果以自己感受到的信号来评判他人的行为,就会变得敏感脆弱难以接近,从而自身作为一种会给人带来刺激的环境,也就会使他人受到污染。

限于篇幅,以上的两种情况讨论不是很严谨,仅供读者参考。读者可以借此大致了解污染在不同人之间的传播方式。
\end{explain}
\indicate{使用污染来处理事务,就会受到进一步的污染。}
\begin{explain}
当事务处理有污染参与时,就会产生“没有事务需要处理”的错觉——要不然是“没有发现/找错了事务”,要不然是“认为事务可以解决”。很多时候会同时产生两方面的错觉。

用来处理事务的污染,有可能是完全不自知的行为,见到了信号就不自主地做了事;也有可能是完全自知的错误认知,手里拿着锤子,眼里都是钉子。只要产生了“没有事务需要处理”的错觉,同时环境中确实有事务需要处理,就一定会产生错误的归因和新的行为。

如果是\indicate{没有发现事务}的情况,那么新的污染也会偏向不自知,也就是以“新的行为”为主。在固定的环境下,行为累积多了就会成为行为模式。可能需要花上很久,才能在某天突然反应过来“我之前的状态出问题了”。

如果是\indicate{认为事务可以解决}的情况,那么新的污染也会偏向自知,也就是以“错误的认知”为主。和行为需要慢慢养成不同的是,认知一形成,立刻就会影响自身的决策。同样,可能需要花上很久,才能意识到“自己之前想错了”。

错误的认知在一些语境下会被称为\indicate{自我欺骗}。本指南不引入这一术语。

受到“不自知的行为”的影响,和“响应某一环境的刺激”机制类似;受到“错误的认知”的影响,和“接受了不完整的教育”机制类似。这是污染在同一人体内的演化方式。
\end{explain}
污染是本指南的核心概念之一。概括地来说,污染代表了自身能力无法覆盖到的部分,是事务处理能力的反面。它是一种相当好用的分析工具,在接下来的篇幅中会被频繁提及。
\begin{explain}
有些读者可能看这些很像\indicate{投射性认同}。确实如此。虽然在词义上略有差别,但\indicate{污染}基本上就是\indicate{认同}的实际表现,一样有\indicate{投射}和\indicate{內摄}等机制。本指南的重点不在于此,故不正式引入投射性认同的概念(以及与其相关的整个体系)。
\end{explain}

\section{意识与自我\label{sec:意识与自我}}
\hfill\begin{minipage}{0.55\textwidth}
\fontsize{8pt}{12pt}\selectfont\fontsize{8pt}{12pt}
\raggedright 曾幻想长大展翅飞翔,如今一副瘦肩膀。\footnote{\bilibili{av10154377}。}

\raggedleft vsinger团队\&和田野《未来的我》

\raggedright 我是最无能的主宰。\footnote{\bilibilid{av4188543};\\\indent \netease{407764392};\\\indent \moegirl{四重罪孽}。\\}

\raggedleft DELA\_P\&雨狸《四重罪孽》
\end{minipage}

%终于,在经过了前面漫长的铺垫以后,我们要进入重头戏“对意识和自我的讨论”了。
我们将一个人自身的所有\indicate{信息、知识、知识体系、行为、决策、行为模式等}统称为\indicate{意识活动}或\indicate{意识现象}\label{def:意识现象}。
\begin{explain}
可能有读者在看前面的时候已经注意到,本指南以一种很别扭的方式,规避了“潜意识”这一概念。除了偶尔用一下“不自知”“下意识”以外,我们把所有潜意识都跳过去了。“行为”这一概念和“潜意识”有显著的区别。本章中不定义潜意识,本指南中也很少提及潜意识。

意识现象是一个复杂系统\footnote{“复杂系统”的定义见\hyperref[def:复杂系统]{3.2.2小节}},潜意识是其中产生的现象,我们现在还没有准备好能处理复杂系统的工具。关于潜意识的讨论见\hyperref[sec:人的基本意识模型]{3.4节}。

引入潜意识无助于构建本章的主要分析框架,只会使讨论变得更混乱。只有自知是不够的,甚至有时自知了反而会产生更坏的后果,比如加剧内耗。对我们影响更大的是理解和控制。
\end{explain}
本节讨论的内容是“人拥有多种行为模式”的事实以及这些行为模式相互的作用。人可以同时处于好几个行为模式之中,但很难同时处于自身拥有的所有行为模式之中。当处于不同行为模式时,我们的能做到的会有差别,能思考的会有差别,得到的认知也会有差别。
\begin{explain}
一些行为只有在特定的行为模式中才能触发。如果状态不对,我们就无法记起某些事情、进入某些思路、使用某些方法、理解某些现象、得出某些结论。大多数情况下这没有到“大脑的自动保护机制”的程度,实际状态更贴近于“注意力不在这方面”“没有耐心”“没意识到”之类。

面对同样的东西,我们在不同的行为模式下,会得出不同的评价。它们有时能解决有时无法解决,有时可控有时不可控,有时可理解有时不可理解。本节描述的一些现象可能看起来像是思维奔逸,甚至人格障碍、精神分裂等,但请勿自我诊断,那只是描述相似而已。
\end{explain}
本指南不深入讨论“自我如何产生”的具体生理原理,只聚焦于“人有自我认知”这一现象及其影响。
\begin{explain}
本节的讨论不涉及“胼胝体切除手术对人的影响”(裂脑人实验)等生理现象带来的启示,也不涉及“自我如何从婴儿期开始不断成熟”的发展心理学理论。一个人不可能因为知道了裂脑人就能将大脑分开思考,也不可能因为“被人告诉自己的心理年龄处于婴儿期”就能成熟起来。我们只关注那些力所能及的现象。
\end{explain}
% 我们将人的意识现象分为两类:将\indicate{直接与外界环境交互}的意识现象称为\indicate{行为现象},将\indicate{不直接与外界环境交互}的意识现象称为\indicate{心理现象}。
% \begin{examples}
% 在事务处理三要素中,负责处理外界输入的\indicate{眼}(调研)总是行为现象,而负责内部分析的\indicate{脑}(分析、计划)总是心理现象。根据具体事务的不同,负责直接处理事务的\indicate{手}(执行)有时是行为现象(如需要改变环境中的某处),有时是心理现象(如需要得出结论)。

% 信息和知识总是心理现象。行为有可能是心理现象,也有可能是行为现象。行为模式大多数情况下是心理现象的行为和行为现象的行为的混合,故本指南不笼统地给行为模式贴标签,而是只判断其组成比例。
% \end{examples}

\subsection{自我}
我们将\indicate{使用思考回路产生对某个人的认知}的过程称为\indicate{评估},将被评估的人称为评估的\indicate{对象}。
\begin{explain}
不是所有的思考回路都能评估某个人,只有少量的思考回路有这个功能。而且,一些思考回路只能评估某些特定的行为模式,而无法评估那些不具有这种行为模式的人。

当使用的思考回路不同时,评估产生的认知也会不相同;当评估对象处于不同的行为模式时,评估产生的认知也会不相同。当我们想到一个人时,总是在使用对这个人的认知来思考。

会评估出不同的认知,是本指南将人按行为模式划分的重要动机之一。在特定的环境下,我们可能只会接触到一个人的某个侧面,而接触不到整体。能评估出复杂、立体的人物形象,反而是不太容易做到的事,要不然需要专业能力,要不然需要机缘巧合。
\end{explain}
如果评估对象是自己,那就将其称为\indicate{自我评估}。我们将\indicate{由自我评估得到的所有认知}称为\indicate{自我}或\indicate{自我认知}。
\begin{explain}
自我评估有些时候会触发两种行为模式,有的时候会使用一种行为模式的思考回路对这种行为模式自身来评估。

每当使用“我”这个人称代词的时候,我们就不可避免地会用到自我认知来思考。那些和自我强相关的东西,如(自己的)命运、需求、目标、倾向、过去、未来、情感、生活等等,也不可避免地会让我们借鉴自我认知。

自我认知可能是经过谨慎的思考得来的,但这并不常见。大多数人都不会像哲学家那样,很彻底地反思“自我”到底是什么。

更多时候,自我认知是完全是下意识的反应,或者是草率得到的结论,和现状高度相关。心情好就会有正面的自我认知,心情差就会有负面的自我认知,想起过去就会有过去的自我认知,想起别人就会有别人的自我认知。
% 
% 如果你正在一路顺风,自我认知就会包含光辉与胜利;如果你正在咬牙坚持,自我认知就会包含顽强与不屈;如果你正在处处碰壁,自我认知就会包含迷茫与无助。如果你正在回忆过去,自我认知就会包含过去的思考;如果你正在想象未来,自我认知就会包含未来的期望;如果你正在代入某个角色,自我认知就会模仿对应的情节。
\end{explain}
不同行为模式下的自我认知有可能截然不同,甚至相互矛盾。
\begin{examples}
你可能在假期刚开始的时候兴致满满地安排自己的活动,但一觉起来却只觉得躺在床上刷手机是最好的;你可能在深夜觉得自己一事无成也看不见一点方向,但第二天继续干活的时候就又鼓足了勇气和精神;你可能看到讲解底层逻辑的视频就觉得自己全都懂了,但面对现实时却又觉得自己根本什么也做不到;你可能在吵架的时候情绪激动非要讨个说法不可,但吵完就会开始发现自己对感情和陪伴的需求;你可能独处的时候觉得最重要的事情就是在一起一辈子,但一见面就会被深深地恶心到发现你们俩并不适合......

这些截然不同,甚至相互矛盾的感受,并没有客观的谁高谁低,并没有客观的“理智与冲动”和“应该与过错”之分。它们是你所拥有的不同的行为模式对你产生的不同影响。
\end{examples}
当你在某种行为模式中使用的自我认知是从一种\indicate{没有发现当前行为模式的思考回路}中得到的,那么这个自我认知就是\indicate{污染}。
\begin{explain}
一种自我认知的适用范围是“形成该自我认知时,发现并加以分析了的行为模式”。只有在对应的行为模式中,才应该去使用对应的自我认知。

如果在使用一种自我认知时,忽视/忘记/没有意识到它的适用范围,那根据它得到的结论和做出的计划很可能是无效的,做事的结果也会偏离预期。你可能高估或者低估了自己的行动力、毅力、专注程度,或其它任意一方面能力。你可能过于乐观或者悲观地判断自己的进度、可用资源、周围人的态度,或其它任意一种外部环境。

相对地,发现了该行为模式,并不意味着自我认知就不是污染。它可能涉及到错误的归因。
\end{explain}

\subsection{理性}
\noindent 我们总是\indicate{使用某种思考回路}去理解其它行为模式。
\begin{explain}
在不同的思考回路下,面对同样的意识现象,我们能理解的部分有所不同,不同的行为模式理解难度不同。如果当前的行为模式不包含思考回路,我们就无法理解任何东西。

如果你不清楚自己某些行为模式的触发条件,发现不了某些关键的思路,那你就无法理解自己的这一部分。当你真正处于这一行为模式,能够观察自己的所作所为时,却又很可能从来不会想到“要理性思考,理解自己”。这会导致你无论何时都无法理解自己的某个行为模式。

在之后的篇幅中,当我们使用“xxx是否可理解”的表达时,一定有“在某种思考回路/行为模式下”的前提条件。在不引起歧义的情况下,也有时会省略掉作为前提条件的思考回路,或者仍然使用“一个人能理解”的习惯表达。读者应该有能力自行补全。
\end{explain}
\indicate{理解}一个人(的意识现象),就是要发现这个人所拥有的所有行为模式(不要求理解和控制每一个行为模式),并且分析清楚这些行为模式如何组合和交替出现。
\begin{explain}
这个定义和理解行为模式的定义如出一辙。行为如何组成行为模式,行为模式就如何组成一个人。它们具有高度相似的层级结构。具体内容参考第三章。

这个定义没有区分“理解自己”和“理解别人”,实际上也没有必要区分。它们遵循着相同的原理,只在一些具体的细节上有所区别。我们用思考回路来分析行为模式,而不在意行为模式具体是谁的。同样,行为和行为模式的理解也不区分是自己的还是他人的。

我们会听到“没有人比自己更理解自己”,也会听到“当局者迷旁观者清”,理解自己和理解别人没有确定的“谁难谁易”。理解的难度取决于你当前应用的思考回路和对方所处的行为模式(这里的“对方”也可能是你自己)。

这听上去像是精神分析,实际上也差不多就是。但本指南尽量避免提及专业概念,故不引入成体系的精神分析术语。
\end{explain}
当理解了一种行为模式后,我们就可以根据其中的行为对待某种环境时的反应,逐步分析出这一行为模式的举动。我们将其称为(在某种环境下)\indicate{模拟}这种行为模式。
\begin{explain}
当理解了一种行为模式以后,我们就可以设想“如果在某种环境中处于该行为模式,会发生什么”。模拟可以选择任意环境,不必要在原本就会触发该行为模式的环境中。

如果这种行为模式同时是可控的,我们就可以根据模拟的结果来决定是否触发某种环境下这种行为模式。在一些情况下,这比“在未知环境下一律控制某个行为模式不触发”要更好。

同时,模拟也不限于自身,可以模拟别人有但自己没有的行为模式。这有助于我们理解别人。能模拟不代表能拥有,可能会因为知识缺失导致某些步骤无法自己完成。
\end{explain}
如果一个人拥有一种思考回路,能够发现并分析自己所拥有的每一种行为模式,并且得到不含污染的自我认知,那么我们就将这种思考回路称为\indicate{理性}\label{def:理性}或\indicate{理性思考},将“使用理性来分析自己的行为模式”称为\indicate{反思}。
\begin{explain}
如果读者喜欢的话,也可以叫\indicate{自我精神分析}。

不同人所拥有的理性是高度相似的。它总是包含一个完整的知识体系(由事务处理能力等内容组成),在此之外还根据个人特征有些许差别。

我们这里定义的理性不直接包含“科学研究的方法论”等内容,这些不在本指南的讨论范围之内。实际上,我们这里定义的理性就是“将科学研究的方法论运用到自我意识上”的行为模式。

可以分析一种行为模式不代表可以理解,如果能力不足,理性有可能得不到任何有效的自我认知。但这只是能力问题或者是现实条件限制,只需要继续丰富自身的知识体系即可。

理性所得到的自我认知,仅对理性来说不是污染。若自身处于别的行为模式中,可能无法理解这些认知。在别的行为模式看来,这些认知和理性本身可能都是污染。

理性有可能只在某些特定环境下存在,包括但不限于面对心理医生时、和朋友谈心时、看书时、深夜、发呆时。
\end{explain}

\subsection{自控的两种方式:意志力法与自我编织\label{sec:自控}}
\noindent 我们总是\indicate{使用某种思考回路}去控制其它行为模式。
\begin{explain}
在不同的思考回路下,面对同样的意识现象,我们能控制的部分有所不同,不同的行为模式控制难度不同。如果当前的行为模式不包含思考回路,我们就无法控制任何东西。

在之后的篇幅中,当我们使用“xxx是否可控”的表达时,一定有“在某种思考回路/行为模式下”的前提条件。在不引起歧义的情况下,也有时会省略掉作为前提条件的思考回路,或者仍然使用“一个人能控制”的习惯表达。读者应该有能力自行补全。
\end{explain}

类似于“行为模式的可控是每一行为链都有行为可控”,我们也可以用同样的方式来定义人是否可控:一个人在某种思考回路/行为模式下\indicate{可控},是指这个人的每种行为模式都在这个思考回路/行为模式下可控。我们将“行为模式触发时控制这个行为模式”称为\indicate{控制}一个人。
\begin{explain}
和前面“理解”时一样,这个定义也无需区分“控制自己”和“控制别人”,这二者没什么本质差别。同样,行为和行为模式的控制也不区分是自己的还是他人的。

如果能有效打断或开启另一个人的每一种行为模式,那就能控制一个人。这常见于父母对小孩、教师对学生、狱警对囚犯等场合。详细讨论参见2.5节。

“控制”本身是一种中性的行为。如果是因为污染而控制,那么它就会偏向负面;如果是因为正确决策而控制,那么它就会偏向正面。
\end{explain}
我们将“用自己的思考回路来控制自己”的行为称为\indicate{自控},本小节后面的内容仅讨论自控。
\begin{explain}
如果某一思考回路A不直接控制行为模式C,而是通过触发行为模式B,使用B控制行为模式C,这对C当然也是有效的控制。如果一个人同时处于好几个思考回路A、B、...中,每个思考回路可以控制一部分行为模式,这也算是这个人可以自控。
\end{explain}
想要达到“控制所有行为模式”的目标有两种方法:一种是站在行为模式层次的“控制每一种行为模式”,另一种是站在意识现象层次的“只控制一部分行为模式,并且清除其它不可控的行为模式”。

\label{def:意志力}我们将\indicate{控制每一种行为模式}的自控法称为\indicate{意志力}自控法,其中,一种思考回路对一种行为模式的\indicate{意志力}是指\indicate{在该行为模式下,这种思考回路可做决策的数量}。
\begin{examples}
控制不同的行为模式所需要的决策数量有很大差别。当你处于某个充满诱惑的环境中时,你可能需要频繁地(比如每搁5秒就)做一次决策,打断自己“想吃东西”“想玩游戏”“想睡觉”的冲动;当你处于某个充满干扰的环境中时,你可能需要频繁地(比如每隔30秒就)做一次决策,强迫自己“继续学习”“继续干活”“继续思考”。而在“只需要触发另一种行为模式,就可以脱离当前行为模式”的时候(如在心情糟糕的时候点杯奶茶喝,或是只要脑子里面想点别的别闲下来)或者“只需要脱离当前环境,就可以脱离当前行为模式”的时候(如看书看累了起身走一走,或是回家/离家),一共就只需要做一次决策。

不同思考回路的意志力特性有不同之处。有些思考回路可能只能应付一定频率以下的决策;有些思考回路可能一天只能做一定数量的决策;有些思考回路只能坚持少数几次决策就会信心崩溃,从此都无法再使用;有些思考回路越做决策越专心,越自控就状态越好。

因此,这里对于意志力的定义比较模糊,仅笼统地称为“数量”,而不具体定义成“总量”“频率”等。请读者根据实际情况,为不同行为模式分别选择合适的理解方式。
\end{examples}
意志力自控法的主要缺点有两个:一是“每次决策都有代价”。行为模式会反复地被触发,从而也会一直需要支付代价。二是“无法补齐能力缺失”。补齐能力的方式参见后文“培养行为模式”。
\begin{explain}
代价有很多种可能的形式,包括但不限于“消耗时间”“分散注意力”“打断思路”“心情变差”“失去兴趣”“错过时机”等。

在处理某一个行为模式时,有可能在坚持了一定时间后,就逐渐脱离了那个行为模式,于是可以专注于自己的目的。靠意志力达到心流状态都是有可能的。这一缺点只在于“无法一劳永逸,每次触发行为模式都需要单独处理”,而不在于“时刻会被打扰”。
\end{explain}
我们将\indicate{选择某些行为模式,并将自身调整为由这些行为模式组成}称为\indicate{自我编织}。当选择的行为模式都是可控的行为模式时,自我编织就是一种自控法。
\begin{explain}
自我编织是一个事务。完成它需要做两方面事情:培养所选的行为模式,去除未被选择的行为模式。培养/去除每一个行为模式都是它的一个子事务。

培养一个行为模式,需要理解这个行为模式,并且针对性地培养组成这一行为模式的行为。去除一个行为模式有两种方法:一种是控制环境,使得这一行为模式总是不会被触发(这种方法仅需发现这个行为模式);一种是消解这一行为模式,即通过改变或去除某些行为,使得行为链不再持续触发,行为模式不再存在(这种方法需要理解这个行为模式)。更详细的讨论参见\hyperref[sec:原生家庭]{2.5节}与\hyperref[sec:人的基本意识模型]{3.4节}。

\indicate{编织}是一种普遍存在的现象,定义见\hyperref[def:编织]{3.2.1小节}。
\end{explain}
自我编织自控法的主要缺点在于“难度高”。我们需要使用一整套完整的理论,才能够稳定地实现目的(但也不能保证实现目的)。
\begin{explain}
不同于意志力自控法,自我编织要求发现自身所有的行为模式(只有这样才能保证按选择调整自身),所以仅有理性才有能力实现自我编织。如果能做到培养/消解某个行为模式,就能一劳永逸地解决这一方面的自控问题。

可能有读者会担心如果把自己的行为模式都换过一遍,那自己还是不是自己。本指南不讨论意识连续性问题(毕竟每个人的自我评估都不一样),只想请读者注意一件事实:人的行为模式本身就会不断地更改,相当多行为模式只会持续很短的时间(比如说一个月)。“无法共情过去的自己”是相当常见的现象。实际上只有“不可控的改变”和“可控的改变”两种选项,而没有“不改变”的选项。
\end{explain}
即使无法全面完成自我编织,即使是做到了培养/消解某个行为模式,也会增加自身的可控程度。在实际操作中的自控,更多的是两种方法的结合。
\begin{examples}
先通过意志力控制住一个浪费时间精力的行为模式,等到时间久了它可能慢慢就消解了;对于更顽固的行为模式,需要专门分析,以找到可以破坏的节点。

而分析行为模式同样有助于找到意志力可以控制的行为,省下更多时间精力用于其它方面的目标。
\end{examples}

\section{总结与讨论}
\subsection{本章总结}
\begin{explain}
本小节准备了两种风格的总结,希望不同口味的读者都能加深对本章的理解。
\end{explain}
\smalltopic{(1)知识点串讲}

我们从外界获取\indicate{信息},分析这些信息就会得到\indicate{认知}\footnote{\rigorous 可能需要确定是先有认知还是先有行为模式。答案是先有初级的行为模式,然后更高级的认知和行为模式才在此基础上交替出现。初级的行为模式由动物性的行为组成,如“进食”“休息”“记忆”“模仿”等。}。信息和认知会影响我们的行为,使我们更加偏向于某种选项,直至形成\indicate{行为}。在合适的环境下,一些行为会连续触发,从而形成\indicate{行为模式}。以上这些,是一个最简单的关于意识的模型。这在很多动物身上也能见到,是一种低层次的\indicate{动物本能}。

但我们是人。我们不想失败。我们不想只能在夜里偷偷哭泣。我们不想让过去的回忆褪色。我们不想只能过毫无价值的一生。我们不想屈服于自己的动物本能,只能做出事与愿违的事情。当我们回顾自身,\indicate{发现}自己的问题的时候,就会去想着\indicate{控制}。人类的\indicate{自我意识}驱动着我们总结出这样的结论:我们相信,只要能解决问题,就不会再经受失败,就会达成所愿。带着这样的\indicate{认知},我们又有了一些新的行为,又形成了一些新的\indicate{行为}和\indicate{行为模式}。

很不幸,事情不一定会向着我们希望的方向发展。对自己和对外界的认知不完善,所有改变的尝试就全都只能是\indicate{污染}。就和那些动物本能一样,我们照样没办法控制这些新东西。自我意识是盲目的,它只能看到目标,但看不到实现目标的道路。靠\indicate{意志力}只能在漆黑一片的荆棘丛中行走,承受着痛苦的折磨,却不知道路还有多长,甚至总是原地打转。路真的会有尽头吗?尽头就是目标了吗?我们不知道。

我们能知道的只有一件事:有其它人\footnote{实际情况可能更复杂,这里使用“人”是为了行文简便。}走到了路的尽头,实现了自己的目标。他们是幸运还是强大,对我们无关紧要;你是见贤思齐、羡慕嫉妒、自惭形秽、愤世嫉俗,也无关紧要。我们只关心一件事:他们是怎么成功的?我们也能向他们那样走向成功吗?于是我们开始了\indicate{归因},开始了模仿。模仿的风险很大,错误的归因也会给自己带来\indicate{污染}。只有那些正确总结了因果关系的\indicate{过程},才是应当前进的正确道路。将这样的道路一段一段拼接起来,就能抵达目标。

这时,我们才终于\indicate{理解}了如何实现自己的目标。但理解了不代表就有能力实现,目标越遥远,越困难,需要补充的能力就越多。为了获取所需的能力,我们首先\footnote{实际情况下,清除污染、总结规律、培养能力三方面行动会交替发生。这里为了叙述流畅做了简化。}需要\indicate{发现}并\indicate{清除}一些\indicate{错误的认知}。这些认知来自于我们身上的污染,为了不再重蹈覆辙,我们也有必要一并\indicate{清除}这些\indicate{污染}。

接下来,我们需要培养新的行为模式。新的行为模式完全由正确的归因组成,我们能够完全理解,充分控制,并且熟练运用。这样的\indicate{知识体系}使我们能够正确地\indicate{分析},正确地\indicate{决策},从而有充足的能力来实现自己的目标。而在这个过程中,我们也慢慢培养出了自己的\indicate{理性}。

\smalltopic{(2)写作思路与分析框架}

第一章中我们详细讨论了事务处理,后续需要详细展开的有两方面:一方面是关于\indicate{能力}的讨论,另一方面是关于使用能力的主体,也就是\indicate{人}的讨论。本章建立的分析框架主要偏向对\indicate{人}的讨论,已经涵盖本指南所有需要用到的概念和思路。而\indicate{能力}方面仅做了简要叙述,完整展开见第三章。

本章围绕\indicate{理解}与\indicate{控制}两个重点,将人的意识活动由低到高区分了\indicate{行为}、\indicate{行为模式}、\indicate{人}三个层次,分别展开了讨论。

本章最重要的切入点是“能发挥出什么能力,取决于一个人在什么状态”。为了区分一个人的不同状态,我们需要明确定义\indicate{行为模式}并据此展开讨论。考虑到思考和认知的重要性,我们还细分出了\indicate{思考回路}这种纯粹在脑内发生的行为模式。

一个人身上不同的行为模式很多时候各行其是,没有什么协调性可言。它们会给出截然相反的判断,做出相互冲突的行动。这是每个人都能在自己身上感受到的事情。做出的计划要是无法执行,那就和没做没啥区别。我们需要稳定地维持能解决事务的行为模式。为此,我们需要\indicate{理解}行为模式的相关原理。

行为模式的组成相当简单清晰:环境和行为的相互触发。相互触发的具体细节通常会很复杂,没有通用的规律,一般需要每个行为模式具体分析。本章讨论了\indicate{行为},外部环境将在下一章讨论。\indicate{行为}的形成也很简单清晰,它是某些信号反复刺激,逐渐熟练后的结果。

理解了行为模式的相关原理后,我们就可以着手细化我们的目标了。如果想要稳定地维持能解决事务的行为模式,而不被其它行为模式干扰或打断,我们就需要\indicate{控制}这些有关的行为模式。行为模式由行为组成,于是想要控制行为模式,只能也只需通过控制行为来实现。行为由信号触发,想要控制行为,就只能也只需通过控制信号来实现。

我们将“维持”细化为了“控制”,但这不代表目标更容易解决了。这一步细化的唯一作用是\indicate{明确了实现目标的难度}。我们必须要发现行为模式,发现关键的行为,并且确认这种行为的触发条件,才能实现有效的控制。这在一些时候可以通过运气或是意志力实现,但这只能应对一些特殊情况。我们需要一种通用的\indicate{分析}方法,来处理所有需要处理的行为和行为模式。于是,我们据此定义了\indicate{理性}。分析能力的强弱,直接决定了我们能否理解和控制每一个不同的行为模式。

也因此,在关于\indicate{能力}的讨论中,我们将能力(\indicate{知识体系})定义为“哪些事务它可以解决,应该使用什么方法解决”。看完第二章的读者应该可以明白,这等价于“其中包含的认知和行为可理解并且可控”。

为了方便搭建分析框架,行文中还引入了“方法”“识别”“过程”等概念。这些概念较为底层,主要为了简化描述和精确规定重要概念,较少在其它地方出现。

本章中还引入了一个较为特殊的概念:\indicate{污染}。这一概念在搭建分析框架时没有太多存在的必要,而在后续的讨论中,应用分析框架时,会体现出其方便之处。

\subsection{再谈死亡\label{sec:再谈死亡}}
\hfill\begin{minipage}{0.55\textwidth}
\fontsize{8pt}{12pt}\selectfont\fontsize{8pt}{12pt}
\raggedright 嗫嚅着故事的终章,迷人的馥郁芬芳。\footnote{\bilibili{av19566463};\\\indent \netease{495834939}。\\}

\raggedleft 汤《时间日记》

\end{minipage}

\smalltopic{(1)关于求死的行为模式}

运用第二章的概念,我们已经能对死亡和自杀这一话题展开更详细且深入的讨论了。

我们在\hyperref[chap:wedge]{序幕}处谈了一大堆和死亡相关的事情。本章的概念可以让我们给出一个更精确的表述:你想死,是因为你有一个行为模式,会在你遇到困难的时候想到“死了就好了”(我们将其称为\indicate{求死}的行为模式);没有死成,是因为你所拥有的行为模式,不足以让你完成“自杀”这一行为。

\indicate{求死}由多种不同的思路结合而成,可能包括恐惧、无助、逃避、绝望、解脱、爱恋、渴求、回忆、梦想、温柔、回答、离去、花束、夏天、歌曲、谎言、晨风、晚霞、乌鸦、酒杯等很多看起来或是很合理或是很离谱的东西,具体组成成分因人而异。如果一个人从未发现过求死这种行为模式,那一般就不可能理解一个无法控制求死的人。即使是发现/经历过以后控制住了,也有很大一部分人无法理解其它人,甚至无法理解过去的自己。

\indicate{绝大多数人所拥有的求死是一种污染}。那些“活不下去”“死了就好了”的念头绝对不可能凭空出现,一定是先想到了“死亡有这种作用”的结论,再去将其当成解决方案。这个结论在人与人之间传来传去,最终失去了所有的背景,只剩下了脍炙人口的一句话。“活不下去”对某些人来说甚至已经降格成了平淡的抱怨,但对另一些人来说就是自身每个侧面最真实透彻的判断,是不敢直视而又充满了诱惑的终极答案。当你听到这些东西,于是思路总是会落到“活不下去了”,总是会绕回“死了就好了”时,你就受到了求死的污染。

这种想法可能来自某些对比。当身边的谁能逃离一种痛苦的环境时,不管是上岸还是远走还是死亡,都会因为“不再痛苦”而被羡慕。至于这个谁到底之后又经历了什么,或者曾经经历过什么,其实无人在意,只有“死了就会脱离这个环境,不再痛苦”的结论留了下来。这种脱离实际环境的认知,混杂着羡慕、期待、发泄、指责等多种感情被广泛传播,就成为了污染源。

这种想法也可能来自某些透彻的思考。人类的历史相当漫长,各路思想家、哲学家、文学家已经将死亡讨论了个遍,很多学者和书中角色基于\indicate{理性}的判断而选择了自杀。求死的行为模式对他们来说不是污染,他们的\indicate{最终决定}被单独取出,被评价为“透彻”“豁达”,并且广为流传。但他们的\indicate{思考过程}并没有跟着一起流传,于是那个被单独取出,被高度评价,被广为流传的最终决定就也成了污染源。

如果我们不用理性来思考,那么会遇到的最大问题是“自己还有其它行为模式”。这些行为模式可能有自己独立的目标和渴望,其中大部分会和“死亡”冲突。它们和求死之间没有什么关系,也不能相互理解,甚至不一定能意识到对方的存在,唯一共同点就是都被塞到了你的身体里,除此之外就和两个陌生人一样。求死劝不动其它的行为模式,你就不能专心自杀。

而这反过来也一样,其它的行为模式也劝不动求死,无法消解它们无法理解的求死。\indicate{其它行为模式对求死的劝告、限制、阻拦几乎都是污染},无论这个行为模式是自己的还是别人的。大部分人没有透彻地想过自己为什么要活下去,只是习惯于这么做,他们\indicate{求生的行为模式也是污染}。即使一个人A有控制自己的求死的能力,能让自己想开一点,开心一些,也很可能没法通过劝告、陪伴、介绍自己的心路历程等方式,让B也想开。这些举动在B看起来很可能无比苍白,完全是站着说话不腰疼。A和B的求死很可能需要用截然不同的方式来打断或消解。

如果你能够使用理性,对自己的所处情况和自身的每种行为模式展开全面深入的分析,并且最后得出“自杀是最好的选择”的结论,那么没有任何东西可以有效地劝住你。虽然大多数人获得了理性以后,实际上不会得到这个结论,但每一个拥有理性的人,在具体了解了你的处境以后,都会得出和你一样的结论,都会认为你的自杀确实是解脱,这个决定确实透彻豁达。你的决定也会得到本指南的尊重和祝福。

你如果靠意志力压制住了其它的行为模式,强忍着痛苦,以卓绝的精神韧性程度成功自杀,那本指南也会称赞你的求死是一种很坚强的行为模式,饱含勇气和决心。但这种称赞不会上升到意识层级,因为你的意识中混杂着污染。面对那些带有污染的行为模式,你只能控制,而不能将其消解。在对人这一层级的的看法上,本指南会惋惜你的离去。

本指南不教唆自杀,也不劝导求生,不对“人对自身生命的处置”持任何态度。本指南只聚焦于“理解现实,处理事务”。虽然这些方法看起来只有活人才能用(这也是本指南劝读者至少在看完全书以后再自杀的原因),但你也可以依靠它们来成功地选择死亡。清除了自己身上的污染以后,路怎么走是个人的事。
\begin{examples}
我把石头还给石头,让胜利的胜利,今夜青稞只属于他自己,一切都在生长。
\raggedleft 海子《日记》
\end{examples}
某种意义上,能让所有行为模式都满意的结果,是让求死带着会触发它的所有东西死去,想留的留下来。

\smalltopic{(2)关于外部干涉}

我估计上一段的讨论内容还是会让很多人觉得我就是在教唆自杀。这其实还挺合理的:这相当于有人指着你的鼻子骂你去死,或者有人指着你的孩子说这小东西该死。

但我们不妨换个视角来想一下:什么样的人看到了这段内容以后,会更想自杀呢?

看了这段内容以后,如果会愤怒或者反感,那肯定就不吃这一套;如果从头到尾都没什么感觉,看这些就和普通的社会调研报告似的,只是感到痛心、担忧,或是只是平静,那也不会有啥事。会出事的有两类人:一类是本来没有求死的行为模式,但看完以后就拥有了;一类是本来就有求死的行为模式,看完以后深感共鸣,觉得自己确实该死。

第一类人其实很少见。准确来说,“在自己的一生中,除了本指南以外,接触不到其它的求死污染”的人很少见。我们生活在一个信息爆炸的时代,各种关于人生意义的思考满天飞(由此我们可以将第一类人并入第二类人中一并讨论)。我们无法将某个人完全隔离在这些东西之外,唯一的可选项就是建立正确的认知。想要得到充分的理解,必然需要充分的分析。
\begin{examples}
大多数人一生都遇不到一次火灾,但消防宣传仍然是很必要的。展开消防宣传,并不意味着在咒你被火烧死。

对死亡的思考可比火灾常见多了,它差不多和诈骗一样常见。开展反诈宣传不意味着歧视你的智商,不意味着觉得你容易被骗。不要上升到人身攻击的高度。反诈是一种纯粹的技术,任何人都可以学习和掌握,也应该学习和掌握。应对“死亡”和“求死”也一样。
\end{examples}
面对第二类人的时候就需要注意一个问题:他们求死的行为模式是哪里来的?他们是因为“厌恶某些东西,恐惧某些东西”,从而才会将死亡视作解脱,还是“没什么原因,只是学别人说话”?后面那种确实可以当做“被别人带坏了”(于是我们需要防范),但前面那种应该怎么对待和处理?以及在此之前还有一个问题:我们应该怎么区分这两种情况?
\begin{explain}
\label{para:无效沟通}坏消息是,我们几乎总是会因为信息不足而无法区分。即使你去当面问,也不会得到有效的回答。原因有三:

一、求生和求死的行为模式在一个人身上可以共存。这两种行为模式无法相互理解,对对方来说相互是污染。一个人在求生行为模式下可以去回应那些关心,但这不能代表求死行为模式的态度。求死者切换到求死行为模式以后,甚至更会觉得大家花在自己身上的精力浪费了,负罪感更强,这反而会刺激求死欲。

二、“自己对自己求死行为模式的描述”不一定是有用的观察。它可能大幅度弱化和忽视了许多关键点,不是有效且正确的信息。一个人自身可能没有任何一种行为模式(无论是求死还是其它)能理解自身的求死。求死者如果不能客观全面地认识自己的状态,那么很可能发自内心地觉得自己没问题,并且对别人也这么说。

三、很多人会预判别人的态度和反应,会为了避免各种类型的麻烦(比如别人的担心、伤感、拯救欲、约束、教育、责罚)而选择撒谎,于是只能得到“我没事”的答复。大部分人无法给出消解求死的有效方案,实际的建议和要求要不然没有作用,要不然折腾人。面对这种情况,求死者选择隐瞒是相当合理的举动。
\end{explain}
如果不能直接观察一个人所拥有的求死的行为模式,就无法确认这个人到底为什么求死,也无从将其解决。方式不对,就会无效,或者起反面效果。发现并理解一个人的某种行为模式是相当困难而专业的事,我们绝对不应该草率地归因,绝不应该草率地断定“某个人身上不存在求死的行为模式”。绝不应该把“偶尔自杀”当成是“受刺激了想不开”处理,绝不应该把“念叨着想死”当成是“被带坏了”。

\subsection{再谈心理咨询}
运用第二章的概念,我们能够比\hyperref[sec:实操1]{之前}更详细且深入地讨论心理咨询了。我们将心理问题重新定义为\indicate{会带来负面效果的污染},而心理咨询的目的,就是清除污染。

我们将\indicate{能够发现、分析、理解、模拟、培养、消解别人的行为模式}的知识体系称为\indicate{心理咨询能力}\label{def:心理咨询能力},从而一名有充足能力的心理咨询师是一位\indicate{有心理咨询能力的人}。心理咨询这种服务,是\indicate{根据来访者的需要,帮助来访者掌握自己}。

由此衍生出了三方面问题:一方面是“如何拥有心理咨询能力”,这其实和上一小节讨论求死的时候提到的内容差不多,大体上来说,切忌先入为主,切忌偏听偏信,要时刻以客观谨慎的态度,从一个人的实际情况出发,展开全面且深入的分析,并且时刻比对自身的认知是否符合实际情况。具体方案参见第三章,此处不做过多讨论。

第二、三则是“如何识别一个人是否有心理咨询能力”和“心理咨询能力如何提供心理咨询服务”。1.3节中这两者均有提及。大多数人既不知道如何识别心理咨询师是否有充足能力,又不知道心理咨询服务如何起作用,从而使得心理咨询的失败率远超心理书籍中介绍的预期(毕竟里面介绍的成功案例居多),进而形成了“心理咨询都没用”的观点。

\smalltopic{(1)如何识别一个人是否有心理咨询能力}

识别一个人是否有心理咨询能力,是一件困难到几乎不可能实现的事情。非常概括地说,能识别出来,几乎一定就意味着自己也有心理咨询能力——但自己都有能力了,还需要找咨询师干什么呢?从别人那打听也差不多:如果你无法确认那个人有心理咨询能力,你就无法判断他的观点靠不靠谱。大部分关于咨询师的评价,靠的都是感觉和信任,而不是证据充足的判断。有这么几种经常会引起误判,\indicate{错认为“对方很靠谱”}的现象:

\indicate{善于倾听}:善于倾听是一种少见的行为模式,它会给人“能够沟通”“温柔亲和”的印象。有时候我们不一定会意识到一个人善于倾听,但也会因为对方易于亲近而乐于与对方接触。不止是心理医生,我们很多时候也会感觉自己身边的朋友有这种特质,能和他们聊得非常投机。

这种行为模式的形成有两种可能性,其中比较常见的一种,是“因为缺失其它行为模式”。如果一个人在某种环境下没有什么主动性,在听到别人倾诉的时候,既不会频繁打断发表自己的见解,也不会觉得厌烦从而起身离去,而只是一直听着,或许会随声附和,或许只是点头微笑,或许是过来蹭饭的,那么这看起来就像是“善于倾听”了。客观上来说,和这样的人相处,向这样的人倾诉,会让人感觉到安全、放松、信任。但这不代表这样的人能够理解你,不代表这样的人拥有心理咨询能力。他们可能根本没在思考,只是行为客观上可以提供陪伴。

另一种,是“明确意识到应该这么做”。而这么做的动机也多种多样:有的就是看中了倾听能起到的安慰效果,从而有意识地这么安抚别人。你身边的朋友和心理咨询师都有可能是这样的人,这样的人能起到的效果和上一种差不多。有的是有意识使用倾听来理解你,这就是咨询师的专业范围了。这种动机会让咨询师问你很多问题,这些问题有很明确的主动性,但同时不预设任何立场。\footnote{当然我们不一定能准确区分一个问题/一串问题是否有立场,同时也不一定能区分是否咨询师真的问了很多问题,这只能作为辅助信息来大致判断。如果你能准确判断,那你也就有心理咨询能力了。}这样的咨询师同时也会使用很多其它手段,总体来说倾听的频率可能不如专门用倾听来安慰的咨询师。一位很有能力的咨询师会在该倾听的时候倾听。

\indicate{善于指引}:相比于善于倾听,善于指引的行为模式会更常见一些,它会给人“聪明可靠”“充满智慧”的印象。我们比较容易意识到一个人善于指引,并且因为对方总是能给出解决方法而乐于与对方接触。除了心理医生,这种特质同样会在身边的朋友身上出现,他们总体来说会更成功一些。

善于指引需要区分不同的方面。这种行为模式的形成需要两个条件:一个是“愿意提供建议”,一个是“在对应方面有充足的能力”。如果以朋友/长辈的标准来评判,善于指引已经相当可靠了。能在你有事情要解决的时候过来帮忙,在你情绪低落的时候给你有效的安慰,在你遇到麻烦的时候帮你梳理思路,在你接触新东西的时候提供教学的人,基本上就是关系最好的人了。这种程度的关系对于很多人来说只存在于梦中。但这不代表这样的人能够理解你,不代表这样的人拥有心理咨询能力。他们可能根本没在思考,只是在响应你展露出来的诉求。

这对于咨询师来说是不够的。无论是对你的问题给出针对性方案,或者是长篇大论地和你讲人生的道理,或者是带你摆沙盘和画房树人,我们能感受到的都只有“咨询室内的氛围融洽”。而咨询室内的氛围如何,和咨询室外的情况无关。在咨询室外你可能仍然保持着旧有的行为模式,没有改变。这些做法有合理之处,一个有心理咨询能力的咨询师也会用,全面地改变旧有的行为模式也得依靠这些方法。但仅凭这些做法本身,仅凭“咨询师善于指引”的特点,是不足以判定咨询师有充足能力的。

\indicate{除此之外},还有一些主要依靠外在条件展露出来的特质,比如说室内布置、音乐、香氛、疗养条件等等。这些相比以上的两方面,更容易让人看出“和心理咨询能力没有直接关系”。

总的来说,虽然“会让人感到安心和可靠”的特质很少见,但不足以让我们判断一个人有充足的心理咨询能力。并且,由此产生的(可能自知,也可能不自知的)依赖和不切实际的希望,可能会对你所拥有的关系(不只是心理咨询师,还有友情、爱情、亲情等)造成持久的打击。你可能会因为“终于找到一个可以理解你的人了”而喜悦,之后又因为“觉得对方应该能理解你,但对方没有”而厌烦、难过、委屈。越是对对方的能力没有客观全面的评估,就越容易出现这样的问题。

\smalltopic{(2)心理咨询能力如何提供心理咨询服务}

本节中我们仅考虑具有充足能力的咨询师和咨询服务。如果不考虑经济等其它外部因素,心理咨询服务的最优水准,大概可以描述为“能够理解来访者的每一种行为模式,并且按需要培养和消解其中的一些,帮助来访者理解和控制其它的行为模式”。其中,对于一些行为模式来说,培养、消解、控制它们需要一些外部条件(比如说住院、疗养、搬家等方式以远离某种环境)。考虑经济条件的话,大多数心理咨询不具备这样的条件,所以在这段讨论中我们尽量避免涉及这一方面,只考虑那些免费或低价的资源,如书籍、视频或其他形式的知识,或是是app或其它形式的计划单。

在这样的条件下,普通心理咨询的最优水准,大概相当于“在来访者的身上增添理性”。我们将心理咨询增添的理性称为\indicate{咨询理性},将来访者本身可能拥有的理性称为\indicate{自身理性}。咨询理性对来访者来说实际上是一种污染,但它一般会比自身理性更加强大,主要起正面作用,能够发现更多的行为模式并做相应的处理。但同时缺点也很明显:\indicate{咨询理性仅能观察和处理咨询室内发生的事},不能跟随来访者,随时观察自身行为,指导自身行动。这会造成两方面问题:

\indicate{咨询理性难以认识某些行为模式}。这里的“认识”包含“发现、理解、模拟”。当然,这不是说“要完全信任咨询师,将自己的信息和盘托出”,这么做有严重的隐私问题。来访者在日常活动时很可能不会有意进行全面的自我评估,从而在观察、认知、表达三方面都有可能出现问题\footnote{读者可以注意到,“观察、认知、表达”其实就是事务处理的三要素在“使用理性”上这一事务上的具体体现。具体展开可以参考在2.4.2小节(2)中关于“\hyperref[para:无效沟通]{当面问也不会得到有效的回答}”的讨论。}。如果仅根据咨询室内的情况来分析和判断,就有可能忽略更重要的根本原因。

因此,咨询师会采用很多种不同的手段,来尽可能全面准确地得知来访者的完整状态。这些手段包括但不限于倾听、安慰、问卷调查、沙盘、情景模拟、催眠、环境布置......如果在某种环境中,心理咨询师能够观察到来访者在用某种行为模式来应对外部环境,那我们就称咨询师在和这种行为模式\indicate{对话}。

对话不是咨询师了解来访者的唯一方式,来访者也可以自己发现某些关键信息(比如“回忆起了某些关键事项”、“某些突发事件的反应和处理”、“应咨询师要求留意某些日常行为”等),并转述给咨询师。在转述时,来访者可能会重新进入当时的行为模式,从而咨询师可以与之对话,了解更多;也从头到尾只是转述,咨询师根据来访者的描述来理解和判断。

来访者要是因自身理性缺陷而无法观察到这些现象,或是观察到了却没有意识到重要性,或是意识到重要性却想不起来转述或羞于转述,咨询理性就没有发挥的空间。在日常生活和咨询时尽可能维持自身理性,会有效加快心理咨询的速度。

\indicate{咨询理性难以改变某些行为模式}。这里的“改变”包含“控制、培养、消解”。在咨询室内,来访者可以和咨询师展开深入全面的交流,对自己的行为有非常清晰透彻的理解,明白哪些事情是一厢情愿,哪些做法会事与愿违,哪些行为会带来痛苦。但一回到日常生活中,失去了咨询理性,就可能又恢复了原样。维持自身理性有助于改变行为模式,但培养自身理性本身也是改变,也很困难。

咨询理性仅在咨询室内存在(有些时候也在其它和咨询师有关的地方存在,比如说和咨询师远程沟通时,或者是回看自己记的笔记时)。在脱离咨询师的环境下,自身理性有可能相当脆弱,很容易因为各种刺激而遗失:有可能是“明明在咨询室已经明白了,但是真的搞砸了还是会忍不住地指责自己”,有可能是“明明知道要冷静,但是真的遇到事情还是会惊慌不知所措”,有可能是“没遇到什么事情,只是睡醒起来就忘了”。遗失了自身理性,就无法想起来要使用正确的方法面对,而只会重蹈覆辙。

因此,咨询师会采用很多不同的手段,来尽可能全面具体地培养来访者的自身理性。这些手段可能包括:一定程度的咨询时间外联络;带有仪式性质的固定行为训练;反复加深记忆的认知方法训练;对来访者行为模式深入的了解、分析、讲解;成体系的心理能力介绍......这些方法中有一部分可以视为污染,但污染在一些时候比理性更能实现有效的控制。

来访者也可以通过一些方法来维持自身理性。一部分方法通过借助外力的方式起作用,比如说使用定时闹钟来替代自律,使用笔记来替代记忆。这类方法省时省力,不需要通过训练就可以起效;但较为机械呆板,无法随机应变。一部分方法通过培养新习惯的方式起作用,比如说遇到麻烦环境时迅速脱离,在情绪不稳时使用吃喝、玩具等方式压惊。这类方法需要一定的培养成本,但面对常见情况时可以有效保持自己的理性,或是记录下来和咨询师分析,或是自己尝试解决和安抚。一部分方法通过维持分析能力的方式起作用,能达到这个水平已经不太需要心理咨询了。

\indicate{总的来说},想要让心理咨询充分发挥作用,自身理性是其中关键的一环,在很多事情上是不可或缺的。维持好自身理性,才能观察到自身更不明显,更不容易被记住的特点,才能系统性地梳理自身的长期行为表现。有了这些细致的认知以后,才能据此展开更详细深入的分析,从而更有效地自控。自身理性越是强大,这种良性循环的效率就越高;自身理性若是过于弱小,这种良性循环就无法建立,心理咨询也会失效。
\begin{explain}
一些人可以时刻维持自身理性,一些人可以在脱离之后迅速重新进入自身理性,一些人在某些环境内(独处、读书、和朋友交谈、心理咨询等)会进入自身理性,一些人可以制作环境(闹钟、纸条、背诵等)以激发自身理性。

在自身理性可以稳定存在的情况下,可以有意识地使用一些手段来增强自身理性的能力(比如心理咨询,再比如阅读书籍、写日记、规划目标、整理优缺点、自我鼓励),以实现更好的分析效果。自身理性不一定能理解这些手段的具体作用(于是它们严格来说属于污染),而是因信任而选择和使用。

理想情况下,随着接受心理咨询的时间逐渐累积,来访者的自身理性会逐渐成长,能够维持更长的时间,分析更深入的问题,还能在脱离之后重新进入。但实际情况中,自身理性在经历了一定的成长后,通常只能维持在较低的水平,有时还会因为遗忘、失落、不信任等原因而消解。只有在遇到某些关键契机以后,自身理性才会又开始一次新的成长。%这主要有三方面相互作用的原因:咨询时间较短(一周一两个小时,仅占清醒时间的1\%)、咨询师能力不足(参见之前的讨论)、自身理性能力不足(参见之前的讨论)。

这些关键契机的形式可能十分不同。有的是面对旧有问题时突然灵机一动,想起来了正确的方法并用上了;有的是在自己常规的方法都用尽的时候,才从记忆深处捞起来了理性;有的是和咨询师谈着谈着,突然想起了过去,获得了新认知;有的是面对重复的日常,突然决定不再继续维持,选择重新开始......

这些关键契机会给我们带来重要的新认知,从而明显影响我们旧有的行为模式。如果心理咨询师能明确识别并充分利用这些关键契机,就会对自身理性的培养起到显著的推动作用。
\end{explain}

\section{实操:原生家庭及其处理方式\label{sec:原生家庭}}
\hfill\begin{minipage}{0.65\textwidth}
\fontsize{8pt}{12pt}\selectfont\fontsize{8pt}{12pt}
\raggedright 我可爱的缺陷者,你的父母没能将你成功抚育。\footnote{原文为“私のかわいい欠落者、あなたの親は、あなたを育てるのに失敗した。”}

\raggedleft 石田翠《东京喰种:re》第52话 夏娃

\raggedright 继承了萨腾努斯的偏执,继承了威诺希的无知,可怜的孩子浑身都布满瑕疵。\footnote{\bilibili{av22271765};\\\indent \netease{554508692}。}

\raggedleft JUSF周存《十二号诛杀者》

\raggedright 一个人被神明剩下,祭奠枯萎的花。\footnote{\bilibili{av3360633}。\\}

\raggedleft 赛亚♂sya《被拯救者的物语》
\end{minipage}

%虽然我也不知道为什么,但近些年只要是心理相关的话题,都避免不了要提一提原生家庭。为了不让本指南显得过于不合群,还是展开来讲一讲吧。

原生家庭这个古老的概念在现代显得有些水土不服,主要原因是现代的孩子不完全生活在家里。一个孩子待在学校的时间比待在家里的时间多得多,在网上的沟通量比在家里的沟通量大得多。家庭仍然会给孩子显著影响,但是并不是所有的影响都是家庭带来的。据此,本指南仍然保留原生家庭这一名词,但在以下的讨论中,采用稍微不同的定义:\indicate{原生家庭}是\indicate{一个人会遇到的所有作为污染源的环境的统称}。
\begin{explain}
由这个定义,我们可以将讨论原生家庭化归为讨论污染。

注意此处对原生家庭的定义不包含其它形式的污染源。把“学校”和“圈子”称为原生家庭大概没什么问题,但是把“一句偶然看到的话”称为原生家庭就太怪了,把“自己得出的结论”称为原生家庭也看着不像人话。

在使用这种定义后,“工作环境”等一般属于成年人的环境也会被称为“原生家庭”。这听起来有点怪,毕竟“原生”这一前缀会给我们带来“出身处”的感觉,我们不太习惯把成年人当孩子看。但是考虑到工作场合确实也在散播“把公司当家”的污染,就其实还好。如果实在觉得奇怪,可以使用“原生学校”“原生公司”“原生圈子”等词汇代替。如果实在觉得“原生”这个词也很怪,可以将其视为心理方面为主的职业病。

读者如果喜欢,也可以把每种传播污染的环境分别单独称为原生家庭,这样一个人就会拥有(原始含义的)原生家庭、学校、公司、圈子等多个原生家庭。这种定义上的细微调整不会影响后续讨论。
\end{explain}
原生家庭的定义是中性的,但是实际应用时大多是贬义。以下篇幅不涉及“良好的原生家庭”的讨论。由于我们还未系统讨论环境对人产生影响的具体机制(相关讨论见\hyperref[sec:人的基本意识模型]{3.4节}),以下篇幅也将不涉及对原生家庭本身的分析,仅聚焦于“原生家庭带来的问题”及其处理方式。
\begin{examples}
由原生家庭带来的污染可以分为两类:一类是本身就没什么好处的;一类是本身在原生家庭之内有好处,但是在原生家庭之外会显得不适应环境的。而绝大多数污染,是这两者的混合。我们既因为污染的坏处而痛恨它,又因为污染的好处而继续保持。有利于家庭关系和睦、有利于自身成长、忍受不了别人的行为、看不下去别人的遭遇......因为这些理由,我们自我安慰,自我欺骗,从而在原生家庭的里外反复逡巡徘徊。
\end{examples}
如果你觉得上面这段话说得很有道理,让你深感共鸣,那说明你没有掌握本章的内容,没有理解本章介绍的分析框架。“好处/坏处”本身是认知,而认知依赖于思考回路,我们所感觉到的“既有好处又有坏处”,是不同的思考回路带来的。当你无法控制这些会给出矛盾结果的思考回路时,无论你根据哪一方做出决定,这个决定都不可能得到自己全心全意的赞同,也不可能被自己坚定地执行下去。如果你想清除原生家庭给你带来的污染,全面的考虑是不可或缺的。
\begin{explain}
清除污染的意思是理解并且控制。从实际效果来看,“控制至完全不触发”等状态也可视为清除污染。读者可根据自身需要调整定义。

清楚污染不是“和解”“接纳”“忍耐”等行为。这些可以作为手段之一,但不能是目标。
\end{explain}
不幸的是,只要你踏上清除污染的道路,就会立刻遇到地狱难度的开局。大部分思考回路没有理解和控制其它行为模式的能力,而如果一个思考回路是被污染而来的,那么想使用它们来理解和控制污染,更是几乎没有可能。
\begin{explain}
一些带有污染的思考回路可能对某些思考回路或者行为模式有强大的控制力。这些思考回路会高效产生坚定的错误认知,这些错误认知会使我们自我怀疑、自我否定、自我攻击、敏感、狂躁、偏执,从而甚至很难完成一些相当简单的事务(比如买水)。

很明显,这种带污染的思考回路才是最应该被控制或者消解的,但是无论是理解它还是控制它,都不是其它受污染的思考回路能做到的。
\end{explain}
另一方面,你所拥有的,也只是一个“想要解脱”“想要幸福”的愿望而已。这个愿望不会告诉你应该怎么达成它自己,不会告诉你怎么清除污染。更糟的是,每当你想起这个愿望的时候,就会产生“如果那样的话......”的幻想(这是一种作为认知的污染);每当你回归现实的时候,就会得出“果然我得不到......”的结论(这也是一种作为认知的污染)。这种恶性循环会掐灭每一次刚点燃的希望之火,你只能收获更多的痛苦。于是,很多人就这样放弃了挣扎,产生了“原生家庭不可逃脱”的认知。他们之中相当一部分完全没有意识到过“自己曾经挣扎过”。
\begin{examples}
网上和其它渠道有很多对于原生家庭的讨论。有些人是先放弃希望再看到讨论,有些人是先看到科普之后尝试并且失败。在放弃挣扎以后再看那些讨论,有可能觉得它们浅薄而苍白,有可能为自己感到浓郁的悲伤,有可能怨恨责怪自己周围的环境......这些不同的经历不在我们的讨论范围之内,这里不做展开。
\end{examples}
有些放弃了逃脱的人,可能还有“环境可以改善”的希望。一部分人会尝试沟通交流,一部分人会希望环境自己变好,一部分人会选择换个环境。但是,绝大多数人不会去关心环境实际上会是什么样。希望原有环境改变,但不去分辨和核实哪些东西可以改变,那些东西不能改变;希望来到新的环境,但不去了解和确认哪些东西确实不同,哪些东西其实一样。愿望能实现多少只能看运气,而大多数情况下是颗粒无收。
\begin{examples}
有些人可能会用身份、责任、感情等方式来劝别人做出改变。依照语境的不同,这可能有道德绑架、投射性认同、爱、操心、牢骚等多种称呼。但人们不是因为拥有某个身份就能履行某种责任的。如果缺失对应的能力,就无法履行责任。关于责任的讨论参见第三章。
\end{examples}
有些外部因素会调动你的愿望,带着你分析,得出一些结论。沿用上一小节的称呼,我们将你自身原有的理性称为\indicate{自身理性},将外部提供的分析能力称为\indicate{外部理性}。

在接下来的篇幅中,为了简便起见,当我们用到处于、进入、脱离等词时,会使用“\indicate{环境}”来统一指代“外部环境、行为模式、目标”;将“找到某个目标的解决方法”“改变某个环境”“维持某种状态”“消解一个带有污染的行为模式”等事务统称为“\indicate{应对}环境”,对应的处理方法称为\indicate{应对方法};将需要分析和应对的环境称为\indicate{目标环境}。

要想有效地清除自身的污染,首先需要维持自身理性的持久存在。这需要我们在有外部理性的情况下,尽可能将其内化为自我理性;在处于自我理性时尽量找资源来充实自身,同时尽量反思;在处于其它环境时尽可能维持理性,以分析和控制目标环境。
\begin{explain}
我们需要让自身理性维持以下这么几种特点:

\indicate{自身理性需要保持稳定。}我们需要尽量避免脱离自身理性。如果目标环境会使我们脱离自身理性,则需要事先想好应对的方式,并且在进入目标环境时能够使用。若我们处于目标环境时脱离了自身理性,应该尽快重新进入。这可以依靠一些外力的提醒,或者培养一些习惯。具体做法可以参考上一节关于自身理性的讨论,或是参考下方讨论。

\indicate{自身理性需要保持纯粹。}我们会同时处于很多种不同的思考回路之中。这些别的思考回路会干扰我们,产生带有污染的认知、情绪等失真的内容。相比于一般的行为模式,思考回路会更容易被触发,更难以控制,于是需要特别关注。除了控制思考回路本身以外,我们还需要将带有污染的认知和自身理性隔离开,避免干扰自己的思考和判断。我们需要坚定“清除污染”的目标,并以此来指导自身行动。

\indicate{自身理性需要保持记忆。}我们维持自身理性时,可以进行有效且深入的思考。这些思考会帮助我们应对目标环境。如果在面对目标环境时无法应用这些思考,或者是自身理性总是会思考同一件事,得出同样的认识,根本没意识到自己之前想过,或者是明明下定决心但是转头就忘了,那么自身理性就不起什么作用。我们可以使用一些辅助记忆的手段来避免这种情况。
\end{explain}
我们将“可以分析目标环境的应对方式,且不会失控地进入该目标环境”的情况称为该目标环境对应的\indicate{安全环境}。
\begin{examples}
依照具体目标环境的不同,安全环境所需要的条件从低到高可能是“处于该目标环境时”“处于某种知识体系中时”“处于自身理性时”“处于某种外部理性时”“不存在”。

一些特殊环境下可能有更复杂的情况。比如说在心理咨询时,咨询师有可能在来访者失控时有效地和目标行为模式对话,并且在来访者平复后将分析和结论反馈给来访者。后续的反馈环节对来访者是安全环境,但整体经历对来访者不是安全环境。安全环境根据人的不同(准确来说,根据自身理性的能力不同)而有所不同。
\end{examples}
一种目标环境的安全环境不一定是“远离该目标环境”。
\begin{examples}
远离目标环境不一定会使你可以分析这种环境,也不一定会使你不失控。“不在环境内,就想不起来对应的细节”的情况更多,于是你无法分析;“偶尔想起来的时候,会想回归这种环境”的情况更多,于是你因此失控。
\end{examples}
对于一般人来说,安全环境仅有“刚脱离目标环境,重新获得理性”时。只有在这种时候才能有鲜活不褪色的记忆和感受,只有在这种时候才能开始分析。我们将“在这种安全环境中分析”称为\indicate{立刻反思}。
\begin{explain}
并非所有行为模式都有脱离的机会,我们可能会一直处于一些行为模式中,尤其是那些可以进行自我评估的思考回路。

并非脱离了目标环境就能恢复理性。如果脱离环境以后去犒劳自己/散心/找人倾诉,那就错过了这次机会。

自身理性很强大的人可以做到在未脱离目标环境时就进入/保持理性,直接使用理性来分析自身,并且指挥自己的行动。但这难度较高。
\end{explain}
想要一脱离目标环境就立刻进入理性,对大部分人来说是很难的事。如果不能稳定地进入理性,就需要意志力来辅助。但之前身处目标环境中时,你可能已经耗尽了全部的力气,精神疲惫至极,没有劲再自己思考了。
\begin{explain}
此时如果有外部理性的帮助,就有可能可以展开有价值的分析,得到可以具体落实的改变。

在目标环境内,或者刚脱离目标环境时,如果能保持基本的理性以观察自身,并且记忆/记录下来具体的感受和行动,就会给后续分析起到相当大的帮助。这比立刻反思要容易一些,但仍然难度较高。
\end{explain}
如果无法立刻反思,那就需要使用一些手段,重新触发该行为模式,进而与之\indicate{对话}。这就需要一些专业技能来辅助完成。我们将“在这种安全环境中分析”称为\indicate{重新面对}。
\begin{explain}
这种操作有可能由心理咨询师主导,有可能看了一些心理书籍后习得了相应的方法,也有可能有些人无师自通。这总是需要一段连续的不受干扰的时间,如咨询、深夜、路上、开小差。

相比于立刻反思,重新面对是相对更常见的方法,我们可以不用卡时间,选择空闲平静的时间段再回顾。虽然分析的难度一般来说会更高,收效一般来说也会更差,但很多时候这是唯一选择。
\end{explain}
在安全环境下分析时,我们可以控制自己进入的目标环境,展开专注的分析。如果有某个环境会触发不可控的行为模式,那么我们需要将其\indicate{隔离}在外。
\begin{examples}
比如,在自己面对某个需要自己能力提升,同时有考核的任务时,有可能会产生“我果然什么都不行”的认知。它和生成它的思考回路会极大影响心情,消耗意志力。当没有考核的压力时,这种思考回路可能就不会触发。这使得我们可以分别处理“能力提升”和“我果然什么都不行”两个目标。

我们同一时间不一定只面对一种环境,不一定只处于一种行为模式之中。只有精准识别每一种行为模式,并且分别处理,才能真正解决问题。
\end{examples}
目标环境分为两种。一种是“缺乏应对的行为模式”,另一种是“会触发带有污染的行为模式”。
\begin{explain}
想应对“缺乏应对的行为模式”的环境,只需要拥有分析目标环境的能力,这种能力可以来自自身或者外界,只要该能力能够给出可行的应对方法,那就可以直接执行。注意,在此过程中需要隔离“我什么都不行”等思考回路,这属于带有污染的行为模式。
\end{explain}
想应对“会触发带有污染的行为模式”的环境,则还需要拥有控制或消解该行为模式的能力。这会困难得多。在安全环境下分析时,我们需要可控地触发或者模拟出我们需要消解的行为模式。这样才能随时打断自己的行为和思路,使用理性来思考“自己在当前情况下应该怎么做”。
\begin{explain}
如果不打断原有的行为模式,就会出现两方面问题:

原有的行为模式可能很占用精力,以至于无法完成思考。有些思考有时会很耗时间,完全不可能靠临场反应得出结论,只能提前想清楚才用得上;有些思考可以靠临场反应,但是如果被分了神就不行。

原有的行为模式可能会提供混乱的认知,以干扰判断。这有可能是因为认知本身错误或带有污染,有可能是因为不同行为模式下的目标不同(并且当前行为模式不知道如何实现该目标,也不知道自己的方法无效)。
\end{explain}
在安全环境下分析时,我们应该尽量根据已知的所有信息,充分考虑可能出现的每种情况,并尽可能想出全面可行的计划。
\begin{explain}
如果有必要,还可以加入一些模拟演练的环节,以增加熟练程度。一开始没有用上新方法没关系,在安全环境下多试几次即可。

一些情况下,确定自己的目标后,就可以据此确定哪些事情应该重点关注,哪些事情不需要管,并且据此来控制自身的行为。但这难度较高。
\end{explain}
分析不一定能给出有效的应对方法。这有两种可能:\indicate{信息缺失}或\indicate{能力不足}。
\begin{explain}
有时在分析时就能直接确定无法得出应对方法;有时在再次面对目标环境时才能发现应对方法行不通。

信息缺失有多种可能,如“实在想不起来具体情况”、“不知道某些做法会带来什么后果”等。这种情况下做的计划不保证能达成自己想要的结果,具体执行时可能出现意料之外的情况。

绝对不应该在信息不足的情况下,给出确定性的判断,期待某种特定结果会发生。无论是觉得“这么做肯定会顺利”还是觉得“这么做肯定会搞砸”都不应该。这种混乱的认知会干扰你的判断。信息不足的情况下,应该对每种可能出现的结果都做好充足的准备,但这难度较高。

如果你能够承受可能带来的结果(可能是经济、关系、情绪等多方面),就可以主动去尝试以获取信息,但这难度较高。无论成功还是失败,尝试都可以为我们提供信息,从而进一步分析、判断、计划。这种情况下,我们可以说\indicate{失败是成功之母}。另一方面,如果失败不能提供信息,或者失败会造成大量损失,那就不应主动失败。
\end{explain}
如果在安全环境下确实分析到位了,那么当你再次面对目标环境时,你应该能按照计划好的方式思考和行动,得到意料之中的结果,实现想要达成的目标。
\begin{explain}
有可能你的意志力会慢慢耗尽,从而你只能在开头的一小段时间内保持理性,而后又会陷入旧有的行为模式中。但这也能说明你使用的方法是正确的,只需要继续提高自己的熟练度,让新习惯覆盖老习惯即可。只有当持续时间总是没有提升,总是会因为相同的原因而失控时,才需要另行处理。
\end{explain}
如果总是无法按计划行事,总是无法达成目标,或者在分析阶段就无处下手,那么就基本可以确定是\indicate{能力不足}。
\begin{explain}
我们以为的安全环境只能做到不失控,而无法做到分析,它实际上不是安全环境。一般有两种情况:要不然是因观察不足、条件限制等原因,无法模拟目标环境(比如说不见到陌生人就没那么害怕),要不然是因能力不足,无法分析并得到结论。

如果有条件,我们可以借助外部环境的帮助,从而补齐展开分析的条件。可以通过他人、录像、日记、随笔等方式来辅助观察;可以通过心理咨询、自己学习等方式来提升能力。

特别需要注意的是,在明确了自身\indicate{能力不足}后,\indicate{不要草率地反复尝试}(除非你想要通过尝试来获取信息)。初期的尝试需要消耗意志力,而挫败感会极大损耗意志力,加深恐惧和无力感。面对一种棘手的环境,在自己怕得无法迈出步伐之前,我们\indicate{只有少数几次机会来尝试新方法,一定不要浪费}。
\end{explain}
如果以上所有方法都无效,那么我们“使用理性,立即化解污染”的尝试即可宣告失败。我们需要使用更加曲折的方式来解决:培养一个新的行为模式。
\begin{explain}
这个新行为模式大部分情况下是污染,这是一种使用污染来对抗污染的手段。
\end{explain}
实际操作起来没有听上去那么糟:我们需要接触一个新环境,在新环境中培养新的行为模式。新环境和目标环境之间有一定的相似之处,但它不会触发/不会频繁触发原有的行为模式。新的行为模式同样可以应对目标环境,其中的行为和行为链会逐渐代替旧有行为模式中的行为和行为链。当新行为模式足够熟练后,旧有行为模式或是无法再被触发,或是触发后无法形成行为链,于是就被消解了。
\begin{examples}
由于换个新环境是很常见的事情,这种方式会自觉或不自觉地被人使用。即使不理解原理,换了新环境后,也有可能自然地感觉到了自由的空气。

很多人上大学之前会觉得家里住着没什么问题,但一个学期以后,就只会感觉到发现家里待着很不舒服;农村的孩子来到城市见过世面以后,就很难再待在自己村子里。

能清晰地感觉到不舒服,就可以使用理性来解决后续问题了。此时也会有一些人选择远离原有的环境,本指南不对此做道德和责任方面的评价,也不将其简单概括为“由俭入奢易,由奢入俭难”等顺口溜。我们只关心最普遍的“因为行为模式的变更,态度和行为产生了变化”的现象。
\end{examples}
这种操作有两方面要求:

一方面是\indicate{新环境需要提供足够多的触发条件}。我们不应该自己思考“如何处理新环境”,不然思考深了就会回到旧有思考回路中(因此会培养出带有污染的新行为模式)。等新行为模式足够熟练后,才有深入思考的机会。

另一方面是\indicate{需要尽量避免触发旧行为模式}。我们应该尽量减少对比新旧环境、新旧行为模式的可能性。如果条件允许,那么直接远离旧环境;如果条件不允许,则应尽量减少自己对旧环境的反应。

达成这两方面条件本身就有一定的难度。即使这两方面都能达成,这种方法也有隐患:我们为自己增加了一个新的原生家庭,受到了新的污染。
\begin{explain}
如果事先调查或了解过,就可以清晰地认识某种环境,清楚环境会培养出什么样的行为模式。这种情况下可以不受什么污染地进入该环境,得到自己想要的。常见的环境可能包括“好学校”“好工作”“自己的房子”“独居”“同居”“心理咨询”之类。

但是,如果只是对环境有一厢情愿的幻想,就有可能会相当吃亏。
\end{explain}
在一些特定情况下,旧行为模式会被巧合性地控制住。在旧环境和旧行为模式的互动中,我们会培养出一些新的行为,这些行为有可能会避免旧行为模式的触发。我们将这种行为称为旧行为模式的\indicate{掩盖}。
\begin{examples}
我们可能在大哭一场后就想不起来之前都受过什么委屈了;我们可能面对某些事情时会害怕但是不知道自己为什么怕;我们可能自杀以后就丢掉了几个月以来的记忆......

如果读者喜欢,也可以将其称为可\indicate{大脑的防御机制/保护机制}。

\indicate{掩盖}的形式十分多样。有可能是刻意忽视了某些信号;有可能是熟练地经历了某个思考回路后,仅输出结果认知;有可能是强行转移注意力......基本上任何形式的行为都有可能成为掩盖。
\end{examples}
“会触发旧行为模式”和“不会触发旧行为模式”相比,“不会触发旧行为模式”相对更有利一些。但掩盖也是严重的污染,需要处理。
\begin{explain}
在时间经过充分久,行为模式改变充分大以后,旧行为模式有可能已经自然消解,此时就可以使用理性来重新发现旧行为模式,将旧行为模式和掩盖一并清除。

如果读者喜欢,也可以将其称为\indicate{看到内在孩童}、\indicate{觉知自己的潜意识}或其它。
\end{explain}
至此,我们已经基本上讨论完了所有清除污染的方式:要不然主动控制,要不然等它逐渐消解。因为篇幅所限,此处的讨论不够具体,更详细的展开参见之后的内容。

拥有成熟自我的标志,就是\indicate{不再受到新的污染}。只要做到这一点,已有的污染就可以慢慢解决,你也就会越来越自由。